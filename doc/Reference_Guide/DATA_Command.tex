% Sandia National Laboratories is a multimission laboratory managed and
% operated by National Technology & Engineering Solutions of Sandia, LLC, a
% wholly owned subsidiary of Honeywell International Inc., for the U.S.
% Department of Energy’s National Nuclear Security Administration under
% contract DE-NA0003525.

% Copyright 2002-2022 National Technology & Engineering Solutions of Sandia,
% LLC (NTESS).


User-defined data table, which can be used to specify sweep points for \texttt{.AC}, \texttt{.DC}, \texttt{.NOISE} or \texttt{.STEP}

\begin{Command}

\format
.DATA [<name>] \\
+ <parameter name> [parameter name]* \\
+ <parameter value> [parameter value]* \\
.ENDDATA

\examples
.data test \\
+ r1   r2 \\
+ 8.0000e+00  4.0000e+00 \\
+ 9.0000e+00  4.0000e+00 \\
.enddata

\arguments

\begin{Arguments}
\argument{name}
Name of the data table.

\argument{parameter name}
Name of sweep parameter.  This can be a device instance parameter, 
  a device model parameter or a user-defined parameter specified using \texttt{.GLOBAL\_PARAM} or a globally scoped \texttt{.PARAM} statement.

\argument{parameter value}
Value of sweep parameter for the given sweep point.  This must be a double precision number.  Each row of the table corresponds to a different sweep step, so multiple parameters can be changed simultaneously.

\end{Arguments}

\comments

Each column of a data table corresponds to a different parameter, and each row corresponds to a different sweep point.

If using \texttt{.DATA} with \texttt{.DC} or \texttt{.STEP}, then any instance parameter, model parameter, 
or user-defined parameter in the global scope.

However, if using \texttt{.DATA} with \texttt{.AC} or \texttt{.NOISE}, then one can sweep the magnitude and phase of an AC source, and linear model parameters (such as resistance and capacitance) in addition to the traditional AC sweep variable, frequency.  Parameters associated with nonlinear models (like transistors) are not allowed.  This is because AC analysis is a linear analysis, performed after the DCOP calculation.  Changing nonlinear device model parameters would result in a different DCOP solution, so changing them during the AC (or NOISE) analysis phase is not valid.

Another caveat, for both \texttt{.AC} and \texttt{.NOISE}, is that all of the frequency
values in the data table must be positive.  If \texttt{.DATA} is used with \texttt{.NOISE}
then the integrals for the total input noise and total output noise will only be calculated,
and sent to stdout, if the frequencies in the data table are monotonically increasing.

\end{Command}
