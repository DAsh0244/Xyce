% Sandia National Laboratories is a multimission laboratory managed and
% operated by National Technology & Engineering Solutions of Sandia, LLC, a
% wholly owned subsidiary of Honeywell International Inc., for the U.S.
% Department of Energy’s National Nuclear Security Administration under
% contract DE-NA0003525.

% Copyright 2002-2021 National Technology & Engineering Solutions of Sandia,
% LLC (NTESS).

\label{modelCommand}
The \texttt{.MODEL} command provides a set of device parameters to be
referenced  by device instances in the circuit.

\begin{Command}

\format
.MODEL <model name> <model type> (<name>=<value>)*

\examples
\begin{alltt}
.MODEL RMOD R (RSH=1)
.MODEL MOD1 NPN BF=50 VAF=50 IS=1.E-12 RB=100 CJC=.5PF TF=.6NS
.MODEL NFET NMOS(LEVEL=1 KP=0.5M VTO=2V)
\end{alltt}

\arguments

\begin{Arguments}

\argument{model name}
The model name used to reference the model.

\argument{model type}

The model type used to define the model.  This determines if the model
is (for example) a resistor, or a MOSFET, or a diode, etc.  For
transistors, there will usually be more than one type possible, such as
NPN and PNP for BJTs, and NMOS and PMOS for MOSFETs.

\argument{\vbox{\hbox{name\hfil}\hbox{value}}}

The name of a parameter and its value.  Most models will have a list of
parameters available for specification.  Those which are not set will
receive default values.  Most will be floating point numbers, but some
can be integers and some can be strings, depending on the definition of
the model.

\end{Arguments}

\comments
\index{\texttt{.MODEL}!subcircuit scoping}The scoping rules for models are:
\begin{XyceItemize}
\item If a \texttt{.MODEL}, statement is included in the main circuit 
netlist, then it is accessible from the main circuit and all subcircuits. 
\item \texttt{.MODEL} statements defined within a subcircuit are scoped 
to that subciruit definition.  So, their models are only accessible within 
that subcircuit definition, as well as within ``nested subcircuits'' also 
defined within that subcircuit definition.
\end{XyceItemize}

Additional illustative examples of scoping are given in the
``Working with Subcircuits and Models'' section of the \Xyce{} Users' 
Guide\UsersGuide. 

A model name can be the same as a device name in \Xyce{}.  However, that
usage will generate a warning message during netlist parsing.  
The reason is that it can lead to ambiguous \texttt{.PRINT} lines 
when a model parameter and instance parameter, for a given device, have
the same name but a different meaning.  For example, \texttt{R1} 
could be used as both a resistor device-name, and as a resistor model-name.
However, \texttt{.PRINT TRAN R1:R} would then be ambiguous.  In addition,
the use of duplicate model and device names is not recommended if those
names will be used within a \Xyce{} expression since that can result in 
an ambiguous expression.

\end{Command}

\subsubsection{\textrmb{LEVEL} Parameter}
\index{model!level parameter}
\index{level parameter}

A common parameter is the \textrmb{LEVEL} parameter, which is set to an
integer value.  This parameter will define exactly which model of the
given type is to be used.  For example, there are many different
available MOSFET models.  All of them will be specified using the same
possible names and types.  The way to differentiate (for example)
between the BSIM3 model and the PSP model is by setting the appropriate
\textrmb{LEVEL}.

\subsubsection{Model Binning}
\index{model!binning}
\index{model binning}

Model binning is supported for MOSFET models.  For model binning, the netlist 
contains a set of similar \texttt{.MODEL} cards which correspond to different 
sizing information (length and width).  They are similar in that they are for the same
model (and same \texttt{LEVEL} number), and have the same prefix.  They are different in that 
they have different \texttt{lmin,lmax,wmin,wmax} parameters, and the name suffix will be
the bin number.  For a MOSFET device instance, \Xyce{} will automatically select the
appropriate binned model, based on the \texttt{L} and \texttt{W} parameters of that 
instance.   It will only seach the models with matching name prefixes, and can only work if
all the binned models have specified all the \texttt{lmin,lmax,wmin,wmax} parameters.

Model binning is not enabled by default.  To enable it, it is necessary to 
specify \texttt{.options parser model\_binning=true}.

\begin{figure}[htbp]
  \begin{centering}
    \shadowbox{
      \begin{minipage}{0.9\textwidth}
        \begin{vquote}
\color{blue}* Model binning example adapted from the BSIM4 test suite\color{black}
m1 2 1 0 b nch L=0.11u W=10.1u NF=5 rgeomod=1 geomod=0
vgs 1 0 1.2
vds 2 0 1.2
Vb b 0 0.0

.dc vds 0.0 1.21 0.02 vgs 0.2 1.21 0.2

.print dc v(2) v(1) i(vds)

* model binning
.model nch.1 nmos(level=14 lmin=0.1u lmax=20u wmin=0.1u wmax=10u)
.model nch.2 nmos(level=14 lmin=0.1u lmax=20u wmin=10u  wmax=100u)

.options parser model\_binning=true

.end
\end{vquote}
\end{minipage}
}
\caption[Model Binning Example]
{Model Binning Example\label{binningExample} }
\end{centering}
\end{figure}

\subsubsection{Model Interpolation}
\label{Model_Interpolation}
\index{model!model interpolation}
\index{model!tempmodel}

\textbf{NOTE:  The temperature interpolation model described in this section has been deprecated and may be removed in a future version of \Xyce{}}.

Traditionally, SPICE simulators handle thermal effects by coding
temperature dependence of model parameters into each device.  These
expressions modify the nominal device parameters given in
the \texttt{.MODEL} card when the ambient temperature is not equal
to \textrmb{TNOM}, the temperature at which parameters were extracted.

These temperature correction equations may be reasonable at temperatures close
to \textrmb{TNOM}, but Sandia users of \Xyce{} have found them inadequate when
simulations must be performed over a wide range of temperatures.  To address
this inadequacy, \Xyce{} implements a model interpolation option that allows
the user to specify multiple \texttt{.MODEL} cards, each extracted from real
device measurements at a different \textrmb{TNOM}.  From these model cards,
\Xyce{} will interpolate parameters based on the ambient temperature using
either piecewise linear or quadratic interpolation.

Interpolation of models is accessed through the model parameter
\textrmb{TEMPMODEL} in the models that support this capability.  In the
netlist, a base model is specified, and is followed by multiple models at other
temperatures.  

Interpolation of model cards in this fashion is implemented in the BJT level 1,
JFET, MESFET, and MOSFETS levels 1-6, 10, and 18.

The use of model interpolation is best shown by example:

\begin{vquote}
Jtest 1a 2a 3 SA2108 TEMP= 40
*
.MODEL SA2108 PJF ( TEMPMODEL=QUADRATIC TNOM = 27
+ LEVEL=2 BETA= 0.003130 VTO = -1.9966 PB = 1.046
+ LAMBDA = 0.00401 DELTA = 0.578; THETA = 0;
+ IS = 1.393E-10          RS = 1e-3)
*
.MODEL SA2108 PJF ( TEMPMODEL=QUADRATIC TNOM = -55
+ LEVEL=2 BETA = 0.00365 VTO = -1.9360 PB = 0.304
+ LAMBDA = 0.00286 DELTA = 0.2540 THETA = 0.0
+ IS = 1.393E-10 RD = 0.0 RS = 1e-3)
* 
.MODEL SA2108 PJF ( TEMPMODEL=QUADRATIC TNOM = 90
+ LEVEL=2 BETA = 0.002770 VTO = -2.0350 PB = 1.507
+ LAMBDA = 0.00528 DELTA = 0.630 THETA = 0.0
+ IS = 1.393E-10          RS = 5.66)
\end{vquote}

Note that the model names are all identical for the three \texttt{.MODEL}
lines, and that they all specify \texttt{TEMPMODEL=QUADRATIC}, but with
different \textrmb{TNOM}.  For parameters that appear in all three
\texttt{.MODEL} lines, the value of the parameter will be interpolated using
the \texttt{TEMP=} value in the device line, which in this example is
40$^\circ$C, in the first line.  For parameters that are not interpolated, such
as \textrmb{RD}, it is not necessary to include these in the second and third
\texttt{.MODEL} lines.

The only valid arguments for \textrmb{TEMPMODEL} are \textrmb{QUADRATIC} and
\textrmb{PWL} (piecewise linear).  The quadratic method includes a limiting
feature that prevents the parameter value from exceeding the range of values
specified in the \texttt{.MODEL} lines.  For example, the \textrmb{RS} value in
the example would take on negative values for most of the interval between -55
and 27, as the value at 90 is very high.  This truncation is necessary as
parameters can easily take on values (such as the negative resistance of
\textrmb{RS} in this example) that will cause a \Xyce{} failure.

With the BJT parameters \textrmb{IS} and \textrmb{ISE}, interpolation is done
not on the parameter itself, but on the the log of the parameter, which
provides for excellent interpolation of these parameters that vary over many
orders of magnitude, and with this type of temperature dependence.

The interpolation scheme used for model interpolation bases the interpolation
on the difference between the ambient temperature and the \textrmb{TNOM} value
of the first model card in the netlist, which can sometimes lead to poorly
conditioned interpolation.  Thus it is often best that the first model card in
the netlist be the one that has the ``middle'' \textrmb{TNOM}, as in the
example above.  This ensures that no matter where in the range of temperature
values the ambient temperature lies, it is a minimal distance from the base
point of the interpolation.
