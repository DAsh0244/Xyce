% Sandia National Laboratories is a multimission laboratory managed and
% operated by National Technology & Engineering Solutions of Sandia, LLC, a
% wholly owned subsidiary of Honeywell International Inc., for the U.S.
% Department of Energy’s National Nuclear Security Administration under
% contract DE-NA0003525.

% Copyright 2002-2020 National Technology & Engineering Solutions of Sandia,
% LLC (NTESS).


%%
%% PDE Device Instance Parameter table.
%%

\index{PDE Devices!instance parameters}
\label{PDE_Instance_Params}
\begin{DeviceParamTable4}{PDE Device Instance Parameters.}

\category{All Levels} \\ \hline

name & The instance name must start with a Z. & -- & -- & 1D, 2D \\ \hline

node & Minimum of 2 connecting circuit nodes. The 2D device may have as
many as 4 nodes, while the 1D device can only have 2.  The node parameter
is a tabular parameter, which specifies all the electrode attributes.  
See table~\ref{PDE_Electrode_Params} for a list. & -- & -- & 1D, 2D \\
\hline
region & Specifies doping regions.  Like the node parameter, this 
is a tabular parameter, containing several attributes..  See 
table~\ref{PDE_Doping_Params} for a list.  & -- & -- & 1D, 2D \\
\hline

area & Cross sectional area of the device. & -- & 1.0 & 1D, 2D \\ \hline

tecplotlevel & Setting for Tecplot output:\par
0 -- no Tecplot files \par
1 -- Tecplot files, each output in a separate file. \par
2 -- Tecplot file, each output appended to a single file. \par
\par\par
Tecplot files will have the \verb+.dat+ suffix, and the 
prefix will be the name of the device instance & -- & 1 & 1D, 2D \\ \hline

gnuplotlevel & Flag for gnuplot output.\par
0 -- no gnuplot files.\par
1 -- gnuplot files.\par
gnuplot is an open source plotting program that is usually installed 
on Linux systems.  gnuplot files will have the 
*Gnu.dat suffix, and the  prefix will be the name of the device instance. 
& -- & 0 & 1D, 2D \\ \hline

txtdatalevel & Flag for volume-averaged text output.\par
0 -- no text files.\par
1 -- text files.\par
txtdataplot files will have the *.txt suffix, and the prefix will be 
the name of the device instance. & -- & 0 & 2D \\ \hline

bulkmaterial & Material of bulk material. & -- & si & 1D, 2D \\ \hline

mobmodel & mobility model. & -- & carr & 1D, 2D \\ \hline

type & P-type or N-type -- this is only relevant if using the default 
dopings & -- & PNP & 1D, 2D \\ \hline

temp & Temperature & K & 300.15 & 1D, 2D \\ \hline

\index{mesh}
nx & Number of mesh points, x-direction. & -- & 11 & 1D, 2D \\ \hline

l, w & Device length and width.  These parameters mean the same thing
for the 1D device. & -- & 1.0e-3 & 1D,2D \\ \hline

graded & Flag for graded junction vs. abrupt junction. (1=graded, 0=abrupt)
& -- & 0 & 1D \\ \hline

wj & Junction width. & -- & 1.0e-4 & 1D \\ \hline

\category{Level 2 (2D) only}\\
\hline\hline

ny & Number of mesh points, y-direction.  Similar to nx (see above). & -- & 11 & 2D \\ \hline


\index{mesh}
\debug{meshfile} & 
\debug{This is a required field for a 2D simulation.  If the user
specifies \texttt{meshfile = internal.mesh}, then \Xyce{} will create a
cartesian mesh.  If the user specifies anything else (for example
\texttt{meshfile = diode.msh}), \Xyce{} will attempt to read in an 
external mesh file (in the example, named \texttt{diode.msh}) which is
in the format of the SG Framework~\cite{kramer-hitchon:1997}.} & \debug{--} & \debug{--} &
\debug{2D} \\ \hline

\debug{x0} & 
\debug{This is the scaling factor for length.  The code will do all of its
scaling internally, so it is generally not necessary to specify it manually.
This is provided primarily for testing purposes.} & 
\debug{--} & 
\debug{max length of device} &
\debug{2D} \\ \hline

\end{DeviceParamTable4}
