% Sandia National Laboratories is a multimission laboratory managed and
% operated by National Technology & Engineering Solutions of Sandia, LLC, a
% wholly owned subsidiary of Honeywell International Inc., for the U.S.
% Department of Energy’s National Nuclear Security Administration under
% contract DE-NA0003525.

% Copyright 2002-2020 National Technology & Engineering Solutions of Sandia,
% LLC (NTESS).


\begin{Device}\label{B_DEVICE}

\device
\begin{alltt}
B<name> <(+) node> <(-) node> V={ABM expression} [device parameters]
B<name> <(+) node> <(-) node> I={ABM expression}
\end{alltt}

\examples
\begin{alltt}
B1 2 0 V=\{sqrt(V(1))\}
B2 4 0 V=\{V(1)*TIME\}
B3 4 2 I=\{I(V1) + V(4,2)/100\}
B4 5 0 V=\{Table \{V(5)\}=(0,0) (1.0,2.0) (2.0,3.0) (3.0,10.0)\}
B5 6 0 V=tablefile("file.dat")
B6 7 0 I=tablefile("file.dat")
\end{alltt}

\comments

The nonlinear dependent source device, also known as the B-source
device, is used in analog behavioral modeling (ABM).  The \texttt{(+)}
and \texttt{(-)} nodes are the output nodes. Positive current flows from
the \texttt{(+)} node through the source to the \texttt{(-)}
node. 

The power supplied or dissipated by the nonlinear dependent source is calculated 
with $I \cdot \Delta V$ where the voltage drop is calculated as $(V_+ - V_-)$ 
and positive current flows from $V_+$ to $V_-$.  Dissipated power has a
positive sign, while supplied power has a negative sign.

The syntax involving the \texttt{tablefile} keyword internally attempts to load the
data in \texttt{"file.dat"} into a \texttt{TABLE} expression.  The data file must
be in plain-text and contain just two pairs of data per line.  For an example see 
the ``Analog Behavioral Modeling'' chapter of the \Xyce{} User's
Guide.

It is important to note that the B-source allows the user to specify
expressions that could have infinite-slope transitions, such as the
following.  (Note: the braces surrounding all expressions are required in this definition.)
\begin{alltt} Bcrtl OUTA 0 V=\{ IF( (V(IN) > 3.5), 5, 0 ) \} \end{alltt}
This can lead to ``timestep too small'' errors when \Xyce{} reaches the
transition point.  Infinite-slope transitions in expressions dependent only on
the \texttt{time} variable are a special case, because \Xyce{} can detect that
they are going to happen in the future and set a ``breakpoint'' to capture
them.  Infinite-slope transitions depending on other solution variables cannot
be predicted in advance, and cause the time integrator to scale back the
timestep repeatedly in an attempt to capture the feature until the timestep is
too small to continue.

One solution to the problem is to modify the expression to allow a continuous transition. 
However, this can become complicated with multiple inputs. The other solution is to specify
device options or instance parameters to allow smooth transitions. The parameter
\texttt{smoothbsrc} enables the smooth transitions. This is done by adding a RC network to the  
output of B sources. For example,

\begin{alltt} Bcrtl OUTA 0 V=\{ IF( (V(IN) > 3.5), 5, 0 ) \} smoothbsrc=1 \end{alltt}

\begin{alltt} .options device  smoothbsrc=1 \end{alltt}

The smoothness of the transition can be controlled by specifying the rc constant of 
the RC network. For example, 

\begin{alltt} Bcrtl OUTA 0 V=\{ IF( (V(IN) > 3.5), 5, 0 ) \} smoothbsrc=1   
 + rcconst = 1e-10 \end{alltt}

Note that this smoothed B-source only applies to voltage sources. The voltage behavioral source supports
two instance parameters \texttt{smoothbsrc} and \texttt{rcconst}. Parameters may be provided as space  
separated \texttt{<parameter>=<value>} specifications as needed. The default value for \texttt{smoothbsrc}
is 0 and the default for \texttt{rcconst} is 1e-9.

See the ``Analog Behavioral Modeling'' chapter of the \Xyce{} User's
Guide for guidance on using the B-source device and ABM expressions,
and the Expressions Section (\ref{ExpressionDocumentation}) for
complete documentation of expressions and expression operators.
One important note is that time-dependent expressions are supported
for the current and voltage parameters of a B source, but
frequency-dependent expressions are not.

B-sources were originally developed primarily to support DC and transient analysis.  
As such, their support for frequency domain analysis (AC and HB) has some limitations.  
The main limitation to be aware of is that time-dependent sources will not work with AC or HB analysis.  
These are sources in which the variable \texttt{TIME} is used in the \texttt{VALUE=} expression. 
The use case of a purely depedent B-source (depends only on other solution variables) will work with AC and HB.  

\end{Device}
