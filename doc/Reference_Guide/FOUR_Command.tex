% Sandia National Laboratories is a multimission laboratory managed and
% operated by National Technology & Engineering Solutions of Sandia, LLC, a
% wholly owned subsidiary of Honeywell International Inc., for the U.S.
% Department of Energy’s National Nuclear Security Administration under
% contract DE-NA0003525.

% Copyright 2002-2019 National Technology & Engineering Solutions of Sandia,
% LLC (NTESS).



Performs Fourier analysis of transient analysis output.

\begin{Command}

\format
.FOUR <freq> <ov> [ov]*

\examples
\begin{alltt}
.FOUR 100K v(5)
.FOUR 1MEG v(5,3) v(3)
.FOUR 20MEG  SENS
.FOUR 40MEG  \{v(3)-v(2)\}
\end{alltt}

\arguments

\begin {Arguments}

\argument{freq}
The fundamental frequency used for Fourier analysis.
Fourier analysis is performed over the last period (\texttt{1/freq}) of the transient simulation.
The DC component and the first nine harmonics are calculated.  

\argument{ov}
The desired solution output, or outputs, to be analyzed. Fourier analysis can be performed
on several outputs for each fundamental frequency, \texttt{freq}. At least one
output must be specified in the {\tt .FOUR} line.  The available outputs are:

\begin{itemize}
\item \texttt{V(<circuit node>)} the voltage at \texttt{<circuit node>}
\item \texttt{V(<circuit node>,<circuit node>)} to output the voltage difference between the first \texttt{<circuit node>} and second \texttt{<circuit node>}  
\item \texttt{I(<device>)} the current through a two terminal device
\item \texttt{I<lead abbreviation>(<device>)} the current into a particular lead of a three or more terminal device (see the Comments, below, for details)
\item \texttt{N(<device parameter>)} a specific device parameter (see the individual devices in Section~\ref{Analog_Devices} for syntax)
\item \texttt{SENS} transient direct sensitivities (see Section~\ref{SensitivityAnalysis} for more details about setting up the \texttt{.SENS} command)
\end{itemize}

\end{Arguments}

\comments
Multiple \texttt{.FOUR} lines may be used in a netlist.  All results from Fourier analysis will be 
returned to the user in a file with the same name as the netlist file suffixed with a \texttt{.FOUR}. 

\texttt{<lead abbreviation>} is a single character designator for individual
leads on a device with three or more leads.  For bipolar transistors these are:
c (collector), b (base), e (emitter), and s (substrate).  For mosfets, lead
abbreviations are: d (drain), g (gate), s (source), and b (bulk).  SOI
transistors have: d, g, s, e (bulk), and b (body).  For PDE devices, the nodes
are numbered according to the order they appear, so lead currents are
referenced like I1(\texttt{<device>}), I2(\texttt{<device>}), etc.

For this analysis, the phase data is always output in degrees.

\end{Command}

