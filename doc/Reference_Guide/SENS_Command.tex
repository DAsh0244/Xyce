% Sandia National Laboratories is a multimission laboratory managed and
% operated by National Technology & Engineering Solutions of Sandia, LLC, a
% wholly owned subsidiary of Honeywell International Inc., for the U.S.
% Department of Energy’s National Nuclear Security Administration under
% contract DE-NA0003525.

% Copyright 2002-2021 National Technology & Engineering Solutions of Sandia,
% LLC (NTESS).


\index{\texttt{.SENS}}
\index{results!sens} \index{sensitivity}
Computes sensitivies for a user-specificed objective function
with respect to a user-specified list of circuit parameters.  


\begin{Command}

\format
.SENS objfunc=<output expression(s)> param=<circuit parameter(s)>  

\examples
\begin{alltt}
.SENS objfunc=\{0.5*(V(B)-3.0)**2.0\} param=R1:R,R2:R
.options SENSITIVITY direct=1 adjoint=1

.SENS objfunc=\{I(VM)\},\{V(3)*V(3)\} param= Q2N2222:bf

.param RES=1k
.SENS objfunc=\{RES*V(3)*V(3)\} param=C1:C

.param res=2
.func powerTestFunc(I) \{res*I*I\}
.SENS objfunc=\{powerTestFunc(I(V1))\} param=R1:R

.global\_param res=2
.SENS objfunc=\{res*I(V1)\} param=R1:R

.global\_param res=3.0k
.SENS objfunc=\{res*I(V1)\} param=res

* AC example using objvars
.sens objvars=2,3 param=r1:r,c1:c,v1:acmag

* AC example using acobjfunc
.sens acobjfunc=\{2.0*V(2)\},\{I(VM)\} param=r1:r,c1:c
\end{alltt}

\comments

This capability can be applied to either DC, transient or AC analysis.  
Both direct and adjoint sensitivities are supported. 
The user can optionally request either direct or adjoint sensitivities, 
or both.  

Although \Xyce{} will allow the user to specify both direct and 
adjoint, one would generally not choose to do both.
The best choice of sensitivity method depends on the problem.  For problems 
with a small number of parameters, and (possibly) lots of objective functions, 
then the direct method is a more efficient choice.  For problems with large 
numbers of parameters, but a small number of objective functions, the 
adjoint method is more efficient.

For all variants of sensitivity analysis, it is necessary to specify 
circuit parameters on the \texttt{.SENS} line in a comma-separated list.  Unlike the SPICE version, 
this capability will not automatically use every parameter in the circuit.
It is also necessary for all variations of sensitivity analysis to specify at least 
one objective function.  This capability will not assume any particular 
objective function.  Also, it is possible to specify multiple
objective functions, in a comma-separated list.

As noted, for transient analysis, both types of sensitivities are supported.
Direct sensitivities are computed at each time step during the forward 
calculation.  Transient adjoint sensitivities, in contrast, must be computed
using a reverse time integration method.  The reverse time integration must be 
performed after the original forward calculation is complete.  As such, transient 
adjoint sensitivity calculations can be thought of as a post-processing step.  
One consequence of this is that transient adjoint output must be specified using 
the \texttt{.PRINT TRANADJOINT} type, rather than the \texttt{.PRINT SENS} 
type.

If transient adjoints are specified, the default behavior for the capability is 
for a transient sensitivity calculation be performed for each time step, even 
if the forward transient simulation consists of millions of steps.  For adjoint 
calculations, this can be problematic, as adjoint methods (noted above) are not 
very efficient when applied to problems with a large number of objective functions.
Each time step, from the point of view of transient adjoints, is effectively a 
separate objective function.  As such, this isn't the best use of adjoints.  
One can specify  a list of time points for which to compute transient adjoint
sensitivities. For many practical problems, the sensitivies at only one or a 
handful of points is needed, so this is a good way to mitigate the computational 
cost of adjoints.  The \Xyce{} Users' Guide~\UsersGuide{} provides an example.

If performing a sensitivity calculation with AC analysis, 
there are two options for the specification of the 
objective function. These options are both different from the DC and TRAN method.
Instead of specifying objective functions with the parameter \texttt{objfunc},
one should either use \texttt{objvars} or \texttt{acobjfunc}.  
The parameter \texttt{objvars} should be followed by a comma separated list of voltage nodes.
The parameter \texttt{acobjfunc} should be followed by a comma separated list of objective functions.
It is not possible to use both specifications in the same netlist.

\end{Command}
