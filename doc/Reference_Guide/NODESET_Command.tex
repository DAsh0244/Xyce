% Sandia National Laboratories is a multimission laboratory managed and
% operated by National Technology & Engineering Solutions of Sandia, LLC, a
% wholly owned subsidiary of Honeywell International Inc., for the U.S.
% Department of Energy’s National Nuclear Security Administration under
% contract DE-NA0003525.

% Copyright 2002-2020 National Technology & Engineering Solutions of Sandia,
% LLC (NTESS).


\index{\texttt{.NODESET}}
\index{initial condition!NODESET}
\index{initial condition}
\label{NODESET_section}

The \texttt{.NODESET} command sets initial conditions for operating point
calculations.  It is similar to \texttt{.IC} (Section~\ref{IC_section}), except 
it is applied as an initial guess, rather than as a firmly enforced
condition. Like \texttt{.IC}, \texttt{.NODESET} initial conditions can be specified
for some or all of the circuit nodes.

Consult the \Xyce{} User's Guide for more guidance.

\begin{Command}

\format
\begin{alltt}
.NODESET < V(<node>)=<value>
.NODESET <node> <value>
\end{alltt}

\examples
\begin{alltt}
.NODESET V(2)=3.1
.NODESET 2 3.1
\end{alltt}

\comments
The \Xyce{} \texttt{.NODESET} command uses a different strategy than either
SPICE or HSPICE.  When \texttt{.NODESET} is specified, \Xyce{} does two solves
for the DC operating point.  One with the \texttt{.NODESET} values held as 
initial conditions (i.e., the same as if it was an .IC solve). The second solve
is then done without any conditions imposed, but with the first solution as an
initial guess. 

The \texttt{.NODESET} capability can only set voltage values, not current values.

The \texttt{.NODESET} capability can not be used, within subcircuits, to set voltage values on global nodes.

\end{Command}

