% Sandia National Laboratories is a multimission laboratory managed and
% operated by National Technology & Engineering Solutions of Sandia, LLC, a
% wholly owned subsidiary of Honeywell International Inc., for the U.S.
% Department of Energy’s National Nuclear Security Administration under
% contract DE-NA0003525.

% Copyright 2002-2022 National Technology & Engineering Solutions of Sandia,
% LLC (NTESS).


An ideal delay device, operating in a manner similar to a
voltage-controlled voltage source, is provided by the YDELAY device.

\begin{Device}

\device
\begin{alltt}
YDELAY <name> <positive node> <negative node>
+ <positive control node> <negative control node>
+ TD=<time delay>
+ [EXTRAPOLATION=<true|false>] [BPENABLED=<true|false>]
+ [LINEARINTERP=<true|false>] 
\end{alltt}

\examples
\begin{alltt}
YDELAY delay1 2 0 1 0 TD=10N
R1 2 0 1
YDELAY delay1 3 0 2 0 TD=10N LINEARINTERP=true
R2 3 0 1
YDELAY delay1 4 0 3 0 TD=10N BPENABLED=FALSE
R4 4 0 1
YDELAY delay1 5 0 4 0 TD=10N EXTRAPOLATION=false
R5 5 0 1
\end{alltt}

\comments

The voltage between the positive and negative control nodes is
reproduced at the positive and negative output nodes delayed by a time
equal to the specified TD parameter.

The device is equivalent in connectivity to a voltage-controlled
voltage source --- the device puts no load on the control nodes, and
its output must be connected to a valid circuit.

Unlike the transmission line, no impedance matching is required, and
reflections due to impedance mismatch do not occur.

These devices may be chained to create outputs at different delays,
but each instance must have its output connected to a valid closed
circuit.  The examples above are chained correctly so that each of the
output nodes is delayed by 10 nanoseconds from the previous stage.

The device functions by storing a history of its input at each
accepted time point.  At each new time step, interpolation is
performed on this history to determine what the signal would have been
at a time TD in the past.  At each step, the device checks its history
to determine if the previous three saved steps include a discontinuity
in the input.  If so, the device assures that \Xyce{} will correctly
resolve the same discontinuity when it appears on the output.

With no special options specified, three-point quadratic interpolation
is used except after a discontinuity, when linear interpolation is
performed.  If \Xyce{} has advanced the time by more than TD and no
discontinuity has occured, then this interpolation is actually
extrapolation.

When \texttt{LINEARINTERP=true} is specified, the history
interpolation used is always linear interpolation.

When \texttt{EXTRAPOLATION=false}, \Xyce{} will never attempt
extrapolation when it has taken a time step larger than TD.  In this
case, the current, unconverged value of the solution is used as the
third interpolation point and the interpolation is recomputed at every
step of the nonlinear solve.

When \texttt{BPENABLED=false}, the device will not set a simulation
breakpoint to force the time integrator to stop exactly TD seconds
after a detected discontinuity on the input.  It will still force a
maximum time step on the time integrator after such a discontinuity,
and other techniques will be applied to assure the discontinuity is
resolved.  This option may result in \Xyce{} rejecting a lot more time
steps and slower simulation than when it is left at its default.

\end{Device}

\subsubsection{Delay device instance parameters}

The instance parameters for the delay device are shown in
Table~\ref{Delay_1_Device_Instance_Params}.

% This table was generated by Xyce:
%   Xyce -doc Delay 1
%
\index{delay element!device instance parameters}
\begin{DeviceParamTableGenerated}{Delay element Device Instance Parameters}{Delay_1_Device_Instance_Params}
BPENABLED & Can this device set discontinuity breakpoints? & -- & true \\ \hline
EXTRAPOLATION & Can this device use extrapolation on history? & -- & true \\ \hline
LINEARINTERP & Should this device use only linear interpolation on history? & -- & false \\ \hline
TD & Time delay & s & 0 \\ \hline
\end{DeviceParamTableGenerated}

