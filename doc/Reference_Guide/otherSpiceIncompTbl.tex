% Sandia National Laboratories is a multimission laboratory managed and
% operated by National Technology & Engineering Solutions of Sandia, LLC, a
% wholly owned subsidiary of Honeywell International Inc., for the U.S.
% Department of Energy’s National Nuclear Security Administration under
% contract DE-NA0003525.

% Copyright 2002-2020 National Technology & Engineering Solutions of Sandia,
% LLC (NTESS).


%%
%% Incompatibilities with other circuit simulators.
%%
\index{PSpice}
\begin{longtable}[h] {>{\raggedright\small}m{2in}|>{\raggedright\let\\\tabularnewline\small}m{3.5in}}
  \caption{Incompatibilities with Other Circuit Simulators.} \\ \hline
  \rowcolor{XyceDarkBlue}
  \color{white}\bf Issue &
  \color{white}\bf Comment \\ \hline \endfirsthead
  \label{Incompat_Other_SPICE}

    \texttt{.DC} sweep output. & The \texttt{.DC} sweep calculation does not
    automatically output the sweep variable.  Only variables explicitly listed on
    the \texttt{.PRINT} line are output.\\ \hline

    MOSFET levels. & In \Xyce{} the MOSFET levels are not the same. In \Xyce{},
    a BSIM3 is MOSFET level 9.  Other simulators have different levels for the 
    BSIM3. \\ \hline

    BSIM SOI v3.2 level. & In \Xyce{} the BSIM SOI (v3.2) is MOSFET level 10.  
    Other simulators have different levels for the BSIM SOI. \\ \hline

    BSIM4 level. & In \Xyce{} the BSIM4 is MOSFET levels 14 and 54.  
    Other simulators have different levels for the BSIM4. \\ \hline

    Syntax for \texttt{.STEP} is different. & The manner of specifying
    a model parameter to be swept is slightly different than in some
    other simulators.  See the \Xyce{} Users' and Reference Guides for
    details.  \\ \hline

    Switch is not the same as SPICE3F5. &  The \Xyce{} switches are not
    compatible with the simple switch implementation in SPICE3F5.  The
    switch in \Xyce{} smoothly transitions between the ON and OFF
    resistances over a small range between the ON and OFF values of the
    control signal (voltage, current, or control expression).  See the
    \Xyce{} Reference Guide for the precise equations that are used to compute the
    switch resistance from the control signal values.  The SPICE3F5 switch
    has a single switching threshold voltage or current, and RON is used
    above threshold while ROFF is used below threshold.  \Xyce{}'s switch is
    considerably less likely to cause transient simulation failures. Results
    similar to SPICE3F5 can be obtained by setting VON and VOFF to the same
    threshold value, but this is not a recommended practice.  \\ \hline

    Piecewise Linear (PWL) source not fully compatible with either HSPICE
    or PSpice. & See Sections \ref{I_DEVICE} and \ref{PWL_SOURCE_SYNTAX_DIFF} 
    of the \Xyce{} Reference Guide for more details. \\ \hline

    Acceptable prefixes in the metric system. &
    The ``atto'' prefix, which is designated by ``a'', is acceptable in HSPICE,
    but is not accepted in \Xyce{}. 
    The use of the ``atto'' prefix in \Xyce{} must be replaced with ``E-18''. \\ \hline

    Hierarchical parameters. & In \Xyce{} hierarchical parameters, \texttt{M} (multiplier
    or multiplicity factor) and \texttt{S} (scale), are not commonly supported.  The \texttt{M} 
    parameter is only supported by the R, L, C and MOSFET device models and some BJT 
    device models (VBIC 1.3 and MEXTRAM).\\ \hline

    \texttt{.MEASURE} has some incompatibilities and differences with HSPICE. &
    See Section \ref{Measure_HSpice_Compatibility} of the \Xyce{} Reference Guide 
    for more details. \\ \hline

    The \texttt{P()} accessor for power may give different results for semiconductor devices
    (D, J, M, Q and Z devices) and the lossless transmission device (T device) than with HSPICE.  
    & See Sections \ref{D_DEVICE}, \ref{J_DEVICE}, \ref{M_DEVICE}, \ref{Q_DEVICE} \ref{T_DEVICE}
    and \ref{Z_DEVICE}  for more details. \\ \hline

    The \Xyce{} \texttt{.OP} statement provides less output than other simulators. & See 
    Section \ref{OP_COMMAND} for more details.    \\ \hline

    Initial conditions for lossless and lossy transmission lines & In SPICE3F5 and PSpice, 
    initial conditions can be set on the initial voltages and currents at each end of the 
    lossless transmission line, but not for the lossy transmission line.  In \Xyce{}, 
    initial conditions can be set on the initial voltages and currents at each end of the
    lossy transmission line, but not for the lossless transmission line  \\ \hline

    Use of vgs(Mxxx) style syntax on the .PRINT line & Some SPICE-style circuit simulators 
    can use the \texttt{.PRINT} line to (for example) print out the vds, vgb, vsd, etc. 
    values for a PMOS transistor (say, \texttt{M1}) using \texttt{.PRINT TRAN vgs(M1) 
    vbs(M1) vds(M1)}.  This is not directly supported in \Xyce{}.  See Section \ref{Q_DEVICE}
    for how this is supported with the \texttt{N()} syntax for the BSIM3 and BSIM4 models.  
    For other transistor devices, use something like this on the \Xyce{} \texttt{.PRINT} 
    line, \texttt{V(ng,ns)} where ng and ns are the names of the circuits nodes attached 
    to the gate and source terminals of the transistor. \\ \hline

    Some devices do not work in frequency-domain analysis & Devices
    that may be expected to work in AC or HB analysis do not at this
    time.  For AC analysis this includes, but is not limited to, the
    lossy transmission line (LTRA) and lossless transmission line
    (TRA).  The LTRA and TRA models will need to be replaced with
    lumped transmission line models (YTRANSLINE) to perform
    small-signal AC analysis.  For harmonic balance, the two
    transmission line models do work correctly in frequency domain.

    Independent behavioral sources, such as a time-dependent B, E, F, G, or H 
    source, will not work correctly with either AC or HB. 
    However, such sources which are purely dependent (only depending on solution 
    variables and not time) will work in AC and HB.
    \\ \hline

    \Xyce{} uses \texttt{FREQ} as the special variable denoting the
    current simulation frequency  & Other simulators may use \texttt{HERTZ}
    instead. \\ \hline

\end{longtable}

