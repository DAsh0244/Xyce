% Sandia National Laboratories is a multimission laboratory managed and
% operated by National Technology & Engineering Solutions of Sandia, LLC, a
% wholly owned subsidiary of Honeywell International Inc., for the U.S.
% Department of Energy’s National Nuclear Security Administration under
% contract DE-NA0003525.

% Copyright 2002-2020 National Technology & Engineering Solutions of Sandia,
% LLC (NTESS).


\index{\texttt{.TRAN}}
\index{analysis!transient} \index{transient analysis}
Calculates the time-domain response of a circuit for a specified duration.

\begin{Command}

\format
\begin{alltt}
.TRAN <initial step value> <final time value>
+ [<start time value> [<step ceiling value>]] [NOOP] [UIC]
+ [\{schedule( <time>, <maximum time step>, ... )\}]
\end{alltt}

\examples
\begin{alltt}
.TRAN 1us 100ms
.TRAN 1ms 100ms 0ms .1ms
.TRAN 0 2.0e-3  \{schedule( 0.5e-3, 0, 1.0e-3, 1.0e-6, 2.0e-3, 0 )\}
\end{alltt}

\arguments

\begin{Arguments}

\argument{initial step value}

Used to calculate the initial time step (see below).

\argument{final time value}

Sets the end time (duration) for the analysis.

\argument{start time value}

Sets the time at which output of the simulation results is to begin.  Defaults to zero.

\argument{step ceiling value}

Sets a maximum time step.  Defaults to ((final time value)-(start time
value))/10, unless there are breakpoints (see below).

\argument{NOOP or UIC}

These two options are synonyms which specify that no operating point
calculation is to be performed, and that the specified initial
condition (from .IC lines or capacitor ``IC'' parameters) should be
used as the transient initial condition instead. Unspecified values
are set to zero.  Finally, the .IC capability can only set voltage
values, not current values.

\argument{schedule(<time>, <maximum time step>, \ldots)}

Specifies a schedule for maximum allowed time steps. The list of
arguments, $t_0$, $\Delta t_0$, $t_1$, $\Delta t_1$, \emph{ etc.\/}
implies that a maximum time step of $\Delta t_0$ will be used while the
simulation time is greater than or equal to $t_0$ and less than $t_1$.
A maximum time step of $\Delta t_1$ will be used when the simulation
time is greater or equal to than $t_1$ and less than $t_2$. This
sequence will continue for all pairs of $t_i$, $\Delta t_i$ that are
given in the \{schedule()\}.  If $\Delta t$ is zero or negative, then no
maximum time step is enforced (other than hardware limits of the host
computer).

\end{Arguments}

\comments

The transient analysis calculates the circuit's response over an
interval of time beginning with \texttt{TIME=0} and finishing at
\texttt{<final time value>}. Use a \texttt{.PRINT}
(print)\index{\texttt{.PRINT}}\index{results!print}\index{\texttt{.PRINT}!\texttt{TRAN}}
statement to get the results of the transient analysis.

Before calculating the transient response \Xyce{} computes a bias
point\index{bias point} for the circuit that is different from the
regular bias point. This is necessary because at the start of a
transient analysis\index{analysis!transient}, the independent sources
can have different values than their DC values. Specifying \texttt{NOOP}
on the \texttt{.TRAN} line causes \Xyce{} to begin the transient
analysis without performing the usual bias point calculation.

The time integration\index{solvers!time integration} algorithms within
\Xyce{} use adaptive time-stepping methods that adjust the time-step
size\index{time step!size} according to the activity in the
analysis. The default ceiling for the internal time step is
\texttt{(<final time value>-<start time value>)/10}.  This default
ceiling value is automatically adjusted if breakpoints are present, to
ensure that there are always at least 10 time steps between breakpoints.
If the user specifies a ceiling value, however, it overrides any
internally generated ceiling values.

\Xyce{} is not strictly compatible with SPICE in its use of the values on the
\texttt{.TRAN} line. In SPICE, the first number on the \texttt{.TRAN} line
specifies the printing interval. In \Xyce{}, the first number is the
\texttt{<initial step value>}, which is used in determining the initial step
size.  The actual initial step size is chosen to be the smallest of three
quantities: the \texttt{<inital step value>}, the \texttt{<step ceiling
value>}, or 1/200th of the time until the next breakpoint.

The third argument to \texttt{.TRAN} simply determines the earliest time
for which results are to be output.  Simulation of the circuit always
begins at \texttt{TIME=0} irrespective of the setting of \texttt{<start time value>}.

\end{Command}
