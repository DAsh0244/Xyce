% Sandia National Laboratories is a multimission laboratory managed and
% operated by National Technology & Engineering Solutions of Sandia, LLC, a
% wholly owned subsidiary of Honeywell International Inc., for the U.S.
% Department of Energy’s National Nuclear Security Administration under
% contract DE-NA0003525.

% Copyright 2002-2020 National Technology & Engineering Solutions of Sandia,
% LLC (NTESS).

The linear (YLIN) device allows an S-, Y-, or Z-parameter model to
be used to define an N-port device.  It is mostly commonly used
as part of a Harmonic Balance (HB) analysis.

\begin{Device}\label{YLIN_DEVICE}

\device
\begin{alltt}
YLIN <name>  <(+) node> <(-) node> [model name]
\end{alltt}

\model
.MODEL <model name> LIN [model parameters]

\examples
\begin{alltt}
YLIN YLIN1 1 0 2 0 YLIN_MOD1
.MODEL YLIN_MOD1 LIN TSTONEFILE=yparams.y2p
\end{alltt}

\parameters
\begin{Parameters}

\param{model name}
  Name of the model defined in a .MODEL line.

\end{Parameters}

\comments
At present, the YLIN device is only supported in the frequency domain for HB analysis, where
the S-parameter model has data at the exact frequencies requested by the .HB analysis.

\end{Device}


\paragraph{Model Parameters}
% This table was generated by Xyce:
%   Xyce -doc Lin 1
%
\index{lin!device model parameters}
\begin{DeviceParamTableGenerated}{LIN Device Model Parameters}{Lin_1_Device_Model_Params}
ISC\_FD & Touchstone file contains frequency-domain short-circuit current data & logical (T/F) & false \\ \hline
%ISC\_TD\_FILE & ISC Time Domain File Name & -- & '' \\ \hline
%ISC\_TD\_FILE\_FORMAT & Format of ISC Time Domain File & -- & 'STD' \\ \hline
TSTONEFILE & Touchstone File Name & -- & '' \\ \hline
\end{DeviceParamTableGenerated}


The Touchstone file name must be specified.  At present, the YLIN device only
accepts Touchstone 2 formatted input files \cite{touchstone2_std_2009}.

For coupling with EM codes, such as EIGER, the YLIN device also accepts
a non-standard version of the Touchstone 2 input files.  If the \texttt{ISC\_FD}
model parameter is set to true then each row of [Network Data] in the input
file also contains additional columns with the ``per-port frequency-domain
short-circuit currents''.   There are then two such additional columns for each
port. The format of (\texttt{RI}, \texttt{MA} or \texttt{DB}) of those additional
columns will be as specified by the Option line in the Touchstone 2 metadata.




