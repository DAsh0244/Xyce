% Sandia National Laboratories is a multimission laboratory managed and
% operated by National Technology & Engineering Solutions of Sandia, LLC, a
% wholly owned subsidiary of Honeywell International Inc., for the U.S.
% Department of Energy’s National Nuclear Security Administration under
% contract DE-NA0003525.

% Copyright 2002-2019 National Technology & Engineering Solutions of Sandia,
% LLC (NTESS).


%%-------------------------------------------------------------------------
%% Purpose        : Main LaTeX Xyce Reference Guide
%% Special Notes  : Graphic files (pdf format) work with pdflatex.  To use
%%                  LaTeX, we need to use postcript versions.
%% Creator        : Scott A. Hutchinson, Computational Sciences, SNL
%% Creation Date  : {05/23/2002}
%%
%%-------------------------------------------------------------------------

The \texttt{POLY} keyword is available in the \texttt{E}, \texttt{F},
\texttt{G}, \texttt{H} and \texttt{B} dependent sources. Based on the same
keyword from SPICE2, \texttt{POLY} provides a compact method of specifying
polynomial expressions in which the variables in the polynomial are specified
followed by an ordered list of polynomial coefficients.  All expressions
specified with \texttt{POLY} are ultimately translated by \Xyce{} into an
equivalent, straightforward polynomial expression in a \texttt{B} source.
Since a straightforward polynomial expression can be easier to read, there is
no real benefit to using \texttt{POLY} except to support netlists imported from
other SPICE-based simulators.

There are three different syntax forms for \texttt{POLY}, which can be
a source of confusion.  The \texttt{E} and \texttt{G} sources
(voltage-dependent voltage or current sources) use one form, the
\texttt{F} and \texttt{H} sources (current-dependent voltage or
current sources) use a second form, and the \texttt{B} source (general
nonlinear source) a third form.  During input processing, any of the
\texttt{E,F,G} or \texttt{H} sources that use nonlinear expressions
are first converted into an equivalent \texttt{B} source, and then any
\texttt{B} sources that use the \texttt{POLY} shorthand are further
converted into standard polynomial expressions.  This section describes
how the compact form will be translated into the final form that is
used internally.

All three formats of \texttt{POLY} express the same three components:
a number of variables involved in the expression ($N$, the number in
parentheses after the \texttt{POLY} keyword), the variables
themselves, and an ordered list of coefficients for the polynomial
terms.  Where they differ is in how the variables are expressed.

\subsubsection{Voltage-controlled sources}

The \texttt{E} and \texttt{G} sources are both voltage-controlled, and
so their \texttt{POLY} format requires specification of two nodes for
each voltage on which the source depends, i.e. the positive and
negative nodes from which a voltage drop is computed.  There must
therefore be twice as many nodes as the number of variables specified
in parentheses after the \texttt{POLY} keyword:

\verb|Epoly 1 2 POLY(3) n1p n1m n2p n2m n3p n3m ...|

In this example, the voltage between nodes 1 and 2 is determined by
a polynomial whose variables are \texttt{V(n1p,n1m)},
\texttt{V(n2p,n2m)}, \texttt{V(n3p,n3m)}.  Not shown in this example
are the polynomial coefficients, which will be described later.


\subsubsection{Current-controlled sources}

The \texttt{F} and \texttt{H} sources are both current-controlled, and
so their \texttt{POLY} format requires specification of one voltage
source name for each current on which the source depends.
There must therefore be exactly as many nodes as the number of variables
specified in parentheses after the \texttt{POLY} keyword:

\verb|Fpoly 1 2 POLY(3) V1 V2 V3 ...|

In this example, the voltage between nodes 1 and 2 is determined by
a polynomial whose variables are \texttt{I(V1)},
\texttt{I(V2)}, and \texttt{I(V3)}.  Not shown in this example
are the polynomial coefficients, which will be described later.

\subsubsection{B sources}

Finally, the most general form of \texttt{POLY} is that used in the
general nonlinear dependent source, the \texttt{B} source. In this variant,
each specific variable must be named explicitly (i.e. not simply by node name or
by voltage source name), because currents and voltages may be mixed as
needed.

\verb|Bpoly 1 2 V={POLY(3) I(V1) V(2,3) V(3) ...}|

\verb|Bpoly2 1 2 I={POLY(3) I(V1) V(2,3) V(3) ...}|


In these examples, the source between nodes 1 and 2 is determined by a
polynomial whose variables are \texttt{I(V1)}, \texttt{V(2,3)}, and
\texttt{V(3)}.  In the first example, the polynomial value determines
the voltage between nodes 1 and 2, and in the second the current.

The \texttt{E, F, G} and \texttt{H} formats are all converted internally
in a first step to the \texttt{B} format.  Thus the following pairs of
sources are exactly equivalent:

\verb|Epoly 1 2 POLY(3) n1p n1m n2p n2m n3p n3m ...|

\verb|BEpoly 1 2 V={POLY(3) V(n1p,n1m) V(n2p,n2m) V(n3p,n3m) ...|

\verb|Fpoly 1 2 POLY(3) V1 V2 V3 ...|

\verb|BFpoly 1 2 V={POLY(3) I(V1) I(V2) I(V3) ...|

After conversion to the \texttt{B} source form, the \texttt{POLY} form
is finally converted to a normal expression using the coefficients and
variables given.

Coefficients are given in a standard order, and the polynomial is
built up by terms until the list of coefficients is exhausted.  The
first coefficient is the constant term of the polynomial, followed by
the coefficients of linear terms, then bi-linear, and so on.  For example:

\verb|Epoly 1 2 POLY(3) n1p n1m n2p n2m n3p n3m 1 .5 .5 .5|

In this example, the constant term is 1.0, and the coefficients of the
three terms linear in the input variables are 0.5.  Thus, this
\texttt{E} source is precisely equivalent to the general \texttt{B}
source:

\verb|BEstandard 1 2 V={1.0 + .5*V(n1p,n1m) + .5*V(n2p,n2m) +.5*V(n3p,n3m)}|

The standard ordering for coefficients is:

POLY(N) $X_1 \ldots X_N C_0\ C_1 \ldots C_N\ C_{11} \ldots C_{1N}\ C_{21} \ldots C_{N1} \ldots C_{NN}\ C_{1^21} \ldots  C_{1^2N} \ldots$

with the polynomial then being:

$$Value = C_0 + \sum_{j=1}^{N} C_j X_j + \sum_{i=1}^N\sum_{j=1}^N C_{ij}X_iX_j + \sum_{i=1}^N\sum_{j=1}^N C_{i^2j} X_i^2X_j + \ldots$$

Here we have used the general form $X_i$ for the $i^{th}$ variable,
which may be either a current or voltage variable in the general case.

It should be reiterated that the \texttt{POLY} format is provided primarily for
support of netlists from other simulators, and that its compactness may be a
disadvantage in readability of the netlist and may be more prone to usage
error.  \Xyce{} users are therefore advised that use of the more
straightforward expression format in the \texttt{B} source may be more
appropriate when crafting original netlists for use in \Xyce{}.  Since \Xyce{}
converts \texttt{POLY} format expressions to the simpler format internally,
there is no performance benefit to use of \texttt{POLY}.
