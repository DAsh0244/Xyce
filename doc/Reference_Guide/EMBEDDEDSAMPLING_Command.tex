% Sandia National Laboratories is a multimission laboratory managed and
% operated by National Technology & Engineering Solutions of Sandia, LLC, a
% wholly owned subsidiary of Honeywell International Inc., for the U.S.
% Department of Energy’s National Nuclear Security Administration under
% contract DE-NA0003525.

% Copyright 2002-2021 National Technology & Engineering Solutions of Sandia,
% LLC (NTESS).

\label{EMBEDDEDSAMPLING_section}
\index{\texttt{.EMBEDDEDSAMPLING}}
\index{analysis!sampling} \index{embedded sampling analysis}
Calculates a full analysis (for \verb|.DC| or \verb|.TRAN| only) over
a distribution of parameter values.  Embedded sampling operates
similarly to \verb|.STEP|, except that the parameter values are
generated from random distributions rather than sweeps, and that the 
loop over parameters happens at the inner-most part of the calculation, 
so all samples are propagated simultaneously.
If used in conjunction 
with projection-based PCE methods, then the sample points are not based on random samples.  
Instead they are based on the quadrature points.

\index{analysis!SAMPLING}
\index{analysis!EMBEDDEDSAMPLING}
\index{EMBEDDEDSAMPLING analysis}
\index{analysis!MC}
\index{MC analysis}
\index{analysis!PCE}
\index{PCE analysis}
\index{analysis!Monte Carlo}
\index{Monte Carlo analysis}
\index{analysis!LHS}
\index{LHS analysis}
\index{Latin Hypercube Sampling analysis}

\index{analysis!EMBEDDEDSAMPLING}
\index{EMBEDDEDSAMPLING analysis}

\begin{Command}
\format
.EMBEDDEDSAMPLING  \\
+ param=<parameter name>,[parameter name]*  \\
+ type=<parameter type>,[parameter type]*  \\
+ means=<mean>,[mean]*  \\
+ std\_deviations=<standard deviation>,[standard deviation]* \\
+ lower\_bounds=<lower bound>,[lower bound]*  \\
+ upper\_bounds=<upper bound>,[upper bound]* \\
+ alpha=<alpha>,[alpha]*  \\
+ beta=<beta>,[beta]*

\examples
\begin{alltt}
.EMBEDDEDSAMPLING
+ param=R1
+ type=normal
+ means=3K
+ std\_deviations=1K

.EMBEDDEDSAMPLING
+ param=R1,R2
+ type=uniform,uniform
+ lower\_bounds=1K,2K
+ upper\_bounds=5K,6K

.EMBEDDEDSAMPLING
+ useExpr=true

.options EMBEDDEDSAMPLES numsamples=10000

.options EMBEDDEDSAMPLES numsamples=25000
+ OUTPUTS=\{R1:R\},\{V(1)\}
+ SAMPLE\_TYPE=MC

.options EMBEDDEDSAMPLES numsamples=1000
+ MEASURES=maxSine
+ SAMPLE\_TYPE=LHS

.options embeddedsamples numsamples=30
+ covmatrix=1e6,1.0e-3,1.0e-3,4e-14
+ OUTPUTS=\{V(1)\},\{R1:R\},\{C1:C\}
\end{alltt}

\arguments

\begin{Arguments}

\argument{param}
Names of the parameters to be sampled.  This may be any of the parameters
that are valid for \verb|.STEP|, including device instance, device model,
or global parameters.  If more than one parameter, then specify as a
comma-separated list.

\argument{type}
Distribution type for each parameter.  This may be uniform, normal or gamma.
If more than one parameter, then specify as a comma-separated list.

\argument{means}
If using normal distributions, the mean for each parameter must be specified.
If more than one parameter, then specify as a comma-separated list.

\argument{std\_deviations}
If using normal distributions, the standard deviation for each parameter
must be specified.  If more than one parameter, then specify as a
comma-separated list.

\argument{lower\_bounds}
If using uniform distributions, the lower bound must be specified.
This is optional for normal distributions.  If used with normal
distributions, may alter the mean and standard deviation.
If more than one parameter, then specify as a comma-separated list.

\argument{upper\_bounds}
If using uniform distributions, the upper bound must be specified.
This is optional for normal distributions.  If used with normal
distributions, may alter the mean and standard deviation.
If more than one parameter, then specify as a comma-separated list.

\argument{alpha}
If using gamma distributions, the alpha value for each parameter
must be specified.  If more than one parameter, then specify as a
comma-separated list.

\argument{beta}
If using gamma distributions, the beta value for each parameter
must be specified.  If more than one parameter, then specify as a
comma-separated list.

\argument{useExpr}
If this argument is set to true, then the sampling algorithm will set up random 
  inputs from expression operators such as \verb|AGAUSS| and \verb|AUNIF|.  In 
  this case it will also ignore the list of parameters on the \verb|.EMBEDDEDSAMPLING| command line.
  For a complete description of expression-based random operators, see the expression
  documentation in section~\ref{ExpressionDocumentation}.

\end{Arguments}

\comments

In addition to the \verb|.EMBEDDEDSAMPLING| command, this analysis
requires a \verb|.options EMBEDDEDSAMPLES| command as well.  The
\verb|.EMBEDDEDSAMPLING| command specifies parameters and their
attributes, either using the \verb|useExpr| option, or with 
comma-separated lists.  The \verb|.options EMBEDDEDSAMPLES| command specifies
analysis options, including the number of samples, the type of
sampling (LHS or MC) 
and the outputs and/or measures for which to compute statistics.  
This line also allows one to specify a non-intrusive Polynomial Chaos 
Expansion (PCE) method (either regression or projection PCE).
To see the details of the \verb|.options EMBEDDEDSAMPLES| command , see table~\ref{EmbeddedSamplesPKG}.

On the \verb|.EMBEDDEDSAMPLING| command line, if not using \verb|useExpr|, 
parameters and their
attributes must be specified using comma-separated lists. The
comma-separated lists must all be the same length.

The \texttt{.PRINT ES} command provides output based on the contents
of those print-lines, and also the \texttt{NUMSAMPLES} and \texttt{OUTPUT}
arguments on the \texttt{.OPTIONS EMBEDDEDSAMPLES} line. If the
\texttt{OUTPUT\_SAMPLE\_STATS} argument on a \texttt{.PRINT ES} line is
set to ``true'' then the statistics for the \texttt{MEAN}, \texttt{MEANPLUS},
\texttt{MEANMINUS}, \texttt{STDDEV} and \texttt{VARIANCE} will be output for each
variable in the \texttt{OUTPUT} argument.  If the \texttt{OUTPUT\_ALL\_SAMPLES}
argument on a \texttt{.PRINT ES} line is set to ``true'' then the values
of all \texttt{NUMSAMPLES} samples, for each variable requested
in the \texttt{OUTPUTS} argument, will be output.

\end{Command}

