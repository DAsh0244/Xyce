% Sandia National Laboratories is a multimission laboratory managed and
% operated by National Technology & Engineering Solutions of Sandia, LLC, a
% wholly owned subsidiary of Honeywell International Inc., for the U.S.
% Department of Energy’s National Nuclear Security Administration under
% contract DE-NA0003525.

% Copyright 2002-2021 National Technology & Engineering Solutions of Sandia,
% LLC (NTESS).

\label{Measure_section}

The \texttt{.MEASURE} statement allows calculation or reporting of simulation 
metrics to an external file, as well as to the standard output and/or a log file.  One can 
measure when simulated signals reach designated values, or when they are equal
to other simulation values.  The \texttt{.MEASURE} statement is supported for 
\texttt{.TRAN}, \texttt{.DC}, \texttt{.AC} and \texttt{.NOISE} analyses.  It can
be used with {\tt .STEP}  in all four cases.  For HSPICE compatibility,
\texttt{.MEAS} is an allowed synonym for \texttt{.MEASURE}.

The syntaxes for the \texttt{.MEASURE} statements are shown below.  The \texttt{AVG},
\texttt{DERIV}, \texttt{EQN},   \texttt{ERR},  \texttt{ERR1}, \texttt{ERR2}, \texttt{FIND-AT},
\texttt{FIND-WHEN}, \texttt{INTEG}, \texttt{MIN}, \texttt{MAX}, \texttt{PP}, \texttt{RMS}
and \texttt{WHEN} measures are supported for all four ``measure modes'' (\texttt{TRAN}, 
\texttt{AC}, \texttt{DC} and \texttt{NOISE}).  The \texttt{TRIG-TARG} measure is only supported for
\texttt{TRAN} measure mode.  Note that each measure type (e.g., \texttt{MAX}) may be listed
twice. This is because only a subset of the allowed ``qualifiers'' (e.g., \texttt{FROM} and
\texttt{TO}) may be supported for the \texttt{AC}, \texttt{DC} and \texttt{NOISE} measure modes.

The \texttt{ERROR} measure is \Xyce{}-specific, and is only supported for \texttt{TRAN} and
\texttt{DC} measure modes.  The \texttt{DUTY}, \texttt{FREQ}, \texttt{FOUR}, \texttt{OFF\_TIME}
and \texttt{ON\_TIME} measures are also \Xyce{}-specific, and are  only supported for
\texttt{TRAN} measure mode.

\begin{Command}
\format
\begin{alltt}
.MEASURE TRAN <result name> AVG <variable>
+ [MIN_THRESH=<value>] [MAX_THRESH=<value>]
+ [FROM=<time>] [TO=<time>] [TD=<time>]
+ [DEFAULT_VAL=<value>] [PRECISION=<value>] [PRINT=<value>]

.MEASURE TRAN <result name> DERIV <variable> AT=<value>
+ [MINVAL=<value>] [DEFAULT_VAL=<value>] 
+ [PRECISION=<value>] [PRINT=<value>]

.MEASURE TRAN <result name> DERIV <variable>
+ WHEN <variable>=<variable\(\sb{2}\)>|<value>
+ [MINVAL=<value>] [FROM=<value>] [TO=<value>] [TD=<value>] 
+ [RISE=r|LAST] [FALL=f|LAST] [CROSS=c|LAST]
+ [DEFAULT_VAL=<value>] [PRECISION=<value>] [PRINT=<value>]

.MEASURE TRAN <result name> DUTY <variable>
+ [ON=<value>] [OFF=<value>] [MINVAL=<value>]
+ [FROM=<value>] [TO=<value>] [TD=<value>]
+ [DEFAULT_VAL=<value>] [PRECISION=<value>] [PRINT=<value>]

.MEASURE TRAN <result name> EQN <variable>
+ [FROM=<value>] [TO=<value>] [TD=<value>]
+ [DEFAULT_VAL=<value>] [PRECISION=<value>] [PRINT=<value>]

.MEASURE TRAN <result name> <ERR|ERR1|ERR2>
+ <variable\(\sb{1}\)> <variable\(\sb{2}\)> [FROM=<value>] [TO=<value>]
+ [MINVAL=<value>] [IGNOR|YMIN=<value>] [YMAX=<value>]
+ [DEFAULT_VAL=<value>] [PRECISION=<value>] [PRINT=<value>]

.MEASURE TRAN <result name> ERROR <variable> FILE=<value>
+ INDEPVARCOL=<value> DEPVARCOL=<value> [COMP_FUNCTION=<value>]
+ [DEFAULT_VAL=<value>] [PRECISION=<value>] [PRINT=<value>]

.MEASURE TRAN <result name> FIND <variable> AT=<value>
+ [MINVAL=<value>] [DEFAULT_VAL=<value>]
+ [PRECISION=<value>] [PRINT=<value>]

.MEASURE TRAN <result name> FIND <variable>
+ WHEN <variable>=<variable\(\sb{2}\)>|<value>
+ [FROM=<value>] [TO=<value>] [TD=<value>] 
+ [RISE=r|LAST] [FALL=f|LAST] [CROSS=c|LAST]
+ [MINVAL=<value>] [DEFAULT_VAL=<value>] 
+ [PRECISION=<value>] [PRINT=<value>]

.MEASURE TRAN <result name> FOUR <variable> AT=freq
+ [NUMFREQ=<value>] [GRIDSIZE=<value>]
+ [FROM=<value>] [TO=<value>] [TD=<value>] 
+ [DEFAULT_VAL=<value>] [PRECISION=<value>] [PRINT=<value>]

.MEASURE TRAN <result name> FREQ <variable>
+ [ON=<value>] [OFF=<value>] [MINVAL=<value>]
+ [FROM=<value>] [TO=<value>] [TD=<value>] 
+ [DEFAULT_VAL=<value>] [PRECISION=<value>] [PRINT=<value>]

.MEASURE TRAN <result name> INTEG <variable>
+ [FROM=<value>] [TO=<value>] [TD=<value>]
+ [DEFAULT_VAL=<value>] [PRECISION=<value>] [PRINT=<value>]

.MEASURE TRAN <result name> MAX <variable>
+ [FROM=<value>] [TO=<value>] [TD=<value>]
+ [RISE=r|LAST] [FALL=f|LAST] [CROSS=c|LAST] [RFC_LEVEL=<value>]
+ [DEFAULT_VAL=<value>] [PRECISION=<value>] 
+ [PRINT=<value>] [OUTPUT=<value>]

.MEASURE TRAN <result name> MIN <variable>
+ [FROM=<value>] [TO=<value>] [TD=<value>]
+ [RISE=r|LAST] [FALL=f|LAST] [CROSS=c|LAST] [RFC_LEVEL=<value>]
+ [DEFAULT_VAL=<value>] [PRECISION=<value>] 
+ [PRINT=<value>] [OUTPUT=<value>]

.MEASURE TRAN <result name> OFF_TIME <variable>
+ [OFF=<value>] [MINVAL=<value>]
+ [FROM=<value>] [TO=<value>] [TD=<value>]
+ [DEFAULT_VAL=<value>] [PRECISION=<value>] [PRINT=<value>]

.MEASURE TRAN <result name> ON_TIME <variable>
+ [ON=<value>] [MINVAL=<value>]
+ [FROM=<value>] [TO=<value>] [TD=<value>]
+ [DEFAULT_VAL=<value>] [PRECISION=<value>] [PRINT=<value>]

.MEASURE TRAN <result name> PP <variable>
+ [FROM=<value>] [TO=<value>] [TD=<value>]
+ [RISE=r|LAST] [FALL=f|LAST] [CROSS=c|LAST] [RFC_LEVEL=<value>]
+ [DEFAULT_VAL=<value>] [PRECISION=<value>] [PRINT=<value>]

.MEASURE TRAN <result name> RMS <variable>
+ [FROM=<value>] [TO=<value>] [TD=<value>]
+ [DEFAULT_VAL=<value>] [PRECISION=<value>] [PRINT=<value>]

.MEASURE TRAN <result name> WHEN <variable>=<variable\(\sb{2}\)>|<value>
+ [FROM=<value>] [TO=<value>] [TD=<value>] 
+ [RISE=r|LAST] [FALL=f|LAST] [CROSS=c|LAST]
+ [MINVAL=<value>] [DEFAULT_VAL=<value>] 
+ [PRECISION=<value>] [PRINT=<value>]

.MEASURE TRAN <result name> TRIG <variable>=<variable\(\sb{2}\)>|<value>
+ [RISE=r1|LAST] [FALL=f1|LAST] [CROSS=c1|LAST]
+ TARG <variable\(\sb{3}\)>=<variable\(\sb{4}\)>|<value> 
+ [RISE=r2|LAST] [FALL=f2|LAST] [CROSS=c2|LAST]
+ [FROM=<value>] [TO=<value>] [TD=<value>] 
+ [DEFAULT_VAL=<value>] [PRECISION=<value>] [PRINT=<value>]

.MEASURE TRAN <result name> TRIG AT=<value>
+ TARG <variable\(\sb{2}\)>=<variable\(\sb{3}\)>|<value> 
+ [RISE=r2|LAST] [FALL=f2|LAST] [CROSS=c2|LAST]
+ [FROM=<value>] [TO=<value>] [TD=<value>] 
+ [DEFAULT_VAL=<value>] [PRECISION=<value>] [PRINT=<value>]

.MEASURE DC <result name> ERROR <variable> FILE=<value>
+ [DEPVARCOL=<value>] [COMP_FUNCTION=<value>]
+ [DEFAULT_VAL=<value>] [PRECISION=<value>] [PRINT=<value>]

.MEASURE <AC|DC|NOISE> <result name> AVG <variable>
+ [FROM=<value>] [TO=<value>]
+ [DEFAULT_VAL=<value>] [PRECISION=<value>] [PRINT=<value>]

.MEASURE <AC|DC|NOISE> <result name> DERIV <variable> AT=<value>
+ [MINVAL=<value>] [DEFAULT_VAL=<value>]
+ [PRECISION=<value>] [PRINT=<value>]

.MEASURE <AC|DC|NOISE> <result name> DERIV <variable>
+ WHEN <variable>=<variable\(\sb{2}\)>|<value>
+ [MINVAL=<value>] [FROM=<value>] [TO=<value>]
+ [RISE=r|LAST] [FALL=f|LAST] [CROSS=c|LAST]
+ [DEFAULT_VAL=<value>] [PRECISION=<value>] [PRINT=<value>]

.MEASURE <AC|DC|NOISE> <result name> EQN <variable>
+ [FROM=<value>] [TO=<value>] 
+ [DEFAULT_VAL=<value>] [PRECISION=<value>] [PRINT=<value>]

.MEASURE <AC|DC|NOISE> <result name> <ERR|ERR1|ERR2>
+ <variable\(\sb{1}\)> <variable\(\sb{2}\)> [FROM=<value>] [TO=<value>]
+ [MINVAL=<value>] [IGNOR|YMIN=<value>] [YMAX=<value>]
+ [DEFAULT_VAL=<value>] [PRECISION=<value>] [PRINT=<value>]

.MEASURE <AC|DC|NOISE> <result name> FIND <variable> AT=<value>
+ [MINVAL=<value>] [DEFAULT_VAL=<value>]
+ [PRECISION=<value>] [PRINT=<value>]

.MEASURE <AC|DC|NOISE> <result name> FIND <variable>
+ WHEN <variable>=<variable\(\sb{2}\)>|<value>
+ [FROM=<value>] [TO=<value>]
+ [RISE=r|LAST] [FALL=f|LAST] [CROSS=c|LAST]
+ [MINVAL=<value>] [DEFAULT_VAL=<value>]
+ [PRECISION=<value>] [PRINT=<value>]

.MEASURE <AC|DC|NOISE> <result name> INTEG <variable>
+ [FROM=<value>] [TO=<value>]
+ [DEFAULT_VAL=<value>] [PRECISION=<value>] [PRINT=<value>]

.MEASURE <AC|DC|NOISE> <result name> MAX <variable>
+ [FROM=<value>] [TO=<value>] 
+ [DEFAULT_VAL=<value>] [PRECISION=<value>] 
+ [PRINT=<value>] [OUTPUT=<value>]

.MEASURE <AC|DC|NOISE> <result name> MIN <variable>
+ [FROM=<value>] [TO=<value>] 
+ [DEFAULT_VAL=<value>] [PRECISION=<value>]
+ [PRINT=<value>] [OUTPUT=<value>]

.MEASURE <AC|DC|NOISE> <result name> PP <variable>
+ [FROM=<value>] [TO=<value>] 
+ [DEFAULT_VAL=<value>] [PRECISION=<value>] [PRINT=<value>]

.MEASURE <AC|DC|NOISE> <result name> RMS <variable>
+ [FROM=<value>] [TO=<value>]
+ [DEFAULT_VAL=<value>] [PRECISION=<value>] [PRINT=<value>]

MEASURE <AC|DC|NOISE> <result name>
+ WHEN <variable>=<variable\(\sb{2}\)>|<value>
+ [FROM=<value>] [TO=<value>]
+ [RISE=r|LAST] [FALL=f|LAST] [CROSS=c|LAST]
+ [MINVAL=<value>] [DEFAULT_VAL=<value>]
+ [PRECISION=<value>] [PRINT=<value>]

\end{alltt}
\index{\texttt{.MEASURE}}
\index{results!measure}

\examples
\begin{alltt}
.MEASURE TRAN hit1_75 WHEN V(1)=0.75 MINVAL=0.02
.MEASURE TRAN hit2_75 WHEN V(1)=0.75 MINVAL=0.08 RISE=2
.MEASURE TRAN avgAll AVG V(1)
.MEASURE TRAN dutyAll DUTY V(1) ON=0.75 OFF=0.25
.MEASURE DC maxV1 MAX V(1)
.MEAS DC minV2 MIN V(2)
.MEASURE AC maxV1R MAX VR(1)
.MEASURE NOISE maxonoise MAX ONOISE
\end{alltt}

\arguments

\begin{Arguments}
\argument{result name}

Measured results are reported to the output and log file.
Additionally, for \texttt{TRAN} measures, the results are stored in
files called \texttt{circuitFileName.mt\#}, where the suffixed number
(\texttt{\#}) starts at \texttt{0} and increases for multiple
iterations (\texttt{.STEP} iterations) of a given simulation. Each
line of this file will contain the measurement name, \texttt{<result
name>}, followed by its value for that run.  The \texttt{<result
name>} must be a legal \Xyce{} character string.  For \texttt{DC} measures,
the results are stored in the files \texttt{circuitFileName.ms\#},
while \texttt{AC} and \texttt{NOISE} measures use the files
\texttt{circuitFileName.ma\#}.

If multiple measures are defined with the same \texttt{<result name>} then
\Xyce{} uses the last such definition, and issues warning messages about
(and discards) any previous measure definitions with the same
\texttt{<result name>}.

\argument{measure type}

\texttt{AVG, DERIV, DUTY, EQN, ERR, ERR1, ERR2, ERROR, FIND, FREQ, FOUR, INTEG, MAX, MIN, OFF\_TIME, ON\_TIME, PP, RMS, WHEN, TRIG, TARG}

The third argument specifies the type of measurement or calculation to
be done. The only exception is the {\tt TARG} clause which comes later
in the argument list, after the {\tt TRIG} clause has been specified.

By default, the measurement is performed over the entire simulation.
The calculations can be limited to a specific measurement window by
using the qualifiers {\tt FROM}, {\tt TO}, {\tt TD}, {\tt RISE}, {\tt
FALL}, {\tt CROSS} and {\tt MINVAL}, which are explained below.

The supported measure types and their definitions are:

\begin{description}
  \item[\tt AVG] Computes the arithmetic mean of {\tt <variable>} for
    the simulation, or within the extent of the measurement window.
    The measurement window can be limited with the qualifiers {\tt FROM},
    {\tt TO} and {\tt TD} for {\tt TRAN} measures, and with {\tt FROM}
    and {\tt TO} for {\tt AC}, {\tt DC} and {\tt NOISE} measures.

  \item[\tt DERIV] Computes the derivative of {\tt <variable>} at a
    user-specified time (by using the {\tt AT} qualifier) or when a
    user-specified condition occurs (by using the {\tt WHEN}
    qualifier). If the {\tt WHEN} qualifier is used then the
    measurement window can be limited with the qualifiers {\tt FROM},
    {\tt TO}, {\tt RISE}, {\tt FALL} and {\tt CROSS} for all measure
    modes.  In addition, the {\tt TD} qualifier is supported for
    {\tt TRAN} measures. The {\tt MINVAL} qualifier is used as a
    comparison tolerance for both {\tt AT} and {\tt WHEN}.  For HSPICE
    compatibility, {\tt DERIVATIVE} is an allowed synonym for {\tt
    DERIV}.

  \item[\tt DUTY] Fraction of time that {\tt <variable>} is greater than
   {\tt ON} and does not fall below {\tt OFF} either for the entire
    simulation, or the measurement window. The qualifier {\tt MINVAL}
    is used as a tolerance on the {\tt ON } and {\tt OFF} values, so
    that the thresholds become ({\tt ON} $-$ {\tt MINVAL}) and ({\tt
    OFF} $-$ {\tt MINVAL}).  The measurement window can be limited
    with the qualifiers {\tt FROM}, {\tt TO}, and {\tt TD} for
   {\tt TRAN} measures.

\item[\tt EQN] Calculates the value of {\tt <variable>} during the simulation.
    The measurement window can be limited with the qualifiers {\tt FROM},
    {\tt TO} and {\tt TD} for {\tt TRAN} measures, and with {\tt FROM}
    and {\tt TO} for {\tt AC}, {\tt DC} and {\tt NOISE} measures.  As noted in the
    ``Additional Examples'' subsection, the variable can use the
    results of other measure statements.

\item[\tt ERRx] Calculates the error between two simulation variables, where
    the {\tt ERR1} and {\tt ERR2} functions (and the use of the {\tt MINVAL},
    {\tt YMIN} and {\tt YMAX} qualifiers in those functions) are defined further
    in the ``Error Functions (ERR1 and ERR2)'' subsection.  The {\tt ERR} measure
    type is a synonym for  the {\tt ERR1} measure type.  The measurement
    window can be limited  with the qualifiers {\tt FROM} and {\tt TO}.

  \item[\tt ERROR] Calculates the norm between the measured waveform and a
   ``comparison waveform'' specified in a file.  The supported norms are
    L1, L2 and INFNORM.  The default norm is the L2 norm.

  \item[\tt FIND-AT] Returns the value of {\tt <variable>} at the
    time when the {\tt AT} clause is satisfied.  The {\tt AT}
    clause is described in more detail later in this list.

  \item[\tt FIND-WHEN] Returns the value of {\tt <variable>} at the
    time when the {\tt WHEN} clause is satisfied.  The {\tt WHEN}
    clause is described in more detail later in this list.

  \item[\tt FOUR] Calculates the fourier transform of the transient
    waveform for {\tt <variable>}, given the fundamental frequency
    {\tt AT}.  All frequencies output by the measure will be multiples
    of that fundamental frequency, and will always start at that
    fundamental frequency. The values of the DC component and the
    first {\tt NUMFREQ-1} harmonics are determined using an
    interpolation of {\tt GRIDSIZE} points.  The default values for
    {\tt NUMFREQ} and {\tt GRIDSIZE} are 10 and 200, respectively.
    The measurement window can be limited with the qualifiers {\tt
    FROM}, {\tt TO} and {\tt TD} for {\tt TRAN} measures.  For this
    measure, the phase data is always output in degrees.

  \item[\tt FREQ] An estimate of the frequency of {\tt <variable>},
    found by cycle counting during the simulation.  Cycles are defined
    through the values of {\tt ON} and {\tt OFF} with {\tt MINVAL}
    being used as a tolerance so that the thresholds becomes ({\tt ON}
    $-$ {\tt MINVAL}) and ({\tt OFF} $+$ {\tt MINVAL}). The
    measurement window can be limited with the qualifiers {\tt FROM},
    {\tt TO} and {\tt TD} for {\tt TRAN} measures.

  \item[\tt INTEG] Calculates the integral of {\tt outVal} through
    second order numerical integration.  The integration window can be
    limited with the qualifiers {\tt FROM}, {\tt TO} and {\tt TD} for
    {\tt TRAN} measures, and with {\tt FROM} and {\tt TO} for {\tt AC},
    {\tt DC} and {\tt NOISE} measures.  For HSPICE compatibility,
    {\tt INTEGRAL} is an allowed synonym for {\tt INTEG}.

  \item[\tt MAX] Returns the maximum value of {\tt <variable>} during
    the simulation.  The measurement window can be limited with the
    qualifiers {\tt FROM}, {\tt TO}, {\tt TD}, {\tt RISE}, {\tt FALL}
    and {\tt CROSS} for {\tt TRAN} measures, and with {\tt FROM} and
    {\tt TO} for {\tt AC}, {\tt DC} and {\tt NOISE} measures.

  \item[\tt MIN] Returns the minimum value of {\tt <variable>} during
    the simulation.  The measurement window can be limited with the
    qualifiers {\tt FROM}, {\tt TO}, {\tt TD}, {\tt RISE}, {\tt FALL}
    and {\tt CROSS} for {\tt TRAN} measures, and with {\tt FROM} and
    {\tt TO} for {\tt AC}, {\tt DC} and {\tt NOISE} measures.

  \item[\tt OFF\_TIME] Returns the time that {\tt <variable>} is below
    {\tt OFF} during the simulation or measurement window, normalized
    by the number of cycles of the waveform during the simulation or
    measurement window.  {\tt OFF} uses {\tt MINVAL} as a tolerance,
    and the threshold becomes ({\tt OFF} $+$ {\tt MINVAL}).  The
    measurement window can be limited with the qualifiers {\tt FROM},
    {\tt TO} and {\tt TD} for {\tt TRAN} measures.

  \item[\tt ON\_TIME] Returns the time that {\tt <variable>} is above
    {\tt ON} during the simulation or measurement window, normalized
    by the number of cycles of the waveform during the simulation or
    measurement window.  {\tt ON} uses {\tt MINVAL} as a tolerance,
    and the threshold becomes ({\tt ON} $-$ {\tt MINVAL}).  The
    measurement window can be limited with the qualifiers {\tt FROM},
    {\tt TO} and {\tt TD} for {\tt TRAN} measures.

  \item[\tt PP] Returns the difference between the maximum value and
    the minimum value {\tt <variable>} during the simulation. The
    measurement window can be limited with the qualifiers {\tt FROM},
    {\tt TO}, {\tt TD}, {\tt RISE}, {\tt FALL} and {\tt CROSS} for
    {\tt TRAN} measures, and with {\tt FROM} and {\tt TO} for {\tt AC},
    {\tt DC} and {\tt NOISE} measures.

  \item[\tt RMS] Computes the root-mean-squared value of {\tt
    <variable>} during the simulation, which is defined as ''the
    square root of the area under the {\tt <variable>} curve, divided by
    the period of interest''.  The measurement window can be
    limited with the qualifiers {\tt FROM}, {\tt TO} and {\tt TD} for
    {\tt TRAN} measures, and with {\tt FROM} and {\tt TO} for {\tt AC},
    {\tt DC} and {\tt NOISE} measures.

  \item[\vbox{\hbox{\tt TRIG\hfil}\hbox{\tt TARG\hfil}}] Measures the
    time between a trigger event and a target event.  The trigger is
    specified with {\tt TRIG <variable>=<variable\(\sb{2}\)>} or {\tt
    TRIG <variable>=<value>} or {\tt TRIG AT=<value>}.  Likewise, the
    the target is specified as {\tt TARG
    <variable\(\sb{3}\)>=<variable\(\sb{4}\)>} or {\tt TRIG
    <variable\(\sb{3}\)>=<value>}.  It is also possible to use this
    measure to find a rise time for variable when the rise time is
    defined as the time to go from some small fraction of the maxima
    to some other fraction of the maxima.  For example, the syntax for
    finding a rise time from 10\% to 90\% of the maxima
    is:\\ \texttt{TRIG V(node) FRAC\_MAX=0.1 TARG V(node)
    FRAC\_MAX=0.9}

  \item[\tt WHEN] Returns the time (or frequency or DC sweep value) when
    {\tt <variable>} reaches {\tt <variable\(\sb{2}\)>} or the constant
    value, {\tt value}.  The measurement window can be limited with the
    qualifiers {\tt FROM}, {\tt TO}, {\tt RISE}, {\tt FALL} and {\tt CROSS}
    for all measure modes.  In addition, the {\tt TD} qualifier is supported
    for {\tt TRAN} measures. The qualifier {\tt MINVAL} acts as a
    tolerance for the comparison.  For example when {\tt <variable\(\sb{2}\)>}
    is specified, the comparison used is when {\tt <variable>} $=$
    {\tt <variable\(\sb{2}\)>} $\pm$ {\tt MINVAL} or when a constant,
    {\tt value} is given: {\tt <variable>} $=$ {\tt value} $\pm$ {\tt
    MINVAL}.  If the conditions specified for finding a given value
    were not found during the simulation then the measure will return
    the default value of {\tt -1}.  The user may change this default
    value with the {\tt DEFAULT\_VAL} qualifier described below.
    Note: The use of {\tt FIND} and {\tt WHEN} in one measure
    statement is also supported.
\end{description}

\argument{\vbox{\hbox{variable\hfil}\hbox{variable\(\sb{n}\)\hfil}\hbox{value}}}

These quantities represents the test for the stated
measurement.  \texttt{<variable>} is a simulation quantity, such as a
voltage or current.  One can compare it to another simulation variable
or a fixed quantity.  Additionally, the \texttt{<variable>} may be
a \Xyce{} expression delimited by \{ \} brackets.  As noted above, an
example is {\tt V(1)=0.75}

\argument{AT=value}
A time {\em at which} the measurement calculation will occur.  This is
used by the {\tt DERIV} and {\tt FIND} measures and the {\tt TRIG} clause.  Note that
ill-considered use of the {\tt FROM}, {\tt TO}, {\tt TD} and {\tt AT}
qualifiers in the same {\tt TRIG-TARG} measure statement can cause an
empty measurement window, and thus a failed measure.  Finally, the {\tt FROM}
and {\tt TO} qualifiers take precedence over the {\tt AT} qualifier for
{\tt DERIV} and {\tt FIND} measures.

\argument{FROM=value}

A time (or frequency or DC sweep value) {\em after which} the
measurement calculation will start.  For {\tt DC} measures, this
qualifier uses the first variable on the {\tt .DC} line.

\argument{TO=value}

A time (or frequency or DC sweep value) {\em at which} the measurement
calculation will stop.  For {\tt DC} measures, this qualifier uses the
first variable on the {\tt .DC} line.

\argument{TD=value}

A time delay before which this measurement should be taken or checked.
Note that ill-considered use of both {\tt FROM} and {\tt TO}
qualifiers and a {\tt TD} qualifier in the same measure statement can
cause an empty measurement window, and thus a failed measure.

\argument{MIN\_THRESH=value}

A minimum threshold value above which the measurement calculation will
be done and below which it will not be done.  This is only used by the
{\tt AVG} measure.

\argument{MAX\_THRESH=value}

A maximum threshold value above which the measurement calculation will
not be done and below which it will be done.  This is only used by the
{\tt AVG} measure.

\argument{RISE=r|LAST}

The number of rises after which the measurement should be checked.  If
\texttt{LAST} is specified, then the last rise found in the simulation
will be used.  It is recommended that only one of the qualifiers {\tt RISE}, 
{\tt FALL} or {\tt CROSS} be used in a given measure statement.  The exception 
is {\tt TRIG-TARG} measures.  In that case, different {\tt RISE}, {\tt FALL}
and {\tt CROSS} criteria can be specified for {\tt TRIG} and {\tt TARG}.

\argument{FALL=f|LAST}

The number of falls after which the measurement should be checked.  If
\texttt{LAST} is specified, then the last fall found in the simulation
will be used.

\argument{CROSS=c|LAST}

The number of zero crossings after which the measurement should be
checked.  If \texttt{LAST} is specified, then the last zero crossing
found in the simulation will be used.

\argument{RFC\_LEVEL=value}

The level used to calculate rises, falls and crosses when the
``level-crossing'' mode is used by measure types that do not support
the {\tt WHEN} qualifier.  So, {\tt RFC\_LEVEL} is used by the
{\tt MAX}, {\tt MIN} and {\tt PP} measures.  Its usage is discussed
further in the subsection on ``Rise, Fall and Cross Qualifiers''.

\argument{MINVAL=value}

For the {\tt DERIV}, {\tt DUTY}, {\tt FIND}, {\tt FREQ}, {\tt OFF\_TIME},
{\tt ON\_TIME} and {\tt WHEN} measures, this is allowed difference between
\texttt{outVal} and the variable to  which it is being compared.  This has
a default value of 1.0e-12.  One may need to specify a larger value to avoid
missing the test condition in a transient run.  The descriptions of those
seven measures detail how {\tt MINVAL} is used by each measure.  For the
{\tt ERR1} and {\tt ERR2} measures, if the absolute value of
{\tt <variable\(\sb{1}\)>} is less than {\tt MINVAL}, then {\tt MINVAL} replaces
the value of the denominator of the {\tt ERR1} or {\tt ERR2} expression. For
all measure types, that support the {\tt FROM}, {\tt TO} and/or {\tt TD}
qualifiers, {\tt MINVAL} also functions as a relative tolerance for the
comparison of the simulation time (or sweep value) to the bounds of the
measurement window. This allows for numerical-roundoff errors if the
{\tt FROM}, {\tt TO} and/or {\tt TD} qualifiers are expressions.

\argument{YMIN=value}

If the absolute value of {\tt <variable\(\sb{1}\)>} in {\tt ERR1} or
{\tt ERR2} measure is less than the {\tt YMIN} value then the {\tt ERR1} or
{\tt ERR2} calculation does not consider that point. The default is 1.0e-15.
{\tt IGNOR} and {\tt IGNORE} are synonyms for {\tt YMIN}.

\argument{YMAX=value}

If the absolute value of {\tt <variable\(\sb{1}\)>} in {\tt ERR1} or
{\tt ERR2} measure is greater than {\tt YMAX} value then the {\tt ERR1} or
{\tt ERR2} calculation does not consider that point. The default is 1.0e15.

\argument{FRAC\_MAX=value}

A fractional value of the maximum value of \texttt{<variable>}.  This
is useful for ensemble runs where the maximum value of a waveform is
not known in advance.  {\tt FRAC\_MAX} is used by the {\tt TRIG} and
{\tt TARG} measures.

\argument{ON=value}

The value at which a signal is considered to be ``on'' for {\tt FREQ},
{\tt DUTY} and {\tt ON\_TIME} measure calculations.  This has a
default value of 0.

\argument{OFF=value}

The value at which a signal is considered to be ``off'' for {\tt
FREQ}, {\tt DUTY} and {\tt ON\_TIME} measure calculations.  This has a
default value of 0.

\argument{DEFAULT\_VAL=value}

If the conditions specified for finding a given value are not found
during the simulation then the measure will return the default value
of {\tt -1} in the \texttt{circuitFileName.mt\#}
(or \texttt{circuitFileName.ms\#} or \texttt{circuitFileName.ma\#})
files.  The measure value in the standard output or log file will be
FAILED.  The default return value for
the \texttt{circuitFileName.mt\#} (or \texttt{circuitFileName.ms\#}
or \texttt{circuitFileName.ma\#}) files is settable by the user for
each measure by adding the qualifier {\tt DEFAULT\_VAL=<retval>} on
that measure line.  As examples, a measure will fail if the condition
specified by a {\tt WHEN} or {\tt AT} qualifier is not found.  It will
also fail if the user specifies a set of {\tt FROM}, {\tt TO} and {\tt
TD} values for a given measure that yields an empty measurement
interval.  If either \texttt{.OPTIONS MEASURE MEASFAIL=<val>}
or \texttt{.OPTIONS MEASURE DEFAULT\_VAL=<val>} are given in the
netlist then those values override the \texttt{DEFAULT\_VAL}
parameters given on individual
\texttt{.MEASURE} lines.  See Section \ref{Options_Reference} for more details.

\argument{PRECISION=value}

The default precision for {\tt .MEASURE} output is 6 digits after the
decimal point.  This argument provides a user configurable precision
for a given {\tt .MEASURE} statement that applies to both
the \texttt{.mt\#} ( or \texttt{.ms\#} or \texttt{.ma\#}) files and
standard output.  If \texttt{.OPTIONS MEASURE MEASDGT=<val>} is given
in the netlist then that value overrides the \texttt{PRECISION}
parameters given on individual \texttt{.MEASURE} lines.

\argument{PRINT=value}

This parameter controls where the {\tt .MEASURE} output appears.  The
default is {\tt ALL}, which produces measure output in both
the \texttt{.mt\#} (or \texttt{.ms\#} or
\texttt{.ma\#}) file and to the standard output.  A value of
{\tt STDOUT} only produces measure output to standard output, while a
value of {\tt NONE} suppresses the measure output to both
the \texttt{.mt\#} (or \texttt{.ms\#} or \texttt{.ma\#}) file and
standard output.  The subsection on ``Suppresing Measure Output''
gives examples and also discuss the interactions of this parameter
with \texttt{.OPTIONS MEASURE MEASPRINT=<val>}.

\argument{OUTPUT=value}

This parameter is only supported for the {\tt MAX} and {\tt MIN}
measures.  The default is {\tt VALUE}.  For {\tt TRAN} measures, a
value of {\tt VALUE} will print the maximum (or minimum) value to
the \texttt {.mt\#} file.  A value of {\tt TIME} will print the time
of the maximum (or minimum) value to the \texttt{.mt\#} file. For {\tt
DC} measures, a value of {\tt SV} will output the value of the first
variable on the {\tt .DC} line to the \texttt{.ms\#} file.  For {\tt AC}
and {\tt NOISE} measures, a value of {\tt FREQ} will print the frequency
at which the maximum (or minimum) value occurs to the \texttt{.ma\#} file.
This parameter does not affect the descriptive output that is printed
to the standard output.  The ``Additional Examples'' subsection gives
an example for the {\tt MAX} measure.

\argument{VAL=value}
This parameter is only implemented for the {\tt TRIG} and {\tt TARG}
measures.  It is not the preferred \Xyce{} syntax. It is only
supported for HSPICE compatibility (see that subsection, below, for
details).

\argument{GOAL=value}

This parameter is not implemented in \Xyce{}, but is included for compatibility
with HSPICE netlists.

\argument{WEIGHT=value}

This parameter is not implemented in \Xyce{}, but is included for compatibility
with HSPICE netlists.

\argument{FILE=value}
The filename for the ``comparison file'' used for the {\tt ERROR}
measure.  This qualifier is required for the {\tt ERROR} measure.

\argument{INDEPVARCOL=value}
The column index, in the ``comparison file'', of the independent
variable (e.g, the simulation time or frequency) used in an {\tt
ERROR} measure.  This qualifier is required for the {\tt TRAN} measure
mode.  For that mode, the {\tt INDEPVARCOL} and {\tt DEPVARCOL}
qualifiers must have different values.  The {\tt INDEPVARCOL}
qualifier is not used for {\tt DC} mode {\tt ERROR} measures, and will
be ``silently ignored'' in that case.  Finally, note that the column
indices in Xyce output files start with 0.

\argument{DEPVARCOL=value}
The column index, in the ``comparison file'', of the dependent
variable used in an {\tt ERROR} measure.  This qualifier is required
for the {\tt ERROR} measure for both measure modes ({\tt DC} and {\tt
TRAN}).  For the {\tt TRAN} measure mode, the {\tt DEPVARCOL} and {\tt
INDEPVARCOL} qualifiers must have different values.  Finally, note
that the column indices in Xyce output files start with 0.

\argument{COMP\_FUNCTION=value} 
This is the norm used by the {\tt ERROR} measure to compare the
simulation values for the measured variable with the corresponding
values in the ``comparison file'' specified with the {\tt FILE}
qualifier.  The allowed values are {\tt L1NORM}, {\tt L2NORM} and {\tt
INFNORM}.  Any other values will default to {\tt L2NORM}.  This
qualifier is optional for the {\tt ERROR} measure, and has a default
value of {\tt L2NORM}.  The descriptive output for each {\tt ERROR}
measure, that is printed to standard output, will explicitly state
which norm was used for each {\tt ERROR} measure.

\end{Arguments}

\end {Command}

\subsubsection{Measure Output}
\label{Measure_Measurement_Output}
\index{measure!measurement output}
As previously mentioned, measured results are reported to the output
and log file.  Additionally, for \texttt{TRAN} measures, the results
are stored in files called
\texttt{circuitFileName.mt\#}, where the suffixed number (\texttt{\#}) starts
at \texttt{0} and increases for multiple iterations (\texttt{.STEP}
iterations) of a given simulation. For \texttt{DC} measures, the results are
stored in the files \texttt{circuitFileName.ms\#}, while \texttt{AC} and
\texttt{NOISE} measures use the files \texttt{circuitFileName.ma\#}.

A user-defined measure can also be output at each time-step via
inclusion in a .PRINT command.  For example, this netlist excerpt
outputs the integral of {\tt V(1)} at each time step.  The measure
value {\tt TINTV1} is then also output at the end of the simulation to
both the standard output and the \texttt{.mt\#} (or \texttt{.ms\#}
or \texttt{.ma\#}) files.
\begin{alltt}
.MEASURE TRAN TINTV1 INTEG V(1)
.PRINT TRAN FORMAT=NOINDEX V(1) TINTV1
\end{alltt}

The output for successful and failed measures to the standard output
(and log files) provides more information than just the measure's
calculated values.  As an example, for a successful and failed {\tt
MAX} measure the standard output would be:

\begin{alltt}
MAXVAL = 0.999758 at time = 0.000249037
Measure Start Time= 0	Measure End Time= 0.001

Netlist warning: MAXFAIL failed. TO value < FROM value
MAXFAIL = FAILED at time = 0
Measure Start Time= 1	Measure End Time= 0.001
\end{alltt}

In general, information on the measurement window, the time(s) that
the measure's value(s) were calculated and a possible cause for a
failed measure are output to standard output for all measures except
for {\tt FOUR}. This information is similar, but not identical, to
HSPICE's verbose output.  For a failed {\tt FOUR} measure, the
standard output will have ``FAILED'', but there may be less
information provided as to why the {\tt FOUR} measure failed.

In this example, the \texttt{circuitFileName.mt\#} file would have the following output:
\begin{alltt}
MAXVAL = 0.999758
MAXFAIL = -1
\end{alltt}

\subsubsection{Measurement Windows}
\label{Measure_Measurement_Windows}
\index{measure!measurement windows}
There is an implicit precedence when multiple qualifiers are specified
to limit the measurement window for a given {\tt .MEASURE} statement
for {\tt TRAN} measures.  In general, \Xyce{} first considers the
time-window criteria of the {\tt FROM}, {\tt TO} and {\tt TD}
qualifiers.  If the simulation time is within that user-specified
time-window then the {\tt RISE}, {\tt FALL}, {\tt CROSS} are
qualifiers are counted and/or the {\tt TRIG}, {\tt TARG} and {\tt
WHEN} qualifiers are evaluated.

The following netlist excerpt shows simple examples where the {\tt
.MEASURE} statement may return the default value because the measure
``failed''.  For {\tt riseSine}, this may occur because {\tt V(1)}
never has an output value of 1.0 because of the time steps chosen
by \Xyce{}. So, careful selection of the threshold values in WHEN,
TRIG and TARG clauses may be needed in some cases.  For {\tt
fallPulseFracMax}, the simulation interval is too short and the {\tt
TARG} value of 0.3 for {\tt V(2)} is not reached within the specified
one-second simulation time.  For {\tt maxSine}, the {\tt FROM}, {\tt
TO} and {\tt TD} values yield an empty time interval, which is
typically an error in netlist entry.
\begin{alltt}
VS  1  0  SIN(0 1.0 0.5 0 0)
VP  2  0  PULSE( 0 10 0.2 0.2 0.2 0.5 2)
R1  1  0  100K
R2  2  0  100K
.TRAN 0  1
.PRINT TRAN FORMAT=NOINDEX V(1) V(2)
.MEASURE TRAN riseSine TRIG V(1)=0 TARG V(1)=1.0
.MEASURE TRAN fallPulseFracMax TRIG V(2) FRAC_MAX=0.97
+ TARG V(2) FRAC_MAX=0.03
.MEASURE TRAN maxSine MAX V(1) FROM=0.2 TO=0.25 TD=0.5
\end{alltt}
The intent in \Xyce{} is for the measurement window to be the
intersection of the {\tt FROM-TO} and {\tt TD} windows, if both are
specified.  As noted above, the use of both {\tt FROM-TO} and {\tt TD}
windows can lead to an empty measurement window.  So, that usage is
not recommended.

\subsubsection{Expression Support}
\index{measure!expression support}
These measure ``qualifiers'' (\texttt{TO}, \texttt{FROM}, \texttt{TD},
\texttt{RISE}, \texttt{FALL}, \texttt{CROSS}, \texttt{AT}, \texttt{OFF},
\texttt{ON}, \texttt{DEFAULT\_VAL} and \texttt{VAL}) support expressions.
The caveat is that the expression must evaluate to a constant at the time
that each measure object is made.  So, that expression can not depend
on solution variables or lead currents.  This limitation matches HSPICE.
It also can not depend on a global parameter.  Finally, it can not depend
on another measure's value, which is an allowed syntax in HSPICE.

Simple examples of allowed syntaxes for qualifiers are as follows, where
all three measures will get the same answer:
\begin{alltt}
.PARAM t1=0.2
.PARAM t2=0.3
.MEASURE TRAN M1 PP V(1) FROM=`0.1+0.2'
.MEASURE TRAN M2 PP V(1) FROM=\{0.1+t1\}
.MEASURE TRAN M3 PP V(1) TO=\{t2\}
\end{alltt}
Expressions should also work in FIND-WHEN, WHEN and TRIG-TARG measures. The
preferred Xyce syntax with curly braces and the three legal HSPICE syntaxes
for expressions should all work.  However, note that the two HSPICE
expression syntaxes shown below are only legal in \Xyce{} \texttt{.MEASURE}
statements.
\begin{alltt}
.PARAM a1=0.1
.PARAM a2=0.7
.MEASURE TRAN M4 FIND V(2) WHEN V(1)=\{a1\}
.MEASURE TRAN M4PAR FIND V(2) WHEN V(1)=PAR(`a1') ; HSPICE exp. syntax
.MEASURE TRAN M4PAREN FIND V(2) WHEN V(1)=(`a1')  ; HSPICE exp. syntax
.MEASURE TRAN M5 WHEN V(1)=\{a1\}
.MEASURE TRAN M6 TRIG \{v(1)-0.1\} VAL=\{a1\} TARG \{v(1)-0.5\} VAL=\{a2\}
\end{alltt}

\subsubsection{Re-Measure}
\label{Measure_ReMeasure}
\index{measure!re-measure}
\Xyce{} can re-calculate (or re-measure) the values for {\tt .MEASURE} and/or
{\tt .FFT} statements using existing \Xyce{} output files.  This is useful for
tuning {\tt .MEASURE} and/or {\tt .FFT} statements to better capture response
metrics for a circuit when the underlying simulation runtime is long.  To use this
functionality, add the command line argument {\tt -remeasure <file>},
where {\tt <file>} is a \Xyce{}-generated {\tt .prn}, {\tt .csv} or
{\tt .csd} output file.

There are several important limitations with {\tt -remeasure}:
\begin{XyceItemize}

  \item The data required by the {\tt .MEASURE} and/or {\tt .FFT} statements
   must have been output in the simulation output file.  When using
  {\tt -remeasure}, \Xyce{} does not recalculate the full solution, but
  uses the data supplied in the output file instead.  Thus, everything
  a {\tt .MEASURE} and/or {\tt .FFT} statement needs to calculate its results
  must be in the output file.  So, the nodal voltages (e.g., node A), lead
  currents (e.g, for device R1) and branch currrents requested by the
  {\tt .MEASURE} statements must have been used, at least once, on the
  {\tt .PRINT} statement in the form of {\tt V(A)}, {\tt N(a)} or {\tt
  I(R1)}.  They can not only appear on the {\tt .PRINT} line within an
  expression or a voltage-difference operator.

  \item Only voltage node values, lead currents and branch currents
  can be used in {\tt .MEASURE} statements while using {\tt
  -remeasure}.  Power values will not be interpreted correctly during
  a re-measure operation.  A work-around for that limitation is
  illustrated below.

  \item {\tt -remeasure} only works with {\tt .tran} or {\tt .dc}
  analyses.  However, it can be used with {\tt .STEP} in both
  cases. It is not currently supported for {\tt .ac} analyses.

  \item For {\tt .tran} analyses, {\tt -remeasure} works with {\tt
  .prn}, {\tt .csv} and {\tt .csd} formatted output data.  However, it
  might only work with {\tt .csv} and {\tt .csd} files generated
  by \Xyce{}.  \item For {\tt .dc} analyses, {\tt -remeasure} works
  with {\tt .prn} and {\tt .csd} formatted output data.  However, it
  might only work with {\tt .csd} files generated by \Xyce{}.

  \item {\tt -remeasure} will fail if the netlist has a {\tt .op}
  statement that precedes the {\tt .tran} or {\tt .dc} statement.
  This can be fixed by either moving the {\tt .op} statement or by
  temporarily commenting the {\tt .op} statement out during {\tt
  -remeasure}.
\end{XyceItemize}

As an example in using {\tt -remeasure}, consider a netlist called
{\tt myCircuit.cir} which had previously been run in \Xyce{} and
produced the output file {\tt myCircuit.cir.prn}.  One could run {\tt
-remeasure} with the following command:
\begin{alltt}
Xyce -remeasure myCircuit.cir.prn myCircuit.cir
\end{alltt}

A work-around for re-measuring power values (e.g., for device R1) is
to use this combination of {\tt .PRINT} and {\tt .MEASURE} lines in
the netlist.  As noted above, expressions will work with re-measure if
all of the quantities used in the expression also appear outside of an
expression on the {\tt .PRINT} line.
\begin{alltt}
R1 a b 1
.PRINT TRAN V(a) V(b) I(R1)
.MEASURE TRAN PR1B MAX \{(V(a)-V(b))*I(R1)\}
\end{alltt}

\subsubsection{RISE, FALL and CROSS Qualifiers}
The \texttt{RISE}, \texttt{FALL} and \texttt{CROSS} qualifiers are
supported for more measures types, and in more ways, in \Xyce{} than
in HSPICE for \texttt{TRAN} meaures.  This sections explains those
differences and supplies some examples.  One key difference is that
\Xyce{} supports two different ``modes'' for these qualifiers for
\texttt{TRAN} meaures.

The first mode is ``level-crossing'', where
the \texttt{RISE}, \texttt{FALL} and \texttt{CROSS} counts are
incremented each time the measured signal (e.g,
\texttt{ V(a)}) crosses the user-specified level (termed
\texttt{crossVal} here).  If we define \texttt{currentVal} and \texttt{lastVal}
as the current and previous values of \texttt{V(a)}, and \texttt{riseCount},
\texttt{fallCount} and \texttt{crossCount} as the number of rises, falls and
crosses that have occurred, then the pseudo-code for this mode is:
\begin{alltt}
if ( (currentVal-crossVal >= 0.0) AND (lastVal-crossVal < 0.0) )
\{
  riseCount++;
\}
else if( (currentVal-crossVal) <= 0.0) AND (lastVal-crossVal > 0.0) )
\{
  fallCount++;
\}
\end{alltt}

For this mode, the \texttt{crossCount} is then incremented if either
the
\texttt{riseCount}  or the \texttt{fallCount} was incremented.  This mode
should work identically to HSPICE for the \texttt{DERIV-WHEN}, \texttt{FIND-WHEN},
\texttt{WHEN} and \texttt{TRIG-TARG} measures.

The second mode is termed ``absolute''. In this mode, \Xyce{} attempts
to auto-detect whether the measured waveform has started a new rise or
fall. However, the \texttt{crossCount} is still evaluated against a
fixed \texttt{crossVal} of 0.  This mode may be useful for pulse
waveforms with sharp rises and falls, where the waveform's maximum (or
minimum) level is not exactly known in advance.  It may not work well
with noisy waveforms.  If we define two Boolean variables
\texttt{isRising} and \texttt{isFalling} then the pseudo-code for this mode is:
\begin{alltt}
if( (currentVal > lastVal) AND !isRising )
\{
  isRising= true;
  isFalling = false;
  riseCount++;
\}
else if( (currentVal < lastVal) AND !isFalling )
\{
  isRising = false;
  isFalling = true;
  fallCount++;
\}
if ( ( (currentVal >= 0.0) AND (lastVal < 0.0) ) OR
     ( (currentVal <= 0.0) AND (lastVal > 0.0) ) )
\{
  crossCount++;
\}
\end{alltt}
The following table shows which of these two modes are supported for which
\Xyce{} measures.
% Sandia National Laboratories is a multimission laboratory managed and
% operated by National Technology & Engineering Solutions of Sandia, LLC, a
% wholly owned subsidiary of Honeywell International Inc., for the U.S.
% Department of Energy’s National Nuclear Security Administration under
% contract DE-NA0003525.

% Copyright 2002-2020 National Technology & Engineering Solutions of Sandia,
% LLC (NTESS).


%%
%% Which Measures support which modes for RISE, FALL and CROSS
%%
\begin{longtable}[h] {>{\raggedright\small}m{1.0in}|>{\raggedright\let\\\tabularnewline\small}m{2.5in}
  |>{\raggedright\let\\\tabularnewline\small}m{2.5in}}
  \caption{RISE, FALL and CROSS Support in .MEASURE.} \\ \hline
  \rowcolor{XyceDarkBlue}
  \color{white}\bf Measure &
  \color{white}\bf Level-Crossing &
  \color{white}\bf Absolute \\ \hline \endfirsthead
  \rowcolor{XyceDarkBlue}
  \color{white}\bf Measure &
  \color{white}\bf Level-Crossing &
  \color{white}\bf Absolute \\ \hline \endhead
  \label{RISE_FALL_CROSS}

  \texttt{DERIV} & The  \texttt{crossVal} is either the value of the 
                 \texttt{WHEN} clause, or it can be set to a fixed 
                 level with \texttt{RFC\_LEVEL}
                 & No \\ \hline
  \texttt{FIND-WHEN} and \texttt{WHEN} & The  \texttt{crossVal} is either the 
         value of the \texttt{WHEN} clause, or it can be set to a fixed level with 
         \texttt{RFC\_LEVEL} & No \\ \hline
  \texttt{MAX} & A fixed \texttt{crossVal} can be set with \texttt{RFC\_LEVEL} 
               & Default, if \texttt{RFC\_LEVEL} is not set \\ \hline
  \texttt{MIN} & A fixed \texttt{crossVal} can be set with \texttt{RFC\_LEVEL} 
               & Default, if \texttt{RFC\_LEVEL} is not set \\ \hline
  \texttt{PP} & A fixed \texttt{crossVal} can be set with \texttt{RFC\_LEVEL} 
              & Default, if \texttt{RFC\_LEVEL} is not set \\ \hline
  \texttt{TRIG} and \texttt{TARG} & The levels are set separately by the values 
                in the \texttt{TRIG} and \texttt{TARG} clauses & If 
                \texttt{FRAC\_MAX} is used\\ \hline
\end{longtable}


As simple examples of these two modes for the \texttt{MAX} measure, consider
the following netlist:
\begin{alltt}
*examples of RFC modes
VPWL1  1  0 PWL(0 0 0.2 0.5 0.4 0 0.6 0.75 0.8 0 1.0 0.75 1.2 0.0)
R1  1  0  100
.TRAN 0 1.2s
.MEASURE TRAN MAX1 MAX V(1) RISE=1
.MEASURE TRAN MAX2 MAX V(1) RISE=1 RFC_LEVEL=0.6
.MEASURE TRAN MAX3 MAX V(1) FALL=1 RFC_LEVEL=0.5
.PRINT TRAN V(1) MAX1 MAX2 MAX3
.END
\end{alltt}
The descriptive output to standard output would then be:
\begin{alltt}
MAX1 = 5.000000e-01 at time = 2.000000e-01
Measure Start Time= 0.000000e+00        Measure End Time= 1.200000e+00
Rise 1: Start Time= 1.000000e-10        End Time= 4.000000e-01

MAX2 = 7.500000e-01 at time = 6.000000e-01
Measure Start Time= 0.000000e+00        Measure End Time= 1.200000e+00
Rise 1: Start Time= 5.600000e-01        End Time= 9.500000e-01

MAX3 = 7.500000e-01 at time = 1.000000e+00
Measure Start Time= 0.000000e+00        Measure End Time= 1.200000e+00
Fall 1: Start Time= 6.700000e-01        End Time= 1.060000e+00
\end{alltt}
The \texttt{MAX1} measure uses the ``absolute'' mode, so the first
rise begins with the very first time-step.  The maximum value in that
first rise interval for measure \texttt{MAX1} then occurs at
time=0.2s.  The \texttt{MAX2} measure uses the ``level-crossing'' mode
with a user-specified \texttt{RFC\_LEVEL} of 0.6V.  So, the first rise
interval for the \texttt{MAX2} measure begins at time=0.56s, and the
maximum value in that first rise interval occurs at time=0.6s.  The
\texttt{MAX3} measure illustrates an important point.  A ``fall'' is not
recorded for the \texttt{MAX3} measure at t=0.2 seconds, but a
``rise'' (and ``cross'') would be recorded, since the value
of \texttt{V(1)} is exactly equal to the
user-specified \texttt{RFC\_LEVEL}. So, the first fall interval for
measure
\texttt{MAX3} begins at time=0.67s, when V(1) first passes through the
user-specified \texttt{RFC\_LEVEL} of 0.5V.

\subsubsection{Additional Examples}
\label{Measure_Additional_Examples}
\index{measure!additional examples}
Pulse width measurements in \Xyce{} can be done as follows, based on
this netlist excerpt.  This may be useful for ensemble runs, where the
maximum value of a one-shot pulse is not known in advance.  The first
syntax uses three measure statements to measure the 50\% pulse width,
and works with noisy waveforms.  The second syntax uses only one
measure statement, but may not always work with noisy waveforms.

\begin{alltt}
* pulse-width measurement example 1
.measure tran rise50FracMax trig v(1) frac_max=0.5 targ v(1) frac_max=1
.measure tran fall50FracMax trig v(1) frac_max=1 targ v(1) frac_max=0.5
.measure tran 50width EQN\{rise50FracMax + fall50FracMax\}

* pulse-width measurement example 2
.measure tran 50widthFracMax trig v(1) frac_max=0.50 targ v(1) frac_max=0.50 FALL=1
\end{alltt}

In some cases, the user may wish to print out both the measure value
and measure time (or the value of the first variable on the {\tt .DC}
line) of a
\texttt{MAX} or \texttt{MIN} measure to the \texttt{.mt0} file.  For a TRAN measure,
this can be done for these two measures with the \texttt{OUTPUT}
keyword as follows:
\begin{alltt}
* printing maximum value and time of maximum value to .mt0 file
.TRAN 0 1
V1 1 0 PWL 0 0 0.5 1 1 0
R1 1 0 1
.MEASURE TRAN MAXVAL MAX V(1)
.MEASURE TRAN TIMEOFMAXVAL V(1) OUTPUT=TIME
\end{alltt}
The output to the \texttt{.mt0} file would be:
\begin{alltt}
MAXVAL = 1.000000e+00
TIMEOFMAXVAL = 5.000000e-01
\end{alltt}
The descriptive output to standard output would be the same for both
measures though.  The measure value and measure time are not
re-ordered in the descriptive output when \texttt{OUTPUT=VALUE} is
used for the \texttt{MAX} or
\texttt{MIN} measures.
\begin{alltt}
MAXVAL = 1.000000e+00 at time = 5.000000e-01
Measure Start Time= 0.000000e+00        Measure End Time= 1.000000e+00

TIMEOFMAXVAL = 1.000000e+00 at time = 5.000000e-01
Measure Start Time= 0.000000e+00        Measure End Time= 1.000000e+00
\end{alltt}

For a DC measure, one would use \texttt{OUTPUT=SV} instead
of \texttt{OUTPUT=TIME}.  In that case, the ``sweep value'' (SV) is
the value of the first variable on the
\texttt{.DC} line. For an AC or NOISE measure, one would use \texttt{OUTPUT=FREQ}.

\subsubsection{Suppresssing Measure Output}
\label{Measure_Suppressing_Measure_Output}
\index{measure!suppressing measure output}
If the \Xyce{} output is post-processed with other programs, such as
Dakota, it may be desirable to only print a subset of the measure
values to the \texttt{.mt\#} (or \texttt{.ms\#} or \texttt{.ma\#})
files, but to print all of the measure output to standard output.  As
an example, these {\tt .MEASURE} statements:
\begin{alltt}
.TRAN 0  2ms
.measure tran minSineOne  min V(1) print=none
.measure tran minSinTwo   min V(2) print=stdout
.measure tran minSinThree min V(3) print=all
.measure tran sinSinFive  min V(4)
\end{alltt}
would produce the following measure output in the {\tt .mt0} file:
\begin{alltt}
MINSINTHREE = -3.851422e-01
MINSINFOUR = -1.998548e+00
\end{alltt}
and the following measure output in standard output:
\begin{alltt}
MINSINTWO = -1.188589e+00 at time = 7.400000e-04
Measure Start Time= 0.000000e+00	Measure End Time= 2.000000e-03

MINSINTHREE = -3.851422e-01 at time = 2.400000e-04
Measure Start Time= 0.000000e+00	Measure End Time= 2.000000e-03

MINSINFOUR = -1.998548e+00 at time = 7.500000e-04
Measure Start Time= 0.000000e+00	Measure End Time= 2.000000e-03
\end{alltt}

\texttt{.OPTIONS MEASURE MEASPRINT=<val>} also provides the option to accomplish
these same effects, but for all of the measure statements in the
netlist.  The interactions between these two features are as follows.
If \texttt{MEASPRINT=ALL} is used, which is the default setting, then
the \texttt{PRINT} qualifier on a given
\texttt{.MEASURE} line will override that setting.  However,
\texttt{MEASPRINT=NONE}
and \texttt{MEASPRINT=STDOUT} will take precedence over
the \texttt{PRINT} qualifiers on individual \texttt{.MEASURE} lines.
Finally, the \texttt{MEASPRINT} option will be ignored during
remeasure, but the \texttt{PRINT} qualifiers on individual measure
lines will be used.

\texttt{.OPTIONS MEASURE MEASOUT=<val>} provides another way to suppress the output
of the \texttt{.mt\#} (or \texttt{.ms\#} or \texttt{.ma\#}) files.
See Section \ref{Options_Reference} for more details.  If given, this
option takes precedence over the \texttt{MEASPRINT} option setting.
However, it is also ignored during remeasure.

\subsubsection{ERROR Functions (ERR1 and ERR2)}
\label{Measure_Error_Functions}
\index{measure! err1 measure}
\index{measure! err2 measure}
This subsection defines the calculation functions for the \texttt{ERR1} and
\texttt{ERR2} measure types.  For the \texttt{ERR1} measure, the measure
value is calculated as follows, where $M_i$ and $C_i$ are the first and
second variables on the measure line and N is the number of time, frequency
or DC sweep values included in the measure calculation:
\begin{equation}
ERR1 = \sqrt{ \frac{1}{N}\sum_{i=1}^{N} \left(\frac{M_i - C_i}{max(MINVAL,\mid M_i \mid)}\right)^2}
\end{equation}

For the \texttt{ERR2} measure, the value is:
\begin{equation}
ERR2 = \frac{1}{N}\sum_{i=1}^{N} \left|\frac{M_i - C_i}{max(MINVAL,\mid M_i \mid)}\right|
\end{equation}

For both measures, if the absolute value of $M_i$ is less than the
\texttt{YMIN} value or greater than the \texttt{YMAX} value then the
{\tt ERR1} or {\tt ERR2} calculation does not consider that point. The
default for \texttt{YMIN} is 1.0-e15.  The default for \texttt{YMAX}
is 1.0e15.

\subsubsection{ERROR Measure}
\label{Measure_Error_Measure}
\index{measure! error measure}
The \Xyce{} ERROR measure is not the functional equivalent of the \texttt{ERR1}
or \texttt{ERR2} measures.  It is intended to solve a different problem, namely
the comparison of data in multiple simulation runs to an assumed ``gold standard''
read in from a file.  It also uses different comparision functions then the
\texttt{ERR1} and \texttt{ERR2} measures..  This  subsection lists some important
caveats with the use of the \texttt{ERROR} measure.
\begin{XyceItemize}

  \item The comparison file, specified with the {\tt FILE} qualifier,
    can be {\tt .prn}, {\tt .csv} and {\tt .csd} formatted output
    data.  However, the {\tt ERROR} measure might only work with {\tt
    .csv} and {\tt .csd} files generated by \Xyce{}.  \item The data
    in the comparison file is assumed to be ``non-step data'', from
    one simulation iteration.  The simulated data can use {\tt .STEP}
    though and the {\tt ERROR} measure values will be re-evaluated for
    each step.  \item For {\tt TRAN} measures, the values of the
    measured waveform are interpolated to the simulation times in the
    comparison waveform.  So, the norm calculation is inherently
    windowed to the time interval of the comparison waveform.  \item
    For {\tt DC} measures, interpolation is not used.  So, the values
    of the simulated and comparion waveforms are compared at the
    values specified by the {\tt DEPVARCOL} qualifier.  Any value for
    the {\tt INDEPVARCOL} qualifier specified on a {\tt DC} measure
    line will be ``silently ignored''.

  \item The time window constraints ({\tt TO}, {\tt FROM} and {\tt TD}
    qualifiers) are not supported for the {\tt ERROR} measure.  So, as
    noted above, the effective window for the norm calculation is set
    by the extent of the comparison waveform.

  \item The values in the column in the comparison file specified with
    the {\tt INDEPVARCOL} qualifier must be monotonically increasing
    for a {\tt TRAN} measure.  Otherwise, \Xyce{} will not run the
    simulation.

  \item The {\tt ERROR} measure currently supports the L1, L2 and
    INFNORM, with the default being the L2 norm.  If anything other
    than L1, L2 or INFNORM is specified, \Xyce{} will default to the
    L2 norm.  The descriptive output for each {\tt ERROR} measure,
    that is printed to standard output, will explicitly state which
    norm was used for each {\tt ERROR} measure.  (Note: The norm value
    is selected with the {\tt COMP\_FUNCTION} qualifier, and the
    allowed values are {\tt L1NORM}, {\tt L2NORM} and {\tt INFNORM}.)

\end{XyceItemize}

As a final note, the \texttt{ERROR} measure can enable the use
of \Xyce{} simulation output in optimization problems, like device
calibration.  However, for internal Sandia users, there may be better
approaches that leverage the combined capabilities of Sandia's Dakota
and \Xyce{} software packages.

\subsubsection{Operator Support for AC Mode Measures}
\label{Measure_AC_Op_Support}
\index{measure!supported operators for AC measures}
All of the operators supported on \texttt{.PRINT AC} lines are supported
for AC measure mode.  The linear parameter operators (e.g., \texttt{SR(1,1)})
are only supported when a \texttt{.LIN} analysis is done, but their values
can be used in \texttt{.MEASURE AC} statements in that case.

One caveat is that AC mode measures that use \texttt{V(a)} will actually
measure \texttt{VR(a)}. The same caveat applies to the use of \texttt{S(1,1)}.
An AC mode measure would measure \texttt{SR(1,1)} instead.

\subsubsection{Operator Support for NOISE Mode Measures}
\label{Measure_AC_Op_Support}
\index{measure!supported operators for AC measures}
All of the operators supported on \texttt{.PRINT NOISE} lines are supported
for NOISE measure mode.  One caveat is that NOISE mode measures that use
\texttt{V(a)} will actually measure \texttt{VR(a)}.

\subsubsection{Behavior for Unsupported Modes and Types}
\label{Measure_Unsupported_Types}
\index{measure!unsupported types}
The \texttt{.MEASURE} statement is supported
for \texttt{.TRAN}, \texttt{.AC}, \texttt{.DC} and \texttt{.NOISE}
analyses. It can be used with {\tt .STEP} in all four cases.
So, \Xyce{} does not support \texttt{HB} measure mode.  If that mode
is included in the netlist then \Xyce{}
parsing will fail and emit error messages.  Similarly, \Xyce{} parsing
will fail if the requested measure type is not supported for a given
measure mode (e.g., \texttt{OFF\_TIME} for a \texttt{AC}, \texttt{DC}
or \texttt{NOISE} measure).

\subsubsection{Compatibility with .DATA}
The \texttt{.DATA} command can be used to specify table-based \texttt{.AC},
\texttt{.DC} or \texttt{.NOISE} sweeps for those three analysis types. For
\texttt{AC} and \texttt{NOISE} measures, the ``swept variable'' then uses
the frequency values in the table specified on the \texttt{.AC} or
\texttt{.NOISE} line.

For \texttt{DC} measures, the swept variable uses the row index in the
table specified on the \texttt{.DC} line.  An example is as follows:
\begin{alltt}
* example of .DATA with DC measures
V1 1 0 1
R1 1 2 1
R2 2 0 1

.data test
+ r1           r2
+ 1.0e+00  4.0e+00
+ 4.0e+00  6.0e+00
+ 6.0e+00  4.0e+00
.enddata

.DC data=test
.print DC V(1) V(2)

.OPTIONS MEASURE MEASDGT=1
.MEASURE DC MAXV2TO MAX V(2) TO=2
.MEASURE DC MAXV2FROM MAX V(2) FROM=2

.END
\end{alltt}

The measure results reported in stdout will be as follows where the respective
maximum values occur for the R1 and R2 values given in the first and second rows
of the \texttt{test} table:
\begin{alltt}
MAXV2TO = 8.0e-01 at Table Row value = 1.0e+00
Measure Start Table Row Value= 1.0e+00  Measure End Table Row Value= 2.0e+00

MAXV2FROM = 6.0e-01 at Table Row value = 2.0e+00
Measure Start Table Row Value= 2.0e+00  Measure End Table Row Value= 3.0e+00
\end{alltt}

All valid measure types will return an answer when a data-based sweep is used
on the \texttt{.AC}, \texttt{.DC} or \texttt{.NOISE} line.  However, the results
for \texttt{AVG}, \texttt{DERIV}, \texttt{FIND}, \texttt{INTEG}, \texttt{RMS}
and \texttt{WHEN} measures may be ``non-physical'' if the frequency values in the
data table are not monotonically increasing.  In addition, for \texttt{DC} measures
the effective step size between table rows is equal to one.

\subsubsection{HSPICE Compatibility}
\label{Measure_HSpice_Compatibility}
\index{measure!HSPICE compatibility}
There are known incompatibilities between the \Xyce{} and HSPICE implementation
of {\tt .MEASURE}.  They include the following:

\begin{XyceItemize}
  \item Since \texttt{.AC} and \texttt{.NOISE} are separate analysis types in
        \Xyce{}, there are separate \texttt{AC} and \texttt{NOISE} measure modes.

  \item Several of the \Xyce{} measure types ({\tt DUTY}, {\tt EQN},
         {\tt FREQ}, {\tt FOUR}, {\tt ON\_TIME}, and {\tt OFF\_TIME})
         and qualifers (e.g., {\tt FRAC\_MAX}) are not found in
         HSPICE.  Several HSPICE measures are not supported
         in \Xyce{}.

  \item The HSPICE qualifers of {\tt REVERSE} and {\tt PREVIOUS} are
  not supported in \Xyce{}.

  \item The HSPICE {\tt .POWER} statement, which prints out a table
        with the AVG, RMS, MIN and MAX measures for each specified
        signal, is not supported in \Xyce{}.

  \item \Xyce{} generally supports more qualifiers ({\tt FROM}, {\tt
        TO}, {\tt TD}, {\tt RISE}, {\tt FALL} and {\tt CROSS}) for the
        measurement windows for a given measure-type.  So, some
        legal \Xyce{} syntaxes may not be legal in HSPICE.

  \item For {\tt TRIG} and {\tt TARG} clauses in \Xyce{}, the {\tt TD}
        qualifier applies to both the {\tt TRIG} and {\tt TARG}
        qualifiers.  HSPICE allows the specification of separate
        time-delays for the {\tt TRIG} and {\tt TARG} clauses.

  \item \Xyce{} will not return a negative value from a {\tt TRIG} and
        {\tt TARG} measure.  The {\tt TARG} clause is only evaluated
        if the {\tt TRIG} clause is satisfied.  This behavior is
        different from HSPICE.

  \item The Xyce {\tt EQN} measure can calculate an expression based
        on other measure values.  So, one of its usages is similar to
        the HSPICE {\tt PARAM} measure.  However, their syntaxes are
        different.

  \item A mismatch between the measure mode and the analysis mode
        (e.g., a {\tt DC} measure in a netlist that uses a {\tt .TRAN}
        analysis statement) will cause a \Xyce{} netlist parsing
        error.  That same mismatch might be silently ignored by
        HSPICE.

  \item How \Xyce{} and HSPICE handle ``steps'' may be different.
        In \Xyce{}, the ``steps'' in the measured data (e.g., the
        generation of new \texttt{.mt\#} or \texttt{.ms\#}
        or \texttt{.ma\#} files) are triggered by the variable(s) on
        the {\tt .STEP} line, but not by the variable(s) on the {\tt
        .DC} line.

  \item Expressions on \texttt{.MEASURE} lines are supported in fewer
        contexts then in HSPICE.  See the ``Expression Support''
        subsection for more details.

  \item The settings for the \texttt{MEASFAIL} and \texttt{MEASOUT}
        options are only used if those options are explicitly given in
        the netlist.  Otherwise, the \Xyce{} defaults will be used.
\end{XyceItemize}

Additional syntax differences between {\tt TRIG} and {\tt TARG}
clauses in \Xyce{} and HSPICE are as follows.  In HSPICE, a {\tt
RISE}, {\tt FALL} or {\tt CROSS} keyword must be specified in the
following measure statement.  Those {\tt RISE} keywords are optional
with this particular syntax example in \Xyce{}.  If they are omitted,
then \Xyce{} uses a default value of 1.
\begin{alltt}
.measure tran riseSine trig v(1)=0.01 RISE=1 targ v(1)=0.99 RISE=1
\end{alltt}
The following HSPICE syntax ({\tt VAL=0.9}) is supported in \Xyce{}
for {\tt TRIG} and {\tt TARG} measures. However, the preferred \Xyce{}
syntax would use {\tt targ v(1)=0.9} instead.
\begin{alltt}
.measure tran riseSine trig v(1) AT=0.0001 targ v(1) VAL=0.9 RISE=1
\end{alltt}

The remainder of this subsection discusses alternate syntaxes
for \Xyce{} measure lines that are supported for improved HSPICE
compatibility.  The definitions of the measure syntaxes given at the
beginning of this \texttt{.MEASURE} section give the preferred
\Xyce{} syntaxes.  However, \texttt{PARAM} (and the equivalent \texttt{EQN})
measure lines are allowed with, or without, the equal sign after
the \texttt{PARAM} keyword.  So, these two \Xyce{} measure statements
are equivalent:
\begin{alltt}
.measure tran noEqualSgn PARAM \{v(1)+1.0\}
.measure tran equalSgn PARAM=\{v(1)+1.0\}
\end{alltt}

There are multiple expression syntaxes that are allowed in various
contexts on HSPICE measure lines.  So, all of these example syntaxes
are allowed in expression contexts on \Xyce{} measure lines.  (Note:
Only the first single-quote-delimited expression format is supported
in all \Xyce{} expression contexts, in addition to the \Xyce{}
curly-braces format.)
\begin{alltt}
.measure tran curlyBraces MAX \{V(1)+1\}
.measure tran singleQuote MAX `V(1)+1'
.measure tran parenSingleQuote MAX (`V(1)+1')
.measure tran parSyntax MAX PAR(`V(1)+1')
\end{alltt}

Undelimited expressions are allowed in some contexts in HSPICE.
However, the syntax for the notLegalInXyce measure shown below
is not allowed in \Xyce{}, since it uses an undelimited expression.
\begin{alltt}
.measure tran PLUS PP PAR(`V(1)+V(2)')
.measure tran notLegalInXyce PARAM PLUS+2.0 ; not legal
\end{alltt}


