% Sandia National Laboratories is a multimission laboratory managed and
% operated by National Technology & Engineering Solutions of Sandia, LLC, a
% wholly owned subsidiary of Honeywell International Inc., for the U.S.
% Department of Energy’s National Nuclear Security Administration under
% contract DE-NA0003525.

% Copyright 2002-2022 National Technology & Engineering Solutions of Sandia,
% LLC (NTESS).


Send analysis results to an output file.
\index{\texttt{.PRINT}}
\index{results!print}

\Xyce{} allows multiple output files to be created during the run and
supports several options for each.

\newenvironment{PrintCommandTable}[1]
               {\renewcommand{\arraystretch}{1.2}
                 \newcommand{\category}[1]{\multicolumn{3}{c}{\smallskip\color{XyceDarkBlue}\em\bfseries ##1}}
                 \begin{longtable}{>{\ttfamily\small}m{3in}<{\normalfont}>{\raggedright\small}m{1.75in}>{\raggedright\let\\\tabularnewline\small}m{1.75in}}
                   \caption{#1} \\ \hline
                   \rowcolor{XyceDarkBlue}
                   \color{white}\normalfont\bf Trigger &
                   \color{white}\bf Files &
                   \color{white}\bf Columns/Description \endfirsthead
                   \caption[]{#1} \\ \hline
                   \rowcolor{XyceDarkBlue}
                   \color{white}\normalfont\bf Trigger &
                   \color{white}\bf Files &
                   \color{white}\bf Columns/Description \endhead}
               {\end{longtable}}

\begin{Command}

\format
\begin{alltt}
.PRINT <print type> [FILE=<output filename>]
+ [FORMAT=<STD|NOINDEX|PROBE|TECPLOT|RAW|CSV|GNUPLOT|SPLOT>]
+ [WIDTH=<print field width>]
+ [PRECISION=<floating point output precision>]
+ [FILTER=<absolute value below which a number outputs as 0.0>]
+ [DELIMITER=<TAB|COMMA>] [TIMESCALEFACTOR=<real scale factor>]
+ [OUTPUT\_SAMPLE\_STATS=<boolean>] [OUTPUT\_ALL\_SAMPLES=<boolean>]
+ <output variable> [output variable]*
\end{alltt}

\examples
\begin{alltt}
.print tran format=tecplot V(1) I(Vsrc) \{V(1)*(I(Vsrc)**2.0)\}

.PRINT TRAN FORMAT=PROBE FILE=foobar.csd V(1) \{abs(V(1))-5.0\}

.PRINT DC FILE=foobar.txt WIDTH=19 PRECISION=15 FILTER=1.0e-10
+ I(VSOURCE5) I(VSOURCE6)

.print tran FORMAT=RAW V(1) I(Vsrc)

R1 1 0 100
X1 1 2 3 MySubcircuit
V1 3 0 1V
.SUBCKT MYSUBCIRCUIT 1 2 3
R1 1 2  100K
R2 2 4  50K
R3 4 3  1K
.ENDS

.PRINT DC V(X1:4) V(2) I(V1)
\end{alltt}


\arguments

\begin{Arguments}

\argument{print type}

A print type is the name of an analysis, one of the analysis specific
print subtypes, or a specialized output command.

% Sandia National Laboratories is a multimission laboratory managed and
% operated by National Technology & Engineering Solutions of Sandia, LLC, a
% wholly owned subsidiary of Honeywell International Inc., for the U.S.
% Department of Energy’s National Nuclear Security Administration under
% contract DE-NA0003525.

% Copyright 2002-2022 National Technology & Engineering Solutions of Sandia,
% LLC (NTESS).

{
\renewcommand{\arraystretch}{1.2}
\begin{tabular}{>{\ttfamily\small}m{1.5in}<{\normalfont}>{\ttfamily\small}m{0.75in}<{\normalfont}m{2.25in}@{}}
  \rowcolor{XyceDarkBlue}
  \color{white}\normalfont\bf Analysis &
  \color{white}\bf Print Type &
  \color{white}\bf Description \\
.AC & AC & Sets default variable list and formats for print subtypes \\ \hline
.AC & AC\_IC & Overrides variable list and format for AC initial conditions \\ \hline
.DC & DC &  \\ \hline
.EMBDEDDEDSAMPLING & ES & \\ \hline
.HB & HB & \\ \hline
.HB & HB\_FD & Overrides variable list and format for HB frequency domain \\ \hline
.HB & HB\_IC & Overrides variable list and format for HB initial conditions \\ \hline
.HB & HB\_STARTUP & Overrides variable list and format for HB start up \\ \hline
.HB & HB\_TD & Overrides variable list and format for HB time domain \\ \hline
.NOISE & Noise & Outputs Noise spectral density curves\\ \hline
.TRAN & TRAN &  \\ \hline
\multicolumn{3}{c}{\smallskip\color{XyceDarkBlue}\em\bfseries Specialized Output Commands} \\
\emph{Homotopy} & HOMOTOPY & Sets variable list and format for homotopy \\ \hline
.SENS & SENS & Sets variable list and format for sensitivity \\ \hline
\end{tabular}
}


A netlist may contain many \texttt{.PRINT} commands, but only commands
with analysis types which are appropriate for the analysis being
performed are processed.  This feature allows you to generate multiple
formats and variable sets in a single analysis run.

For analysis types that generate multiple output files, the print
subtype allows you to specify variables and output parameters for each
of those output files.  If there is no \texttt{.PRINT <subtype>} provided in the
net list, the variables and parameters from the analysis type will be
used.

\argument{FORMAT=<STD|NOINDEX|PROBE|TECPLOT|RAW|CSV|GNUPLOT|SPLOT>}

The output format may be specified using the \texttt{FORMAT} option.
The \texttt{STD} format outputs the data divided up into data columns.
The \texttt{NOINDEX} format is the same as the \texttt{STD} format
except that the index column is omitted. The \texttt{PROBE} format
specifies that the output should be formatted to be compatible with the
\index{PSpice!Probe} PSpice Probe plotting utility.  The
\texttt{TECPLOT} format specifies that the output should be formatted to
be compatible with the Tecplot plotting program.  The \texttt{RAW}
format specifies that the output should comply with the SPICE binary
rawfile format.  The {\bf -a} command line option, in conjunction with
\texttt{FORMAT=RAW} on the \texttt{.PRINT} line, can then be used to output an
ASCII rawfile.  The \texttt{CSV} format specifies that the output file
should be a comma-separated value file with a header indicating the
variables printed in the file.  It is similar to, but not identical to
using \texttt{DELIMITER=COMMA}; the latter will also print a footer that
is not compatible with most software that requires CSV format.  The \texttt{GNUPLOT}
(or \texttt{SPLOT}) format is the same as the \texttt{STD} format except
that if \texttt{.STEP} is used then two (or one) blank lines are inserted
before the data for steps 1,2,3,... where the first step is step 0. The
\texttt{SPLOT} format is useful for when the ``splot'' command in gnuplot
is used to produce 3D perspective plots.

\argument{FILE=<output filename>}

Specifies the name of the file to which the output will be written.
See the ``Results Output and Evaluation Options'' section of the 
\Xyce{} Users' Guide\UsersGuide for more information on how this
feature works for analysis types (e.g., AC and HB) that can produce 
multiple output files. 

\argument{WIDTH=<print field width>}

Controls the output width used in formatting the output.

\argument{PRECISION=<floating point precision>}

Number of floating point digits past the decimal for output data.

\argument{FILTER=<filter floor value>}

Used to specify the absolute value below which output variables will be
printed as \texttt{0.0}.

\argument{DELIMITER=<TAB|COMMA>}

Used to specify an alternate delimiter in the STD or NOINDEX format
output.

\argument{TIMESCALEFACTOR=<real scale factor>}

Specify a constant scaling factor for time.  Time is normally printed in
units of seconds, but if one would like the units to be milliseconds,
then set TIMESCALEFACTOR=1000.

\argument{OUTPUT\_SAMPLE\_STATS=<boolean>}
Output the sample statistics for an \texttt{EMBEDDEDSAMPLING}
analysis.This argument is only supported for \texttt{.PRINT ES}.
Its default value is true.  Section \ref{EMBEDDEDSAMPLING_section}
has more details.

\argument{OUTPUT\_ALL\_SAMPLES=<boolean>}
Output all of the sample values for an \texttt{EMBEDDEDSAMPLING}
analysis.  This argument is only supported for \texttt{.PRINT ES}.
Its default value is false.  Section \ref{EMBEDDEDSAMPLING_section}
has more details.

\argument{<output variable>}

Following the analysis type and other options is a list of output
variables. There is no upper bound on the number of output variables.
The output is divided up into data columns and output according to any
specified options (see options given above).  Output variables can be
specified as:

\begin{XyceItemize}
\item \texttt{V(<circuit node>)} to output the voltage at \texttt{<circuit node>}
\item \texttt{V(<circuit node>,<circuit node>)} to output the voltage difference between the first \texttt{<circuit node>} and second \texttt{<circuit node>}
\item \texttt{I(<device>)} to output current through a two terminal device
\item \texttt{I<lead abbreviation>(<device>)} to output current into a particular lead of a three or more terminal device (see the Comments, below, for details) \index{lead currents}
\item \texttt{P(<device>)} or \texttt{W(<device>)} to output the power dissipated/generated in a device. 
At this time, not all devices support power calculations. In addition, the results for semiconductor devices
(D, J, M, Q and Z devices) and the lossless transmission device (T device) may differ from other simulators.  
Consult the Features Supported by Xyce Device Models table
in section \ref{Analog_Devices} and the individual sections on each device for more details.  Finally,
power calculations are not supported for any devices for \texttt{.AC} and \texttt{.NOISE} analyses.
\item \texttt{N(<device internal variable>)} to output a specific device's internal variable. (The comments section
below has more detail on this syntax.)
\item \texttt{\{expression\}} to output an expression
\item \texttt{<device>:<parameter>} to output a device parameter
\item \texttt{<model>:<parameter>} to output a model parameter
\end{XyceItemize}
When the analysis type is AC, HB or Noise, additional output variable formats are available:
\begin{XyceItemize}
\item \texttt{VR(<circuit node>)} to output the real component of voltage response at a point in the circuit
\item \texttt{VI(<circuit node>)} to output the imaginary component of voltage response at a point in the circuit
\item \texttt{VM(<circuit node>)} to output the magnitude of voltage response
\item \texttt{VP(<circuit node>)} to output the phase of voltage response in degrees
\item \texttt{VDB(<circuit node>)} to output the magnitude of voltage response in decibels.
\item \texttt{VR(<circuit node>,<circuit node>)} to output the real component of voltage response between two nodes in the circuit
\item \texttt{VI(<circuit node>,<circuit node>)} to output the imaginary component of voltage response between two nodes in the circuit
\item \texttt{VM(<circuit node>,<circuit node>)} to output the magnitude of voltage response between two nodes in the circuit
\item \texttt{VP(<circuit node>,<circuit node>)} to output the phase of voltage response between two nodes in the circuit in degrees
\item \texttt{VDB(<circuit node>,<circuit node>)} to output the magnitude of voltage response between two nodes in the circuit, in decibels
\item \texttt{IR(<device>)} to output the real component of the current through a two terminal device
\item \texttt{II(<device>)} to output the imaginary component of the current through a two terminal device
\item \texttt{IM(<device>)} to output the magnitude of the current through a two terminal device
\item \texttt{IP(<device>)} to output the phase of the current through a two terminal device in degrees
\item \texttt{IDB(<device>)} to output the magnitude of the current through a two terminal device in decibels.
\end{XyceItemize}

In AC and Noise analyses, outputting a voltage node without any of these
optional designators results in output of the real and imaginary parts of the signal.
Note that under AC and Noise analyses, current variables are only supported for
devices that have ``branch currents''that are part of the solution vector. This includes
the V, E, H and L devices.  It also includes the voltage-form of the B device.

Note that when using the variable list for time domain output, usage of
frequency domain functions like \texttt{VDB} can result in -Inf output being
written to the output file.  This is easily solved by defining the time
domain equivalent command to specify the correct output for time domain
data.

Further explanation of the current specifications is given in comments section below.

When a \texttt{.LIN} analysis is done then additional output variable formats are available
via the \texttt{.PRINT AC} line, where \texttt{<index1>} and \texttt{<index2>} must both
be greater than 0 and also both less than or equal to the number of ports in the netlist:
\begin{XyceItemize}
\item \texttt{SR(<index1>,<index2>)} to output the real component of an S-parameter
\item \texttt{SI(<index1>,<index2>)} to output the imaginary component of an S-parameter
\item \texttt{SM(<index1>,<index2>)} to output the magnitude of an S-parameter
\item \texttt{SP(<index1>,<index2>)} to output the phase of an S-parameter in degrees
\item \texttt{SDB(<index1>,<index2>)} to output the magnitude of an S-parameter in decibels.
\item \texttt{YR(<index1>,<index2>)} to output the real component of a Y-parameter
\item \texttt{YI(<index1>,<index2>)} to output the imaginary component of a Y-parameter
\item \texttt{YM(<index1>,<index2>)} to output the magnitude of a Y-parameter
\item \texttt{YP(<index1>,<index2>)} to output the phase of a Y-parameter in degrees
\item \texttt{YDB(<index1>,<index2>)} to output the magnitude of a Y-parameter in decibels.
\item \texttt{ZR(<index1>,<index2>)} to output the real component of a Z-parameter
\item \texttt{ZI(<index1>,<index2>)} to output the imaginary component of a Z-parameter
\item \texttt{ZM(<index1>,<index2>)} to output the magnitude of a Z-parameter
\item \texttt{ZP(<index1>,<index2>)} to output the phase of a Z-parameter in degrees
\item \texttt{ZDB(<index1>,<index2>)} to output the magnitude of a Z-parameter in decibels.
\end{XyceItemize}

When the analysis type is Noise, additional output variable formats are available
via the \texttt{.PRINT NOISE} line for devices that support stationary noise.
\begin{XyceItemize}
\item \texttt{INOISE} to output the input noise contributions
\item \texttt{ONOISE} to output the output noise contributions
\item \texttt{DNI(<deviceName>)} to output the input noise contribution from device <deviceName>
\item \texttt{DNI(<deviceName>,<noiseSource>)} to output the input noise contribution from
                source <noiseSource> for device <deviceName>
\item \texttt{DNO(<deviceName>)} to output the output noise contribution from device <deviceName>
\item \texttt{DNO(<deviceName>,<noiseSource>)} to output the output noise contribution from
                source <noiseSource> for device <deviceName>
\end{XyceItemize}

\end{Arguments}

\comments

\begin{XyceItemize}
\item Currents are positive flowing from node 1 to node 2 for two node
  devices, and currents are positive flowing into a particular lead for
  multi-terminal devices.

\item \texttt{<circuit node>} is simply the name of any node in your
  top-level circuit, or \texttt{<subcircuit name>:<node>} to reference
  nodes that are internal to a subcircuit.

\item \texttt{<device>} is the name of any device in your top-level
  circuit, or \texttt{<subcircuit name>:<device>} to reference devices
  that are internal to a subcircuit.

\item \texttt{<lead abbreviation>} is a single character designator for
  individual leads on a device with three or more leads.  For bipolar
  transistors these are: c (collector), b (base), e (emitter), and s
  (substrate).  For MOSFETs, JFETs, and MESFETs, lead abbreviations
  are: d (drain), g (gate), s (source), and for MOSFETS and JFETs,
  b (bulk).  In addition to these standard leads, SOI
  and CMG MOSFETs have e (bulk) nodes and SOI transistors have
  optional b (body) nodes whose lead currents may also be printed in
  this manner.  For PDE devices, the nodes are numbered according to
  the order they appear, so lead currents are referenced like
  I1(\texttt{<device>}), I2(\texttt{<device>}), etc.  In \Xyce{},
  a \texttt{.PRINT} line request like \texttt{I(Q1)} is a parsing
  error for a multi-terminal device.  Instead, an explicit lead
  current designator like \texttt{IC(Q1)} must be used.

\item The "lead current" method of printing from devices in Xyce is done
  at a low level with special code added to each device; the method is
  therefore only supported in specific devices that have this extra
  code.  So, if \texttt{.PRINT I(Y)} does not work, for a device called
  Y, then you will need to attach an ammeter (zero-volt voltage source)
  in series with that device and print the ammeter's current instead.

\item Lead currents of subcircuit ports are not supported.  However,
  access is provided via specific node names (e.g.,
  \texttt{X1:internalNodeName}) or specific devices (e.g.,
  \texttt{X1:V3}) inside the subcircuit.

\item For STD formatted output, the values of the output variables are
  output as a series of columns (one for each output variable).

\item When the command line option \texttt{-r <raw-file-name>} is used,
  all of the output is diverted to the \emph{raw-file-name} file as a
  concatenation of the plots, and each plot includes all of the
  variables of the circuit instead of the variable list(s) given on the
  \texttt{.PRINT} lines in the netlist.  Using the \texttt{-a} options in
  conjunction with the \texttt{-r} option results in a raw file that is
  output all in ASCII characters.

\item Any output going to the same file from one simulation of \Xyce{}
  results in the concatenation of output.  However, if a simulation is
  re-run then the original output will we over-written.

\item During analysis a number of output files may be generated.  The
  selection of which files are created depends on a variety of factors,
  most obvious of which is the \texttt{.PRINT} command. See
  section~\ref{Netlist_Commands} for more details.

\item Frequency domain values are output as complex values for Raw,
  TecPlot and Probe formats when a complex variable is printed.  For STD
  and CSV formats, the output appears in two columns, the real part
  followed by the imaginary part.  The print variables 
  \texttt{VR, VI, VM, VDB} and \texttt{VP} print the scalar values 
  for the real part, imaginary part,
  magnitude, magnitude in decibels, and phase, respectively.

\item When outputting a device or model parameter, it is usually
  necessary to specify both the device name and the parameter name,
  separated by a colon.  For example, the saturation current of a
  diode model \texttt{DMOD} would be requested
  as \texttt{DMOD:IS}. Section~\ref{Print_Device_Info} on ``Device
  Parameters and Internal Variables'' below gives more details and
  provides an example.

\item The \texttt{N()} syntax is used to access internal solution variables 
that are not normally visible from the netlist, such as voltages on
internal nodes and/or branch currents within a given device.  The
internal solution variables for each \Xyce{} device are not given in
the Reference Guide sections on those devices.  However, if the user
runs \texttt{Xyce -namesfile <filename> <netlist>} then \Xyce{} will
output into the first filename a list of all solution variables
generated by that netlist.  Section~\ref{Print_Device_Info} on
``Device Parameters and Internal Variables'' below gives more details
and provides an example.

\item The \texttt{DNI()} and \texttt{DNO()} syntax is used to print out the
  individual input and output noise contributions for each noise source within
  a device. The user can get a listing of the noise source names for each device in
  a netlist by running \texttt{Xyce -noise\_names\_file <filename> <netlist>}.
  The \Xyce{} Users' Guide\UsersGuide provides an example.

\item If multiple \texttt{.PRINT} lines are given for the same analysis type,
  the same output file name, and the same format, the variable lists
  of all matching \texttt{.PRINT} lines are merged together in the order
  found, and the resulting output is the same as if all the print line
  variable lists had been specified on a single \texttt{.PRINT} line.

\item Attempting to specify multiple \texttt{.PRINT} lines for the same
  analysis type to the same file with different specifications
  of \texttt{FORMAT} is an error.

\item \Xyce{} should emit a warning or error message, similar to 
  ``Could not open filename'' if: 1) the name of the output file is 
  actually a directory name; or 2) the output file is in a 
  subdirectory that does not already exist.  \Xyce{} 
  will not create new subdirectories.

\item The output filename specified with the -r command line option,
  to produce raw file output, should take precedence over
  a {\tt FILE=} parameter specified on a {\tt .PRINT} line.

\item The print statements for some analysis types could result in multiple
  output files.  For example, \texttt{.PRINT HB} will produce both
  frequency- and time-domain output, and place these in different files.
  The default name of these files is the name of the netlist followed by
  a data type suffix, followed by a format-specific extension.

  In \Xyce{}, if a \texttt{FILE} option is given to such a print
  statement, only the ``primary'' data for that analysis type is sent to
  the named file.  The secondary data is still sent to the default file
  name.  This behavior may be subject to change in future releases.

  For analysis types that can produce multiple files,
  special \texttt{.PRINT} lines have been provided to allow the user to
  control the handling of the additional files.  These additional print
  line specifiers are enumerated in the analysis-specific sections
  below.

  If one desires that all outputs for a given analysis type be given
  user-defined file names, it is necessary to use additional print lines
  with additional \texttt{FILE} options.  For example, if one uses
  a \texttt{FILE} option to a \texttt{.PRINT HB} line, only
  frequency-domain data will be sent to the named file.  To redirect the
  time-domain data to a file with a user-defined name, add
  a \texttt{.PRINT HB\_TD} line.  See the individual analysis types
  below for details of what additional print statements are available.
\end{XyceItemize}

\end{Command}

\subsubsection{Print AC Analysis}
\index{\texttt{.PRINT}!AC Analysis}
\index{\texttt{.PRINT}!\texttt{AC}}
% Sandia National Laboratories is a multimission laboratory managed and
% operated by National Technology & Engineering Solutions of Sandia, LLC, a
% wholly owned subsidiary of Honeywell International Inc., for the U.S.
% Department of Energy’s National Nuclear Security Administration under
% contract DE-NA0003525.

% Copyright 2002-2019 National Technology & Engineering Solutions of Sandia,
% LLC (NTESS).

AC Analysis generates two output files, the primary output is in the
frequency domain and the initial conditions output is in the time domain.

Note that when using the \texttt{.PRINT AC} to create the variable list
for DC type output, usage of frequency domain functions like \texttt{VDB} can result
in -Inf output being written to the output file.  This is easily solved
by defining a \texttt{.PRINT AC\_IC} command to specify the correct
output for initial condition data.

Homotopy output can also be generated.

{
\begin{PrintCommandTable}{Print AC Analysis Type}
.PRINT AC & \emph{circuit-file}.FD.prn & INDEX FREQ \\ \hline
.PRINT AC FORMAT=GNUPLOT & \emph{circuit-file}.FD.prn & INDEX FREQ \\ \hline
.PRINT AC FORMAT=NOINDEX & \emph{circuit-file}.FD.prn & FREQ \\ \hline
.PRINT AC FORMAT=CSV & \emph{circuit-file}.FD.csv & FREQ \\ \hline
.PRINT AC FORMAT=RAW & \emph{circuit-file}.raw & FREQ \\ \hline
Xyce -a \newline .PRINT AC FORMAT=RAW & \emph{circuit-file}.raw & FREQ \\ \hline
.PRINT AC FORMAT=TECPLOT & \emph{circuit-file}.FD.dat & FREQ \\ \hline
.PRINT AC FORMAT=PROBE & \emph{circuit-file}.csd & -- \\ \hline


\multicolumn{3}{c}{\smallskip\color{XyceDarkBlue}\em\bfseries Add \texttt{.OP} To Netlist To Enable AC\_IC Output} \\ \hline
.PRINT AC\_IC & \emph{circuit-file}.TD.prn & INDEX TIME \\ \hline
.PRINT AC\_IC FORMAT=GNUPLOT & \emph{circuit-file}.TD.prn & INDEX TIME \\ \hline
.PRINT AC\_IC FORMAT=CSV & \emph{circuit-file}.TD.csv & TIME \\ \hline
.PRINT AC\_IC FORMAT=RAW & \emph{circuit-file}.raw & TIME \\ \hline
Xyce -a \newline .PRINT AC\_IC FORMAT=RAW & \emph{circuit-file}.raw & TIME \\ \hline
.PRINT AC\_IC FORMAT=TECPLOT & \emph{circuit-file}.TD.dat & TIME \\ \hline
.PRINT AC\_IC FORMAT=PROBE & \emph{circuit-file}.TD.csd & -- \\ \hline

\multicolumn{3}{c}{\smallskip\color{XyceDarkBlue}\em\bfseries Command Line Raw Override Output} \\ \hline
Xyce -r & \emph{circuit-file}.raw & All circuit variables printed \\ \hline
Xyce -r -a & \emph{circuit-file}.raw & All circuit variables printed \\ \hline

\multicolumn{3}{c}{\smallskip\color{XyceDarkBlue}\em\bfseries Additional Output Available} \\ \hline
.OP & \emph{log file} & Operating point data \\ \hline
.SENS \newline .PRINT SENS & \multicolumn{2}{c}{see~\nameref{Print_Sensitivity}} \\ \hline
.OPTIONS NONLIN CONTINUATION=<method> \newline .PRINT HOMOTOPY & \multicolumn{2}{c}{see~\nameref{Print_Homotopy}} \\ \hline
\end{PrintCommandTable}
}



\subsubsection{Print DC Analysis}
\index{\texttt{.PRINT}!DC Analysis}
\index{\texttt{.PRINT}!\texttt{DC}}
% Sandia National Laboratories is a multimission laboratory managed and
% operated by National Technology & Engineering Solutions of Sandia, LLC, a
% wholly owned subsidiary of Honeywell International Inc., for the U.S.
% Department of Energy’s National Nuclear Security Administration under
% contract DE-NA0003525.

% Copyright 2002-2020 National Technology & Engineering Solutions of Sandia,
% LLC (NTESS).

DC Analysis generates output based on the format specified by the \texttt{.PRINT} command.

Homotopy and sensitivity output can also be generated.

{
\begin{PrintCommandTable}{Print DC Analysis Type}
.PRINT DC & \emph{circuit-file}.prn & INDEX \\ \hline
.PRINT DC FORMAT=GNUPLOT & \emph{circuit-file}.prn & INDEX \\ \hline
.PRINT DC FORMAT=SPLOT & \emph{circuit-file}.prn & INDEX \\ \hline
.PRINT DC FORMAT=NOINDEX & \emph{circuit-file}.prn & -- \\ \hline
.PRINT DC FORMAT=CSV & \emph{circuit-file}.csv & -- \\ \hline
.PRINT DC FORMAT=RAW & \emph{circuit-file}.raw & -- \\ \hline
Xyce -a \newline .PRINT DC FORMAT=RAW & \emph{circuit-file}.raw & -- \\ \hline
.PRINT DC FORMAT=TECPLOT & \emph{circuit-file}.dat & -- \\ \hline
.PRINT DC FORMAT=PROBE & \emph{circuit-file}.csd & -- \\ \hline
\category{Command Line Raw Override Output} \\
Xyce -r & \emph{circuit-file}.raw & All circuit variables \\ \hline
Xyce -r -a & \emph{circuit-file}.raw & All circuit variables \\ \hline
\category{Additional Output Available} \\
.OP & \emph{log file} & Operating point data \\ \hline
.SENS \newline .PRINT SENS & \multicolumn{2}{c}{see~\nameref{Print_Sensitivity}} \\ \hline
.OPTIONS NONLIN CONTINUATION=<method> \newline .PRINT HOMOTOPY & \multicolumn{2}{c}{see~\nameref{Print_Homotopy}} \\ \hline
\end{PrintCommandTable}
}


\subsubsection{Print Harmonic Balance Analysis}
\index{\texttt{.PRINT}!Harmonic Balance Analysis}
\index{\texttt{.PRINT}!\texttt{HB}}
% Sandia National Laboratories is a multimission laboratory managed and
% operated by National Technology & Engineering Solutions of Sandia, LLC, a
% wholly owned subsidiary of Honeywell International Inc., for the U.S.
% Department of Energy’s National Nuclear Security Administration under
% contract DE-NA0003525.

% Copyright 2002-2022 National Technology & Engineering Solutions of Sandia,
% LLC (NTESS).

HB Analysis generates one output file in the frequency domain and one in the
time domain based on the format specified by the \texttt{.PRINT}
command.  Additional startup and initial conditions output can be
generated based on \texttt{.OPTIONS} commands.

Note that when using the \texttt{.PRINT HB} to create the variable list
for time domain output, usage of frequency domain functions like \texttt{VDB} can
result in -Inf output being written to the output file.  This is easily
solved by defining a \texttt{.PRINT HB\_TD}, \texttt{.PRINT HB\_IC} and
\texttt{.PRINT HB\_STARTUP} commands to specify the correct output for
the time domain data.

If \texttt{.STEP} is used with HB then the Initial Condition (IC) data will initially
be output to a ``tmp file'' (e.g., \texttt{<netlist-name>.hb\_ic.prn.tmp}).
If that IC data meets the required tolerance then it will be copied to the end
of the \texttt{<netlist-name>.hb\_ic.prn} file, and the tmp file will be deleted.

Homotopy output can also be generated.

{
\begin{PrintCommandTable}{Print HB Analysis Type}
.PRINT HB & \emph{circuit-file}.HB.TD.prn \newline \emph{circuit-file}.HB.FD.prn  \newline \emph{circuit-file}.hb\_ic.prn & INDEX TIME \newline INDEX FREQ \newline INDEX TIME \\ \hline
.PRINT HB FORMAT=GNUPLOT & \emph{circuit-file}.HB.TD.prn \newline \emph{circuit-file}.HB.FD.prn  \newline \emph{circuit-file}.hb\_ic.prn & INDEX TIME \newline INDEX FREQ \newline INDEX TIME \\ \hline
.PRINT HB FORMAT=SPLOT & \emph{circuit-file}.HB.TD.prn \newline \emph{circuit-file}.HB.FD.prn  \newline \emph{circuit-file}.hb\_ic.prn & INDEX TIME \newline INDEX FREQ \newline INDEX TIME \\ \hline
.PRINT HB FORMAT=NOINDEX & \emph{circuit-file}.HB.TD.prn \newline \emph{circuit-file}.HB.FD.prn  \newline \emph{circuit-file}.hb\_ic.prn & TIME \newline FREQ \newline TIME \\ \hline
.PRINT HB FORMAT=CSV & \emph{circuit-file}.HB.TD.csv \newline \emph{circuit-file}.HB.FD.csv  \newline \emph{circuit-file}.hb\_ic.csv &  TIME \newline FREQ \newline TIME \\ \hline
.PRINT HB FORMAT=TECPLOT & \emph{circuit-file}.HB.TD.dat \newline \emph{circuit-file}.HB.FD.dat  \newline \emph{circuit-file}.hb\_ic.dat &  TIME \newline FREQ \newline TIME \\ \hline

.PRINT HB\_FD & \emph{circuit-file}.HB.FD.prn & INDEX FREQ \\ \hline
.PRINT HB\_FD FORMAT=GNUPLOT& \emph{circuit-file}.HB.FD.prn & INDEX FREQ \\ \hline
.PRINT HB\_FD FORMAT=SPLOT& \emph{circuit-file}.HB.FD.prn & INDEX FREQ \\ \hline
.PRINT HB\_FD FORMAT=NOINDEX & \emph{circuit-file}.HB.FD.prn & FREQ \\ \hline
.PRINT HB\_FD FORMAT=CSV & \emph{circuit-file}.HB.FD.csv &  FREQ \\ \hline
.PRINT HB\_FD FORMAT=TECPLOT & \emph{circuit-file}.HB.FD.dat & FREQ \\ \hline

.PRINT HB\_TD & \emph{circuit-file}.HB.TD.prn & INDEX TIME \\ \hline
.PRINT HB\_TD FORMAT=GNUPLOT & \emph{circuit-file}.HB.TD.prn & INDEX TIME \\ \hline
.PRINT HB\_TD FORMAT=SPLOT & \emph{circuit-file}.HB.TD.prn & INDEX TIME \\ \hline
.PRINT HB\_TD FORMAT=NOINDEX & \emph{circuit-file}.HB.TD.prn & TIME \\ \hline
.PRINT HB\_TD FORMAT=CSV & \emph{circuit-file}.HB.TD.csv & TIME \\ \hline
.PRINT HB\_TD FORMAT=TECPLOT & \emph{circuit-file}.HB.TD.dat & TIME \\ \hline

\multicolumn{3}{c}{\smallskip\color{XyceDarkBlue}\em\bfseries  Startup Period} \\ \hline
.OPTIONS HBINT STARTUPPERIODS=<n> \newline .PRINT HB\_STARTUP & \emph{circuit-file}.startup.prn & INDEX TIME \\ \hline
.OPTIONS HBINT STARTUPPERIODS=<n> \newline .PRINT HB\_STARTUP FORMAT=GNUPLOT & \emph{circuit-file}.startup.prn & INDEX TIME \\ \hline
.OPTIONS HBINT STARTUPPERIODS=<n> \newline .PRINT HB\_STARTUP FORMAT=SPLOT & \emph{circuit-file}.startup.prn & INDEX TIME \\ \hline
.OPTIONS HBINT STARTUPPERIODS=<n> \newline .PRINT HB\_STARTUP FORMAT=NOINDEX & \emph{circuit-file}.startup.prn & TIME \\ \hline
.OPTIONS HBINT STARTUPPERIODS=<n> \newline .PRINT HB\_STARTUP FORMAT=CSV \newline .OPTIONS HBINT STARTUPPERIODS=<n> & \emph{circuit-file}.startup.csv & TIME \\ \hline
.OPTIONS HBINT STARTUPPERIODS=<n> \newline .PRINT HB\_STARTUP FORMAT=TECPLOT \newline .OPTIONS HBINT STARTUPPERIODS=<n> & \emph{circuit-file}.startup.dat & TIME \\ \hline

\multicolumn{3}{c}{\smallskip\color{XyceDarkBlue}\em\bfseries  Initial Conditions} \\ \hline
.OPTIONS HBINT SAVEICDATA=1 \newline .PRINT HB\_IC & \emph{circuit-file}.hb\_ic.prn  & INDEX TIME \\ \hline
.OPTIONS HBINT SAVEICDATA=1 \newline .PRINT HB\_IC FORMAT=GNUPLOT & \emph{circuit-file}.hb\_ic.prn  & INDEX TIME \\ \hline
.OPTIONS HBINT SAVEICDATA=1 \newline .PRINT HB\_IC FORMAT=SPLOT & \emph{circuit-file}.hb\_ic.prn  & INDEX TIME \\ \hline
.OPTIONS HBINT SAVEICDATA=1 \newline .PRINT HB\_IC FORMAT=NOINDEX & \emph{circuit-file}.hb\_ic.prn  & TIME \\ \hline
.OPTIONS HBINT SAVEICDATA=1 \newline .PRINT HB\_IC FORMAT=CSV  & \emph{circuit-file}.hb\_ic.csv & TIME \\ \hline
.OPTIONS HBINT SAVEICDATA=1 \newline .PRINT HB\_IC FORMAT=TECPLOT & \emph{circuit-file}.hb\_ic.dat & TIME \\ \hline

\multicolumn{3}{c}{\smallskip\color{XyceDarkBlue}\em\bfseries Additional Output Available} \\ \hline
.OP & \emph{log file} & Operating point data \\ \hline
.SENS \newline .PRINT SENS & \multicolumn{2}{c}{see~\nameref{Print_Sensitivity}} \\ \hline
.OPTIONS NONLIN CONTINUATION=<method> \newline .PRINT HOMOTOPY & \multicolumn{2}{c}{see~\nameref{Print_Homotopy}} \\ \hline
\end{PrintCommandTable}
}


\subsubsection{Print Noise Analysis}
\index{\texttt{.PRINT}!Noise Analysis}
\index{\texttt{.PRINT}!\texttt{NOISE}}
% Sandia National Laboratories is a multimission laboratory managed and
% operated by National Technology & Engineering Solutions of Sandia, LLC, a
% wholly owned subsidiary of Honeywell International Inc., for the U.S.
% Department of Energy’s National Nuclear Security Administration under
% contract DE-NA0003525.

% Copyright 2002-2021 National Technology & Engineering Solutions of Sandia,
% LLC (NTESS).

NOISE Analysis generates two output files, the primary output is in the
frequency domain and the initial conditions output is in the time domain.

{
\begin{PrintCommandTable}{Print NOISE Analysis Type}
.PRINT NOISE & \emph{circuit-file}.NOISE.prn & INDEX FREQ \\ \hline
.PRINT NOISE FORMAT=GNUPLOT & \emph{circuit-file}.NOISE.prn & INDEX FREQ \\ \hline
.PRINT NOISE FORMAT=SPLOT & \emph{circuit-file}.NOISE.prn & INDEX FREQ \\ \hline
.PRINT NOISE FORMAT=NOINDEX & \emph{circuit-file}.NOISE.prn & FREQ \\ \hline
.PRINT NOISE FORMAT=CSV & \emph{circuit-file}.NOISE.csv & FREQ \\ \hline
.PRINT NOISE FORMAT=TECPLOT & \emph{circuit-file}.NOISE.dat & FREQ \\ \hline
%.PRINT NOISE FORMAT=PROBE & \emph{circuit-file}.FD.csd & -- \\ \hline

\multicolumn{3}{c}{\smallskip\color{XyceDarkBlue}\em\bfseries Additional Output Available} \\ \hline
.OP & \emph{log file} & Operating point data \\ \hline
.OPTIONS NONLIN CONTINUATION=<method> \newline .PRINT HOMOTOPY & \multicolumn{2}{c}{see~\nameref{Print_Homotopy}} \\ \hline
\end{PrintCommandTable}
}



\subsubsection{Print Transient Analysis}
\index{\texttt{.PRINT}!Transient Analysis}
\index{\texttt{.PRINT}!\texttt{TRAN}}
% Sandia National Laboratories is a multimission laboratory managed and
% operated by National Technology & Engineering Solutions of Sandia, LLC, a
% wholly owned subsidiary of Honeywell International Inc., for the U.S.
% Department of Energy’s National Nuclear Security Administration under
% contract DE-NA0003525.

% Copyright 2002-2021 National Technology & Engineering Solutions of Sandia,
% LLC (NTESS).


Transient Analysis generates time domain output based on the format specified by the \texttt{.PRINT} command.

Homotopy and sensitivty output can also be generated.
{
\begin{PrintCommandTable}{Print Transient Analysis Type}
.PRINT TRAN & \emph{circuit-file}.prn & INDEX TIME \\ \hline
.PRINT TRAN FORMAT=GNUPLOT & \emph{circuit-file}.prn & INDEX TIME \\ \hline
.PRINT TRAN FORMAT=SPLOT & \emph{circuit-file}.prn & INDEX TIME \\ \hline
.PRINT TRAN FORMAT=NOINDEX & \emph{circuit-file}.prn & TIME \\ \hline
.PRINT TRAN FORMAT=CSV & \emph{circuit-file}.csv & TIME \\ \hline
.PRINT TRAN FORMAT=RAW & \emph{circuit-file}.raw & TIME \\ \hline
Xyce -a \newline .PRINT TRAN FORMAT=RAW & \emph{circuit-file}.raw & TIME \\ \hline
.PRINT TRAN FORMAT=TECPLOT & \emph{circuit-file}.dat & TIME \\ \hline
.PRINT TRAN FORMAT=PROBE & \emph{circuit-file}.csd & -- \\ \hline
\multicolumn{3}{c}{\smallskip\color{XyceDarkBlue}\em\bfseries Command Line Raw Override Output} \\ \hline
Xyce -r raw-file-name & \emph{raw-file-name} & All circuit variables printed \\ \hline
Xyce -r raw-file-name -a & \emph{raw-file-name} & All circuit variables printed \\ \hline
\multicolumn{3}{c}{\smallskip\color{XyceDarkBlue}\em\bfseries Additional Output Available} \\ \hline
.OP & \emph{log file} & Operating point data \\ \hline
.SENS \newline .PRINT SENS & \multicolumn{2}{c}{see~\nameref{Print_Sensitivity}} \\ \hline
.OPTIONS NONLIN CONTINUATION=<method> \newline .PRINT HOMOTOPY & \multicolumn{2}{c}{see~\nameref{Print_Homotopy}} \\ \hline
\end{PrintCommandTable}
}


\subsubsection{Print Homotopy}
\label{Print_Homotopy}
\index{\texttt{.PRINT}!Homotopy Analysis}
% Sandia National Laboratories is a multimission laboratory managed and
% operated by National Technology & Engineering Solutions of Sandia, LLC, a
% wholly owned subsidiary of Honeywell International Inc., for the U.S.
% Department of Energy’s National Nuclear Security Administration under
% contract DE-NA0003525.

% Copyright 2002-2019 National Technology & Engineering Solutions of Sandia,
% LLC (NTESS).

Homotopy output is generated by the inclusion of the \newline
\texttt{.OPTIONS NONLIN CONTINUATION=<method>} command.

{
\begin{PrintCommandTable}{Print Homotopy}
.OPTIONS NONLIN CONTINUATION=<method> \newline .PRINT HOMOTOPY & 
\emph{circuit-file}.HOMOTOPY.prn & INDEX TIME \\ \hline
.OPTIONS NONLIN CONTINUATION=<method> \newline .PRINT HOMOTOPY FORMAT=GNUPLOT & 
\emph{circuit-file}.HOMOTOPY.prn & INDEX TIME \\ \hline
.OPTIONS NONLIN CONTINUATION=<method> \newline .PRINT HOMOTOPY FORMAT=SPLOT &
\emph{circuit-file}.HOMOTOPY.prn & INDEX TIME \\ \hline
.OPTIONS NONLIN CONTINUATION=<method> \newline .PRINT HOMOTOPY FORMAT=NOINDEX &
\emph{circuit-file}.HOMOTOPY.prn & TIME \\ \hline
.OPTIONS NONLIN CONTINUATION=<method> \newline .PRINT HOMOTOPY FORMAT=CSV & 
\emph{circuit-file}.HOMOTOPY.csv & TIME \\ \hline
.OPTIONS NONLIN CONTINUATION=<method> \newline .PRINT HOMOTOPY FORMAT=TECPLOT & 
\emph{circuit-file}.HOMOTOPY.dat & TIME \\ \hline
\end{PrintCommandTable}
}


\subsubsection{Print Sensitivity}
\label{Print_Sensitivity}
\index{\texttt{.PRINT}!Homotopy Analysis}
% Sandia National Laboratories is a multimission laboratory managed and
% operated by National Technology & Engineering Solutions of Sandia, LLC, a
% wholly owned subsidiary of Honeywell International Inc., for the U.S.
% Department of Energy’s National Nuclear Security Administration under
% contract DE-NA0003525.

% Copyright 2002-2019 National Technology & Engineering Solutions of Sandia,
% LLC (NTESS).

Sensitivity is enabled by inclusion of the \newline .SENS command.

Steady-state sensitivities (adjoint or direct) and transient direct sensitivities
will be handled by the .PRINT SENS command. Transient adjoint, on the other hand,
is handled by the .PRINT TRANADJOINT command.

For transient sensitivity output, a \texttt{TIME} column will be included for the
STD, GNUPLOT, SPLOT, NOINDEX and CSV formats.  For AC sensitivity output, a \texttt{FREQ}
column will be included for the STD, GNUPLOT, SPLOT, NOINDEX and CSV formats.

{
\begin{PrintCommandTable}{Print Sensitivities for .TRAN and .DC}
.SENS objfunc={<$obj$>} p=[$p_1$1][,$p_n$]* \newline
.PRINT SENS & \emph{circuit-file}.SENS.prn & $obj$ d{$obj$}/d($p_1$) d{$obj$}/d($p_n$) \newline \\ \hline
.SENS objfunc={<$obj$>} p=[$p_1$1][,$p_n$]* \newline
.PRINT SENS FORMAT=GNUPLOT & \emph{circuit-file}.SENS.prn & $obj$ d{$obj$}/d($p_1$) d{$obj$}/d($p_n$) \newline \\ \hline
.SENS objfunc={<$obj$>} p=[$p_1$1][,$p_n$]* \newline
.PRINT SENS FORMAT=SPLOT & \emph{circuit-file}.SENS.prn & $obj$ d{$obj$}/d($p_1$) d{$obj$}/d($p_n$) \newline \\ \hline
.SENS objfunc={<$obj$>} p=[$p_1$1][,$p_n$]* \newline
.PRINT SENS FORMAT=NOINDEX & \emph{circuit-file}.SENS.prn & $obj$ d{$obj$}/d($p_1$) d{$obj$}/d($p_n$) \newline \\ \hline
.SENS objfunc={<$obj$>} p=[$p_1$1][,$p_n$]* \newline
.PRINT SENS FORMAT=CSV & \emph{circuit-file}.SENS.csv & $obj$ d{$obj$}/d($p_1$) d{$obj$}/d($p_n$) \newline \\ \hline
.SENS objfunc={<$obj$>} p=[$p_1$][,$p_n$]* \newline
.PRINT SENS FORMAT=TECPLOT & \emph{circuit-file}.SENS.dat & $obj$ d{$obj$}/d($p_1$) d{$obj$}/d($p_n$) \newline \\ \hline

\end{PrintCommandTable}
}

{
\begin{PrintCommandTable}{Print Sensitivities for .AC}
.SENS objfunc={<$obj$>} p=[$p_1$1][,$p_n$]* \newline
.PRINT SENS & \emph{circuit-file}.FD.SENS.prn & $obj$ d{$obj$}/d($p_1$) d{$obj$}/d($p_n$) \newline \\ \hline
.SENS objfunc={<$obj$>} p=[$p_1$1][,$p_n$]* \newline
.PRINT SENS FORMAT=GNUPLOT & \emph{circuit-file}.FD.SENS.prn & $obj$ d{$obj$}/d($p_1$) d{$obj$}/d($p_n$) \newline \\ \hline
.SENS objfunc={<$obj$>} p=[$p_1$1][,$p_n$]* \newline
.PRINT SENS FORMAT=SPLOT & \emph{circuit-file}.FD.SENS.prn & $obj$ d{$obj$}/d($p_1$) d{$obj$}/d($p_n$) \newline \\ \hline
.SENS objfunc={<$obj$>} p=[$p_1$1][,$p_n$]* \newline
.PRINT SENS FORMAT=NOINDEX & \emph{circuit-file}.FD.SENS.prn & $obj$ d{$obj$}/d($p_1$) d{$obj$}/d($p_n$) \newline \\ \hline
.SENS objfunc={<$obj$>} p=[$p_1$1][,$p_n$]* \newline
.PRINT SENS FORMAT=CSV & \emph{circuit-file}.FD.SENS.csv & $obj$ d{$obj$}/d($p_1$) d{$obj$}/d($p_n$) \newline \\ \hline
.SENS objfunc={<$obj$>} p=[$p_1$][,$p_n$]* \newline
.PRINT SENS FORMAT=TECPLOT & \emph{circuit-file}.FD.SENS.dat & $obj$ d{$obj$}/d($p_1$) d{$obj$}/d($p_n$) \newline \\ \hline

\end{PrintCommandTable}
}

{
\begin{PrintCommandTable}{Print Transient Adjoint Sensitivities}
.SENS objfunc={<$obj$>} p=[$p_1$1][,$p_n$]* \newline
.PRINT TRANADJOINT & \emph{circuit-file}.TRADJ.prn & $obj$ d{$obj$}/d($p_1$) d{$obj$}/d($p_n$) \newline \\ \hline
.SENS objfunc={<$obj$>} p=[$p_1$1][,$p_n$]* \newline
.PRINT TRANADJOINT FORMAT=NOINDEX & \emph{circuit-file}.TRADJ.prn & $obj$ d{$obj$}/d($p_1$) d{$obj$}/d($p_n$) \newline \\ \hline
.SENS objfunc={<$obj$>} p=[$p_1$1][,$p_n$]* \newline
.PRINT TRANADJOINT FORMAT=CSV & \emph{circuit-file}.TRADJ.csv & $obj$ d{$obj$}/d($p_1$) d{$obj$}/d($p_n$) \newline \\ \hline
.SENS objfunc={<$obj$>} p=[$p_1$][,$p_n$]* \newline
.PRINT TRANADJOINT FORMAT=TECPLOT & \emph{circuit-file}.TRADJ.dat & $obj$ d{$obj$}/d($p_1$) d{$obj$}/d($p_n$) \newline \\ \hline

\end{PrintCommandTable}
}


\subsubsection{Print Embedded Sampling Analysis}
\index{\texttt{.PRINT}!EMBEDDEDSAMPLING Analysis}
\index{\texttt{.PRINT}!\texttt{ES}}
% Sandia National Laboratories is a multimission laboratory managed and
% operated by National Technology & Engineering Solutions of Sandia, LLC, a
% wholly owned subsidiary of Honeywell International Inc., for the U.S.
% Department of Energy’s National Nuclear Security Administration under
% contract DE-NA0003525.

% Copyright 2002-2021 National Technology & Engineering Solutions of Sandia,
% LLC (NTESS).

EMBEDDEDSAMPLING Analysis generates one output file, with the information for
each output variable grouped in a set of contiguous columns, based on the format
specified by the \texttt{.PRINT} command.  The arguments \texttt{OUTPUT\_SAMPLE\_STATS}
and \texttt{OUTPUT\_ALL\_SAMPLES} are specific to \texttt{.PRINT ES} lines. Section
\ref{EMBEDDEDSAMPLING_section} has more details on their usage.

For a transient analysis, a TIME column will also be included for the STD,
GNUPLOT, SPLOT, NOINDEX and CSV formats.  The TIME variable will also be
included in the TECPLOT format for that case.

{
\begin{PrintCommandTable}{Print EMBEDDEDSAMPLING Analysis Type}
.PRINT ES & \emph{circuit-file}.ES.prn & INDEX \\ \hline
.PRINT ES FORMAT=GNUPLOT & \emph{circuit-file}.ES.prn & INDEX \\ \hline
.PRINT ES FORMAT=SPLOT & \emph{circuit-file}.ES.prn & INDEX \\ \hline
.PRINT ES FORMAT=NOINDEX & \emph{circuit-file}.ES.prn & -- \\ \hline
.PRINT ES FORMAT=CSV & \emph{circuit-file}.ES.csv & -- \\ \hline
.PRINT ES FORMAT=TECPLOT & \emph{circuit-file}.ES.dat & -- \\ \hline
\end{PrintCommandTable}
}



\subsubsection{Print Intrusive PCE Analysis}
\index{\texttt{.PRINT}!PCE Analysis}
\index{\texttt{.PRINT}!\texttt{PCE}}
% Sandia National Laboratories is a multimission laboratory managed and
% operated by National Technology & Engineering Solutions of Sandia, LLC, a
% wholly owned subsidiary of Honeywell International Inc., for the U.S.
% Department of Energy’s National Nuclear Security Administration under
% contract DE-NA0003525.

% Copyright 2002-2021 National Technology & Engineering Solutions of Sandia,
% LLC (NTESS).

PCE Analysis generates one output file, with the information for
each output variable grouped in a set of contiguous columns, based on the format
specified by the \texttt{.PRINT} command.  The arguments \texttt{OUTPUT\_SAMPLE\_STATS}
and \texttt{OUTPUT\_ALL\_SAMPLES} are specific to \texttt{.PRINT PCE} lines. Section
\ref{PCE_section} has more details on their usage.

For a transient analysis, a TIME column will also be included for the STD,
GNUPLOT, SPLOT, NOINDEX and CSV formats.  The TIME variable will also be
included in the TECPLOT format for that case.

{
\begin{PrintCommandTable}{Print PCE Analysis Type}
.PRINT PCE & \emph{circuit-file}.PCE.prn & INDEX \\ \hline
.PRINT PCE FORMAT=GNUPLOT & \emph{circuit-file}.PCE.prn & INDEX \\ \hline
.PRINT PCE FORMAT=SPLOT & \emph{circuit-file}.PCE.prn & INDEX \\ \hline
.PRINT PCE FORMAT=NOINDEX & \emph{circuit-file}.PCE.prn & -- \\ \hline
.PRINT PCE FORMAT=CSV & \emph{circuit-file}.PCE.csv & -- \\ \hline
.PRINT PCE FORMAT=TECPLOT & \emph{circuit-file}.PCE.dat & -- \\ \hline
\end{PrintCommandTable}
}



\subsubsection{Parameter Stepping}
\label{Print_Step}
\index{\texttt{.PRINT}!Parameter Stepping}
% Sandia National Laboratories is a multimission laboratory managed and
% operated by National Technology & Engineering Solutions of Sandia, LLC, a
% wholly owned subsidiary of Honeywell International Inc., for the U.S.
% Department of Energy’s National Nuclear Security Administration under
% contract DE-NA0003525.

% Copyright 2002-2021 National Technology & Engineering Solutions of Sandia,
% LLC (NTESS).

During parameter stepping, enabled with the \texttt{.STEP} command, the
output generated by each of analysis types varies.  Generally the FORMAT
indicates this variation, however some combinations of analysis and
format can result in additional variation.

The following table lists how the output differs for each analysis type
and format.

{
\renewcommand{\arraystretch}{1.2}
\begin{longtable}{>{\ttfamily\small}m{1in}<{\normalfont\small}>{\ttfamily\small}m{1.5in}<{\normalfont\footnotesize}m{2.25in}@{}}
  \rowcolor{XyceDarkBlue}
  \color{white}\normalfont\bf Print Type &
  \color{white}\bf Format &
  \color{white}\bf Description \endfirsthead
  \rowcolor{XyceDarkBlue}
  \color{white}\normalfont\bf Print Type &
  \color{white}\bf Format &
  \color{white}\bf Description \endhead
AC & STD & 1, 3, 4, 11, 12, 13 \\ \hline
AC & GNUPLOT & 1, 3, 4, 11, 12, 13, 20 \\ \hline
AC & SPLOT & 1, 3, 4, 11, 12, 13, 21 \\ \hline
AC & CSV & 4, 11 \\ \hline
AC & PROBE & 16 \\ \hline
AC & TECPLOT & 4, 12, 13, 18 \\ \hline
AC & RAW & 19 \\ \hline
AC & RAW (Xyce -a) & 19 \\ \hline
AC\_IC & STD & 1, 4, 11, 12, 13 \\ \hline
AC\_IC & GNUPLOT & 1, 4, 11, 12, 13, 20 \\ \hline
AC\_IC & SPLOT & 1, 4, 11, 12, 13, 21 \\ \hline
AC\_IC & CSV & 4, 11 \\ \hline
AC\_IC & PROBE & 16 \\ \hline
AC\_IC & TECPLOT & 12, 13, 18 \\ \hline
AC\_IC & RAW & 19 \\ \hline
AC\_IC & RAW (Xyce -a) & 19 \\ \hline
DC & STD & 1, 11, 12 \\ \hline
DC & GNUPLOT & 1, 11, 12, 20 \\ \hline
DC & SPLOT & 1, 11, 12, 21 \\ \hline
DC & CSV & 11 \\ \hline
DC & PROBE & 17 \\ \hline
DC & TECPLOT & 4, 12, 13, 18 \\ \hline
DC & RAW & 19 \\ \hline
DC & RAW (Xyce -a) & 19 \\ \hline
HB\_TD & STD & 1, 2, 4, 11, 12, 13 \\ \hline
HB\_TD & GNUPLOT & 1, 2, 4, 11, 12, 13, 20 \\ \hline
HB\_TD & SPLOT & 1, 2, 4, 11, 12, 13, 21 \\ \hline
HB\_TD & CSV & 11 \\ \hline
HB\_TD & TECPLOT & 12, 13, 18 \\ \hline
HB\_FD & STD & 1, 3, 4, 11, 12, 13 \\ \hline
HB\_FD & GNUPLOT & 1, 3, 4, 11, 12, 13, 20 \\ \hline
HB\_FD & SPLOT & 1, 3, 4, 11, 12, 13, 21 \\ \hline
HB\_FD & CSV & 4, 11 \\ \hline
HB\_FD & TECPLOT & 4, 12, 13, 18 \\ \hline
HB\_IC & STD & 1, 2, 4, 11, 12, 13 \\ \hline
HB\_IC & GNUPLOT & 1, 2, 4, 11, 12, 13, 20 \\ \hline
HB\_IC & SPLOT & 1, 2, 4, 11, 12, 13, 21 \\ \hline
HB\_IC & CSV & 11 \\ \hline
HB\_IC & TECPLOT & 12, 13, 18 \\ \hline
HB\_STARTUP & STD & 1, 2, 4, 11, 12, 13 \\ \hline
HB\_STARTUP & GNUPLOT & 1, 2, 4, 11, 12, 13, 20 \\ \hline
HB\_STARTUP & SPLOT & 1, 2, 4, 11, 12, 13, 21 \\ \hline
HB\_STARTUP & CSV & 11 \\ \hline
HB\_STARTUP & TECPLOT & 12, 13, 18 \\ \hline
TRAN & STD & 1, 2, 11, 12 \\ \hline
TRAN & GNUPLOT & 1, 2, 11, 12, 20 \\ \hline
TRAN & SPLOT & 1, 2, 11, 12, 21 \\ \hline
TRAN & CSV & 2, 11 \\ \hline
TRAN & PROBE & 17 \\ \hline
TRAN & TECPLOT & 2, 4, 12, 13, 18 \\ \hline
TRAN & RAW & 2, 19 \\ \hline
TRAN & RAW (Xyce -a) & 2, 19 \\ \hline
\multicolumn{3}{c}{\smallskip\color{XyceDarkBlue}\em\bfseries Specialized Output Commands} \\
HOMOTOPY & STD & 1, 2, 4, 11, 15 \\ \hline
HOMOTOPY & GNUPLOT & 1, 2, 4, 11, 15, 20 \\ \hline
HOMOTOPY & SPLOT & 1, 2, 4, 11, 15, 21 \\ \hline
HOMOTOPY & CSV & 2, 11 \\ \hline
HOMOTOPY & PROBE & 17 \\ \hline
HOMOTOPY & TECPLOT & 2, 4, 15, 18 \\ \hline
SENSITIVITY & STD & 1, 2, 11, 14 \\ \hline
SENSITIVITY & GNUPLOT & 1, 2, 11, 14, 20 \\ \hline
SENSITIVITY & SPLOT & 1, 2, 11, 14, 21 \\ \hline
SENSITIVITY & CSV & 2, 11 \\ \hline
SENSITIVITY & TECPLOT & 2, 14, 18 \\ \hline
\end{longtable}
}

{
\renewcommand{\arraystretch}{1.2}
\begin{longtable}{>{\ttfamily\small}m{1in}<{\normalfont\small}>{\ttfamily\small}m{3.75in}@{}}
  \rowcolor{XyceDarkBlue}
  \color{white}\normalfont\bf Description & 
  \endfirsthead
  \rowcolor{XyceDarkBlue}
  \color{white}\normalfont\bf Description & 
  \endhead
1 & INDEX column added to output variable list \\
2 & TIME column added to output variable list \\
3 & FREQ column added to output variable list \\
4 & Frequency domain data written as Re($var$) and Im($var$) \\
11 & INDEX resets to zero at start of each .STEP \\
12 & Prints 'End of Xyce(TM) Parameter Sweep' at end of .STEP simulation \\
13 & Prints 'End of Xyce(TM) Simulation' at end of non-.STEP simulation \\
14 & Prints 'End of Xyce(TM) Sensitivity Simulation' at end of simulation \\
15 & Prints 'End of Xyce(TM) Homotopy Simulation' at end of simulation \\
16 & Two '\#;' at the end of each .STEP (BUG) \\
17 & One '\#;' at end of each .STEP \\
18 & New ZONE for each .STEP, and AUXDATA for each .STEP parameter \\
19 & Prints 'Plotname: Step Analysis: Step $s$ of $n$ params' at the start of each .STEP \\
20 & Inserts two blank lines before the data for steps 1,2,3,... where 
     the first step is step 0 \\
21 & Inserts one blank line before the data for steps 1,2,3,... where
     the first step is step 0 \\
\end{longtable}
}



\subsubsection{Print Wildcards}
\label{Print_Wildcards}
\index{\texttt{.PRINT}!Wildcards}
Wildcards are supported on \texttt{.PRINT} lines, as described below.  In particular, 
\texttt{V(*)} will print all of the node voltages in the circuit for all
analysis modes.  The \texttt{P(*)}  and \texttt{W(*)} wildcards are supported
for analysis modes (TRAN and DC) that support power calculations.

For TRAN and DC analysis modes, \texttt{I(*)} will print all of the
currents.  This includes both solution variables, which generally means those
associated  with voltage sources and inductors that are not coupled through
a mutual inductance device, and the lead currents associated with most
other devices.  For TRAN and DC, the \texttt{I(*)} wildcard also supports
lead currents for the multi-terminal J, M and Z devices via
\texttt{IB(*)}, \texttt{ID(*)}, \texttt{IG(*)} and \texttt{IS(*)}, and 
for the multi-terminal Q device via \texttt{IB(*)}, \texttt{IC(*)}, \texttt{IE(*)}
and \texttt{IS(*)}.  The \texttt{IE(*)} wildcard is also supported for SOI
and CMG devices.  A  request for \texttt{I(*)} will not return any of the lead currents
for  J, M, Q or Z devices.  Wildcards of the form \texttt{I1(*)},
that use numerical designators, are only supported for the T and YGENEXT devices.
Finally, as an example, a request for IC(*) in a netlist that does not contain
any Q devices will be silently ignored. 

For AC and NOISE analysis modes, the \texttt{I(*)} operator will only output
the branch currents, since lead currents are not supported for those two
analysis modes.  The \texttt{VR(*)}, \texttt{VI(*)}, \texttt{VP(*)},
\texttt{VM(*)}, \texttt{VDB(*)}, \texttt{IR(*)}, \texttt{II(*)},
\texttt{IP(*)}, \texttt{IM(*)} and \texttt{IDB(*)} wildcards are also
supported for these two analysis modes.

There is also support for the * character (meaning ``zero or more
characters'') and the ? character (meaning ``any one character'') in more
complex wildcards, where the * and/or ? characters can be in
any positions in the wildcard specification.  For example, \texttt{V(X1*)}
will output the voltage at all nodes in subcircuit X1 for all analysis
modes.  As another example, \texttt{V(1?)} will output the voltage at all
nodes that have two-character names that start with the character 1.  These
more complex wildcards should work for all supported voltage operators.

Similarly, \texttt{P(X1*)} or \texttt{W(X1*)} will output the
power for all devices, that support power calculations, in subcircuit X1.
Devices that don't support power calculations
will be silently omitted.  Alternately, \texttt{P(R?)} or \texttt{W(R?)}
will output the power for all resistors that have two-character names.

More complex wildcards are also supported for all valid current operators.
The caveats are that for DC and TRAN analyses, the wildcard will include
both branch and lead currents.  For AC and NOISE analyses, the wildcard will
only include branch currents. 

\subsubsection{Device Parameters and Internal Variables}
\label{Print_Device_Info}
\index{\texttt{.PRINT}!Device Parameters and Internal Variables}
This subsection describes how to print out device parameters and device 
internal variables, via a simple V-R circuit example. In particular, 
the example given below gives illustrative examples of how to print out the voltage
at a node ({\tt V(1)}), the current through a device ({\tt I(V1)}), the current through
a device using using an internal solution variable ({\tt N(V1\_branch)}), a device 
parameter ({\tt R1:R}) and the power dissipated by a device ({\tt P(R1)}).  It also shows
how device parameters and internal variables can be used in a \Xyce{} expression.
\begin{alltt}
* filename is example.cir
.DC V1 1 2 1
V1 1 0 1
R1 1 0 2
.PRINT DC FORMAT=NOINDEX PRECISION=2 WIDTH=8 
+ V(1) I(V1) N(V1_branch) R1:R P(R1) \{R1:R*N(V1_branch)*I(V1)\}
.END
\end{alltt}
The \Xyce{} output would then be (where the {\tt NOFORMAT}, {\tt WIDTH} and {\tt PRECISION}
arguments were used mainly to format the example output for this guide):
\begin{alltt}
   V(1)        I(V1)    N(V1\_BRANCH)    R1:R        P(R1)    \{R1:R*N(V1\_BRANCH)*I(V1)\}
   1.00e+00   -5.00e-01   -5.00e-01    2.00e+00    5.00e-01    5.00e-01
   2.00e+00   -1.00e+00   -1.00e+00    2.00e+00    2.00e+00    2.00e+00
\end{alltt}
The internal solution variables for each \Xyce{} device are typically not given in the Reference 
Guide sections on those devices.  However, if for the example given above, the user runs \texttt{Xyce 
-namesfile example\_names example.cir} then the file \texttt{example\_names} would contain a list
of the two solution variables that are accessible with the {\tt N()} syntax on a {\tt .PRINT} line.
In this simple example, they are the voltage at Node 1 and the branch current through the voltage
source {\tt V1}.  If {\tt V1} was in a subcircuit then the \texttt{example\_names} file would have shown
the ``fully-qualified'' device name, including the subcircuit names.
\begin{alltt}
HEADER
	0	   v1_branch
	1	           1
\end{alltt}
Additional (and more useful) examples for using the N() syntax to print out:
\begin{XyceItemize}
\item The $M$, $R$, $B$ and $H$ internal variables for mutual inductors are given in Section
   \ref{K_DEVICE}.  This includes an example where the mutual inductor is in a sub-circuit.
\item The $g_{m}$ (tranconductance), $V_{th}$, $V_{ds}$, $V_{gs}$, $V_{bs}$, and $V_{dsat}$ 
   internal variables for the BSIM3 and BSIM4 models for the MOSFET are given in 
   Section \ref{M_DEVICE}.
\end{XyceItemize}
In these two cases, only the $M$ and $R$ variables for the mutual inductors are
actually solution variables.  However, the {\tt -namesfile} approach can still be
used to determine the fully-qualified \Xyce{} device names required to use the {\tt N()}
syntax.

