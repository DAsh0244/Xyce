% Sandia National Laboratories is a multimission laboratory managed and
% operated by National Technology & Engineering Solutions of Sandia, LLC, a
% wholly owned subsidiary of Honeywell International Inc., for the U.S.
% Department of Energy’s National Nuclear Security Administration under
% contract DE-NA0003525.

% Copyright 2002-2019 National Technology & Engineering Solutions of Sandia,
% LLC (NTESS).


\label{SUBCKTsection}
The \index{\texttt{.SUBCKT}} \texttt{.SUBCKT} statement begins a subcircuit
definition by giving its name, the number and order of its nodes and the names
and default parameters that direct its behavior.  The \texttt{.ENDS} statement
signifies the end of the subcircuit definition. See Section~\ref{XDevicesection}
for more information on using subcircuits with the \texttt{X} device.

%%%
%%% Subcircuit
%%%

\begin{Command}

\format
\begin{alltt}
.SUBCKT <name> [node]*
+ [PARAMS: [<name>=<value>]* ]
\ldots
.ENDS
\end{alltt}

\examples
\begin{alltt}
.SUBCKT OPAMP 10 12 111 112 13
\ldots
.ENDS

.SUBCKT FILTER1 INPUT OUTPUT PARAMS: CENTER=200kHz,
+ BANDWIDTH=20kHz
\ldots
.ENDS

.SUBCKT PLRD IN1 IN2 IN3 OUT1
+ PARAMS: MNTYMXDELY=0 IO_LEVEL=1
\ldots
.ENDS

.SUBCKT 74LS01 A B Y
+ PARAMS: MNTYMXDELY=0 IO_LEVEL=1
\ldots
.ENDS
\end{alltt}

\arguments

\begin{Arguments}
\argument{name}
The \index{subcircuit!name} name used to reference a subcircuit.

\argument{node}
An optional list of nodes. This is not mandatory since it is
feasible to define a subcircuit without any interface nodes.

\argument{PARAMS:}
Keyword that provides values to subcircuits as arguments for use as
expressions in the subcircuit.  Parameters defined in the \texttt{PARAMS:}
section may be used in expressions within the body of the subcircut and
will take the default values specified in the subcircuit definition unless
overridden by a \texttt{PARAMS:} section when the subcircuit is
instantiated.

\end{Arguments}

\comments
A subcircuit  designation ends\index{subcircuit!designation} with a
\texttt{.ENDS} command. The entire netlist between \texttt{.SUBCKT} and
\texttt{.ENDS} is part of the definition. Each time the subcircuit is called
via an \texttt{X} device, the entire netlist in the subcircuit definition
replaces the \texttt{X} device.

There must be an equal number of nodes in the subcircuit call and in its
definition.  As soon as the subcircuit is called, the actual nodes (those in
the calling statement) substitute for the argument nodes (those in the
defining statement).

Node zero\index{subcircuit!node zero} cannot be used in this node list, as
it is the global ground node.

Subcircuit references may be nested\index{subcircuit!nesting} to any
level.  Subcircuits definitions may also be nested;
a \texttt{.SUBCKT} statement and its closing \texttt{.ENDS} may
appear between another \texttt{.SUBCKT}/\texttt{.ENDS} pair.  A
subcircuit defined inside another subcircuit definition is local to
the outer subcircuit and may not be used at higher levels of the
circuit netlist.

Subcircuits should include only device instantiations and possibly these
statements:
\begin{XyceItemize}
\item \texttt{.MODEL} (model definition)
\item \texttt{.PARAM} (parameter)
\item \texttt{.FUNC} (function)
\end{XyceItemize}

\index{\texttt{.MODEL}!subcircuit scoping}\index{\texttt{.PARAM}!subcircuit scoping}\index{\texttt{.FUNC}!subcircuit scoping}
Models, parameters, and functions defined within a subcircuit are
scoped\index{subcircuit!scoping} to that definition.  That is they are only
accessible within the subcircuit definition in which they are included.
Further, if a \texttt{.MODEL}, \texttt{.PARAM} or a \texttt{.FUNC} statement
is included in the main circuit netlist, it is accessible from the main
circuit as well as all subcircuits.

Node, device, and model names are scoped\index{subcircuit!scoping} to the
subcircuit in which they are defined.  It is allowable to use a name in a
subcircuit that has been previously used in the main circuit netlist. When
the subcircuit is flattened (expanded into the main netlist), all of its
names are given a prefix via the subcircuit instance name.  For example,
\texttt{Q17} becomes \texttt{X3:Q17} after expansion.  After expansion, all
names are unique.  The single exception occurs in the use of global node
names, which are not expanded.

Additional illustative examples of scoping are given in the
``Working with Subcircuits and Models'' section of the \Xyce{} Users' 
Guide\UsersGuide.  Those examples apply to models and functions also.

\end{Command}
