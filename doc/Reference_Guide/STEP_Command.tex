% Sandia National Laboratories is a multimission laboratory managed and
% operated by National Technology & Engineering Solutions of Sandia, LLC, a
% wholly owned subsidiary of Honeywell International Inc., for the U.S.
% Department of Energy’s National Nuclear Security Administration under
% contract DE-NA0003525.

% Copyright 2002-2020 National Technology & Engineering Solutions of Sandia,
% LLC (NTESS).


\index{\texttt{.STEP}}
\index{analysis!step} \index{step parametric analysis}
Calculates a full analysis (\verb|.DC|, \verb|.TRAN|, \verb|.AC|, etc.) over a range of
parameter values.  This type of analysis is very similar to .DC analysis.
Similar to .DC analysis, .STEP supports sweeps which are
linear, decade logarithmic, octave logarithmic, a list of values, or over a multivariate data table.

\begin{description}

\item[\tt LIN] Linear sweep \\
The sweep variable is swept linearly from the starting to the ending value.

\item[\tt OCT] Sweep by octaves \\
The sweep variable is swept logarithmically by octaves.

\item[\tt DEC] Sweep by decades \\
The sweep variable is swept logarithmically by decades.

\item[\tt LIST] Sweep over specified values \\
The sweep variable is swept over an enumerated list of values.

\item[\tt DATA] Sweep over table of multivariate values\\
The sweep variables are swept over the rows of a table.  

\end{description}

\subsubsection{Linear Sweeps}
\index{analysis!STEP!Linear sweeps} \index{STEP analysis!Linear sweeps}

\begin{Command}

\format
.STEP [LIN] <parameter name> <initial> <final> <step>

\examples
\begin{alltt}
.STEP R1 45 50 5
.STEP V1 20 10 -1
.STEP LIN V1 20 10 -1
.STEP TEMP -45 -55 -10
.STEP C101:C 45 50 5
.STEP DLEAK:IS 1.0e-12 1.0e-11 1.0e-12

.global_param v1_val=10
V1 1 0 DC \{v1_val\}
.STEP v1_val 20 10 -1

.data table
+ c1 r1
+ 1e-8  1k
+ 2e-8  0.5k
+ 3e-8  0.25k
.enddata
.STEP data=table
\end{alltt}

\arguments

\begin{Arguments}

\argument{parameter name}
Name of the parameter to be swept.  This may be the special parameter
name \texttt{TEMP} (the ambient simulation temperature), a device
name, device instance or model parameter name, or global parameter
name as defined in a \texttt{.global\_param} statement.  It may not be
the name of a parameter defined in a \texttt{.param} statement.

If a device name is given, the primary parameter for that device is
taken as the parameter; in the first two examples above, the primary
parameters of the devices R1 and V1 are stepped (resistance and DC
voltage, respectively).  The C, L and I devices are then the other
devices with primary parameters, which are the capacitance, inductance 
and DC current, respectively.

To specify a device instance parameter other than the device's primary
parameter, or if the device has no primary parameter, use the 
syntax \texttt{<device name>}:\texttt{<parameter name>}, as
in the fourth example above.

To sweep a device model parameter, use the syntax \texttt{<model
name>}:\texttt{<parameter name>}, as in the fifth example above.

\argument{initial}
Initial value for the parameter.

\argument{final}
Final value for the parameter.

\argument{step}
Value that the parameter is incremented at each step.

\end{Arguments}

\comments

For linear sweeps, the LIN keyword is optional.

STEP parameter analysis will sweep a parameter from its initial value to
its final value, at increments of the step size.  At each step of this
sweep, it will conduct a full analysis (\texttt{.DC}, \texttt{.TRAN},
\texttt{.AC}, etc.) of the circuit.

The specification is similar to that of a \texttt{.DC} sweep, except
that unlike \texttt{.DC}, only one parameter may be swept on
each \texttt{.STEP} line.  Multiple \texttt{.STEP} lines may be
specified, forming nested step loops.  The variables will be stepped
in order such that the first \texttt{.STEP} line that appears in the
netlist will be the innermost loop, and the last \texttt{.STEP} line
will be the outermost.

Output, as designated by a \texttt{.PRINT} statement, is slightly more
complicated in the case of a \texttt{.STEP} simulation.  If the user
has specified a \texttt{.PRINT} line in the input file, \Xyce{} will
output two files.  All steps of the sweep will be output to a single file as
usual, but with the results of each step appearing one after another
with the ``Index'' column starting over at zero.  Additionally, a file
with a ``.res'' suffix will be produced indicating what parameters
were used for each iteration of the step loops; this file will always
be in columnar text format, irrespective of any \texttt{FORMAT=}
option specified on \texttt{.PRINT} lines.  If \texttt{.RESULT} lines
(see section~\ref{.RESULT}) appear in the netlist, the ``.res'' file
will also contain columns for each expression given
on the \texttt{.RESULT} lines, and the value of the result expression
will be printed for each step taken.

Note that analysis lines in \Xyce{} do not currently support use of
expressions to define their parameters (e.g., end times
for \texttt{.TRAN} analysis, or fundamental frequencies
for \texttt{.HB} analysis), and so it is not possible to use stepped
parameters to vary how the analysis will be run at each step.  If each
step requires different analysis parameters, this would have to be
accomplished by performing separate runs of \Xyce{}.

If the stop value is smaller than the start value, the step value
should be negative.  If a positive step value is given in this case,
only a single point (at the start value) will be performed, and a
warning will be emitted.

\end{Command}

\subsubsection{Decade Sweeps}
\index{analysis!STEP!Decade sweeps} \index{STEP analysis!Decade sweeps}

\begin{Command}

\format
.STEP DEC <sweep variable name> <start> <stop> <points>

\examples

\begin{alltt}
.STEP DEC VIN 1 100 2
.STEP DEC R1 100 10000 3 
.STEP DEC TEMP 1.0 10.0 3
\end{alltt}

\comments
The stop value should be larger than the start value.  If a stop value
smaller than the start value is given, only a single point at the
start value will be performed, and a warning will be emitted.

\end{Command}

\subsubsection{Octave Sweeps}
\index{analysis!STEP!Octave sweeps} \index{STEP analysis!Octave sweeps}

\begin{Command}

\format
.STEP OCT <sweep variable name> <start> <stop> <points>

\examples

\begin{alltt}
.STEP OCT VIN 0.125 64 2 
.STEP OCT TEMP 0.125 16.0 2 
.STEP OCT R1 0.015625 512 3
\end{alltt}

\comments
The stop value should be larger than the start value.  If a stop value
smaller than the start value is given, only a single point at the
start value will be performed, and a warning will be emitted.

\end{Command}

\subsubsection{List Sweeps}
\index{analysis!STEP!List sweeps} \index{STEP analysis!List sweeps}

\begin{Command}

\format
\begin{alltt}
.STEP <sweep variable name> LIST <val> <val> <val>\ldots
\end{alltt}

\examples
\begin{alltt}
.STEP VIN LIST 1.0 2.0 10. 12.0 
.STEP TEMP LIST 8.0 21.0
\end{alltt}

\end{Command}

\subsubsection{Data Sweeps}
\index{analysis!DC!Data Sweeps} \index{DC analysis!Data sweeps}

\begin{Command}
\format
\begin{alltt}
.STEP DATA=<data table name> 
\end{alltt}

\examples
.STEP data=resistorValues

.data resistorValues \\
+ r1   r2 \\
+ 8.0000e+00  4.0000e+00 \\
+ 9.0000e+00  4.0000e+00 \\
.enddata

\end{Command}
