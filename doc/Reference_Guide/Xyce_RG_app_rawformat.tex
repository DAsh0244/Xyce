% Sandia National Laboratories is a multimission laboratory managed and
% operated by National Technology & Engineering Solutions of Sandia, LLC, a
% wholly owned subsidiary of Honeywell International Inc., for the U.S.
% Department of Energy’s National Nuclear Security Administration under
% contract DE-NA0003525.

% Copyright 2002-2020 National Technology & Engineering Solutions of Sandia,
% LLC (NTESS).


\chapter{Rawfile Format}
\label{rawformat}
\index{rawfile}

The rawfile format produced by \Xyce{} closely follows SPICE3
conventions.  Differences are noted in section~\ref{rawformatnotes}. 
Details on the both the ASCII and binary formats are provided here for
reference. 


\section{ASCII Format}
\label{rawformatascii}
\index{rawfile!ASCII}

The ASCII file format can be created using the \texttt{-a} flag on the command
line. See Chapter \ref{cmd_line_args} for more information.

The ASCII format standard dictates that the file consist of lines or sets of
lines introduced by a keyword. The \texttt{Title} and \texttt{Date} lines
should be the first in the file, and should occur only once. They are followed
by the \texttt{Plotname}, \texttt{Flags}, \texttt{No.~Variables}, and
\texttt{No.~Points} lines for each plot. 

Listed next are sets of \texttt{Variables}, and \texttt{Values} lines. Let
\emph{numvars} be the number of variables (as specified in the
\texttt{No.~Variables} line), and \emph{numpts} be the number of points (as
shown on the \texttt{No.~Points} line). After the \texttt{Variables} keyword
there must be \emph{numvars} declarations of outputs, and after the
\texttt{Values} keyword, there must be \emph{numpts} lines, each consisting of
\emph{numvars} values. 

Finally, \Xyce{} also allows for a \texttt{Version} line to be placed after the
\texttt{No.~Points} line for compatability with various software programs.

See Table~\ref{table_rawformatascii} for a summary of the above.

\begin{longtable}[h] {>{\raggedright\small}m{1.5in}|>{\raggedright\let\\\tabularnewline\small}m{4.5in}}
  \caption{\Xyce{} ASCII rawfile format.} \\ \hline
  \rowcolor{XyceDarkBlue}
  \color{white}\bf Issue & 
  \color{white}\bf Comment \\ \hline \endfirsthead  
  \label{table_rawformatascii}

    \texttt{Title:} & 
    An arbitrary string describing the circuit.
    \\ \hline

    \texttt{Date:} & 
    A free-format date string.
    \\ \hline

    \texttt{Plotname:} & 
    A string describing the analysis type.
    \\ \hline

    \texttt{Flags:} & 
    A string describing the data type (\emph{real} or \emph{complex}).
    \\ \hline

    \texttt{No.~Variables:} & 
    The number of variables.
    \\ \hline

    \texttt{No.~Points:} & 
    The number of points.
    \\ \hline
    
    \texttt{Version:} (optional) &
    The version of \Xyce{} used to generate this output. By default the version is not output in the header.  It can 
    be output with the \texttt{.options output outputversioninrawfile=true} option.
    \\ \hline

    \texttt{Variables:} & 
    A newline followed by multiple lines, one for each variable, of the form 
    \texttt{[tab] <index> [tab] <name> [tab] <type>}.
    \\ \hline

    \texttt{Values:} & 
    A newline followed by multiple lines, for each point and variable, of the form
    \texttt{[tab] <value> }
    with an integer index preceeding each set of points.  Complex values are output 
    as \texttt{[tab] <real component>, <imaginary component> }.
    \\ \hline

  %\end{tabularx}
\end{longtable}



\section{Binary Format}
\label{rawformatbinary}
\index{rawfile!binary}

The binary format is similar to the ASCII format, except that strings are null
terminated rather than newline terminated. In addition, all the \texttt{values}
lines are stored in a binary format. The binary storage of real values as
double precision floats is architecture specific.

See Table~\ref{table_rawformatbinary} for a summary of the binary table format.

\begin{longtable}[h] {>{\raggedright\small}m{1.5in}|>{\raggedright\let\\\tabularnewline\small}m{4.5in}}
  \caption{\Xyce{} binary rawfile format.} \\ \hline
  \rowcolor{XyceDarkBlue}
  \color{white}\bf Issue & 
  \color{white}\bf Comment \\ \hline \endfirsthead  
  \label{table_rawformatbinary}

    \texttt{Title:} & 
    An arbitrary string describing the circuit.
    \\ \hline

    \texttt{Date:} & 
    A free-format date string.
    \\ \hline

    \texttt{Plotname:} & 
    A string describing the analysis type.
    \\ \hline

    \texttt{Flags:} & 
    A string describing the data type (\emph{real} or \emph{complex}).
    \\ \hline

    \texttt{No.~Variables:} & 
    The number of variables.
    \\ \hline

    \texttt{No.~Points:} & 
    The number of points.
    \\ \hline
    
    \texttt{Version:} (optional) &
    The version of \Xyce{} used to generate this output. By default the version is not output in the header.  It can 
    be output with the \texttt{.options output outputversioninrawfile=true} option.
    \\ \hline

    \texttt{Variables:} & 
    A newline followed by multiple lines, one for each variable, of the form 
    \texttt{[tab] <index> [tab] <name> [tab] <type>}.
    \\ \hline

    \texttt{Binary:} & 
    Each real data point is stored contiguously in \texttt{sizeof(double)} byte blocks. 
    Complex values are output as real and imaginary components in a block of size 
    \texttt{2*sizeof(double)} byte blocks.
    \\ \hline

\end{longtable}


\section{Special Notes}
\label{rawformatnotes}

\begin{XyceItemize}
\item Complex data points are only output under \texttt{AC} analysis.
\item \texttt{Commands} and \texttt{Options} lines are not used.
\item Binary header is formatted ASCII.
\item \Xyce{} can output an optional \texttt{Version} line in the header.
\end{XyceItemize}



%%% Local Variables:
%%% mode: latex
%%% End:

% END of Xyce_RG_app_rawfile.tex ************
