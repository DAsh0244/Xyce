% Sandia National Laboratories is a multimission laboratory managed and
% operated by National Technology & Engineering Solutions of Sandia, LLC, a
% wholly owned subsidiary of Honeywell International Inc., for the U.S.
% Department of Energy’s National Nuclear Security Administration under
% contract DE-NA0003525.

% Copyright 2002-2022 National Technology & Engineering Solutions of Sandia,
% LLC (NTESS).


%%
%% LOCA Solver Options Table
%%
\index{solvers!continuation!options}
\begin{OptionTable}{Options for Continuation and Bifurcation Tracking Package.}
\label{ContinuationPKG}
STEPPER & Stepping algorithm to use:
\begin{XyceItemize}
\item 0 (Natural or Zero order continuation)
\item 1 (Arc-length continuation)
\end{XyceItemize} &
0 (Natural) \\ \hline

PREDICTOR &
Predictor algorithm to use:
\begin{XyceItemize}
\item 0 (Tangent)
\item 1 (Secant)
\item 2 (Random)
\item 3 (Constant)
\end{XyceItemize} &
0 (Tangent) \\ \hline

STEPCONTROL &
Algorithm used to adjust the step size between continuation steps:
\begin{XyceItemize}
\item 0 (Constant)
\item 1 (Adaptive)
\end{XyceItemize} &
0 (Constant) \\ \hline

CONPARAM &
Parameter in which to step during a continuation run &
VA:V0 \\ \hline

INITIALVALUE & Starting value of the continuation parameter &
0.0 \\ \hline

MINVALUE & Minimum value of the continuation parameter &
-1.0E20 \\ \hline

MAXVALUE & Maximum value of the continuation parameter &
1.0E20 \\ \hline

BIFPARAM & Parameter to compute during bifurcation tracking runs &
VA:V0 \\ \hline

MAXSTEPS & Maximum number of continuation steps (includes failed steps) & 20 \\ \hline

MAXNLITERS & Maximum number of nonlinear iterations allowed (set this parameter equal to the \texttt{MAXSTEP} parameter in the  \texttt{NONLIN} option block & 20 \\ \hline

INITIALSTEPSIZE & Starting value of the step size & 1.0 \\ \hline

MINSTEPSIZE & Minimum value of the step size & 1.0E20 \\ \hline

MAXSTEPSIZE & Maximum value of the step size & 1.0E-4 \\ \hline

AGGRESSIVENESS & Value between 0.0 and 1.0 that determines how aggressive the step size control algorithm should be when increasing the step size.  0.0 is a constant step size while 1.0 is the most aggressive. & 0.0 \\ \hline

RESIDUALCONDUCTANCE & If set to a nonzero (small) number, this parameter will
force the GMIN stepping algorithm  to stop and declare victory once the
artificial resistors have a conductance that is smaller  than this number.
This should only be used in transient simulations, and \emph{ONLY} if it is
absolutely necessary to get past the DC operating point calculation. It is
almost always better to fix the circuit so that residual conductance is not
necessary. & 0.0 \\ \hline

\end{OptionTable}
