% Sandia National Laboratories is a multimission laboratory managed and
% operated by National Technology & Engineering Solutions of Sandia, LLC, a
% wholly owned subsidiary of Honeywell International Inc., for the U.S.
% Department of Energy’s National Nuclear Security Administration under
% contract DE-NA0003525.

% Copyright 2002-2021 National Technology & Engineering Solutions of Sandia,
% LLC (NTESS).

\label{FFT}
\index{fft}\index{results!fft}

Performs Fast Fourier Transform analysis of transient analysis output.

\begin{Command}

\format
\begin{alltt}
.FFT <ov> [NP=<value>] [WINDOW=<value>] [ALFA=<value>]
+ [FORMAT=<value>] [START=<value>] [STOP=<value>]
+ [FREQ=value] [FMIN=value] [FMAX=value]
\end{alltt}

\examples
\begin{alltt}
.FFT v(1)
.FFT v(1,2) NP=512 WINDOW=HANN
.FFT \{v(3)-v(2)\} START=1 STOP=2
\end{alltt}

\arguments

\begin {Arguments}

\argument{ov}
The desired solution output to be analyzed. Only one
output variable can be specified on a {\tt .FFT} line. However,
multiple {\tt .FFT} lines with the same output variable but, for
example, different windowing functions may be used in a netlist.
The available outputs are:

\begin{itemize}
  \item \texttt{V(<circuit node>)} the voltage at \texttt{<circuit node>}
  \item \texttt{V(<circuit node>,<circuit node>)} to output the voltage
    difference between the first \texttt{<circuit node>} and second
    \texttt{<circuit node>}
  \item \texttt{I(<device>)} the current through a two terminal device
  \item \texttt{I<lead abbreviation>(<device>)} the current into a particular lead
    of a three or more terminal device (see the Comments, below, for details)
  \item \texttt{N(<device parameter>)} a specific device parameter (see the
    individual devices in Section~\ref{Analog_Devices} for syntax)
\end{itemize}

\argument{NP}
The number of points in the FFT.  This value must be a power of 2.  If
the user-entered value is not a power of two then it will be rounded to
the nearest power of 2.  The minimum allowed value of {\tt NP} is 4.
The default value is 1024.

\argument{WINDOW}
The windowing function that will be applied to the sampled waveform
values.  The allowed values are as follows, where table~\ref{FFT_Window_Funcs}
gives their exact definitions:

\begin{itemize}
  \item \texttt{RECT} (or \texttt{RECTANGULAR}) = rectangular window (default)
  \item \texttt{BART} (or \texttt{BARTLETT}) = Bartlett (triangular) window
  \item \texttt{BARTLETTHANN} = Bartlett-Hann window
  \item \texttt{BLACK} = Blackman window
  \item \texttt{BLACKMAN} = ``Conventional Blackman'' window
  \item \texttt{HAMM} (or \texttt{HAMMING}) = Hamming window
  \item \texttt{HANN} (or \texttt{HANNING}) = Hanning window
  \item \texttt{HARRIS} (or \texttt{BLACKMANHARRIS}) = Blackman-Harris window
  \item \texttt{NUTTALL} = Nuttall window
  \item \texttt{COSINE2} = Power-of-cosine window, with exponent 2.
  \item \texttt{COSINE4} = Power-of-cosine window, with exponent 4.
  \item \texttt{HALFCYCLESINE} = Half-cycle sine window
  \item \texttt{HALFCYCLESINE3} = Half-cycle sine window, with exponent 3.
  \item \texttt{HALFCYCLESINE6} = Half-cycle sine window, with exponent 6.
\end{itemize}

\argument{ALFA}
This parameter is supported for HSPICE compatibility. It currently
has no effect though, since the {\tt GAUSS} and {\tt KAISER} windows
are not supported.

\argument{FORMAT}
The allowed values are {\tt NORM} and {\tt UNORM}.  If {\tt NORM} is
selected then the magnitude values will be normalized to the largest
magnitude.  If {\tt UNORM} is selected then the actual magnitude values
will be output instead.  The default value for this parameter depends
on the \texttt{.OPTIONS FFT FFT\_MODE} setting.  If \texttt{FFT\_MODE=0},
which is the default options setting, then the default value is {\tt NORM}.
If \texttt{FFT\_MODE=1} then the default value is {\tt UNORM}.

\argument{START}
The start time for the FFT analysis.  The default value is the
start time for the transient analysis. {\tt FROM} is an allowed
synonym for {\tt START}.

\argument{STOP}
The end time for the FFT analysis.  The default value is the
end time for the transient analysis. {\tt TO} is an allowed
synonym for {\tt STOP}.

\argument{FREQ}
The ``first harmonic'' of the frequencies provided in the output
file.  The default value for {\tt FREQ} is {\tt 1/(STOP - START)}.
If {\tt FREQ} is given then it is rounded to the nearest integer
multiple of the default value.  The \Xyce{} Users' Guide\UsersGuide
provides an example.

\argument{FMIN}
This parameter can use to adjust the harmonics included in the
``additional metrics'' defined in Section \ref{FFT_metrics}.  It
has a default value of 1.

\argument{FMAX}
This parameter can use to adjust the harmonics included in the
``additional metrics'' defined in Section \ref{FFT_metrics}.  It
has a default value of \texttt{NP/2}.

\end{Arguments}

\comments
Multiple \texttt{.FFT} lines may be used in a netlist.  All results from FFT analyses
will be returned to the user in a file with the same name as the netlist file suffixed
with \texttt{.fftX} where {\tt X} is the step number.

\texttt{<lead abbreviation>} is a single character designator for individual
leads on a device with three or more leads.  For bipolar transistors these are:
c (collector), b (base), e (emitter), and s (substrate).  For mosfets, lead
abbreviations are: d (drain), g (gate), s (source), and b (bulk).  SOI
transistors have: d, g, s, e (bulk), and b (body).  For PDE devices, the nodes
are numbered according to the order they appear, so lead currents are
referenced like I1(\texttt{<device>}), I2(\texttt{<device>}), etc.

For this analysis, the phase data is always output in degrees.

In \Xyce{}, \texttt{WINDOW=TRIANGULAR} is not an allowed synonym for
\texttt{WINDOW=BART}.  This is to avoid confusion with other analysis
packages such as SciPy and Matlab.

\end{Command}

\subsubsection{.OPTIONS FFT FFT\_MODE}
\label{FFT_MODE}
\index{fft!options}
The setting for \texttt{.OPTIONS FFT FFT\_MODE} is used to control whether the
\Xyce{} FFT processing and output are more compatible with HSPICE (0) or
Spectre (1).  This setting affects the format of the window functions, the
conversion from two-sided to one-sided results, and whether the default output
for the magnitude values is normalized, or not.  The default setting for
\texttt{FFT\_MODE} is 0.

If \texttt{FFT\_MODE=0} then symmetric window functions are used.  If
\texttt{FFT\_MODE=1} then periodic window functions are used.  The next
subsection provides more details on that difference.

If \texttt{FFT\_MODE=0} then the two-sided to one-sided conversion doubles
the magnitudes of the 1,2,...,NP harmonics. If \texttt{FFT\_MODE=1} then that
conversion only doubles the magnitudes of the 1,2,...,(NP-1) harmonics.

The default value for the \texttt{FORMAT} parameter depends on the
\texttt{.OPTIONS FFT FFT\_MODE} setting.  If \texttt{FFT\_MODE=0} then
the default value for that parameter is {\tt NORM}. If \texttt{FFT\_MODE=1}
then the default value is {\tt UNORM}.

\subsubsection{Window Functions}
\index{fft!window functions}
Table~\ref{FFT_Window_Funcs} gives the definitions of the window functions
implemented in \Xyce{}.  For HSPICE compatibility, the {\tt BLACK} window type
is actually the ``-67 dB Three-Term Blackman-Harris window''~\cite{Doerry2017}
rather than the ``Conventional Blackman Window'' used by Spectre, Matlab and SciPy.
The Convential Blackman Window can be selected with the {\tt BlACKMAN} window
type instead.  The definition of the {\tt BART} window type ~\cite{oppenheimSchafer}
was chosen to match Spectre, Matlab and SciPy.  The \Xyce{} definition may differ
from HSPICE.

As mentioned in in the previous subsection, the choice of symmetric vs. periodic
window functions can be selected via the use of \texttt{.OPTIONS FFT FFT\_MODE=<0|1>}.
If symmetric windows are used then $L = N-1$ in the formulas in
table~\ref{FFT_Window_Funcs}, where $N$ is the number of points in the FFT.
If periodic windows are used then $L=N$.

\begin{longtable}[h] {>{\raggedright\small}m{1.15in}|>{\raggedright\let\\\tabularnewline\small}m{1.35in}
  |>{\raggedright\let\\\tabularnewline\small}m{3.5in}}
  \caption{.FFT Window Function Definitions (N = Number of Points).} \\ \hline
  \rowcolor{XyceDarkBlue}
  \color{white}\bf Value &
  \color{white}\bf Description &
  \color{white}\bf Definition \\ \hline \endfirsthead
  \rowcolor{XyceDarkBlue}
  \color{white}\bf Value &
  \color{white}\bf Description &
  \color{white}\bf Definition \\ \hline \endhead
  \label{FFT_Window_Funcs}

  RECT & Rectangular &
    $w(i)=1, \hspace*{0.1in}\textrm{for}~0 \leq i < N$ \\ \hline
  BART & Barlett~\cite{oppenheimSchafer} &
    $w(i) = \left\{\begin{array}{ll}\frac{2 i}{L},\hspace*{0.1in}\textrm{if}~i < 0.5 \cdot (N-1) \\
                                     2-\frac{2 i}{L},\hspace*{0.1in}\textrm{otherwise}\end{array}\right.$\\ \hline
  BARTLETTHANN & Bartlett-Hann~\cite{Doerry2017}&
    $w(i) = 0.62 - 0.48 \cdot |\frac{i}{L} -0.5| + 0.38 \cdot cos(2 \pi \cdot (\frac{i}{L} -0.5)), \hspace*{0.1in}\textrm{for}~0 \leq i < N $ \\ \hline
  HANN & Hanning~\cite{oppenheimSchafer} &
    $w(i) = sin^2(\frac{\pi i}{L}), \hspace*{0.1in}\textrm{for}~0 \leq i < N$ \\ \hline
  HAMM & Hamming~\cite{oppenheimSchafer} &
    $w(i) = 0.54 - 0.46 \cdot cos(\frac{2 \pi i}{L}), \hspace*{0.1in}\textrm{for}~0 \leq i < N $ \\ \hline
  BLACKMAN & ``Conventional Blackman window''~\cite{Doerry2017} &
    $w(i) = 0.42-0.5 \cdot cos(\frac{2 \pi i}{L})+0.08 \cdot cos(\frac{4 \pi i}{L}), \hspace*{0.1in}\textrm{for}~0 \leq i < N$  \\ \hline
  BLACK & -67 dB Three-Term Blackman-Harris window~\cite{Doerry2017} &
    $w(i) = 0.42323-0.49755 \cdot cos(\frac{2 \pi i}{L})+0.07922 \cdot cos(\frac{4 \pi i}{L}), \hspace*{0.1in}\textrm{for}~0 \leq i < N$  \\ \hline
  HARRIS & -92 dB Four-Term Blackman-Harris window~\cite{Doerry2017} &
    $w(i) = 0.35875 - 0.48829 \cdot cos(\frac{2 \pi i}{L}) + 0.14128 \cdot cos(\frac{4 \pi i}{L}) - 0.01168 \cdot cos(\frac{6 \pi i}{L}), \hspace*{0.1in}\textrm{for}~0 \leq i < N$  \\ \hline
  NUTTALL & Four-Term Nuttall, Minimum Sidelobe (Blackman-Nuttall)~\cite{Doerry2017} &
    $w(i) = 0.3635819 - 0.4891775 \cdot cos(\frac{2 \pi i}{L}) + 0.1365995 \cdot cos(\frac{4 \pi i}{L}) - 0.0106411 \cdot cos(\frac{6 \pi i}{L}), \hspace*{0.1in}\textrm{for}~0 \leq i < N$  \\ \hline
   COSINE2 & Power-of-cosine window, with exponent 2 &
   $w(i) = 0.5 - 0.5 \cdot cos(\frac{2 \pi i}{L}), \hspace*{0.1in}\textrm{for}~0 \leq i < N $ \\ \hline
   COSINE2 & Power-of-cosine window, with exponent 4 &
   $w(i) = 0.375-0.5 \cdot cos(\frac{2 \pi i}{L})+0.125 \cdot cos(\frac{4 \pi i}{L}), \hspace*{0.1in}\textrm{for}~0 \leq i < N$ \\ \hline
   HALFCYCLESINE & Half-cycle sine window & $w(i) = sin(\frac{\pi i}{L}) $ \\ \hline
   HALFCYCLESINE3 & Half-cycle sine window, with exponent 3 & $w(i) = sin^{3}(\frac{\pi i}{L}) $ \\ \hline
   HALFCYCLESINE6 & Half-cycle sine window, with exponent 6 & $w(i) = sin^{6}(\frac{\pi i}{L}) $ \\ \hline
\end{longtable}

\subsubsection{Additional FFT Metrics}
\index{fft!additional metrics}
\label{FFT_metrics}
The following additional metrics will be sent to stdout if \texttt{.OPTIONS FFT FFTOUT=1}
is used in the netlist.  These definitions are also used by the corresponding measure
types on \texttt{.MEASURE FFT} lines.

Define the integer index of the ``first harmonic'' ($f_{0}$) as follows (where it has a
default value of 1 and the $ROUND()$ function rounds to the nearest integer):

$f_{0}=\left\{\begin{array}{ll}ROUND(\frac{FREQ}{STOP-START}),\hspace*{0.1in}\textrm{if FREQ given} \\
 1,\hspace*{1.4in}\textrm{otherwise}\end{array}\right.$

Finally, define $mag[i]$ as the magnitude of the FFT coefficient at index $i$, and $N$ as
the number of points in the FFT.

The Signal to Noise-plus-Distortion Ratio (SNDR) is calculated as follows, where the summation
in the denominator includes all of the frequencies except for the first harmonic frequency $f_{0}$:

$SNDR = 20 \cdot log10(\frac{mag[f_{0}]}{sqrt(\sum mag[i]*mag[i])}),\hspace*{0.1in}\textrm{for}~1 \leq i \leq 0.5 \cdot N ~\textrm{, and}~i \neq f_{0}$

The Effective Number of Bits (ENOB) is calculated as follows, where the units is ``bits'':

$ENOB = \frac{SNDR - 1.76}{6.02}$

For the Signal to Noise Ratio (SNR) metric define the upper frequency limit for the SNR metric as follows,
with the caveat that if $f_{2}$ is less than $f_{0}$ then its value is set to $f_{0}$:

$f_{2}=\left\{\begin{array}{ll}ROUND(\frac{FMAX}{STOP-START}),\hspace*{0.1in}\textrm{if FMAX given}\\
   0.5 \cdot N,\hspace*{1.1in}\textrm{otherwise}\end{array}\right.$

The SNR is then calculated as follows, where the summation in the denominator only uses indexes that
are either not an integer-multiple of $f_{0}$ or that are greater than the upper frequency limit of $f_{2}$.
(Note: for the default case of $f_{0}$ = 1 and \texttt{FMAX} not given there are no ``noise frequencies''.)

$SNR = 20 \cdot log10(\frac{mag[f_{0}]}{sqrt(\sum mag[i]*mag[i])}),\hspace*{0.1in}\textrm{for}~i > f_{2} ~\textrm{ or }~i\%f_{0} \ne 0$

For the Total Harmonic Distortion (THD) metric define the upper frequency limit ($f_{2}$) in
the summation in the numerator as:

$f_{2}=\left\{\begin{array}{ll}ROUND(\frac{FMAX}{STOP-START}),\hspace*{0.1in}\textrm{if FMAX given} \\ 0.5 \cdot N,\hspace*{1.1in}\textrm{otherwise}\end{array}\right.$

The THD is then calculated using only the indexes that are multiples of $f_{0}$.  So, if $f_{0}$=2
then the summation in the numerator would only include indexes 4,6,8,etc.

$THD = 20 \cdot log10(\frac{sqrt(\sum mag[i]*mag[i])}{mag[f_{0}]}),\hspace*{0.1in}\textrm{for}~2 \cdot f_{0} \leq i \leq f_{2} ~\textrm{, and}~i\%f_{0}=0$

For Spurious Free Distortion Ratio (SFDR) metric define the upper and lower frequency
limits ($f_{2}$ and $f_{1}$) considered in the denominator of the calculation as:

$f_{2}=\left\{\begin{array}{ll}ROUND(\frac{FMAX}{STOP-START}),\hspace*{0.1in}\textrm{if FMAX given} \\
 0.5 \cdot N,\hspace*{1.1in}\textrm{otherwise}\end{array}\right.$

and:

$f_{1}=\left\{\begin{array}{ll}ROUND(\frac{FMIN}{STOP-START}),\hspace*{0.1in}\textrm{if FMIN given} \\
 f_{0},\hspace*{1.35in}\textrm{if FMIN not given, and}~f_{2} \geq f_{0}\\
 1,\hspace*{1.4in}\textrm{if FMIN not given, and}~f_{2} < f_{0}\end{array}\right.$

The SFDR is then calculated as:

$SFDR = 20 \cdot log10(\frac{mag[f_{0}]}{MAX(mag[i])}), \hspace*{0.1in}\textrm{for}~f_{1} \leq i \leq f_{2}~\textrm{, and}~i \neq f_{0}$

\subsubsection{Re-Measure}
\label{FFT_ReMeasure}
\index{fft!remeasure}
\Xyce{} can re-calculate (or re-measure) the values for {\tt .FFT} statements
using existing \Xyce{} output files.  Section~\ref{Measure_ReMeasure} discusses
this topic in more detail for both {\tt .MEASURE} and {\tt .FFT} statements. One
additional caveat is that \texttt{FFT\_ACCURATE} is set to 0 during the re-measure
operation.  This should have no impact on the accuracy of the re-measured results
if the output file was previously generated with \texttt{FFT\_ACCURATE} set to 1.

\subsubsection{Compatibility Between OUTPUT and FFT Options}
\index{fft!options}
Some \texttt{.OPTIONS OUTPUT} settings are incompatible with using the default
setting of \texttt{.OPTIONS FFT FFT\_ACCURATE=1}.  The use of
\texttt{.OPTIONS OUTPUT OUTPUTTIMEPOINTS} is compatible.  However, the use of
\texttt{.OPTIONS OUTPUT INITIAL\_INTERVAL} is not.  In that latter case,
\texttt{.OPTIONS FFT FFT\_ACCURATE} will be automatically set to 0.
