% Sandia National Laboratories is a multimission laboratory managed and
% operated by National Technology & Engineering Solutions of Sandia, LLC, a
% wholly owned subsidiary of Honeywell International Inc., for the U.S.
% Department of Energy’s National Nuclear Security Administration under
% contract DE-NA0003525.

% Copyright 2002-2020 National Technology & Engineering Solutions of Sandia,
% LLC (NTESS).


%%
%% Time Integration Package Table
%%

\begin{OptionTable}{Options for Time Integration Package.}
\label{TimeIntPKG}
METHOD & Time integration method.  This parameter is only
relevant when running \Xyce{} in transient mode.  Supported methods:
\begin{XyceItemize}
\item trap or 7 (variable order Trapezoid)
\item gear or 8 (Gear method) 
\end{XyceItemize} &
trap or 7 (variable order Trapezoid) \\ \hline
RELTOL\index{\texttt{RELTOL}}  & Relative error tolerance & 
1.0E-03 \\ \hline
ABSTOL\index{\texttt{ABSTOL}}  & Absolute error tolerance & 1.0E-06 \\
\hline
RESTARTSTEPSCALE\index{\texttt{RESTARTSTEPSCALE}}  & 
This parameter is a scalar which determines how small the initial
time step out of a breakpoint should be.  In the current version of the
time integrator, the first step after a breakpoint isn't subjected to
much error analysis, so for very stiff circuits, this step can be 
problematic. & 0.005 \\ \hline
NLNEARCONV\index{\texttt{NLNEARCONV}}  & 
This flag sets if ``soft'' 
failures of the nonlinear solver, when the convergence criteria are almost, 
but not quite, met, should result in a "success" code being returned from
the nonlinear solver to the time integrator.
If this is enabled, it is expected that the error analysis performed by the 
time integrator will be the sole determination of whether or not the time 
step is considered a ``pass'' or a ``fail''.  This is on by default, but
occasionally circuits need tighter convergence criteria.  
& 0 (FALSE)  \\ \hline
NLSMALLUPDATE\index{\texttt{NLSMALLUPDATE}} &   
This flag is another ``soft'' nonlinear solver failure flag.  In
this case, if the flag is set, time steps in which the nonlinear solver 
stalls, and is using updates that are numerically tiny, can be considered
to have converged by the nonlinear solver.  If this flag is set, 
the time integrator is responsible for determining
if a step should be accepted or not.
& 1 (TRUE) \\ \hline

RESETTRANNLS & The nonlinear solver resets its settings for the
transient part of the run to something more efficient (basically a simpler set
of options with smaller numbers for things like max Newton step).  If this is
set to false, this resetting is turned off. Normally should be left as
default. & 1 (TRUE) \\ \hline

MAXORD & This parameter determines the maximum order of integration
that time integrators will attempt.  Setting this option
does not guarantee that the integrator will integrate at this order, it just
sets the maximum order the integrator will attempt.  In order to guarantee a
particular order is used, see the option \texttt{MINORD} below.  & 
2 for variable order Trapezoid and Gear \\ \hline

MINORD & This parameter determines the minimum order of integration
that  time integrators will attempt to maintain.  The integrator will
start at Backward Euler and move up in order as quickly as possible to achive
\texttt{MINORD} and then it will keep the order above this.  If \texttt{MINORD}
is set at 2 and \texttt{MAXORD} is set at 2, then the integrator will move to
second order as quickly as possible and stay there.  & 1 \\ \hline

NEWLTE & This parameter determines the reference value for relative
convergence criterion in the local truncation error based time step control.
The  supported choices
\begin{XyceItemize}
\item 0. The reference value is the current value on each node.
\item 1. The reference value is the maximum of all the signals at the current time.
\item 2. The reference value is the maximum of all the signals over all past time.
\item 3. The reference value is the maximum value on each signal over all past time.
\end{XyceItemize}   & 1 \\ \hline

NEWBPSTEPPING & This flag sets a new time stepping method after a break point. 
Previously, \Xyce{} treats each breakpoint identically to the DCOP point, in which
the intitial time step out of the DCOP is made to be very very small, because
the LTE calculation is unreliable.  As a result, \Xyce{} takes an incredibly small
step out of each breakpoint and then tries to grow the stepsize from there. 
When \texttt{NEWBPSTEPPING} is set, \Xyce{} can take a reasonable
large step out of every non-DCOP breakpoint, and then just relies on the step
control to ensure that the step is small enough.  

Note that the new time stepping method after a break point does not work
well with the old LTE calculation since the old LTE calculation is
conservative and it tends to reject the first time step out of a break
point. We recommend to use newlte if you choose to use the new time
stepping method out of a break point. & 1 (TRUE) \\ \hline

MASKIVARS & This parameter masks out current variables in the local truncation error (LTE) based time step
control. & 0 (FALSE) \\  \hline

ERROPTION & This parameter determines if Local Truncation Error (LTE)
control is turned on or not.  If \texttt{ERROPTION} is  on, then step-size
selection is based on the number of Newton iterations nonlinear solve.  
For Trapezoid and Gear, if the number of nonlinear
iterations is below \texttt{NLMIN} then the step is doubled.  If the number
of nonlinear iterations is above \texttt{NLMAX} then the step is cut by one
eighth.  In between, the step-size is left alone.  Because this option can
lead to very large time-steps, it is very important   to specify an appropriate
\texttt{DELMAX} option.  If the circuit has breakpoints, then the option
\texttt{MINTIMESTEPSBP} can also help to adjust the maximum time-step by
specifying the minimum number of time points between breakpoints. & 0 (Local Truncation Error is used)  \\ \hline

NLMIN &  This parameter determines the lower bound for the desired
number of nonlinear iterations during a Trapezoid time or Gear integration solve with
\texttt{ERROPTION}=1.
& 3  \\ \hline

NLMAX & This parameter determines the upper bound for the desired
number of nonlinear iterations during a Trapezoid time or Gear integration solve with
\texttt{ERROPTION}=1.
& 8  \\ \hline

DELMAX & This parameter determines the maximum time step-size used
with \texttt{ERROPTION}=1.  If a maximum time-step is also specified on the
\texttt{.TRAN} line, then the minimum of that value and \texttt{DELMAX} is
used.
& 1e99 \\ \hline

MINTIMESTEPSBP & This parameter determines the minimum number of
time-steps to use between breakpoints.  This enforces a maximum time-step
between breakpoints equal to the distance between the last breakpoint and the
next breakpoint divided by \texttt{MINTIMESTEPSBP}.
& 10  \\ \hline

TIMESTEPSREVERSAL & This parameter determines whether time-steps are
rejected based upon the step-size selection strategy in \texttt{ERROPTION}=1.
If it is set to 0, then a step will be accepted with successful nonlinear
solves independent of whether the number of nonlinear iterations is between
\texttt{NLMIN} and \texttt{NLMAX}.  If it is set to 1, then when the number of
nonlinear iterations is above \texttt{NLMAX}, the step will be rejected and the
step-size cut by one eighth and retried.  If \texttt{ERROPTION}=0 (use LTE) then
\texttt{TIMESTEPSREVERSAL}=1 (reject steps) is set.  
& 0 (do not reject steps) \\ \hline

DOUBLEDCOPSTEP \index{PDE Devices!time integration parameters} \index{TCAD Devices!time integration parameters} & 
TCAD devices by default will solve an extra "setup" problem to mitigate
some of the convergence problems that TCAD devices often exhibit.
This extra setup problem solves a nonlinear Poisson equation first to establish 
an initial guess for the full drift-diffusion(DD) problem.  The name of this 
parameter refers to the fact that the code is solving two DC operating point 
steps instead of one.  To solve only the nonlinear Poisson problem, then set 
\texttt{DOUBLEDCOP=nl\_poisson}.  To solve only the drift-diffusion problem 
(skipping the nonlinear Poisson), set \texttt{DOUBLEDCOP=drift\_diffusion}.
To explicitly set the default behavior, then set \texttt{DOUBLEDCOP=nl\_poisson, drift\_diffusion}.
& 
Default value, for TCAD circuits, is a combination: nl\_poisson, drift\_diffusion.
Default value, for non-TCAD circuits is a moot point.  If no TCAD devices are present
in the circuit, then there will not be an extra DCOP solve.
\\ \hline

BREAKPOINTS & This parameter specifies a comma-separated list of timepoints that 
should be used as breakpoints.  They do not replace the existing breakpoints 
already being set internally by Xyce, but instead will add to them.
& N/A  \\ \hline

\debug{BPENABLE}\index{\texttt{BPENABLE}} & \debug{Flag for
  turning on/off breakpoints (1 = ON, 0 = OFF).  It is unlikely anyone would
  ever set this to FALSE, except to help debug the breakpoint capability.}
& \debug{1 (TRUE)} \\ \hline

\debug{EXITTIME}\index{\texttt{EXITTIME}} & \debug{If this is set
  to nonzero, the code will check the simulation time at the end of each step.
  If the total time exceeds the exittime, the code will ungracefully exit.
  This is a debugging option, the point of which is the have the code stop at a
  certain time during a run without affecting the step size control.  If not
  set by the user, it isn't activated.}& \debug{-} \\ \hline

\debug{EXITSTEP}\index{\texttt{EXITSTEP}} &
\debug{Same as \texttt{EXITTIME}, only applied to step number.
The code will exit at the specified step.  If not set by the user,
it isn't activated.} &
\debug{-} \\ \hline

\index{solvers!time integration!options}
\end{OptionTable}
