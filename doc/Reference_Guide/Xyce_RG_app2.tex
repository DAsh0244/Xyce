% Sandia National Laboratories is a multimission laboratory managed and
% operated by National Technology & Engineering Solutions of Sandia, LLC, a
% wholly owned subsidiary of Honeywell International Inc., for the U.S.
% Department of Energy’s National Nuclear Security Administration under
% contract DE-NA0003525.

% Copyright 2002-2021 National Technology & Engineering Solutions of Sandia,
% LLC (NTESS).


%%-------------------------------------------------------------------------
%% Purpose        : Second Section LaTeX Xyce Reference Guide
%% Special Notes  : Graphic files (pdf format) work with pdflatex.  To use
%%                  LaTeX, we need to use postcript versions.  Not sure why.
%% Creator        : Scott A. Hutchinson, Computational Sciences, SNL
%% Creation Date  : {05/23/2002}
%%
%%-------------------------------------------------------------------------

\chapter{Command Line Arguments}
\label{cmd_line_args}
\index{command line!arguments}

\Xyce{} supports a handful of command line arguments which must be given {\em
  before} the netlist filename.  While most of these are intended for general
use, others simply give access to new features that, while supported, are not 
enabled by default.  The general usage is as follows:
\begin{vquote}
  Xyce [arguments] <netlist filename>
\end{vquote}

Table~\ref{cmd_line_arg_list} gives a list of supported command line
options.\footnote{Note that the ``-h'' option might list command line options not present 
in this table.  These extra options are generally deprecated and should not be used.  Only 
the options listed in the table are considered supported features.}

% Sandia National Laboratories is a multimission laboratory managed and
% operated by National Technology & Engineering Solutions of Sandia, LLC, a
% wholly owned subsidiary of Honeywell International Inc., for the U.S.
% Department of Energy’s National Nuclear Security Administration under
% contract DE-NA0003525.

% Copyright 2002-2020 National Technology & Engineering Solutions of Sandia,
% LLC (NTESS).

%%
%% Command line arguments table.
%%
\small
\begin{longtable}[htbp]
{|>{\setlength{\hsize}{0.5\hsize}}Y|
>{\setlength{\hsize}{1.0\hsize}}Y|
>{\setlength{\hsize}{0.75\hsize}}Y|
>{\setlength{\hsize}{0.55\hsize}}Y|} 

\caption [List of \Xyce{} command line arguments.]{List of \Xyce{} command line arguments. \label{cmd_line_arg_list}} \index{command line} \\
\hline

\rowcolor{XyceDarkBlue}{\color{white}\bf Argument} &
{\color{white}\bf Description} &
{\color{white}\bf Usage} &
{\color{white}\bf Default} \endhead \hline

-h &
Help option. Prints usage and exits. &
\verb+-h+ &
- \\ \hline

-v &
Prints the version banner and exits. &
\verb+-v+ &
- \\ \hline


-delim &
Set the output file field delimiter. &
%% have to break line to keep table neat
\verb+-delim+
\verb+<TAB|COMMA|string>+ &
- \\ \hline

-o &
Place the results into specified file. &
\verb+-o  <file>+ &
- \\ \hline

-l &
Place the log output into specified file. &
\verb+-l  <file>+ &
- \\ \hline

-r &
Output a binary rawfile. &
\verb+-r  <file>+ &
- \\ \hline

-a &
Use with \verb+-r+ to output a readable (ASCII) rawfile. &
\verb+-r  <file> -a+ &
- \\ \hline

-nox &
Use the NOX nonlinear solver. &
\verb+-nox  <ON|OFF>+ &
on \\ \hline

-linsolv &
Set the linear solver. &
%% have to break line to keep table neat
\verb+-linsolv+
\verb+<KLU|+
\verb+KSPARSE|SUPERLU|+
\verb+AZTECOO|BELOS>+ &
klu(serial) and aztecoo(parallel) \\ \hline

-param &
Print a terse summary of model parameters, device parameters and default values. &
\verb+-param+ &
- \\ \hline

-syntax &
Check netlist syntax and exit.&
\verb+-syntax+ &
- \\ \hline

-norun &
Check netlist syntax and topology and exit. &
\verb+-norun+ &
- \\ \hline

-quiet &
Suppress some of the simulation-progress messages
sent to stdout for transient simulations. &
\verb+-quiet+ &
- \\ \hline

-namesfile &
Output a list of all solution variables generated by the netlist
into \tt{<filename>} &
%% have to break line to keep table neat
\verb+-namesfile+
\verb+<filename>+ &
- \\ \hline

-randseed &
Set random number seed for expression library's random number functions
 and also \texttt{.SAMPLING} analysis. &
\verb+-randseed <number>+ &
If not provided, Xyce will select a seed using the system ``time'' function.  \\ \hline

-maxord &
Maximum time integration order. &
\verb+-maxord <1..5>+ &
- \\ \hline

-jacobian\_test &
Jacobian matrix diagnostic. &
\verb+-jacobian_test+ &
- \\ \hline
\end{longtable}



A few other command line options are available that are typically only used in Xyce development.  
For example the options {\tt -param}, {\tt -info}, {\tt -doc} and {\tt -doc\_cat} are 
used to generate the device tables in this guide.  The options {\tt -jacobian\_test} 
and {\tt -namesfile} can be useful in debugging new devices in \Xyce{}.  The option
 {\tt -namesfile} is also useful for determining the ``fully qualified node
names'', including the subcircuit hierarchy, for nodes and internal variables for mutual
inductors.  The {\tt .PRINT} section \ref{.PRINT} has more information on, and examples for,
the {\tt -namesfile} command line option.

\chapter{Runtime Environment}
\label{runtime}

\section{Running Xyce in Serial}

After ensuring that the directory into which \Xyce{} was installed is
in your PATH variable, one merely executes the code by running the
command, \texttt{Xyce} with the desired netlist name appended.

\section{Running Xyce in Parallel}

Open MPI must be installed on the host machine.  It may be download from 

\texttt{http://www.open-mpi.org/}.  Consult the documentation for help with installation.

After ensuring that the both the directory into which \Xyce{} was
installed and the directory in which mpirun is found are in your PATH
variable, one merely executes the code by running the command,
\texttt{mpirun [mpirun options] Xyce [xyce options]} with the desired
netlist name appended.

\section{Running Xyce on Sandia HPC and CEE Platforms}



This version of \Xyce{} has been installed centrally on Sandia HPC
and CEE platforms, and requires metagroup access.  Contact the \Xyce{} team
for details on how to obtain this access.

Once you have registered for metagroup membership, the central
installs of \Xyce{} may be accessed by a module load.

\verb+module load Xyce+ adds all required modules and sets all
required environment variables to access the normal version of
\Xyce{}.  \verb+module load XyceRad+ does the same thing for the
version \Xyce{} containing Sandia proprietary models.

\verb+module help Xyce+ provides some additional information about
what the module does.

Consult the system documentation for help with submitting jobs on
these platforms.

\verb+https://computing.sandia.gov/+

\chapter{Setting Convergence Parameters for Xyce}
\label{Conv_Guidelines}
\index{convergence}\index{\Xyce{}!convergence}\index{parameters!convergence}

Because the solution algorithms and methods within \Xyce{} are different than
those used by other circuit simulation tools (e.g., SPICE), the overall
convergence behavior is sometimes different, as are the parameters which control
this behavior.

\section{Adjusting Transient Analysis Error Tolerances}
\index{transient analysis!error tolerances}\index{solvers!time
  integration!options}

\Xyce{} uses a variable order trapezoid integration as its default
scheme, and this method may also be requested explicitly with the
\textrmb{TIMEINT} option \texttt{METHOD=trap} or
\texttt{METHOD=7}. Trapezoid time-stepping is second order accurate
and does not have any numerical dissipation in its local truncation
error. Variable order trapezoid integration dynamically uses Backward
Euler (BE) and trapezoid rule. When \texttt{ERROPTION=1} is set with
\texttt{METHOD=7}, trapezoid rule is used almost exclusively (BE
only used at breakpoints). See table~\ref{TimeIntPKG} for details.

Another time integration option is the second-order Gear method.  It
may be selected with the \textrmb{TIMEINT} option \texttt{METHOD=gear} or
\texttt{METHOD=8}. See table~\ref{TimeIntPKG} for details.

\subsection{Setting \textrmb{RELTOL} and \textrmb{ABSTOL}}
\index{\Xyce{}!\texttt{RELTOL}}\index{\Xyce{}!\texttt{ABSTOL}}
In \Xyce{}, both the time integration package and the nonlinear solver package
have \textrmb{RELTOL} and \textrmb{ABSTOL} settings.  Some general guidelines for
settings parameters are~\cite{Petzold:1996}:
\begin{XyceItemize}
\item Use the \emph{same} \textrmb{RELTOL} and \textrmb{ABSTOL} values for both
  the \textrmb{TIMEINT} and the \\ \textrmb{NONLIN-TRAN} \textrmb{.OPTIONS}
  statements.
\item For a conservative approach (i.e., safe), set \texttt{RELTOL=1.0E-(\emph{m}+1)}
  where \texttt{\emph{m}} is the desired number of significant digits of
  accuracy.
\item Set \textrmb{ABSTOL} to the smallest value at which the solution
  components (either voltage or current) are essentially insignificant.
\item Note that the above suggests that $\textrmb{ABSTOL}<\textrmb{RELTOL}$.
\end{XyceItemize}

The current defaults for these parameters are \texttt{ABSTOL=1.0E-6} and
\texttt{RELTOL = 1.0E-3}.  For a complete list of the time integration
parameters, see chapter ~\ref{Netlist_Commands}.

\section{Adjusting Nonlinear Solver Parameters (in transient mode)}
\index{solvers!nonlinear!options}

In \Xyce{}, the nonlinear solver options for transient analysis are set using
the \texttt{.OPTIONS NONLIN-TRAN} line in a netlist.  This subsection gives
some guidelines for setting this parameters.
\begin{XyceItemize}
\item For guidelines on setting \textrmb{RELTOL} and \textrmb{ABSTOL}, see above.
\item \textrmb{RHSTOL} -- This is the maximum residual error for each nonlinear
  solution.  \Xyce{} uses this as a ``safety'' check on nonlinear
  convergence.  Typically, \texttt{1.0E-2} (the default) works well.
\item \textrmb{DELTAXTOL} -- This is the weighted update norm tolerance and is
  the primary check for nonlinear convergence.  Since it is weighted (i.e.,
  normalized using \textrmb{RELTOL} and \mbox{\textrmb{ABSTOL}}), a value of
  \texttt{1.0} would give it the same accuracy as the time integrator.  For
  robustness, the default is \texttt{0.33} but sometimes a value of
  \texttt{0.1} may help prevent ``time-step too small'' errors.  A value of
  \texttt{0.01} is considered quite small.
\item \textrmb{MAXSTEP} -- This is the maximum number of Newton (nonlinear)
  steps for each nonlinear solve.  In transient analysis, the default is
  \texttt{20} but can be increased to help prevent ``time-step too small''
  errors.  This is roughly equivalent to \textrmb{ITL4} in SPICE.
\end{XyceItemize}

\chapter{Quick Reference for Users of Other SPICE 
Circuit Simulators}
\label{PSpice_Ref}
\index{PSpice}
\index{Users of PSpice}

This chapter describes many of the differences between \Xyce{} and
other SPICE-like circuit simulaters.  The primary focus is on the difference
between Orcad PSpice and \Xyce{}, with an eye towards providing the ability for those
familiar with using PSpice to begin using \Xyce{} quickly.  

This chapter is likely not complete, and \Xyce{} users might also consult
specific sections of this Reference Guide about particular \Xyce{} commands.
Those sections may have additional information on Xyce's incompatibilities with
other circuit simulators, and how to work around them.  

\section{Differences Between Xyce and PSpice}
This section is focused on the differences between \Xyce{} and PSpice.
However, some of this discussion also applies to other SPICE-like
circuit simulators. 

\subsection{Command Line Options}
Command line arguments are supported in \Xyce{} but they are different than
those of PSpice.   For a complete reference, see Chapter~\ref{cmd_line_args}.

\subsection{Device Support}
Most, but not all, devices commonly found in other
circuit simulation tools are supported.  \Xyce{} also contains enhanced versions of
many semiconductor devices that simulate various environmental effects.  For the
complete list, please see the
Analog Device Summary in Table~\ref{Device_Summary}.

\subsection{ .OPTIONS Support}
For the specific devices or models that are supported in \Xyce{}, most of the
standard netlist inputs are the same as those in standard SPICE.
However, the \textrmb{.OPTIONS} command has several additional features used to
expose capabilities specific to \Xyce{}.  In particular, \Xyce{} does not
support the standard PSpice format \textrmb{.OPTIONS} line in netlists.
Instead, options for each supported package are called according to the
following format.

\begin{Command}
\format
.OPTIONS <pkg> [<tag>=<value>]*

\arguments

\begin{Arguments}
\argument{DEVICE}       Device Model
\argument{TIMEINT}      Time Integration
\argument{NONLIN}       Nonlinear Solver
\argument{NONLIN-TRAN}  Transient Nonlinear Solver
\argument{NONLIN-HB}    HB Nonlinear Solver
\argument{LOCA}         Continuation/Bifurcation Tracking
\argument{LINSOL}       Linear Solver
\argument{LINSOL-HB}    HB Linear Solver
\argument{OUTPUT}       Output
\argument{RESTART}      Restart
% \argument{MPDEINT}      MPDE
\argument{HBINT}        Harmonic Balance (HB)
\argument{SENSITIVITY}  Direct and Adjoint sensitivity analysis
\end{Arguments}

\end{Command}

\index{\texttt{.OPTIONS}}

For a complete description of the supported options in \Xyce{}, see
section~\ref{Options_Reference}.  

Known caveats are that the \texttt{ABSTOL} options have different definitions 
in PSpice and \Xyce{}.  Also, a PSpice \texttt{.OPTIONS VNTOL=<value>} line can 
be mapped into these two \Xyce{} lines:
\begin{vquote}
.OPTIONS NONLIN ABSTOL=<value> 
.OPTIONS NONLIN_TRAN ABSTOL=<value>
\end{vquote}

The PSpice \texttt{ITL1} and \texttt{ITL4} options are similar to the \Xyce{} \texttt{MAXSTEPS}.
In PSpice, \texttt{ITL1} affects \texttt{.DC} analyses, while \texttt{ITL4} 
affects \texttt{.TRAN} analyses.  In \Xyce{}, \texttt{.OPTIONS NONLIN}
refers to options for \texttt{.DC} analyses, while \texttt{.OPTIONS NONLIN-TRAN}
refers to options for \texttt{.TRAN} analyses.  So, a feasible mapping is 
PSpice \texttt{.OPTIONS ITL1=20} becomes \texttt{.OPTIONS NONLIN MAXSTEP=20}
in \Xyce{}.  Similarly, PSpice \texttt{.OPTIONS ITL4=20} becomes 
\texttt{.OPTIONS NONLIN-TRAN MAXSTEP=20} in \Xyce{}.  However, given that
PSpice and Xyce{} use different default values for \texttt{ITL1} and \texttt{ITL4} 
vs. \texttt{MAXSTEPS}, the best approach may be to not translate the \texttt{ITL1} and 
\texttt{ITL4} lines into the corresponding \Xyce{} netlist.

\subsection{.PROBE vs. .PRINT}
\Xyce{} does not support the ``\textrmb{.PROBE}'' statement.  Output of Probe-format
files, in .csd format, that are readable by PSpice is done using the \textrmb{.PRINT} netlist statement.
See section~\ref{.PRINT} for the syntax for \texttt{FORMAT=PROBE}.  That section
also describes wildcard support and access to subcircuit nodes in \Xyce{}, both of
which are different than PSpice.

\Xyce{} does not support PSpice style abbreviations in the \textrmb{.PRINT}
statement.  For example, to print out the value of the voltage at node
\texttt{A} in a transient simulation you must request \mbox{\texttt{.PRINT TRAN V(A)}},
not \mbox{\texttt{.PRINT TRAN A}}.  \Xyce{} also does not support \texttt{N()} as a
synonym for \texttt{V()} on \texttt{.PRINT} lines.

\subsection{Converting PSpice ABM Models for Use in Xyce}
\label{Converting_ABM}
\index{device!ABM device!PSpice equivalent}

\Xyce{} is almost fully compatible with PSpice with respect to analog
behavioral models.  This includes the
\texttt{E}\index{PSpice!\texttt{E} device},
\texttt{F}\index{PSpice!\texttt{F} device},
\texttt{G}\index{PSpice!\texttt{G} device}, and
\texttt{H}\index{PSpice!\texttt{H} device} 
device types.  A notable exception to this compatibility is in the use of lead and device
currents in expressions used in controlled source definitions.  That feature is not 
supported in \Xyce{}.  In addition, the \texttt{FREQ},\texttt{LAPLACE} and 
\texttt{CHEBYSHEV} forms for E and G sources or the \texttt{ERROR} qualifier are 
not supported in \Xyce{}..
  
\subsection{Usage of \textrmb{.STEP} Analysis}

The implementation of \textrmb{.STEP} in \Xyce{} is not the
same as that of PSpice.  See section~\ref{.STEP} for the syntax and
function of the \textrmb{.STEP} function in \Xyce{}.

\subsubsection{Model Parameter Sweeps}

PSpice requires extra keywords to apply a \textrmb{.STEP} statement to a
model parameter.  \Xyce{} handles model parameters differently, and is
actually somewhat more flexible than PSpice.  Unfortunately, this means
that the two specifications are not compatible.

A model parameter in PSpice would be handled like this:
\begin{vquote}
R1 1 2 RMOD 1
.model RMOD RES(R=30)
.step RES RMOD(R) 30 50 5
\end{vquote}
The equivalent way to specify this in \Xyce{} would be:
\begin{vquote}
R1 1 2 RMOD 1
.model RMOD RES(R=30)
.step RMOD:R 30 50 5
\end{vquote}
Note that \Xyce{} does not require the \texttt{RES} keyword on the
\texttt{.STEP} line.  In PSpice, this keyword is needed to specify 
what type of model is being used.  \Xyce{} actually has more flexibility
than PSpice in this regard---any model or instance variable can be set on
the \texttt{.STEP} line using the same syntax.

\Example{\texttt{.step D101:IS 1.0e-3 5.0e-3 1.0e-3}}

In this example, \texttt{D101} is the name of a model or instance, 
and \texttt{IS} is the name of the parameter within that model or instance.

\subsection{Behavioral Digital Devices}
There are at least four significant differences.  First, the instance line
syntax for the \Xyce{} digital behavioral devices differs from PSpice.
Second,  \Xyce{} uses one model card for the timing and Input/Output (I/O)
characteristics, while PSpice uses separate model cards for timing and I/O
characteristics.  The model cards also have different parameters. 
Third, \Xyce{} does support the \texttt{DIGINITSTATE} option. However, it has a
different default value than in PSpice.  So, the DCOP calculations for flip-flops and latches may
be different in some cases between \Xyce{} and PSpice.  Finally, closely spaced input
transitions to a gate (e.g., ones spaced by less than the \texttt{DELAY}
parameter of the \Xyce{} model) may produce different behaviors in \Xyce{}
and PSpice.  Please consult Section \ref{U_DEVICE} for more details.

\subsection{Power Dissipation}
PSpice supports printing the power dissipation of a device via syntax like
\texttt{W(<name>)}.  At this time, not all \Xyce{} devices support power calculations. 
In addition, the \Xyce{} results for the FET semiconductor devices (J, M and Z devices) may 
differ from the PSpice results.  Consult the Features Supported by Xyce Device Models table
in Section \ref{Analog_Devices} and the individual sections on each device for more details.  
Additional limitations on lead current and power
calculations in \Xyce{} are given in Section \ref{leadCurrentPowerCalculations}.

Example work-arounds are as follows, using either the
node voltage at Node 2 or the lead current through Resistor 2:

\begin{vquote}
.DC V1 0 5 1
.param R2VAL=10
V1 1 0 5V
R1 1 2 10
R2 2 0 \{R2VAL\}
.PRINT DC V(2) \{V(2)*V(2)/R2VAL\} \{I(R2)*I(R2)*R2VAL\}
\end{vquote}

\subsection{Dependent Sources with TABLE Syntax}
\label{DS_TABLE_SYNTAX_DIFF}
The documented PSpice syntax for the \texttt{TABLE} form of the E and G sources
is identical to the \Xyce{} syntax for those two devices.  As an example, consider 
this E-source netlist line which conforms to the documented PSpice and Xyce 
syntaxes:
\begin{vquote}
E5  5 0 TABLE {V(1,0)} = (-2,-3) (2,3)
\end{vquote}
There is an equal sign between the expression \texttt{\{V(1,0)\}} and the list of
value pairs (e.g., before \texttt{(-2,-3)}).  There is also a comma between the two values
in each set of value pairs.  However, it has been observed that some PSpice versions 
will accept variants of the documented PSpice syntax.  As examples, PSpice might 
use this \texttt{TABLE} syntax, where the equal sign between the expression and the 
list of value pairs is missing and there is an extra set of parentheses around 
the list of value pairs:
\begin{vquote}
TABLE \{EXPR\} ((x1,y1) (x2,y2) \ldots (xn, yn))
\end{vquote}
PSpice might also specify the \texttt{TABLE} syntax without the commas between the 
two values in each set of value-pairs. For example, this is a legal syntax in some 
PSpice versions:
\begin{vquote}
TABLE \{EXPR\} = (x1 y1) (x2 y2) \ldots (xn yn)
\end{vquote}
So, the generic solution is to change these alternative PSpice syntaxes (and possibly 
others) to conform with the \Xyce{} E and G source \texttt{TABLE} syntax, which is
(see also Sections \ref{E_DEVICE} and \ref{G_DEVICE}):
\begin{vquote}
TABLE \{EXPR\} = (x1,y1) (x2,y2) \ldots (xn, yn)
\end{vquote}

\subsection{MODEL STATEMENTS}
\label{MODEL_STATEMENT_SYNTAX_DIFF}
In PSpice, some \texttt{.MODEL} statements may have commas separating the list
of parameters, which causes problems in \Xyce{}.  A simple workaround is to
replace those commas with spaces in the corresponding \Xyce{} \texttt{.MODEL}
statements.

In PSpice, some \texttt{.MODEL} statements may not have parentheses
surrounding the list of parameters.  While Xyce also does not require parentheses
in model cards, parentheses are accepted.  The only \Xyce{} requirement is that if
they are used then they must be paired with one left parenthesis before all of the
parameters and one right parentheses after all of the parameters. It is an error
to have unmatched parentheses.  

PSpice syntaxes where only a subset of the model 
parameters are enclosed within parentheses are also not supported in \Xyce{}. 
A PSpice example is:
\begin{vquote}
.model somebjt NPN Is=1e-16 (Xti=3 Bf=100) Eg=1.11 NC=2
\end{vquote}

Nested parentheses, as is often seen when a \texttt{DEV} (deviation) is specified 
for a parameter in a PSpice model statement, are also not allowed in \Xyce{}.  A PSpice
example is:
\begin{vquote}
.model someotherbjt NPN(Is=1e-16 Xti=3 (Bf=100 DEV=5\%) Eg=1.11 NC=2)
\end{vquote}

The previous PSpice example also raises the issue of model parameters that are supported
in PSpice but not in \Xyce{}.  It that case, \Xyce{} will issue a warning about the
invalid parameter and the simulation will run.

Another common issue is a PSpice model parameter (e.g., \texttt{BV=}) without a value.  
That PSpice syntax error is often silently ignored in PSpice, but flagged as a parsing 
error in a \Xyce{} netlist.  

Temperature coefficient (\texttt{TC}) specifications can be a problem also.  The 
documented PSpice syntax is this, with a comma between the two values.  

\begin{vquote}
TC=0.1,0.1
\end{vquote}

However, it has been observed that some PSpice versions allow the TC parameter to 
omit the comma between those two values.  That is not legal in \Xyce{}.

\subsection{.NODESET and .IC Statements}
\Xyce{} and PSpice differ in their capabilities to handle \texttt{.NODESET} and 
\texttt{.IC} statements in subcircuits.  See sections ~\ref{NODESET_section} 
and ~\ref{IC_section} for more details.

\subsection{Piecewise Linear Sources}
\label{PWL_SOURCE_SYNTAX_DIFF}
The preferred \Xyce{} syntax for PWL sources does not use parentheses or commas within
the time-voltage pair listing.  See Section \ref{I_DEVICE} for more details.

The \Xyce{} PWL source does not support the PSpice \texttt{.IN} format for file input.  
See Section \ref{I_DEVICE} for the ASCII text and \texttt{.csv} formats supported 
by \Xyce{} for file input.

The \Xyce{} repeat \texttt{R=<value>} syntax for PWL sources is not compatible with 
the PSpice \texttt{REPEAT} syntax for PWL sources.  Some work-arounds are as follows.
This PSpice \texttt{REPEAT FOREVER} syntax:
\begin{vquote}
VPWL1 1 0 PWL REPEAT FOREVER (0,0) (0.5,1) (1,0)
+ ENDREPEAT
\end{vquote}
is equivalent to this \Xyce{} syntax:
\begin{vquote}
VPWL1 1 0 PWL 0 0 0.5 1 1 0 R=0
\end{vquote}
Similarly, if the PSpice source has its time-voltage pairs in a \texttt{.csv}
file, and the specified waveform starts at time=0, then this PSpice syntax:
\begin{vquote}
VPWL2 2 0 PWL 
+ REPEAT FOREVER
+ FILE "data.csv"
+ ENDREPEAT
\end{vquote}
is equivalent to this \Xyce{} syntax:
\begin{vquote}
VPWL2 2 0 PWL file "data.csv" R=0
\end{vquote}
For more general PSpice \texttt{REPEAT} syntaxes, and especially for the PSpice
\texttt{REPEAT for N} syntax, the user might have to manually duplicate the 
PSpice waveform in a \texttt{.csv} file.

\subsection{.AC Output}
The \Xyce{} .csd file format for a \texttt{.AC} analysis is different than the
PSpice format, but is still viewable in the PSpice A/D waveform viewer.  This
PSpice \texttt{.PROBE} statement:
\begin{vquote}
.PROBE/CSDF V([1b]) VR([1b]) VI([1b])
\end{vquote}
will produce \#N and \#C lines in its netlistName.csd file like this, where the
real and imaginary parts of V(1b) are output for each data point on the \#C line.
The end-user can then use the PSpice A/D UI to choose to plot the VR and VI
quantities.
\begin{vquote}
#N
`V(1b)' `V(1b)' `V(1b)'
#C 1.0000000000E01 3
2.470E-02/-1.552E-01:1 2.470E-02/-1.552E-01:2 2.470E-02/-1.552E-01:3
\end{vquote}

This corresponding \Xyce{} .PRINT AC statement:
\begin{vquote}
.PRINT AC FORMAT=PROBE V(1b) VR(1b) VI(1b)
\end{vquote}
will produce \#N and \#C lines in its netlistName.csd file like this, where the real
and imaginary parts of V(1b) are still output on the \#C line.  However, in \Xyce{},
the VR() and VI() operators return real-valued quantities as shown below.  This
\Xyce{} formatted file is still viewable in PSpice A/D.
\begin{vquote}
#N
`V(1b)' `VR(1b)' `VI(1b)'
#C 1.0000000000E01 3
2.470e-02/-1.552e-01:1 2.470e-02/0.000e+00:2 -1.552e-01/0.000e+00:3
\end{vquote}

\subsection{Additional differences}
Some other differences between \Xyce{} and PSpice are described in 
Table~\ref{Incompat_PS}.  Users should also consult 
Table~\ref{Incompat_Other_SPICE}, since that table lists more general
incompatibilities that span multiple circuit simulators.

% Sandia National Laboratories is a multimission laboratory managed and
% operated by National Technology & Engineering Solutions of Sandia, LLC, a
% wholly owned subsidiary of Honeywell International Inc., for the U.S.
% Department of Energy’s National Nuclear Security Administration under
% contract DE-NA0003525.

% Copyright 2002-2019 National Technology & Engineering Solutions of Sandia,
% LLC (NTESS).


%%
%% Differences between PSpice and Xyce table.
%%

\index{PSpice}
\begin{longtable}[h] {>{\raggedright\small}m{2in}|>{\raggedright\let\\\tabularnewline\small}m{4in}}
  \caption{Incompatibilities with PSpice.} \\ \hline
  \rowcolor{XyceDarkBlue}
  \color{white}\bf Issue & 
  \color{white}\bf Comment \\ \hline \endfirsthead
  \label{Incompat_PS}

\texttt{.VECTOR}, \texttt{.WATCH}, and \texttt{.PLOT}  output 
control analysis are not supported. & \Xyce{} does 
not support these commands.  \\ \hline


\texttt{.PZ} analysis is not supported. & \Xyce{} does not support this command.  
\\ \hline


\texttt{.DISTO} analysis is not supported. & \Xyce{} does not support this command.  
\\ \hline

\texttt{.TF} analysis is not supported. & \Xyce{} does not support this command.  
\\ \hline

\texttt{.AUTOCONVERGE} is not supported. & \Xyce{} does not support this command.  
\\ \hline

\texttt{.SENS} analysis is supported, but has a different syntax than PSpice. & The \Xyce{} version of \texttt{.SENS} requires that the user specify exactly which parameters are the subject of the sensitivity analysis.  Additionally, \Xyce{} can compute sensitivities in transient as well as the \texttt{.DC} case (unlike PSpice).
\\ \hline

\texttt{.NOISE} analysis is supported, but not all devices supported. & The \Xyce{} version of \texttt{.NOISE} is new enough that not all noise models have been implemented.
\\ \hline

\texttt{.MC} and \texttt{.WCASE} statistical analyses are not supported.  
& \Xyce{} does not support these commands.  
\\ \hline

\texttt{.DISTRIBUTION}, which defines a user distribution for tolerances, is not supported.  
& \Xyce{} does not support this command.  This command goes along with 
\texttt{.MC} and \texttt{.WCASE} statistical analyses, which are also not supported. \\ \hline

\texttt{.LOADBIAS} and \texttt{.SAVEBIAS} initial condition commands are not supported.  
& \Xyce{} does not support these commands.  \\ \hline

\texttt{.ALIASES}, \texttt{.ENDALIASES}, are not supported.
& \Xyce{} does not support these commands.  \\ \hline

\texttt{.STIMULUS} is not supported.  & \Xyce{} does not support this command.  \\ \hline

\texttt{.TEXT} is not supported.  & \Xyce{} does not support this command.  \\ \hline

\texttt{.PROBE} does not work & \Xyce{} does not support this.  Use the 
\texttt{FORMAT=PROBE} option of .PRINT instead.  See 
section~\ref{.PRINT} for syntax.\\ \hline

\texttt{.OP} only produces output in serial & .OP is supported in \Xyce{}, but will not
produce the extra output normally associated with the .OP statement, if running a parallel build.\\ \hline

Pulsed source rise time of zero & A requested pulsed source rise/fall time of
zero really is zero in \Xyce{}.  In other simulators, requesting a zero
rise/fall time causes them to use the printing interval found on the tran
line.\\ \hline

Mutual Inductor Model & Not the same as PSpice.  This is a Sandia developed
model. \\ \hline

\texttt{.PRINT} line shorthand & Output variables have to be specified as a
V(node) or I(source). Listing the node alone will not work. \\ \hline

BSIM3 level & In \Xyce{} the BSIM3 level=9.  In PSpice the BSIM3 is
level=7. \\ \hline

Interactive mode & \Xyce{} does not have an interactive mode.  \\ \hline

Time integrator default tolerances & \Xyce{} has much tighter default solver
tolerances than some other simulators (e.g., PSpice), and thus often takes
smaller time steps.  As a result, it will often take a greater number of total
time steps for a given time interval.  To have \Xyce{} take time steps
comparable to those of PSpice, set the \texttt{RELTOL} and \texttt{ABSTOL} time
integrator options to larger values (e.g., \texttt{RELTOL=1.0E-2, ABSTOL=1.0E-6}).
\\ \hline

{\tt.OPTIONS} statements \index{\texttt{.OPTIONS}} & \Xyce{} does 
not support PSpice style
\texttt{.OPTION} statements. In \Xyce{}, the various packages all (potentially)
have their own separate \texttt{.OPTIONS} line in the netlist.  For a complete
description, see section~\ref{Options_Reference}.  \\ \hline

\texttt{DTMAX} & \Xyce{} does support a maximum time step-size
control on the .tran line, but we discourage its use. The time 
integration\index{solvers!time integration}
\index{algorithm!time integration} algorithms within
\Xyce{} use adaptive time-stepping methods that adjust the time-step
size\index{time step!size} according to the activity in the analysis.  If the
simulator is not providing enough accuracy, the \texttt{RELTOL} and
\texttt{ABSTOL} parameters should be decreased for both the time integration
package (\texttt{.OPTIONS TIMEINT}) and the transient nonlinear solver package
(\texttt{.OPTIONS NONLIN-TRAN}).  We have found that in most cases specifying 
the same maximum timestep that PSpice requires for convergence actually 
slows \Xyce{} down by preventing it from taking larger timesteps when the 
behavior warrants.  \\ \hline

%Analog Behavioral Models & PSpice supports analog behavioral
%modeling in the same way as PSpice through the E, F, G and H devices,
%but \Xyce{} supports only the general ``B'' source.  \Xyce{}'s E and G
%sources are purely SPICE 3F5 compatible linear sources.  For a
%complete description, including rules for conversion to the \Xyce{}
%equivalent, see section~\ref{PSpice_Ref}.  \\ \hline

\texttt{.TRAN} ``\texttt{UIC}'' keyword & PSpice requires the use
of a keyword \texttt{UIC} on the \texttt{.TRAN} line in order to use
initial conditions via \texttt{IC} keywords on instance lines.  Doing so 
also tells PSpice not to perform an operating point calculation. In
\Xyce{}, \texttt{UIC} is ignored and produces a warning message.  \Xyce{} 
always uses initial conditions specified with
\texttt{IC} keywords, and the case of inductors and capacitors automatically 
inserts a fictitious voltage source around the device that guarantees 
the correct potential drop across the device during the operating point.  
If the user desires that \Xyce{} not perform an operating point calculation, 
but rather use an initial condition for a transient run of all zero 
voltages, then the user should specify \texttt{NOOP} instead. \\ \hline

Temperature specification & Device temperatures in \Xyce{} are 
specified through the \texttt{.OPTIONS DEVICE} line.  PSpice 
allows a \texttt{.TEMP} line that is not recognized (and is ignored) 
by \Xyce{}. \\ \hline

Lead currents for lossless transmission lines & PSpice uses \texttt{A} and
\texttt{B} to reference the two terminals of the lossless tranmission line.  
So, \Xyce{} uses \texttt{I1()} and \texttt{I2()}, while PSpice uses \texttt{IA()} 
and \texttt{IB()} to access the lead currents for the device. \\ \hline

Extended ASCII characters in \texttt{.LIB} files & The use of those characters 
is fine in \Xyce{} comment lines.  It may be best to replace them with the 
printable equivalent on other \Xyce{} netlist lines though. 
\end{longtable}


%%% Local Variables:
%%% mode: latex
%%% End:


\subsection{Translating Between PSpice and Xyce Netlists}
Some internal Sandia users have found the following checklist to be helpful
in getting their PSpice netlists to run in \Xyce{}.  Additional changes may
be needed in some cases.

For the .cir file:

\begin{XyceItemize}
\item Change .LIB references to point to the modified libraries generated
   for use with \Xyce{}.
\item Change \texttt{PROBE} and \texttt{PROBE64} statements to \texttt{PRINT <Sim Type>}
\item Find cases where the PSpice netlist used \texttt{N()} rather
   than \texttt{V()}.
\item \texttt{.DC} has the keyword \texttt{PARAM} in PSpice.  If it
   exists then remove it in the \Xyce{} netlist.
\item \texttt{.OPTIONS TNOM=X} is changed to \texttt{.OPTIONS
   DEVICE TNOM=X} in the \Xyce{} netlist.
\item \texttt{.TEMP args} does not exist in \Xyce{}.  The equivalent
   \Xyce{} statement is \texttt{.STEP TEMP LIST args}
\item The default time integrator tolerances can make \Xyce{} take
   smaller timesteps on some circuits, and therefore have slower
   simulation times.  The \Xyce{} timesteps can be increased at the
   expense of time integration accuracy by loosening the
   integrator tolerances.  Some users find that  \texttt{.OPTIONS 
   TIMEINT RELTOL=1e-2 ABSTOL=1e-4} leads to time steps more like PSpice's.
\item Move any \texttt{.IC} and \texttt{.NODESET} statements to the top-level,
   and use the fully qualified node names in those statements.
\item Adjust the syntax for any PWL sources, if needed, per 
   Section \ref{PWL_SOURCE_SYNTAX_DIFF}.
\end{XyceItemize}

For the .lib file:
\begin{XyceItemize}
\item Add \texttt{LEVEL=2} parameter to diode models.
\item Fix the parentheses and comma differences between PSpice and 
   \Xyce \texttt{.MODEL} statements per Section \ref{MODEL_STATEMENT_SYNTAX_DIFF}.
\item Find and modify any nested expression statements.  This may 
   entail replacing ``\{'' with ``('' in the expression in the \Xyce{} netlist.
\item Fix the table syntax for dependent sources, as discussed in
Section \ref{DS_TABLE_SYNTAX_DIFF}.

\end{XyceItemize}

\section{Differences Between Xyce and Other SPICE Simulators}
This section covers some known differences between \Xyce{} and other SPICE-like
circuit simulators, besides PSpice, as listed in Table~\ref{Incompat_Other_SPICE}.
However, users of those other simulators (e.g., SPICE3F5, HSPICE, ngspice, ...)
should also check the previous subsection on PSpice, since some of that
discussion also applies here.

% Sandia National Laboratories is a multimission laboratory managed and
% operated by National Technology & Engineering Solutions of Sandia, LLC, a
% wholly owned subsidiary of Honeywell International Inc., for the U.S.
% Department of Energy’s National Nuclear Security Administration under
% contract DE-NA0003525.

% Copyright 2002-2020 National Technology & Engineering Solutions of Sandia,
% LLC (NTESS).


%%
%% Incompatibilities with other circuit simulators.
%%
\index{PSpice}
\begin{longtable}[h] {>{\raggedright\small}m{2in}|>{\raggedright\let\\\tabularnewline\small}m{3.5in}}
  \caption{Incompatibilities with Other Circuit Simulators.} \\ \hline
  \rowcolor{XyceDarkBlue}
  \color{white}\bf Issue &
  \color{white}\bf Comment \\ \hline \endfirsthead
  \label{Incompat_Other_SPICE}

    \texttt{.DC} sweep output. & The \texttt{.DC} sweep calculation does not
    automatically output the sweep variable.  Only variables explicitly listed on
    the \texttt{.PRINT} line are output.\\ \hline

    MOSFET levels. & In \Xyce{} the MOSFET levels are not the same. In \Xyce{},
    a BSIM3 is MOSFET level 9.  Other simulators have different levels for the 
    BSIM3. \\ \hline

    BSIM SOI v3.2 level. & In \Xyce{} the BSIM SOI (v3.2) is MOSFET level 10.  
    Other simulators have different levels for the BSIM SOI. \\ \hline

    BSIM4 level. & In \Xyce{} the BSIM4 is MOSFET levels 14 and 54.  
    Other simulators have different levels for the BSIM4. \\ \hline

    Syntax for \texttt{.STEP} is different. & The manner of specifying
    a model parameter to be swept is slightly different than in some
    other simulators.  See the \Xyce{} Users' and Reference Guides for
    details.  \\ \hline

    Switch is not the same as SPICE3F5. &  The \Xyce{} switches are not
    compatible with the simple switch implementation in SPICE3F5.  The
    switch in \Xyce{} smoothly transitions between the ON and OFF
    resistances over a small range between the ON and OFF values of the
    control signal (voltage, current, or control expression).  See the
    \Xyce{} Reference Guide for the precise equations that are used to compute the
    switch resistance from the control signal values.  The SPICE3F5 switch
    has a single switching threshold voltage or current, and RON is used
    above threshold while ROFF is used below threshold.  \Xyce{}'s switch is
    considerably less likely to cause transient simulation failures. Results
    similar to SPICE3F5 can be obtained by setting VON and VOFF to the same
    threshold value, but this is not a recommended practice.  \\ \hline

    Piecewise Linear (PWL) source not fully compatible with either HSPICE
    or PSpice. & See Sections \ref{I_DEVICE} and \ref{PWL_SOURCE_SYNTAX_DIFF} 
    of the \Xyce{} Reference Guide for more details. \\ \hline

    Acceptable prefixes in the metric system. &
    The ``atto'' prefix, which is designated by ``a'', is acceptable in HSPICE,
    but is not accepted in \Xyce{}. 
    The use of the ``atto'' prefix in \Xyce{} must be replaced with ``E-18''. \\ \hline

    Hierarchical parameters. & In \Xyce{} hierarchical parameters, \texttt{M} (multiplier
    or multiplicity factor) and \texttt{S} (scale), are not commonly supported.  The \texttt{M} 
    parameter is only supported by the R, L, C and MOSFET device models and some BJT 
    device models (VBIC 1.3 and MEXTRAM).\\ \hline

    \texttt{.MEASURE} has some incompatibilities and differences with HSPICE. &
    See Section \ref{Measure_HSpice_Compatibility} of the \Xyce{} Reference Guide 
    for more details. \\ \hline

    The \texttt{P()} accessor for power may give different results for semiconductor devices
    (D, J, M, Q and Z devices) and the lossless transmission device (T device) than with HSPICE.  
    & See Sections \ref{D_DEVICE}, \ref{J_DEVICE}, \ref{M_DEVICE}, \ref{Q_DEVICE} \ref{T_DEVICE}
    and \ref{Z_DEVICE}  for more details. \\ \hline

    The \Xyce{} \texttt{.OP} statement provides less output than other simulators. & See 
    Section \ref{OP_COMMAND} for more details.    \\ \hline

    Initial conditions for lossless and lossy transmission lines & In SPICE3F5 and PSpice, 
    initial conditions can be set on the initial voltages and currents at each end of the 
    lossless transmission line, but not for the lossy transmission line.  In \Xyce{}, 
    initial conditions can be set on the initial voltages and currents at each end of the
    lossy transmission line, but not for the lossless transmission line  \\ \hline

    Use of vgs(Mxxx) style syntax on the .PRINT line & Some SPICE-style circuit simulators 
    can use the \texttt{.PRINT} line to (for example) print out the vds, vgb, vsd, etc. 
    values for a PMOS transistor (say, \texttt{M1}) using \texttt{.PRINT TRAN vgs(M1) 
    vbs(M1) vds(M1)}.  This is not directly supported in \Xyce{}.  See Section \ref{Q_DEVICE}
    for how this is supported with the \texttt{N()} syntax for the BSIM3 and BSIM4 models.  
    For other transistor devices, use something like this on the \Xyce{} \texttt{.PRINT} 
    line, \texttt{V(ng,ns)} where ng and ns are the names of the circuits nodes attached 
    to the gate and source terminals of the transistor. \\ \hline

    Some devices do not work in frequency-domain analysis & Devices
    that may be expected to work in AC or HB analysis do not at this
    time.  For AC analysis this includes, but is not limited to, the
    lossy transmission line (LTRA) and lossless transmission line
    (TRA).  The LTRA and TRA models will need to be replaced with
    lumped transmission line models (YTRANSLINE) to perform
    small-signal AC analysis.  For harmonic balance, the two
    transmission line models do work correctly in frequency domain.

    Independent behavioral sources, such as a time-dependent B, E, F, G, or H 
    source, will not work correctly with either AC or HB. 
    However, such sources which are purely dependent (only depending on solution 
    variables and not time) will work in AC and HB.
    \\ \hline

    \Xyce{} uses \texttt{FREQ} as the special variable denoting the
    current simulation frequency  & Other simulators may use \texttt{HERTZ}
    instead. \\ \hline

\end{longtable}

  

\section{DC Operating Point Calculation Failures in Xyce}
This section discusses various netlist problems that can cause \Xyce{} to 
fail to get a DC Operating Point (DCOP).  Some of this discussion is
``tutorial'' in nature, but helps illustrate the issues.

\subsection{Incompatible Voltage Constraints at Circuit Nodes}
The \Xyce{} DCOP calculation will fail if the netlist specifies incompatible voltage
constraints at a given node in the circuit.  This netlist fragment will cause \Xyce{} 
to fail to get a DCOP because the two voltage sources obviously cannot both apply 
their assigned voltage at \texttt{Node1}.
\begin{vquote}
VA Node1 0 1
VB Node1 0 2
\end{vquote}
This configuration is also not allowed because there is an infinite number of ways that
the two voltage sources can supply current to the rest of the circuit and still maintain
the requested voltage.
\begin{vquote}
VA Node1 0 1
VB Node1 0 1
\end{vquote}
With those two netlist fragments as background, the next two examples illustrate 
a ``\Xyce{}-unique'' way that DCOP failure can occur.  This happens because initial 
conditions on capacitors in \Xyce{} are enforced with additional voltage sources
during the DCOP.  So, these two netlist fragments are identical to the two cases 
given above, and will both cause a DCOP failure in \Xyce{}.  A similar problem can
occur with other \Xyce{} devices that allow initial conditions, for voltage drops across
the device, to be set.
\begin{vquote}
VA node1 0 1
CB node1 0 1.0pf IC=2
\end{vquote}
or
\begin{vquote}
VA node1 0 1
CB node1 0 1.0pf IC=1
\end{vquote}
\subsection{Multiple Voltage Constraints From Subcircuits or at Global Nodes}
Similar incompatible voltage constraints can be caused by subcircuit definitions, if the 
subcircuits enforce voltage constraints on one (or more) of their interface nodes.  An
example netlist fragment is given below.  In this example, subcircuits \texttt{X1} and \texttt{X2}
are trying to enforce incompatible constraints at \texttt{Node1} in the top-level circuit.  
This is notionally identical to the first example in the previous subsection.  However, 
these incompatibilities can be harder to find if the subcircuit definitions are located in
different library files.
\begin{vquote}
X1 node1 0 MySubcircuitA
X2 node1 0 MySubcircuitB
.SUBCKT MYSUBCIRCUITA 1 2
VA 1 0 1 
R1A 1 internalNodeA 0.5
R2A internalNodeA 2 0.5
.ENDS 
.SUBCKT MYSUBCIRCUITB 3 4
VB 3 0 2 
R1B 3 internalNodeB 0.5
R2B internalNodeB 4 0.5
.ENDS 
\end{vquote}

Global nodes that have voltage sources applied to them from separate parts 
of the circuit (e.g, from within subcircuit definitions) can cause yet another
version of the DCOP failure modes given in the previous subsection.  If these
two netlist statements are given in different subcircuit definitions then a 
\Xyce{} DCOP failure will occur.
\begin{vquote}
Vpin1 \$G\_GlobalNode1 0 1
Vpin2 \$G\_GlobalNode1 0 2  
\end{vquote}
Of course, the examples given above can occur in varied combinations.

\subsection{NODESET and IC Statements in Subcircuits}
As previously noted, \Xyce{} does not support \texttt{.NODESET} and 
\texttt{.IC} statements in subcircuits.  This is a common cause of DCOP failure
in \Xyce{} when the same netlist converges in PSpice.  See sections ~\ref{NODESET_section} 
and ~\ref{IC_section} for more details on how to move those \texttt{.NODESET} and 
\texttt{.IC} statements to the ``top-level'' in the \Xyce{} netlist.

\subsection{No DC Path to Ground for a Current Flow}
A \Xyce{} DCOP failure can occur if there is no DC path to ground at a node but
a current flow must occur.  This can happen because of a typographic
error during netlist entry. An simple example is as follows, where the netlist 
line for R1 has 0 (``oh'') rather then 0 (``zero'').  It can also happen when
all of the current into a subcircuit must flow through capacitors.
\begin{vquote}
I1 1 0 1
R1 1 O 1
C1 1 0 2pF
\end{vquote}

\subsection{Inductor Loops}
An inductor loop with no DC path to ground will also typically cause a DCOP
failure.  A simple example is:
\begin{vquote}
V1 1 0 1
R1 1 2 1
L1 2 3 2uH
L2 2 3 2uH
R3 3 0 1
\end{vquote}

\subsection{Infinite Slope Transistions}
It is possible for a user to specify expressions that could have infinite-slope 
transitions with B-, E-, F-, G- and H-sources. A common example is IF statements
within those source definitions.  This can often lead to ``timestep too  small''
errors when Xyce reaches the transition point.  In some cases, it can also cause 
DCOP failures.  See Section \ref{B_DEVICE} and the ``Analog Behavioral Modeling'' 
(ABM) chapter of the Xyce Users' Guide for guidance on using the B-source device 
and ABM expressions.  Those recommendations also apply to the E-, F-, G- 
and H-sources.

\subsection{Simulation Settings}
Automatic source stepping was added to \Xyce{} in version 6.3.  \Xyce{} also automatically 
does Gmin stepping when the DCOP calculation fails to converge.  In addition, the time 
integration options normally do not affect the DCOP calculation.  So, adjusting the simulation
settings for \Xyce{} typically has no effect on the DCOP calculation.  However, if both of
the automatic homotopy methods mentioned above do not work, and none of the other
netlist issues mentioned above exist, then Xyce does have other homotopy methods 
available. See the \Xyce{} Users Guide \cite{Xyce_Users_Guide_6_5} for more details.

\chapter{Quick Reference for Microsoft Windows Users}
\label{Windows_Ref}
\index{Microsoft Windows}
\index{Users of Xyce on Microsoft Windows}

\Xyce{} is supported on Microsoft Windows.  However, the primary targets
for \Xyce{} are high-performance supercomputers and workstations, which
are almost always running a variant of Unix.  All of \Xyce{} 
developement is done on Unix platforms.  Bearing this in mind, there are
occasionally issues with using a Unix application on a Windows platform.
Some of these issues are described in the table below.

% Sandia National Laboratories is a multimission laboratory managed and
% operated by National Technology & Engineering Solutions of Sandia, LLC, a
% wholly owned subsidiary of Honeywell International Inc., for the U.S.
% Department of Energy’s National Nuclear Security Administration under
% contract DE-NA0003525.

% Copyright 2002-2019 National Technology & Engineering Solutions of Sandia,
% LLC (NTESS).


%%
%%  Table of Windows Issues.
%%

\index{Microsoft Windows}
\label{Windows_Issues}
\begin{longtable}[h] {>{\raggedright\small}m{2in}|>{\raggedright\let\\\tabularnewline\small}m{4in}}
  \caption{Issues for Microsoft Windows.} \\ \hline
  \rowcolor{XyceDarkBlue}
  \color{white}\bf Issue &
  \color{white}\bf Comment \\ \hline \endfirsthead
  \caption[]{Issues for Microsoft Windows.} \\ \hline
  \rowcolor{XyceDarkBlue}
  \color{white}\bf Issue &
  \color{white}\bf Comment \\ \hline \endhead

File names are case-sensitive &
\Xyce{} will expect library files, which are referenced in the netlist, to have
exactly the same case as the actual filename.  If not, \Xyce{} will be unable
to find the library file.  \\ \hline

Windows endline characters are different from other OS's &
The characters that mark the end of a line in Windows are a carriage return
followed by a Line Feed (\texttt{CR+LF}).  In Unix-like systems (including
Linux and OS X), the character is simply a Line Feed (\texttt{LF}).  Moving a
file between the two systems does not usually cause issues, but users should be
aware of the difference in case problems arise.  \\ \hline

\Xyce{} is unable to read proprietary file formats. &
Programs such as Microsoft Word by default use file formats that \Xyce{} cannot
recognize.  It is best not to use such programs to create netlists, unless
netlists are saved as *.txt files.   If you must use a Microsoft editor, it is
better to use Microsoft Notepad.  In general, the best solution is to use a
Unix-style editor, such as Vi, Gvim, or Emacs.  \\ \hline

\end{longtable}


%%% Local Variables:
%%% mode: latex
%%% End:


%%% Local Variables:
%%% mode: latex
%%% End:

% END of Xyce_RG_app2.tex ************
