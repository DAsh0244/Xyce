% Sandia National Laboratories is a multimission laboratory managed and
% operated by National Technology & Engineering Solutions of Sandia, LLC, a
% wholly owned subsidiary of Honeywell International Inc., for the U.S.
% Department of Energy’s National Nuclear Security Administration under
% contract DE-NA0003525.

% Copyright 2002-2022 National Technology & Engineering Solutions of Sandia,
% LLC (NTESS).


\label{IC_section}
\index{\texttt{.IC}}
\index{\texttt{.DCVOLT}}
\index{initial condition!IC}
\index{initial condition!DCVOLT}
\index{initial condition}

The \texttt{.IC/.DCVOLT} command sets initial conditions for operating point calculations.
These operating point conditions will be enforced the entire way through the
nonlinear solve.  Initial conditions can be given for some or all of the
circuit nodes.

As the conditions are enforced for the entire solve, only the nodes not
specified with \texttt{.IC} statements will change over the course of the
operating point calculation.

Note that it is possible to specify conditions that are not solvable.
Consult the \Xyce{} Users' Guide~\UsersGuide{} for more guidance.

\begin{Command}
\format
\begin{alltt}
.IC V(<node>)=<value>
.IC <node> <value>
.DCVOLT V(<node>)=<value>
.DCVOLT <node> <value>
\end{alltt}

\examples
\begin{alltt}
.IC V(2)=3.1
.IC 2 3.1
.DCVOLT V(2)=3.1
.DCVOLT 2 3.1
\end{alltt}

\comments
The \texttt{.IC} capability can only set voltage values, not current values.

The \texttt{.IC} capability can not be used within subcircuits to set
voltage values on global nodes.

\end{Command}

