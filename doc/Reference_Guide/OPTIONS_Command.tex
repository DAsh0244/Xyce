% Sandia National Laboratories is a multimission laboratory managed and
% operated by National Technology & Engineering Solutions of Sandia, LLC, a
% wholly owned subsidiary of Honeywell International Inc., for the U.S.
% Department of Energy’s National Nuclear Security Administration under
% contract DE-NA0003525.

% Copyright 2002-2021 National Technology & Engineering Solutions of Sandia,
% LLC (NTESS).


\label{Options_Reference}

\index{\texttt{.OPTIONS}} \index{analysis!options}  \index{analysis!control parameters} \index{solvers!control parameters}

Set various simulation limits, analysis control parameters
and output parameters.  In general, they use the following format:

\begin{Command}
\format
.OPTIONS <pkg> [<name>=<value>]*

\examples
.OPTIONS TIMEINT ABSTOL=1E-8

\arguments

\begin{Arguments}

\argument{pkg}

\begin{basedescript}{
    \desclabelstyle{\multilinelabel}
    \desclabelwidth{1.5in}
    \renewcommand{\makelabel}[1]{\tt #1\hfill}}
  \item[\tt DEVICE]       Device Model
  \item[\tt TIMEINT]      Time Integration
  \item[\tt NONLIN]       Nonlinear Solver
  \item[\tt NONLIN-TRAN]  Transient Nonlinear Solver
  \item[\tt NONLIN-HB]    HB Nonlinear Solver
  \item[\tt LOCA]         Continuation/Bifurcation Tracking
  \item[\tt LINSOL]       Linear Solver
  \item[\tt LINSOL-HB]    HB Linear Solver
  \item[\tt LINSOL-AC]    AC Linear Solver
  \item[\tt OUTPUT]       Output
  \item[\tt RESTART]      Restart
  \item[\tt SAMPLES]      Sampling analysis and non-intrusive Polynomial Chaos (PCE)
  \item[\tt EMBEDDEDSAMPLES]  EmbeddedSampling and non-intrusive Polynomial Chaos (PCE) 
  \item[\tt PCES]         Fully intrusive Polynomial Chaos (PCE)
  \item[\tt SENSITIVITY]  Direct and Adjoint sensitivities
  \item[\tt HBINT]        Harmonic Balance (HB)
  \item[\tt DIST]         Distribution
  \item[\tt MEASURE]      Measure
  \item[\tt PARSER]       Parsing
\end{basedescript}

\argument{\vbox{\hbox{name\hfil}\hbox{value}}}
The name of the parameter and the value it will be assigned.

\end{Arguments}

\comments

Exceptions to this format are the \texttt{OUTPUT} and \texttt{RESTART}
options, which use their own format. They are defined under their
respective descriptions.

The designator \texttt{pkg} refers loosely to a {\em module} in the
code.  Thus, the term is used here as identifying a specific module to
be controlled via {\em options} set in the netlist input file.

\end{Command}

\subsubsection{\texttt{.OPTIONS DEVICE} (Device Package Options)}
\index{device!package options}\index{\texttt{.OPTIONS}!\texttt{DEVICE}}

The device package parameters listed in Table~\ref{DevicePKG} outline the options
available for specifying device specific parameters.  Some of these (\texttt{DEFAS,
DEFAD, TNOM} etc.) have the same meaning as they do for
the \texttt{.OPTION} line from Berkeley SPICE (3f5).  Parameters which
apply globally to all device models will be specified here.
Parameters specific to a particular device instance or model are
specified in section~\ref{Analog_Devices}.

% Sandia National Laboratories is a multimission laboratory managed and
% operated by National Technology & Engineering Solutions of Sandia, LLC, a
% wholly owned subsidiary of Honeywell International Inc., for the U.S.
% Department of Energy’s National Nuclear Security Administration under
% contract DE-NA0003525.

% Copyright 2002-2021 National Technology & Engineering Solutions of Sandia,
% LLC (NTESS).


%%
%% Device Selections Table
%%

\begin{OptionTable}{Options for Device Package}
\label{DevicePKG}%
DEFAD & MOS Drain Diffusion Area & 0.0 \\ \hline
DEFAS & MOS Source Diffusion Area & 0.0 \\ \hline
DEFL & MOS Default Channel Length & 1.0E-4 \\ \hline
DEFW & MOS Default Channel Width & 1.0E-4 \\ \hline
DIGINITSTATE & This option controls the behavior of the Digital Latch (DLTCH), D Flip-Flop (DFF),
JK Flip-Flop (JKFF) and T Flip-Flop (TFF) behavioral digital devices during the DC Operating Point (DCOP)
calculations. See ~\ref{U_DEVICE} for more details.  & 3 \\ \hline 
GMIN & Minimum Conductance & 1.0E-12 \\ \hline
MINRES & This is a minimum resistance to be used in place of the default zero value of semiconductor device internal resistances.  It is only used when model specifications (\texttt{.MODEL} cards) leave the parameter at its default value of zero, and is not used if the model explicitly sets the resistance to zero.   & 0.0 \\ \hline
MINCAP & This is a minimum capacitance to be used in place of the default zero value of semiconductor device internal capacitances.  It is only used when model specifications (\texttt{.MODEL} cards) leave the parameter at its default value of zero, and is not used if the model explicitly sets the capacitance to zero.   & 0.0 \\ \hline
TEMP & Temperature & 27.0 $^\circ$C (300.15K) \\ \hline
TNOM & Nominal Temperature & 27.0 $^\circ$C (300.15K) \\ \hline
% TVR: I can't find any use of this parameter in Xyce!
%\debug{\texttt{SCALESRC}} & \debug{Scaling factor for source scaling} & \debug{0.0} \\ \hline
\debug{NUMJAC} & \debug{Numerical Jacobian flag (only use for small problems)}
& \debug{0 (FALSE)} \\ \hline
VOLTLIM & Voltage limiting & 1 (TRUE) \\ \hline
B3SOIVOLTLIM & BSIMSOI3 Voltage limiting.  This flag is similar to VOLTLIM, except that it only 
  applies to the BSIMSOI version 3  (the newer versions of the BSIM SOI do not have voltage limiting).  
  Turning this off will often improve numerical robustness.  Unlike VOLTLIM, turning this off 
  does not disable the initial condition code in the BSIMSOI model.  & 1 (TRUE) \\ \hline

\debug{icFac} 
& \debug{This is a multiplicative factor which is applied to right-hand side vector loads of 
.IC initial conditions during the DCOP phase. }
& \debug{10000.0} \\ \hline

LAMBERTW\index{Lambert-W Function} & This flag determines if the Lambert-W 
function should be applied in place of
exponentials in hard-to-solve devices.  This capability is implemented in the diode
and BJT.  Try this for BJT circuits that have convergence problems.  For best effect, this option
should be tried with voltlim turned off.  A detailed explanation of the Lambert-W function,
and its application to device modeling can be found in reference ~\cite{Banwell}. & 0 (FALSE) \\ \hline

MAXTIMESTEP & Maximum time step size & 1.0E+99 \\ \hline

SMOOTHBSRC & This flag enables smooth transitions by adding a RC network to the output of ABM devices    &    0  \\ \hline


RCCONST & This option controls the smoothness of the transitions if the
SMOOTHBSRC flag is enabled. This is done by specifying the RC constant of the
RC network & 1e-9 \\ \hline

\\ \hline
\multicolumn{3}{|c|}{\color{XyceDarkBlue}\em\bfseries MOSFET Homtopy parameters} \\ \hline

VDSSCALEMIN & Scaling factor for Vds    & 0.3      \\ \hline
VGSTCONST   & Initial value for Vgst    & 4.5 Volt \\ \hline
LENGTH0     & Initial value for length  & 5.0e-6   \\ \hline
WIDTH0      & Initial value for width   & 200.0e-6 \\ \hline
TOX0        & Initial value for oxide thickness & 6.0e-8  \\ \hline

\\ \hline
\multicolumn{3}{|c|}{\color{XyceDarkBlue}\em\bfseries Debug output parameters} \\ \hline

\debug{DEBUGLEVEL} & \debug{The higher this number, the more info is output} & \debug{1} \\ \hline
\debug{DEBUGMINTIMESTEP} & \debug{First time-step debug information is output} & \debug{0} \\ \hline
\debug{DEBUGMAXTIMESTEP} & \debug{Last time-step of debug output} & \debug{65536} \\ \hline
\debug{DEBUGMINTIME} & \debug{Same as \texttt{DEBUGMINTIMESTEP} except controlled by time (sec.) instead of step number} & \debug{0.0} \\ \hline
\debug{DEBUGMAXTIME} & \debug{Same as \texttt{DEBUGMAXTIMESTEP} except controlled by time (sec.) instead of step number} & \debug{100.0} \\ \hline
\end{OptionTable}


\subsubsection{\texttt{.OPTIONS TIMEINT} (Time Integration Options)}
\index{solvers!time integration!options}
\index{\texttt{.OPTIONS}!\texttt{TIMEINT}}

The time integration parameters listed in Table~\ref{TimeIntPKG} give the
available options for helping control the time integration algorithms for
transient analysis.

Time integration options are set using the \texttt{.OPTIONS TIMEINT} command.

% Sandia National Laboratories is a multimission laboratory managed and
% operated by National Technology & Engineering Solutions of Sandia, LLC, a
% wholly owned subsidiary of Honeywell International Inc., for the U.S.
% Department of Energy’s National Nuclear Security Administration under
% contract DE-NA0003525.

% Copyright 2002-2019 National Technology & Engineering Solutions of Sandia,
% LLC (NTESS).


%%
%% Time Integration Package Table
%%

\begin{OptionTable}{Options for Time Integration Package.}
\label{TimeIntPKG}
METHOD & Time integration method.  This parameter is only
relevant when running \Xyce{} in transient mode.  Supported methods:
\begin{XyceItemize}
\item trap or 7 (variable order Trapezoid)
\item gear or 8 (Gear method) 
\end{XyceItemize} &
trap or 7 (variable order Trapezoid) \\ \hline
RELTOL\index{\texttt{RELTOL}}  & Relative error tolerance & 
1.0E-03 \\ \hline
ABSTOL\index{\texttt{ABSTOL}}  & Absolute error tolerance & 1.0E-06 \\
\hline
RESTARTSTEPSCALE\index{\texttt{RESTARTSTEPSCALE}}  & 
This parameter is a scalar which determines how small the initial
time step out of a breakpoint should be.  In the current version of the
time integrator, the first step after a breakpoint isn't subjected to
much error analysis, so for very stiff circuits, this step can be 
problematic. & 0.005 \\ \hline
NLNEARCONV\index{\texttt{NLNEARCONV}}  & 
This flag sets if ``soft'' 
failures of the nonlinear solver, when the convergence criteria are almost, 
but not quite, met, should result in a "success" code being returned from
the nonlinear solver to the time integrator.
If this is enabled, it is expected that the error analysis performed by the 
time integrator will be the sole determination of whether or not the time 
step is considered a ``pass'' or a ``fail''.  This is on by default, but
occasionally circuits need tighter convergence criteria.  
& 0 (FALSE)  \\ \hline
NLSMALLUPDATE\index{\texttt{NLSMALLUPDATE}} &   
This flag is another ``soft'' nonlinear solver failure flag.  In
this case, if the flag is set, time steps in which the nonlinear solver 
stalls, and is using updates that are numerically tiny, can be considered
to have converged by the nonlinear solver.  If this flag is set, 
the time integrator is responsible for determining
if a step should be accepted or not.
& 1 (TRUE) \\ \hline

RESETTRANNLS & The nonlinear solver resets its settings for the
transient part of the run to something more efficient (basically a simpler set
of options with smaller numbers for things like max Newton step).  If this is
set to false, this resetting is turned off. Normally should be left as
default. & 1 (TRUE) \\ \hline

MAXORD & This parameter determines the maximum order of integration
that time integrators will attempt.  Setting this option
does not guarantee that the integrator will integrate at this order, it just
sets the maximum order the integrator will attempt.  In order to guarantee a
particular order is used, see the option \texttt{MINORD} below.  & 
2 for variable order Trapezoid and Gear \\ \hline

MINORD & This parameter determines the minimum order of integration
that  time integrators will attempt to maintain.  The integrator will
start at Backward Euler and move up in order as quickly as possible to achive
\texttt{MINORD} and then it will keep the order above this.  If \texttt{MINORD}
is set at 2 and \texttt{MAXORD} is set at 2, then the integrator will move to
second order as quickly as possible and stay there.  & 1 \\ \hline

NEWLTE & This parameter determines the reference value for relative
convergence criterion in the local truncation error based time step control.
The  supported choices
\begin{XyceItemize}
\item 0. The reference value is the current value on each node.
\item 1. The reference value is the maximum of all the signals at the current time.
\item 2. The reference value is the maximum of all the signals over all past time.
\item 3. The reference value is the maximum value on each signal over all past time.
\end{XyceItemize}   & 1 \\ \hline

NEWBPSTEPPING & This flag sets a new time stepping method after a break point. 
Previously, \Xyce{} treats each breakpoint identically to the DCOP point, in which
the intitial time step out of the DCOP is made to be very very small, because
the LTE calculation is unreliable.  As a result, \Xyce{} takes an incredibly small
step out of each breakpoint and then tries to grow the stepsize from there. 
When \texttt{NEWBPSTEPPING} is set, \Xyce{} can take a reasonable
large step out of every non-DCOP breakpoint, and then just relies on the step
control to ensure that the step is small enough.  

Note that the new time stepping method after a break point does not work
well with the old LTE calculation since the old LTE calculation is
conservative and it tends to reject the first time step out of a break
point. We recommend to use newlte if you choose to use the new time
stepping method out of a break point. & 1 (TRUE) \\ \hline

MASKIVARS & This parameter masks out current variables in the local truncation error (LTE) based time step
control. & 0 (FALSE) \\  \hline

ERROPTION & This parameter determines if Local Truncation Error (LTE)
control is turned on or not.  If \texttt{ERROPTION} is  on, then step-size
selection is based on the number of Newton iterations nonlinear solve.  
For Trapezoid and Gear, if the number of nonlinear
iterations is below \texttt{NLMIN} then the step is doubled.  If the number
of nonlinear iterations is above \texttt{NLMAX} then the step is cut by one
eighth.  In between, the step-size is left alone.  Because this option can
lead to very large time-steps, it is very important   to specify an appropriate
\texttt{DELMAX} option.  If the circuit has breakpoints, then the option
\texttt{MINTIMESTEPSBP} can also help to adjust the maximum time-step by
specifying the minimum number of time points between breakpoints. & 0 (Local Truncation Error is used)  \\ \hline

NLMIN &  This parameter determines the lower bound for the desired
number of nonlinear iterations during a Trapezoid time or Gear integration solve with
\texttt{ERROPTION}=1.
& 3  \\ \hline

NLMAX & This parameter determines the upper bound for the desired
number of nonlinear iterations during a Trapezoid time or Gear integration solve with
\texttt{ERROPTION}=1.
& 8  \\ \hline

DELMAX & This parameter determines the maximum time step-size used
with \texttt{ERROPTION}=1.  If a maximum time-step is also specified on the
\texttt{.TRAN} line, then the minimum of that value and \texttt{DELMAX} is
used.
& 1e99 \\ \hline

MINTIMESTEPSBP & This parameter determines the minimum number of
time-steps to use between breakpoints.  This enforces a maximum time-step
between breakpoints equal to the distance between the last breakpoint and the
next breakpoint divided by \texttt{MINTIMESTEPSBP}.
& 10  \\ \hline

TIMESTEPSREVERSAL & This parameter determines whether time-steps are
rejected based upon the step-size selection strategy in \texttt{ERROPTION}=1.
If it is set to 0, then a step will be accepted with successful nonlinear
solves independent of whether the number of nonlinear iterations is between
\texttt{NLMIN} and \texttt{NLMAX}.  If it is set to 1, then when the number of
nonlinear iterations is above \texttt{NLMAX}, the step will be rejected and the
step-size cut by one eighth and retried.  If \texttt{ERROPTION}=0 (use LTE) then
\texttt{TIMESTEPSREVERSAL}=1 (reject steps) is set.  
& 0 (do not reject steps) \\ \hline

DOUBLEDCOPSTEP \index{PDE Devices!time integration parameters} \index{TCAD Devices!time integration parameters} & 
TCAD devices by default will solve an extra "setup" problem to mitigate
some of the convergence problems that TCAD devices often exhibit.
This extra setup problem solves a nonlinear Poisson equation first to establish 
an initial guess for the full drift-diffusion(DD) problem.  The name of this 
parameter refers to the fact that the code is solving two DC operating point 
steps instead of one.  To solve only the nonlinear Poisson problem, then set 
\texttt{DOUBLEDCOP=nl\_poisson}.  To solve only the drift-diffusion problem 
(skipping the nonlinear Poisson), set \texttt{DOUBLEDCOP=drift\_diffusion}.
To explicitly set the default behavior, then set \texttt{DOUBLEDCOP=nl\_poisson, drift\_diffusion}.
& 
Default value, for TCAD circuits, is a combination: nl\_poisson, drift\_diffusion.
Default value, for non-TCAD circuits is a moot point.  If no TCAD devices are present
in the circuit, then there will not be an extra DCOP solve.
\\ \hline

BREAKPOINTS & This parameter specifies a comma-separated list of timepoints that 
should be used as breakpoints.  They do not replace the existing breakpoints 
already being set internally by Xyce, but instead will add to them.
& N/A  \\ \hline

\debug{BPENABLE}\index{\texttt{BPENABLE}} & \debug{Flag for
  turning on/off breakpoints (1 = ON, 0 = OFF).  It is unlikely anyone would
  ever set this to FALSE, except to help debug the breakpoint capability.}
& \debug{1 (TRUE)} \\ \hline

\debug{EXITTIME}\index{\texttt{EXITTIME}} & \debug{If this is set
  to nonzero, the code will check the simulation time at the end of each step.
  If the total time exceeds the exittime, the code will ungracefully exit.
  This is a debugging option, the point of which is the have the code stop at a
  certain time during a run without affecting the step size control.  If not
  set by the user, it isn't activated.}& \debug{-} \\ \hline

\debug{EXITSTEP}\index{\texttt{EXITSTEP}} &
\debug{Same as \texttt{EXITTIME}, only applied to step number.
The code will exit at the specified step.  If not set by the user,
it isn't activated.} &
\debug{-} \\ \hline

\index{solvers!time integration!options}
\end{OptionTable}



\subsubsection{\texttt{.OPTIONS NONLIN} (Nonlinear Solver Options)}
\index{solvers!nonlinear!options}\index{\texttt{.OPTIONS}!\texttt{NONLIN}}\index{\texttt{.OPTIONS}!\texttt{NONLIN-TRAN}}

The nonlinear solver parameters listed in Table~\ref{NonlinPKG} provide
methods for controlling the nonlinear solver for \index{DC analysis}
\index{analysis!DC} DC, \index{transient analysis}
\index{analysis!transient} transient and harmonic balance. Note that the
nonlinear solver options for DCOP, transient and harmonic balance are
specified in separate options statements, using \texttt{.OPTIONS
 NONLIN}, \texttt{.OPTIONS NONLIN-TRAN} and \texttt{.OPTIONS
 NONLIN-HB}, respectively. The defaults for  \texttt{.OPTIONS
 NONLIN} and \texttt{.OPTIONS NONLIN-TRAN} are specified in the
third and fourth columns of Table~\ref{NonlinPKG}.  The defaults for 
\texttt{.OPTIONS NONLIN-HB} are the same as the default settings given for
\texttt{NONLIN-TRAN} with two exceptions. For \texttt{NONLIN-HB}, the default
for \texttt{ABSTOL} is 1e-9 and the default for \texttt{RHSTOL} is 1e-4.

% Sandia National Laboratories is a multimission laboratory managed and
% operated by National Technology & Engineering Solutions of Sandia, LLC, a
% wholly owned subsidiary of Honeywell International Inc., for the U.S.
% Department of Energy’s National Nuclear Security Administration under
% contract DE-NA0003525.

% Copyright 2002-2020 National Technology & Engineering Solutions of Sandia,
% LLC (NTESS).


%%
%% Nonlinear Solver Options Table
%%
\index{solvers!nonlinear!options}
\begin{OptionTable4}{Options for Nonlinear Solver Package.}
\label{NonlinPKG}
NOX & Use NOX nonlinear solver. & 1 (TRUE) & 0 (FALSE) \\ \hline

NLSTRATEGY & Nonlinear solution strategy.  Supported Strategies:
\begin{XyceItemize}
\item 0 (Newton)
\item 1 (Gradient)
\item 2 (Trust Region)
\end{XyceItemize} &
0 (Newton) &
0 (Newton) \\ \hline

SEARCHMETHOD &
Line-search method used by the nonlinear solver.  Supported
line-search methods:
\begin{XyceItemize}
\item 0 (Full Newton - no line search)
\item 1 (Interval Halving)
\item 2 (Quadratic Interpolation)
\item 3 (Cubic Interpolation)
\item 4 (More'-Thuente)
\end{XyceItemize} &
0 (Full Newton) & 
0 (Full Newton) \\ \hline

CONTINUATION & Enables the use of Homotopy/Continuation algorithms for the nonlinear solve.  Options are:
\begin{XyceItemize}
\item 0 (Standard nonlinear solve)
\item 1 (Natural parameter homotopy.  See LOCA options list)
\item 2/mos (Specialized dual parameter homotopy for MOSFET circuits)
\item 3/gmin (GMIN stepping, similar to that of SPICE)
\end{XyceItemize} & 
0 (Standard nonlinear solve) &
0 (Standard nonlinear solve) \\ \hline

ABSTOL\index{\texttt{ABSTOL}} & Absolute residual vector tolerance &
1.0E-12 & 1.0E-06 \\ \hline

RELTOL\index{\texttt{RELTOL}} & Relative residual vector tolerance &
1.0E-03 & 1.0E-02 \\ \hline

DELTAXTOL & Weighted nonlinear-solution update norm convergence
tolerance & 1.0 & 0.33 \\ \hline

RHSTOL & Residual convergence tolerance (unweighted 2-norm) &
1.0E-06 & 1.0E-02 \\ \hline

SMALLUPDATETOL & Minimum acceptable norm for weighted nonlinear-solution update & 
1.0E-06 & 1.0E-06 \\ \hline

MAXSTEP\index{\texttt{MAXSTEP}} & Maximum number of Newton steps & 200 & 20 \\ \hline

MAXSEARCHSTEP & Maximum number of line-search steps & 2 & 2 \\ \hline

IN\_FORCING & Inexact Newton-Krylov forcing flag & 
0 (FALSE) &
0 (FALSE) \\ \hline

AZ\_TOL &  Sets the minimum allowed linear solver tolerance. Valid only if \texttt{IN\_FORCING}=1.  & 
1.0E-12 &
1.0E-12 \\ \hline

RECOVERYSTEPTYPE &  If using a line search, this option determines the type of step to take if the line search fails. Supported strategies:
\begin{XyceItemize}
\item 0 (Take the last computed step size in the line search algorithm)
\item 1 (Take a constant step size set by \texttt{RECOVERYSTEP})
\end{XyceItemize} & 
0 &
0 \\ \hline

RECOVERYSTEP & Value of the recovery step if a constant step length is selected & 
1.0 &
1.0 \\ \hline

\debug{DEBUGLEVEL} & \debug{The higher this number, the more info is output} & 
\debug{1} &
\debug{1} 
\\ \hline

\debug{DEBUGMINTIMESTEP} & \debug{First time-step debug information is output} & 
\debug{0} &
\debug{0}
\\ \hline

\debug{DEBUGMAXTIMESTEP} & \debug{Last time-step of debug output} &
\debug{99999999} &
\debug{99999999} \\
\hline

\debug{DEBUGMINTIME} & \debug{Same as \texttt{DEBUGMINTIMESTEP} except controlled by
time (sec.) instead of step number} & 
\debug{0.0}  &
\debug{0.0} 
\\ \hline

\debug{DEBUGMAXTIME} & \debug{Same as \texttt{DEBUGMAXTIMESTEP} except controlled by
time (sec.) instead of step number} & 
\debug{1.0E+99} &
\debug{1.0E+99} 
\\ \hline

\end{OptionTable4}


\index{solvers!nonlinear-transient!options}

\subsubsection{\texttt{.OPTIONS LOCA} (Continuation and Bifurcation Tracking Package Options)}
\index{solvers!continuation!options}
\index{solvers!homotopy!options}

The continuation selections listed in Table~\ref{ContinuationPKG}
provide methods for controlling continuation and bifurcation analysis.
These override the defaults and any that were set simply in the
continuation package.  This option block is only used if the nonlinear
solver or transient nonlinear solver enable continuation through the
\texttt{CONTINUATION} flag.

There are two specialized homotopy methods that can be set in the nonlinear
solver options line.  One is MOSFET-based homotopy, which is specific to MOSFET
circuits.  This is specified using \texttt{continuation=2} or
\texttt{continuation=mos}.  The other is GMIN stepping, which is specified
using \texttt{continuation=3} or \texttt{continuation=gmin}.  For either of
these methods, while it is possible to modify their default LOCA options, it is
generally not necessary to do so. Note that \Xyce{} automatically attempts GMIN
stepping if the inital attempt to find the DC operating point fails. If GMIN
stepping is specified in the netlist, \Xyce{} will not attempt to find a DC
operating point without GMIN stepping. 

LOCA options are set using the \texttt{.OPTIONS LOCA} command.\index{\texttt{.OPTIONS}!\texttt{LOCA}}

% Sandia National Laboratories is a multimission laboratory managed and
% operated by National Technology & Engineering Solutions of Sandia, LLC, a
% wholly owned subsidiary of Honeywell International Inc., for the U.S.
% Department of Energy’s National Nuclear Security Administration under
% contract DE-NA0003525.

% Copyright 2002-2022 National Technology & Engineering Solutions of Sandia,
% LLC (NTESS).


%%
%% LOCA Solver Options Table
%%
\index{solvers!continuation!options}
\begin{OptionTable}{Options for Continuation and Bifurcation Tracking Package.}
\label{ContinuationPKG}
STEPPER & Stepping algorithm to use:
\begin{XyceItemize}
\item 0 (Natural or Zero order continuation)
\item 1 (Arc-length continuation)
\end{XyceItemize} &
0 (Natural) \\ \hline

PREDICTOR &
Predictor algorithm to use:
\begin{XyceItemize}
\item 0 (Tangent)
\item 1 (Secant)
\item 2 (Random)
\item 3 (Constant)
\end{XyceItemize} &
0 (Tangent) \\ \hline

STEPCONTROL &
Algorithm used to adjust the step size between continuation steps:
\begin{XyceItemize}
\item 0 (Constant)
\item 1 (Adaptive)
\end{XyceItemize} &
0 (Constant) \\ \hline

CONPARAM &
Parameter in which to step during a continuation run &
VA:V0 \\ \hline

INITIALVALUE & Starting value of the continuation parameter &
0.0 \\ \hline

MINVALUE & Minimum value of the continuation parameter &
-1.0E20 \\ \hline

MAXVALUE & Maximum value of the continuation parameter &
1.0E20 \\ \hline

BIFPARAM & Parameter to compute during bifurcation tracking runs &
VA:V0 \\ \hline

MAXSTEPS & Maximum number of continuation steps (includes failed steps) & 20 \\ \hline

MAXNLITERS & Maximum number of nonlinear iterations allowed (set this parameter equal to the \texttt{MAXSTEP} parameter in the  \texttt{NONLIN} option block & 20 \\ \hline

INITIALSTEPSIZE & Starting value of the step size & 1.0 \\ \hline

MINSTEPSIZE & Minimum value of the step size & 1.0E20 \\ \hline

MAXSTEPSIZE & Maximum value of the step size & 1.0E-4 \\ \hline

AGGRESSIVENESS & Value between 0.0 and 1.0 that determines how aggressive the step size control algorithm should be when increasing the step size.  0.0 is a constant step size while 1.0 is the most aggressive. & 0.0 \\ \hline

RESIDUALCONDUCTANCE & If set to a nonzero (small) number, this parameter will
force the GMIN stepping algorithm  to stop and declare victory once the
artificial resistors have a conductance that is smaller  than this number.
This should only be used in transient simulations, and \emph{ONLY} if it is
absolutely necessary to get past the DC operating point calculation. It is
almost always better to fix the circuit so that residual conductance is not
necessary. & 0.0 \\ \hline

\end{OptionTable}


\subsubsection{\texttt{.OPTIONS LINSOL} (Linear Solver Options)}
\index{solvers!linear!options}

\Xyce{} uses both sparse direct
solvers\index{solvers!linear!sparse-direct} as well as Krylov iterative
methods\index{solvers!linear!iterative (preconditioned Krylov methods)}
for the solution of the linear equations generated by Newton's method.
For the advanced users, there are a variety of options that can be set
to help improve these solvers.  Transformations of the linear system
have a ``\verb+TR_+'' prefix on the flag.  Many of the options for the
Krylov solvers are simply passed through to the underlying
Trilinos/AztecOO\index{Trilinos}\index{solvers!linear!Trilinos}\index{Aztec}\index{solvers!linear!Aztec}
solution settings and thus have an ``\verb+AZ_+'' prefix on the flag.

Linear solver options are set using the \texttt{.OPTIONS LINSOL}
command.\index{\texttt{.OPTIONS}!\texttt{LINSOL}}

% Sandia National Laboratories is a multimission laboratory managed and
% operated by National Technology & Engineering Solutions of Sandia, LLC, a
% wholly owned subsidiary of Honeywell International Inc., for the U.S.
% Department of Energy’s National Nuclear Security Administration under
% contract DE-NA0003525.

% Copyright 2002-2022 National Technology & Engineering Solutions of Sandia,
% LLC (NTESS).


%%
%% Linear Solver Options Table
%%
\index{solvers!linear!options}
\begin{OptionTable}{Options for Linear Solver Package.}  
\label{LinSolPKG}
type & Determines which linear solver will be used.
\begin{XyceItemize}
\item KLU
\item KSparse
\item SuperLU (optional) 
\item AztecOO
\item Belos
\item ShyLU (optional) 
\end{XyceItemize}
Note that while KLU, KSparse, and SuperLU (optional) are available for parallel execution they will solve the linear system in serial.  Therefore they will be useful for moderate problem sizes but will not scale in memory or performance for large problems  &
KLU (Serial, Parallel $< 10^4$ unknowns) AztecOO, (Parallel, $\geq 10^4$ unknowns) \\ \hline

prec\_type & Determines which preconditioner will be used with an iterative linear solver
\begin{XyceItemize}
\item Ifpack
\end{XyceItemize}
A preconditioner will not be used if a direct solver (KLU, KSparse, SuperLU) is specified.
& Ifpack (Ifpack\_IlukGraph)\\ \hline

use\_aztec\_precond & Triggers use of native AztecOO preconditioners for the iterative linear solves & 0 (FALSE) \\ \hline

use\_ifpack\_factory & Use Ifpack factory to create preconditioner instead of using Ifpack\_IlukGraph & 0 (FALSE) \\ \hline

ifpack\_type & If using the Ifpack factory, {\tt use\_ifpack\_factory=1}, which preconditioner is created
\begin{XyceItemize}
\item Amesos (Additive Schwarz w/ KLU subdomain solve)
\item ILU
\item ILUT
\end{XyceItemize}
& Amesos \\ \hline

ShyLU\_rthresh & Relative dropping threshold for Schur complement preconditioner (ShyLU only) & 1.0E-03 \\ \hline

\multicolumn{3}{|c|}{\color{XyceDarkBlue}\em\bfseries Transformation parameters} \\ \hline\hline

TR\_partition & Perform load-balance partitioning on the linear system & 0 (NONE, Serial) \hspace{3em} 1 (Isorropia, Parallel) \\ \hline

TR\_partition\_type & Type of load-balance partitioning on the linear system & HYPERGRAPH \\ \hline

TR\_singleton\_filter & Triggers use of singleton filter for linear system & 0 (FALSE, Serial) 1 (TRUE, Parallel) \\ \hline

TR\_amd & Triggers use of approximate minimum-degree (AMD) ordering for linear system & 0 (FALSE, Serial) \hspace{3em} 1 (TRUE, Parallel) \\ \hline

TR\_global\_btf & Triggers use of block triangular form (BTF) ordering for linear system, requires \texttt{TR\_amd}=0 and \texttt{TR\_partition}=0 & 0 (FALSE) \\ \hline

TR\_reindex & Reindexes linear system parallel global indices in lexigraphical order, recommended with singleton filter  & 1 (TRUE) \\ \hline

TR\_solvermap & Triggers remapping of column indices for parallel runs, recommended with singleton filter & 1 (TRUE) \\ \hline

\multicolumn{3}{|c|}{\color{XyceDarkBlue}\em\bfseries Iterative linear solver parameters} \\ \hline\hline

adaptive\_solve & Triggers use of AztecOO adaptive solve algorithm for preconditioning of iterative linear solves & 0 (FALSE) \\ \hline

AZ\_max\_iter & Maximum number of iterative solver iterations & 200 \\ \hline

AZ\_precond & AztecOO iterative solver preconditioner flag (used only when \texttt{use\_aztec\_precond}=1) &
\texttt{AZ\_dom\_decomp (14)}
\\ \hline

AZ\_solver & Iterative solver type & \texttt{AZ\_gmres (1)} \\ \hline

AZ\_conv & Convergence type & \texttt{AZ\_r0 (0)} \\ \hline

AZ\_pre\_calc & Type of precalculation & \texttt{AZ\_recalc (1)} \\ \hline

AZ\_keep\_info & Retain calculation info & \texttt{AZ\_true (1)} \\ \hline

AZ\_orthog & Type of orthogonalization & \texttt{AZ\_modified (1)} \\ \hline

AZ\_subdomain\_solve & Subdomain solution for domain decomposition preconditioners & \texttt{AZ\_ilut (9)} \\ \hline

AZ\_ilut\_fill & Approximate allowed fill-in factor for the ILUT preconditioner & 2.0 \\ \hline

AZ\_drop & Specifies drop tolerance used in conjunction with LU or ILUT preconditioners & 1.0E-03 \\ \hline

AZ\_reorder & Reordering type & \texttt{AZ\_none (0)} \\ \hline

AZ\_scaling & Type of scaling & \texttt{AZ\_none (0)} \\ \hline

AZ\_kspace & Maximum size of Krylov subspace & 50 \\ \hline

AZ\_tol & Convergence tolerance & 1.0E-9 \\ \hline

AZ\_output & Output level & AZ\_none (0) \\
& & 50 (if verbose build) \\ \hline

AZ\_diagnostics & Diagnostic information level & AZ\_none (0) \\ \hline

AZ\_overlap & Schwarz overlap level for ILU preconditioners & 0 \\ \hline

AZ\_rthresh & Diagonal shifting relative threshold for ILU preconditioners & 1.0001 \\ \hline

AZ\_athresh & Diagonal shifting absolute threshold for ILU preconditioners & 1.0E-04 \\ \hline

output\_ls & Write out linear system matrix and right-hand-side vector, post-transformation, to Matrix Market file every \# solves
& 0 (no output) \\ \hline

output\_base\_ls & Write out linear system matrix and right-hand-side vector, pre-transformation, to Matrix Market file every \# solves
& 0 (no output) \\ \hline

output\_failed\_ls & Write out linear system matrix and right-hand-side vector to Matrix Market file every \# solves when linear solver fails (only available for direct solvers)
& 0 (no output) \\ \hline
\end{OptionTable}

%%% Local Variables:
%%% mode: latex
%%% End:


\subsubsection{\texttt{.OPTIONS LINSOL-HB} (Linear Solver Options)}

For harmonic balance (HB) analysis, \Xyce{} provides both iterative 
and direct methods for the solution of the steady state.  Only matrix-free techniques
are available for preconditioning the HB Jacobian with an iterative linear solver.
The direct linear solver explicitly forms the HB Jacobian and solves the complex-valued
linear system with the requested solver.
For HB analysis, a reduced number of linear solver options are available,  
and are set using the \texttt{.OPTIONS LINSOL-HB} command.\index{\texttt{.OPTIONS}!\texttt{LINSOL-HB}}

% Sandia National Laboratories is a multimission laboratory managed and
% operated by National Technology & Engineering Solutions of Sandia, LLC, a
% wholly owned subsidiary of Honeywell International Inc., for the U.S.
% Department of Energy’s National Nuclear Security Administration under
% contract DE-NA0003525.

% Copyright 2002-2019 National Technology & Engineering Solutions of Sandia,
% LLC (NTESS).


%%
%% Linear Solver Options Table
%%
\index{solvers!linear!options}
\label{HBLinSolPKG}
\begin{OptionTable}{Options for Linear Solver Package for HB.}  \\ \hline

TYPE & Determines which linear solver will be used.
\begin{XyceItemize}
\item AztecOO 
\item Belos
\item Direct
\end{XyceItemize}
&
AztecOO\\ \hline

DIRECT\_SOLVER & Determines which direct linear solver will be used.
\begin{XyceItemize}
\item LAPACK
\end{XyceItemize}
&
LAPACK\\ \hline

AZ\_kspace & Maximum size of Krylov subspace & 50 \\ \hline

AZ\_max\_iter & Maximum number of iterative solver iterations & 200 \\ \hline

AZ\_tol & Convergence tolerance & 1.0E-9 \\ \hline

\texttt{prec\_type} & Preconditioning type & block\_jacobi \\ \hline

\end{OptionTable}

%%% Local Variables:
%%% mode: latex
%%% End:


\subsubsection{\texttt{.OPTIONS LINSOL-AC} (Linear Solver Options)}

For AC analysis, \Xyce{} provides both iterative and direct methods for the
solution of the linear equations. For the advanced users, there are a variety 
of options that can be set to help improve these solvers.  Transformations 
of the linear system have a ``\verb+TR_+'' prefix on the flag.  Many of the 
options for the Krylov solvers are simply passed through to the underlying
Trilinos/AztecOO\index{Trilinos}\index{solvers!linear!Trilinos}\index{Aztec}\index{solvers!linear!Aztec}
solution settings and thus have an ``\verb+AZ_+'' prefix on the flag.

Linear solver options are set using the \texttt{.OPTIONS LINSOL-AC}
command.\index{\texttt{.OPTIONS}!\texttt{LINSOL-AC}}  The available options
are the same as those for \texttt{.OPTIONS LINSOL}.


\subsubsection{\texttt{.OPTIONS OUTPUT} (Output Options)}

The \index{results!output options} \index{\texttt{.OPTIONS}!\texttt{OUTPUT}} \verb+.OPTIONS OUTPUT+
command can be used to allow control of the output frequency of data to files specified
by \index{\texttt{.TRAN}} \verb+.PRINT TRAN+ commands.  

One method is to specify output intervals.  The format for this method is:
\begin{vquote}
.OPTIONS OUTPUT INITIAL_INTERVAL=<interval> [<t0> <i0> [<t1> <i1>]* ]
\end{vquote}
where \verb+INITIAL_INTERVAL=<interval>+ specifies the starting interval time
for output and \verb+<tx> <ix>+ specifies later simulation times \verb+<tx>+
where the output interval will change to \verb+<ix>+. The solution is output at the
exact intervals requested; this is done by interpolating the solution
to the requested time points.

Another useful method for controlling the output frequency is to specify discrete output
points.  
\begin{vquote}
.OPTIONS OUTPUT OUTPUTTIMEPOINTS=<t0>,<t1>,* 
\end{vquote}
If this option is used, then only the specified time points will appear in the output file.
No other points will be output, so files using this method can be very sparse.  For this type
of output, the output values are not interpolated.  Instead, the specified output points are 
set as breakpoints in the time integrator, so the output values are computed directly.

In addition to controlling the frequency of output, it is also possible to use
output options to suppress the header from standard format output files, and the footer
from both standard and tecplot format output files.
\begin{vquote}
.OPTIONS OUTPUT PRINTHEADER=<boolean> PRINTFOOTER=<boolean>
\end{vquote}
where setting the \texttt{PRINTHEADER} variable to ``false'' will suppress the header and 
\texttt{PRINTFOOTER} variable to ``false'' will suppress the footer.  The \texttt{PRINTHEADER}
option is only applicable to \texttt{.PRINT <analysis> FORMAT=<STD|GNUPLOT|SPLOT>} files.
The \texttt{PRINTFOOTER} option is only applicable to
\texttt{.PRINT <analysis> FORMAT=<STD|GNUPLOT|SPLOT|TECPLOT>} files.

It is possible to add a \texttt{STEPNUM} column as the first column in the output file.
\begin{vquote}
.OPTIONS OUTPUT ADD\_STEPNUM\_COL=<boolean>
\end{vquote}
where setting the \texttt{ADD\_STEPNUM\_COL} variable to ``true'' will add the
\texttt{STEPNUM} column.  The default is ``false''. This option is applicable to
\texttt{FORMAT=<STD|NOINDEX|GNUPLOT|SPLOT>} for any \texttt{.PRINT} line that
supports \texttt{FORMAT=STD} output.

The default \Xyce{} output for phase operators, such as \texttt{VP()}, \texttt{IP()},
\texttt{SP()}, \texttt{YP()} and \texttt{ZP()}, is in degrees.  For compatibility with
other simulators like Spice3f5 and ngspice, it is possible to change that operator
output to use radians instead:
\begin{vquote}
.OPTIONS OUTPUT PHASE\_OUTPUT\_RADIANS=<boolean>
\end{vquote}
The default value for this option is FALSE.  If set to TRUE then the phase output
will be in radians instead of degrees.  This option also applies to the format
for AC sensitivity output.  It does not affect the output from a \texttt{.FOUR}
analysis or a texttt{.FOUR} measure though.  Those two outputs are always in degrees.

\subsubsection{\texttt{.OPTIONS RESTART} (Checkpointing Options)}

The \index{restart} \index{\texttt{.OPTIONS}!\texttt{RESTART}} \verb+.OPTIONS RESTART+ command is
used to control all \index{checkpoint} checkpoint output and restarting.

The checkpointing form of the \texttt{.OPTIONS RESTART} command takes the following format:
\Format{\par\tt .OPTIONS RESTART [PACK=<0|1>] JOB=<job prefix> \linebreak
+ [INITIAL\_INTERVAL=<initial interval time> [<t0> <i0> [<t1> <i1>]* ]]}

\texttt{PACK=<0|1>} indicates whether the restart data will be byte packed
or not.  Parallel restarts must always be packed while Windows/MingW
runs are always not packed.  Otherwise, data will be packed by default unless
explicitly specified.
\texttt{JOB=<job prefix>} identifies the prefix for restart files.  The
actual restart files will be the job name with the current simulation time
appended (e.g. \texttt{name1e-05} for \texttt{JOB=name} and simulation time
1e-05 seconds).  Furthermore, \texttt{INITIAL\_INTERVAL=<initial interval
  time>} identifies the initial interval time used for restart output.  The
\texttt{<tx> <ix>} intervals identify times \texttt{<tx>} at which the output
interval \texttt{(ix)} should change.  This functionality is identical to
that described for the \texttt{.OPTIONS OUTPUT} command.

\paragraph{Examples}

To generate checkpoints at every time step (default):

\Example{\texttt{.OPTIONS RESTART JOB=checkpt}}

To generate checkpoints every 0.1 $\mu s$:

\Example{\texttt{.OPTIONS RESTART JOB=checkpt INITIAL\_INTERVAL=0.1us}}

To generate unpacked checkpoints every 0.1 $\mu s$:

\Example{\texttt{.OPTIONS RESTART PACK=0 JOB=checkpt INITIAL\_INTERVAL=0.1us}}

To specify an initial interval of 0.1 $\mu s$, at 1 $\mu s$ change to interval
of 0.5 $\mu s$, and at 10 $\mu s$ change to interval of 0.1 $\mu s$:

\Example{\par\texttt{.OPTIONS RESTART JOB=checkpt INITIAL\_INTERVAL=0.1us 1.0us\linebreak + 0.5us 10us 0.1us}}

\subsubsection{\texttt{.OPTIONS RESTART} (Restarting Options)}

To restart from an existing restart file\index{restart!file}, specify the file
by either \texttt{FILE=<restart file name>} to explicitly use a restart file or
by \texttt{JOB=<job name> START\_TIME=<specified name>} to specify a file
prefix and a specified time.  The time must exactly match an output file time
for the simulator to correctly identify the correct file.  To continue
generating restart output files, \texttt{INITIAL\_INTERVAL=<interval>} and
following intervals can be appended to the command in the same format as
described above.  New restart files will be packed according to the previous
restart file read in.  

The restarting form of the \texttt{.OPTIONS RESTART} command takes the following format:

\Format{\par\tt .OPTIONS RESTART FILE=<restart file name>|JOB=<job name> START\_TIME=<time> \linebreak
    + [ INITIAL\_INTERVAL=<interval> [<t0> <i0> [<t1> <i1>]* ]]}

\paragraph{Examples}

Example restarting from checkpoint file at 0.133 $\mu s$:
\Example{\texttt{.OPTIONS RESTART JOB=checkpt START\_TIME=0.133us}}

To restart from checkpoint file at 0.133 $\mu s$:
\Example{\texttt{.OPTIONS RESTART FILE=checkpt0.000000133}}

Restarting from 0.133 $\mu s$ and continue checkpointing at 0.1 $\mu s$
intervals:
\Example{\par\texttt{
    .OPTIONS RESTART FILE=checkpt0.000000133 JOB=checkpt\_again\linebreak
    + INITIAL\_INTERVAL=0.1us
  }}

\subsubsection{\texttt{.OPTIONS RESTART}: special notes for use with two-level-Newton}
\index{restart!two-level}

Large parallel problems which involve power supply parasitics often
require a two-level solve, in which different parts of the problem
are handled separately.  In most respects, restarting a two-level
simulation is similar to restarting a conventional simulation.
However, there are a few differences:

\begin{XyceItemize}
\item When running with a two-level algorithm, \Xyce{} requires (at least) two
different input files.  In order to do a restart of a two-level \Xyce{}
simulation, it is necessary to have an \texttt{.OPTIONS RESTART} statement
in each file.

\item It is necessary for the statements to be consistent.  For example,
the output times must be exactly the same, meaning the initial intervals
must be exactly the same.

\item \Xyce{} will \emph{not} check to make sure that the restart
options used in different files match, so it is up to the user to ensure
matching options.

\item Finally, as each netlist that is part of a two-level solve will have its
own \texttt{.OPTIONS RESTART} statement, that means that each netlist
will generate and/or use its own set of restart files.  As a result,
the restart file name used by each netlist must be unique.
\end{XyceItemize}

\subsubsection{\texttt{.OPTIONS SAMPLES} (Sampling options)}
\index{solvers!sampling!options}

The sampling selections listed in Table~\ref{SamplesPKG}
provide methods for controlling Monte Carlo and Latin Hypercube Sampling methods.

SAMPLES options are set using the \texttt{.OPTIONS SAMPLES} command.\index{\texttt{.OPTIONS}!\texttt{SAMPLES}} 
They are only used if the netlist also includes a \texttt{.SAMPLING} statement. 

% Sandia National Laboratories is a multimission laboratory managed and
% operated by National Technology & Engineering Solutions of Sandia, LLC, a
% wholly owned subsidiary of Honeywell International Inc., for the U.S.
% Department of Energy’s National Nuclear Security Administration under
% contract DE-NA0003525.

% Copyright 2002-2020 National Technology & Engineering Solutions of Sandia,
% LLC (NTESS).


%%
%% Sampling Options Table
%%
\index{solvers!sampling!options}
\begin{OptionTable}{Options for Sampling Package.} \label{SamplesPKG}
NUMSAMPLES   & Total number of samples & 0 \\ \hline
SAMPLE\_TYPE & Sampling type (MC or LHS) & MC \\ \hline
OUTPUTS      & Comma separated list of outputs (anything that would be a valid \texttt{.RESULT} output command) & -- \\ \hline
MEASURES     & Comma separated list of measure names (must refer to \texttt{.MEASURE} commands in the netlist) & -- \\ \hline
COVMATRIX    & Covariance matrix specified in row major form as comma-separated double precision numbers. & -- \\ \hline
SEED         & Random seed & See footnote.\footnotemark[1] \\ \hline
\end{OptionTable}

\footnotetext[1]{The seed can also be set using command line option, -randseed.   The command line seed will override the netlist seed value.  If the seed is not set in either the netlist or on the command line, then Xyce generates a seed internally.  In all cases, \Xyce{} will output text to the console indicating what seed is being used.}  



\subsubsection{\texttt{.OPTIONS EMBEDDEDSAMPLES} (Embedded Sampling options)}
\index{solvers!embeddedsampling!options}

The sampling selections listed in Table~\ref{EmbeddedSamplesPKG}
provide methods for controlling Embedded Sampling methods.

EMBEDDEDSAMPLES options are set using the \texttt{.OPTIONS EMBEDDEDSAMPLES}
command.\index{\texttt{.OPTIONS}!\texttt{EMBEDDEDSAMPLES}}  They are only used if the
netlist also includes a \texttt{.EMBEDDEDSAMPLING} statement.

% Sandia National Laboratories is a multimission laboratory managed and
% operated by National Technology & Engineering Solutions of Sandia, LLC, a
% wholly owned subsidiary of Honeywell International Inc., for the U.S.
% Department of Energy’s National Nuclear Security Administration under
% contract DE-NA0003525.

% Copyright 2002-2021 National Technology & Engineering Solutions of Sandia,
% LLC (NTESS).

%%
%% Embedded Sampling Table
%%
\index{solvers!embeddedsampling!options}
\begin{OptionTable}{Options for Embedded Sampling Package.} \label{EmbeddedSamplesPKG}
NUMSAMPLES   & Total number of samples & 0 \\ \hline
SAMPLE\_TYPE & Sampling type (MC or LHS) & MC \\ \hline
OUTPUTS      & Comma separated list of outputs (anything that would be a valid \texttt{.RESULT} output command) & -- \\ \hline
COVMATRIX    & Covariance matrix specified in row major form as comma-separated double precision numbers. & -- \\ \hline
SEED         & Random seed & See footnote.\footnotemark[1] \\ \hline
\end{OptionTable}

\footnotetext[1]{The seed can also be set using command line option, -randseed.   The command line seed will override the netlist seed value.  If the seed is not set in either the netlist or on the command line, then Xyce generates a seed internally.  In all cases, \Xyce{} will output text to the console indicating what seed is being used.}



\subsubsection{\texttt{.OPTIONS PCES} (Fully intrusive PCE options)}
\index{solvers!pce!options}
\index{solvers!PCE!options}

The sampling selections listed in Table~\ref{PCEPKG}
provide methods for controlling Embedded Sampling methods.

PCES options are set using the \texttt{.OPTIONS PCES}
command.\index{\texttt{.OPTIONS}!\texttt{PCES}}  They are only used if the
netlist also includes a \texttt{.PCE} statement.

% Sandia National Laboratories is a multimission laboratory managed and
% operated by National Technology & Engineering Solutions of Sandia, LLC, a
% wholly owned subsidiary of Honeywell International Inc., for the U.S.
% Department of Energy’s National Nuclear Security Administration under
% contract DE-NA0003525.

% Copyright 2002-2022 National Technology & Engineering Solutions of Sandia,
% LLC (NTESS).

%%
%% Fully intrusive PCE Table
%%
\index{solvers!pce!options}
\begin{OptionTable}{Options for PCE Package.} \label{PCEPKG}
OUTPUTS      & Comma separated list of outputs (anything that would be a valid \texttt{.PRINT} output variable) & -- \\ \hline
COVMATRIX    & Covariance matrix specified in row major form as comma-separated double precision numbers. & -- \\ \hline
SAMPLE\_TYPE & Sampling type (MC or LHS). This is only used if resampling is enabled. & MC \\ \hline
SEED         & Random seed. This is only used if resampling is enabled. & See footnote.\footnotemark[1] \\ \hline
OUTPUT\_SAMPLE\_STATS   &  Compute and outputs statistics for specified outputs. & -- \\ \hline
RESAMPLE    & Once the PCE coefficients are obtained, perform sampling on the PCE approximation & -- \\ \hline
OUTPUT\_PCE\_COEFFS    & Output the PCE coefficients & -- \\ \hline
SPARSE\_GRID    &  Use sparse grid methods if using projection PCE.    & -- \\ \hline
STDOUTPUT    &  Send sampling and PCE output to the terminal   & -- \\ \hline
\end{OptionTable}

\footnotetext[1]{The seed can also be set using command line option, -randseed.   The command line seed will override the netlist seed value.  If the seed is not set in either the netlist or on the command line, then Xyce generates a seed internally.  In all cases, \Xyce{} will output text to the console indicating what seed is being used.}



\subsubsection{\texttt{.OPTIONS SENSITIVITY} (Direct and Adjoint Sensitivity Options)}
\index{solvers!sensitivty!options}

The sensitivity selections listed in Table~\ref{SensitivityPKG}
provide methods for controlling direct and adjoint sensitivity analysis.

SENSITIVITY options are set using the \texttt{.OPTIONS SENSITIVITY} command.\index{\texttt{.OPTIONS}!\texttt{SENSITIVITY}} 
They are only used if the netlist also includes a \texttt{.SENS} statement. 

% Sandia National Laboratories is a multimission laboratory managed and
% operated by National Technology & Engineering Solutions of Sandia, LLC, a
% wholly owned subsidiary of Honeywell International Inc., for the U.S.
% Department of Energy’s National Nuclear Security Administration under
% contract DE-NA0003525.

% Copyright 2002-2020 National Technology & Engineering Solutions of Sandia,
% LLC (NTESS).


%%
%% Sensitivity Solver Options Table
%%
\index{solvers!sensitivity!options}
\begin{OptionTable}{Options for Sensitivity Package.} \label{SensitivityPKG}
ADJOINT & Flag to enable adjoint sensitivity calculation & false \\ \hline
DIRECT & Flag to enable direct sensitivity calculation & false \\ \hline
OUTPUTSCALED & Flag to enable output of scaled sensitivities & false \\ \hline
OUTPUTUNSCALED & Flag to enable output of unscaled sensitivities & true \\ \hline
STDOUTPUT & Flag to enable output of sensitivies to std output & false \\ \hline
ADJOINTBEGINTIME & Start time for set of time steps over which to compute transient adjoints. & 0.0 \\ \hline
ADJOINTFINALTIME & End time for set of time steps over which to compute transient adjoints. & 1.0e+99 \\ \hline
ADJOINTTIMEPOINTS & List of comma-separated time points at which to compute transient adjoints. & -- \\ \hline
\end{OptionTable}



\subsubsection{\texttt{.OPTIONS HBINT} (Harmonic Balance Options)}
\index{solvers!hb!options}\index{\texttt{.OPTIONS}!\texttt{HBINT}}

The Harmonic Balance parameters listed in Table~\ref{hbPKG} give the available
options for helping control the algorithm for harmonic balance analysis.

Harmonic Balance options are set using the \texttt{.OPTIONS HBINT} command.

% Sandia National Laboratories is a multimission laboratory managed and
% operated by National Technology & Engineering Solutions of Sandia, LLC, a
% wholly owned subsidiary of Honeywell International Inc., for the U.S.
% Department of Energy’s National Nuclear Security Administration under
% contract DE-NA0003525.

% Copyright 2002-2022 National Technology & Engineering Solutions of Sandia,
% LLC (NTESS).


\index{solvers!hb!options}
\begin{OptionTable}{Options for HB.}
\label{hbPKG}

NUMFREQ & Number of harmonics to be calculated for each tone. It must have the same number of entries as .HB
statement & 10\\ \hline

STARTUPPERIODS & Number of periods to integrate through before calculating the initial conditions.  This option is only used when TAHB=1.& 0\\ \hline

SAVEICDATA & Write out the initial conditions to a file. & 0\\ \hline

TAHB &  This flag sets transient assisted HB. When TAHB=0, transient analysis is not performed to get an initial guess. When TAHB=1, it uses transient analysis to get an initial guess. For multi-tone HB simulation, the initial guess is generated by a single tone transient simulation. The first tone following \verb|.HB| is used to determine the period for the transient simulation.
For multi-tone HB simulation, it should be set to the frequency that produces the most nonlinear response 
by the circuit. When tahb = 2, the DC op is used as an initial guess & 1  \\ \hline

VOLTLIM &  This flag sets voltage limiting for HB. During the initial guess calculation, which normally uses transient simulation, the voltage limiting flag is determined by .options device voltlim. During the HB phase, the voltage limiting flag is determined by .options hbint voltlim. & 1 \\ \hline

INTMODMAX & The maximum intermodulation product order used in the spectrum. & 
the largest value in the NUMFREQ list. \\ \hline

NUMTPTS & Number of time points in the output & The total number of frequencies (positive, negative and DC). \\ \hline 
\end{OptionTable}

%%% Local Variables:
%%% mode: latex
%%% End:


\subsubsection{\texttt{.OPTIONS DIST} (Parallel Distribution Options)}
\index{dist!options}\index{\texttt{.OPTIONS}!\texttt{DIST}}

The parameters listed in Table~\ref{distPKG} give the available
options for controlling the parallel distribution used in \Xyce{}.
There are three choices for distribution strategy.

The default distribution strategy is ``first-come, first-served''
(\texttt{STRATEGY=0}), which divides the devices found in the netlist
into equal sized groups (in the order they are parsed) and distributes
a group to each processor.  This does not take into account the
connectivity of the circuit or balance device model computation, and
therefore can exhibit parallel imbalance for post-layout circuits that
have a substantial portion of parasitic devices.

The ``flat round-robin'' strategy (\texttt{STRATEGY=1}) will generate
the same distribution as the default strategy, but every parallel
processor will participate in reading its portion of the netlist.
This strategy provides a more scalable setup than the default
strategy, but can only be applied to flattened (non-hierarchical)
netlists.

The ``device balanced'' strategy (\texttt{STRATEGY=2}) will evenly
divide each of the device types over the number of parallel
processors, so each processor will have a balanced number of each
model type.  This allieviates the parallel imbalance in the device
model computation that can be experienced with post-layout circuits.
However, it does not take into account the circuit connectivity, so
the communication will not be minimized by this strategy.

% Sandia National Laboratories is a multimission laboratory managed and
% operated by National Technology & Engineering Solutions of Sandia, LLC, a
% wholly owned subsidiary of Honeywell International Inc., for the U.S.
% Department of Energy’s National Nuclear Security Administration under
% contract DE-NA0003525.

% Copyright 2002-2022 National Technology & Engineering Solutions of Sandia,
% LLC (NTESS).


\index{dist!options}
\begin{OptionTable}{Options for Parallel Distribution.}
\label{distPKG}

STRATEGY & Parallel device distribution strategy
\begin{XyceItemize}
\item 0 (First-Come, First-Served)
\item 1 (Flat Round-Robin)
\item 2 (Device Balanced)
\end{XyceItemize}
& 0 \\ \hline
\end{OptionTable}

%%% Local Variables:
%%% mode: latex
%%% End:


\subsubsection{\texttt{.OPTIONS FFT} (FFT Options)}
\index{fft!options}\index{\texttt{.OPTIONS}!\texttt{FFT}}
The parameters listed in Table~\ref{fftPKG} give the available
options for controlling all of the \texttt{.FFT} statements in
a given \Xyce{} netlist.

If \texttt{FFT\_ACCURATE} is set to 1 (true), which is the default, then
\Xyce{} will insert breakpoints at the sample times requested by the
collection of \texttt{.FFT} lines in the netlist.  This has been found
to improve the accuracy of the \texttt{.FFT} analyses, at the possible
expense of simulation speed.  If \texttt{FFT\_ACCURATE} is set to
0 (false), then interpolation is used to determine the output variable
values at the specified sample times.  If the \texttt{-remeasure} command
line option is used to recalculate the \texttt{.MEASURE FFT} and/or
\texttt{.FFT} statements for a \texttt{.TRAN} analysis, then
\texttt{FFT\_ACCURATE} is set to 0 during the re-measure operation.  Finally,
if \texttt{.OPTIONS OUTPUT INITIAL\_INTERVAL} is used in the netlist
then \texttt{.OPTIONS FFT FFT\_ACCURATE} will also be set to 0.

If \texttt{FFTOUT} is set to 1 then additional metrics are output to both stdout
and the \verb+<netlistName>.fft0+ file for each \texttt{.FFT} line.  In
addition a sorted list of the 30 largest harmonics is output to stdout.
Those additional metrics are as follows, where Section \ref{FFT_metrics}
provides detailed definitions for these metrics:

\begin{itemize}
  \item Effective Number of Bits (ENOB)
  \item Spurious Free Dynamic Range (SFDR)
  \item Signal to Noise Ratio (SNR)
  \item Signal to Noise-and-Distortion Ratio (SNDR)
  \item Total Harmonic Distorion (THD)
\end{itemize}

The setting for \texttt{FFT\_MODE} is used to control whether the \Xyce{} FFT
processing and output are more compatible with HSPICE (0) or Spectre (1).
This setting affects the format of the window functions, the conversion from
two-sided to one-sided results, and whether the default output for the
magnitude values is normalized, or not.  Section~\ref{FFT_MODE} gives
more details.

% Sandia National Laboratories is a multimission laboratory managed and
% operated by National Technology & Engineering Solutions of Sandia, LLC, a
% wholly owned subsidiary of Honeywell International Inc., for the U.S.
% Department of Energy’s National Nuclear Security Administration under
% contract DE-NA0003525.

% Copyright 2002-2021 National Technology & Engineering Solutions of Sandia,
% LLC (NTESS).

\index{fft!options}
\begin{OptionTable}{Options for FFT.}
\label{fftPKG}

FFT\_ACCURATE & Insert breakpoints at the sample times requested by the
collection of .FFT lines in the netlist & 1 (true) \\ \hline

FFTOUT & Output additional metrics (ENOB, SFDR, SNR, SNDR and THD) for
each .FFT line & 0 (false) \\ \hline

FFT\_MODE & Controls whether the FFT calculations and output format are
more compatible with HSPICE (0) or Spectre (1) & 0 \\ \hline
\end{OptionTable}

%%% Local Variables:
%%% mode: latex
%%% End:


\subsubsection{\texttt{.OPTIONS MEASURE} (Measure Options)}
\index{measure!options}\index{\texttt{.OPTIONS}!\texttt{MEASURE}}
The parameters listed in Table~\ref{measurePKG} give the available
options for controlling all of the \texttt{.MEASURE} statements in
a given \Xyce{} netlist.  The \texttt{MEASDGT}, \texttt{MEASFAIL}
and \texttt{MEASOUT} options are included for HSPICE compatibility.

If given in the netlist, the setting for \texttt{MEASOUT} controls whether 
the \texttt{.mt\#} (or \texttt{.ms\#} or \texttt{.ma\#}) files are made (1) or not (0). 
The \texttt{MEASOUT} setting takes precedence over the \texttt{MEASPRINT} setting 
(which is a Xyce-specific option) if both are given in the netlist.
See Section \ref{Measure_Suppressing_Measure_Output} for more details then on 
how the \texttt{MEASPRINT} option interacts with the individual 
\texttt {.MEASURE} statements and the \texttt{-remeasure} command 
line option.

If given in the netlist, the setting for the \texttt{MEASDGT} overrides the 
\texttt{PRECISION} qualifiers given on individual \texttt{.MEASURE} lines. 
The default value for the \texttt{MEASDGT} option is different from in HSPICE.

The \Xyce{} behavior for failed measures can be controlled via the \texttt{MEASFAIL}
and \texttt{DEFAULT\_VAL} options, as well as with the \texttt{DEFAULT\_VAL} 
qualifiers on individual \texttt{.MEASURE} lines.  The order of precedence (from 
high-to-low) is the \texttt{MEASFAIL} option, the \texttt{DEFAULT\_VAL} option, 
and the \texttt{DEFAULT\_VAL} qualifier on individual \texttt{.MEASURE} lines.  
If the \texttt{MEASFAIL} option is given and set equal to 0 then \Xyce{} outputs 
``0'' in the \texttt{.mt\#} ( or \texttt{.ms\#} or \texttt{.ma\#}) files for a failed measure.  
If the \texttt{MEASFAIL} option is given and set equal to 1 (or any other non-zero 
value) then \Xyce{} outputs ``FAILED'' in the \texttt{.mt\#} ( or \texttt{.ms\#} or \texttt{.ma\#}) 
files for a failed measure.  If given in the netlist, the setting for the 
\texttt{DEFAULT\_VAL} option overrides the \texttt{DEFAULT\_VAL} qualifiers given 
on individual \texttt{.MEASURE} lines.  The \texttt{DEFAULT\_VAL} option and the 
\texttt{DEFAULT\_VAL} qualifiers can be set to any real number.  For all of these 
cases, \Xyce{} will print ``FAILED'' to the standard output for a failed measure.
As a final note, the \texttt{FOUR} measure is a special case since it produces multiline
output.  Failed \texttt{FOUR} measures will be reported as ``FAILED'' in the
\texttt{.mt\#} ( or \texttt{.ms\#} or \texttt{.ma\#}) files, irrespective of the various
\texttt{MEASFAIL} and \texttt{DEFAULT\_VAL} settings.

The \texttt{USE\_CONT\_FILES} option controls whether each \texttt{AC\_CONT},
\texttt{DC\_CONT}, \texttt{NOISE\_CONT} or \texttt{TRAN\_CONT} mode measure uses
a separate output file for its results, or not. Section~\ref{Measure_CONT_Measurement_Output}
provides more details and an example netlist for this options setting.

For backwards compatibility with previous \Xyce{} versions, \texttt{USE\_LTTM} has
been added.  This option defaults to 0, which uses the new version of the \texttt{TRIG-TARG}
measure; while setting it to 1 will use the old version of the \texttt{TRIG-TARG} measure
for all \texttt{TRIG-TARG} measures in the netlist.  If the \texttt{FRAC\_MAX} qualifier
is used on a \texttt{TRIG-TARG} line then \Xyce{} will automatically default to
\texttt{USE\_LTTM=1} for that particular measure line.  It is anticipated that this
option setting will be removed at some point.

% Sandia National Laboratories is a multimission laboratory managed and
% operated by National Technology & Engineering Solutions of Sandia, LLC, a
% wholly owned subsidiary of Honeywell International Inc., for the U.S.
% Department of Energy’s National Nuclear Security Administration under
% contract DE-NA0003525.

% Copyright 2002-2021 National Technology & Engineering Solutions of Sandia,
% LLC (NTESS).


\index{measure!options}
\begin{OptionTable}{Options for MEASURE.}
\label{measurePKG}

DEFAULT\_VAL & Default value for ``failed measures'' in the .mt\# 
( or .ms\# or .ma\#) files. & -1 \\ \hline
MEASDGT & Precision for all .MEASURE statements.  This value applies to the
output to both the .mt\# ( or .ms\# or .ma\#) files and the standard output. & 6 \\ \hline
MEASFAIL & Specify output format for failed measures & 1 \\ \hline
MEASOUT & Control whether the .mt0 file is made or not & 1 \\ \hline
MEASPRINT & Measure Output
\begin{XyceItemize}
\item ALL (Output measure information to both file(s) and stdout)
\item STDOUT (Output measure information to stdout only)
\item NONE (Suppress all measure output)
\end{XyceItemize}
& ALL \\ \hline
USE\_CONT\_FILES & Specifies whether ``continuous'' mode measures use
separate output files for each such measure & 1 (TRUE) \\ \hline
\end{OptionTable}

%%% Local Variables:
%%% mode: latex
%%% End:


\subsubsection{\texttt{.OPTIONS PARSER} (Parser Options)}
\index{parser!options}\index{\texttt{.OPTIONS}!\texttt{PARSER}}
The parameter listed in Table~\ref{parserPKG} gives the available
option for netlist parsing.

% Sandia National Laboratories is a multimission laboratory managed and
% operated by National Technology & Engineering Solutions of Sandia, LLC, a
% wholly owned subsidiary of Honeywell International Inc., for the U.S.
% Department of Energy’s National Nuclear Security Administration under
% contract DE-NA0003525.

% Copyright 2002-2020 National Technology & Engineering Solutions of Sandia,
% LLC (NTESS).


\index{parser!options}
\begin{OptionTable}{Options for Parsing.}
\label{parserPKG}
MODEL\_BINNING & Enable model binning during netlist parsing.  See 
Section \ref{modelCommand} for more details on how model binning 
works in \Xyce{}. & FALSE \\ \hline
\end{OptionTable}

%%% Local Variables:
%%% mode: latex
%%% End:



