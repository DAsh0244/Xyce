% Sandia National Laboratories is a multimission laboratory managed and
% operated by National Technology & Engineering Solutions of Sandia, LLC, a
% wholly owned subsidiary of Honeywell International Inc., for the U.S.
% Department of Energy’s National Nuclear Security Administration under
% contract DE-NA0003525.

% Copyright 2002-2019 National Technology & Engineering Solutions of Sandia,
% LLC (NTESS).



\index{\texttt{.SAVE}}
\index{output!save} \index{save operating point conditions}
Stores the operating point of a circuit in the specified file for use in subsequent simulations.  The data may be saved as \texttt{.IC} or \texttt{.NODESET} lines.

\begin{Command}

\format
\begin{alltt}
.SAVE [TYPE=<IC|NODESET>] [FILE=<filename>] [LEVEL=<all|none>]
+ [TIME=<save\_time>]
\end{alltt}

\examples
\begin{alltt}
.SAVE TYPE=IC FILE=mycircuit.ic
.SAVE TYPE=NODESET FILE=myothercircuit.ic

.include mycircuit.ic
\end{alltt}

\comments

  The file created by \texttt{.SAVE} will contain \texttt{.IC} or
  \texttt{.NODESET} lines containing all the voltage node values at the DC
  operating point of the circuit. The default \textrmb{TYPE} is
  \texttt{NODESET}. The default \texttt{filename} is
  \texttt{\emph{netlist.cir}.ic}.
  
  The resulting file may be used in subsequent simulations to obtain quick DC
  convergence simply by including it in the netlist, as in the third example
  line above.  \Xyce{} has no corresponding \texttt{.LOAD} statement.
  
  The \textrmb{LEVEL} parameter is included for compatability with HSPICE
  netlists. If \texttt{none} is specified, then no save file is created. The
  default \textrmb{LEVEL} is \texttt{all}.
  
  \textrmb{TIME} is also an HSPICE compatibility parameter. This is unsupported
  in \Xyce{}. \Xyce{} outputs the save file only at time=0.0.

\end{Command}

