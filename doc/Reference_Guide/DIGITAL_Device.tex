% Sandia National Laboratories is a multimission laboratory managed and
% operated by National Technology & Engineering Solutions of Sandia, LLC, a
% wholly owned subsidiary of Honeywell International Inc., for the U.S.
% Department of Energy’s National Nuclear Security Administration under
% contract DE-NA0003525.

% Copyright 2002-2022 National Technology & Engineering Solutions of Sandia,
% LLC (NTESS).

 
%%
%% Behavioral Digital Description Table
%%
\begin{Device}\label{YTypeDigitalDevice}

\device
\begin{alltt}
Y<type> <name> [low output node] [high output node]
+ [input reference node] <input node>* <output node>*
+  <model name> [device parameters]
\end{alltt}

\model
.MODEL <model name> DIG [model parameters]

\examples
\begin{alltt}
YAND MYAND in1 in2 out DMOD IC=TRUE
YNOT THENOT in out DMOD
YNOR ANOR2 vlo vhi vref in1 in2 out DDEF
.model DMOD DIG (
+ CLO=1e-12  CHI=1e-12
+ S0RLO=5  S0RHI=5  S0TSW=5e-9
+ S0VLO=-1  S0VHI=1.8
+ S1RLO=200  S1RHI=5  S1TSW=5e-9
+ S1VLO=1  S1VHI=3
+ RLOAD=1000
+ CLOAD=1e-12
+ VREF=0 VLO=0 VHI=3
+ DELAY=20ns )
.MODEL DDEF DIG
\end{alltt}

\parameters
\begin{Parameters}

\param{type}
Type of digital device.  Supported devices are: NOT, BUF, AND, NAND, OR, NOR, XOR,
NXOR, DFF, JKFF, TFF, DLTCH and ADD.  (Note: INV is now the preferred synonym for NOT.  
The NOT device type will be deprecated in future \Xyce{} releases.)  For Y-type digital
devices, all devices have two input nodes and one output node, except for NOT, DFF and
ADD.  NOT has one input and one output.  ADD has three inputs (in1, in2, carryIn) and 
two outputs (sumOut and carryOut).  DFF has four inputs (PREB, CLRB, Clock and
Data) and two outputs ($Q$ and $\bar{Q}$). TFF has two inputs (T and Clock) and two
outputs ($Q$ and $\bar{Q}$).  The TFF uses ``positive'' (``rising'') edge clocking.
The JKFF has five inputs (PREB, CLRB, Clock, J and K) and two outputs ($Q$ and $\bar{Q}$).
The JKFF uses ``negative'' (``falling'') edge clocking.
DLTCH has four inputs (PREB, CLRB, Enable and Data) and two outputs ($Q$ and $\bar{Q}$). 

\param{name} 
Name of the device instance.  This must be present, and when combined
with the \texttt{Y<type>}, must be unique in the netlist.  In the
examples, MYAND, THENOT and ANOR2 have been used as names for the three
devices.

\param{low output node}

Dominant node to be connected to the output node(s) to establish low
output state.  This node is connected to the output by a resistor and
capacitor in parallel, whose values are set by the model.  If specified
by the model, this node must be omitted from the instance line and a
fixed voltage \textrmb{VLO} is used instead.

\param{high output node}

Dominant node to be connected to the output node(s) to establish high
output state.  This node is connected to the output by a resistor and
capacitor in parallel, whose values are set by the model.  If specified
by the model, this node must be omitted from the instance line and a fixed
voltage \textrmb{VHI} is used instead.

\param{input reference node}

This node is connected to the input node by a resistor and capacitor in
parallel, whose values are set by the model.  Determination if the input
state is based on the voltge drop between the input node and this node.
If specified by the model, this node must be omitted from the instance line and
a fixed voltage \textrmb{VREF} is used instead.

\param{input nodes, output nodes}

Nodes that connect to the circuit.

\param{model name}

Name of the model defined in a .MODEL line.

\param{device parameters}

Parameter listed in Table~\ref{Digital_1_Device_Instance_Params} may be
provided as \texttt{<parameter>=<value>} specifications as needed.  For
devices with more than one output, multiple output initial states may be
provided as Boolean values in either a comma separated list (e.g.
IC=TRUE,FALSE for a device with two outputs) or individually 
(e.g. IC1=TRUE IC2=FALSE or IC2=FALSE).  Finally, the IC specification
must use TRUE and FALSE rather than T and F.


\end{Parameters}
\end{Device}

\paragraph{Device Parameters}

%%
%% Digital Device Param Table
%%
% This table was generated by Xyce:
%   Xyce -doc Digital 1
%
\index{behavioral digital!device instance parameters}
\begin{DeviceParamTableGenerated}{Behavioral Digital Device Instance Parameters}{Digital_1_Device_Instance_Params}
IC1 & Vector of initial values for output(s) & logical (T/F) & false \\ \hline
IC2 &  & -- & false \\ \hline
\end{DeviceParamTableGenerated}


\paragraph{Model Parameters}

%%
%% Digital Model Param Table
%%
% This table was generated by Xyce:
%   Xyce -doc Digital 1
%
\index{behavioral digital!device model parameters}
\begin{DeviceParamTableGenerated}{Behavioral Digital Device Model Parameters}{Digital_1_Device_Model_Params}
CHI & Capacitance between output node and high reference & F & 1e-06 \\ \hline
CLO & Capacitance between output node and low reference & F & 1e-06 \\ \hline
CLOAD & Capacitance between input node and input reference & F & 1e-06 \\ \hline
DELAY & Delay time of device & s & 1e-08 \\ \hline
RLOAD & Resistance between input node and input reference & $\mathsf{\Omega}$ & 1000 \\ \hline
S0RHI & Low state resitance between output node and high reference & $\mathsf{\Omega}$ & 100 \\ \hline
S0RLO & Low state resistance between output node and low reference & $\mathsf{\Omega}$ & 100 \\ \hline
S0TSW & Switching time transition to low state & s & 1e-08 \\ \hline
S0VHI & Maximum voltage to switch to low state & V & 1.7 \\ \hline
S0VLO & Minimum voltage to switch to low state & V & -1.5 \\ \hline
S1RHI & High state resistance between output node and high reference & $\mathsf{\Omega}$ & 100 \\ \hline
S1RLO & High state resistance between output node and low reference & $\mathsf{\Omega}$ & 100 \\ \hline
S1TSW & Switching time transition to high state & s & 1e-08 \\ \hline
S1VHI & Maximum voltage to switch to high state & V & 7 \\ \hline
S1VLO & Minimum voltage to switch to high state & V & 0.9 \\ \hline
VHI & Internal high state supply voltage & V & 0 \\ \hline
VLO & Internal low state supply voltage & V & 0 \\ \hline
VREF & Internal reference voltage for inputs & V & 0 \\ \hline
\end{DeviceParamTableGenerated}


\paragraph{Model Description}

The input interface model consists of the input node connected with a resistor and
capacitor in parallel to the digital ground node.  The values of these are: \textrmb{RLOAD}
and \textrmb{CLOAD}.  

The logical state of any input node is determined by comparing the voltage relative
to the reference to the range for the low and high state.  The range for the low
state is \textrmb{S0VLO} to \textrmb{S0VHI}.  Similarly, the range for the high state
is \textrmb{S1VLO} to \textrmb{S1VHI}.  The state of an input node will remain fixed as
long as its voltage stays within the voltage range for its current state.  That input node
will transition to the other state only when its state goes outside the range of its
current state.

The output interface model is more complex than the input model, but shares the same
basic configuration of a resistor and capacitor in parallel to simulate loading.  For
the output case, there are such connections to two nodes, the digital ground node and the
digital power node.  Both of these nodes must be specified on the instance line.

The capacitance to the high node is specified by \textrmb{CHI}, and the capacitance to the low
node is \textrmb{CLO}.  The resistors in parallel with these capacitors are variable, and have
values that depend on the state.  In the low state (S0), the resistance values are:
\textrmb{S0RLO} and \textrmb{S0RHI}.  In the high state (S1) ,the resistance values are: 
\textrmb{S1RLO} and \textrmb{S1RHI}.  Transition to the high state occurs exponentially over
a time of \textrmb{S1TSW}, and to the low state \textrmb{S0TSW}.

The device's delay is given by the model parameter \textrmb{DELAY}.  Any input changes
that affect the device's outputs are propagated after this delay.

Another caveat is that closely spaced input transitions to the \Xyce{} digital behavioral
models may not be accurately reflected in the output states.  In particular, input-state
changes spaced by more than \texttt{DELAY} seconds have independent effects on the output
states. However, two input-state changes (S1 and S2) that occur within \texttt{DELAY} seconds
(e.g., at time=t1 and time=t1+0.5*\texttt{DELAY}) have the effect of masking the effects
of S1 on the device's output states, and only the effects of S2 are propagated to the
device's output states.

\paragraph{DCOP Calculations for Flip-Flops and Latches}
The behavior of the digital devices during the DC Operating Point (DCOP) calculations
can be controlled via the \texttt{IC1} and \texttt{IC2} instance parameters and the
\texttt{DIGINITSTATE} device option.  See ~\ref{U_DEVICE} and ~\ref{Options_Reference} for 
more details on these instance parameters and device option.  

\paragraph{Converting Y-Type Digital Devices to U-Type Digital Devices}
\Xyce{} is migrating the digital behavioral devices to U devices.  The goal is increased
compatibility with PSpice netlists.  This subsection gives four examples of how to 
convert an existing \Xyce{} netlist using Y-type digital devices to the corresponding U device
syntaxes.  The conversion process depends on whether the device has a fixed number of inputs
or a variable number of inputs. In all cases, the the model parameters \textrmb{VREF},
\textrmb{VLO} and \textrmb{VHI} should be omitted from the U device model card.  For
U devices, the nodes \textrmb{vlo} and \textrmb{vhi} are always specified on the 
instance line.

Example 1: Fixed number of inputs, Y-device model card contains \textrmb{VREF},
\textrmb{VLO} and \textrmb{VHI}.  Assume \textrmb{VREF}=\textrmb{VLO}.

\begin{alltt}
YNOT THENOT in out DMOD
.model DMOD DIG (
+ CLO=1e-12  CHI=1e-12
+ S0RLO=5  S0RHI=5  S0TSW=5e-9
+ S0VLO=-1  S0VHI=1.8
+ S1RLO=200  S1RHI=5  S1TSW=5e-9
+ S1VLO=1  S1VHI=3
+ RLOAD=1000
+ CLOAD=1e-12
+ VREF=0 VLO=0 VHI=3
+ DELAY=20ns )

* Digital power node.  Assume digital ground node = GND
V1 DPWR 0 3V 
UTHENOT INV DPWR 0 in out DMOD1
.model DMOD1 DIG (
+ CLO=1e-12  CHI=1e-12
+ S0RLO=5  S0RHI=5  S0TSW=5e-9
+ S0VLO=-1  S0VHI=1.8
+ S1RLO=200  S1RHI=5  S1TSW=5e-9
+ S1VLO=1  S1VHI=3
+ RLOAD=1000
+ CLOAD=1e-12
+ DELAY=20ns )
\end{alltt}

Example 2: Fixed number of inputs, Y-device instance line contains \textrmb{vlo},
\textrmb{vhi} and \textrmb{vref}.  Assume \textrmb{vref}=\textrmb{vlo}.

\begin{alltt}
YNOT THENOT vlo vhi vref in out DMOD1
UTHENOT INV vhi vlo in out DMOD1
\end{alltt}

Example 3: Variable number of inputs, Y-device model card contains \textrmb{VREF},
\textrmb{VLO} and \textrmb{VHI}.  Assume \textrmb{VREF}=\textrmb{VLO}.

\begin{alltt}
YAND MYAND in1 in2 out DMOD
UMYAND AND(2) DPWR 0 in1 in2 out DMOD1
\end{alltt}

Example 4: Variable number of inputs, Y-device instance line contains \textrmb{vlo},
\textrmb{vhi} and \textrmb{vref}.  Assume \textrmb{vref}=\textrmb{vlo}.

\begin{alltt}
YAND MYAND vlo vhi vref in1 in2 out DMOD1
UMYAND AND(2) vhi vlo in1 in2 out DMOD1
\end{alltt}


