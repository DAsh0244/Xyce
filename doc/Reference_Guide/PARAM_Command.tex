% Sandia National Laboratories is a multimission laboratory managed and
% operated by National Technology & Engineering Solutions of Sandia, LLC, a
% wholly owned subsidiary of Honeywell International Inc., for the U.S.
% Department of Energy’s National Nuclear Security Administration under
% contract DE-NA0003525.

% Copyright 2002-2019 National Technology & Engineering Solutions of Sandia,
% LLC (NTESS).


%%
%% Function Table
%%

User defined parameter that can be used in expressions throughout the netlist.

\begin{Command}

\format
.PARAM [<name>=<value>]*

\examples
\begin{alltt}
.PARAM A_Param=1K
.PARAM B_Param=\{A_Param*3.1415926535\}
\end{alltt}

\arguments

\begin{Arguments}
\argument{\vbox{\hbox{name\hfil}\hbox{value}}}

The name of a parameter and its value.
\medskip

\end{Arguments}

\comments
Parameters defined using \verb+.PARAM+ are evaluated when the netlist is read in, and therefore must evaluate to constants at the time the netlist is parsed.  It is therefore illegal to use any time- or solution-dependent terms in parameter definitions, including the TIME variable or any nodal voltages.   Since they must be constants, these parameters may also not be used in \texttt{.STEP} loops.

There are several reserved words that may not be used as names for parameters.  These parameters are:
\begin{XyceItemize}
\item \verb+Time+ 
\item \verb+Vt+
\item \verb+Temp+
\item \verb+GMIN+
\end{XyceItemize}

\index{\texttt{.PARAM}!subcircuit scoping}The scoping rules for parameters are:
\begin{XyceItemize}
\item If a \texttt{.PARAM}, statement is included in the main circuit 
netlist, then it is accessible from the main circuit and all subcircuits. 
\item \texttt{.PARAM} statements defined within a subcircuit are scoped 
to that subciruit definition.  So, their parameters are only accessible within 
that subcircuit definition, as well as within ``nested subcircuits'' also 
defined within that subcircuit definition.
\end{XyceItemize}

Additional illustative examples of scoping are given in the
``Working with Subcircuits and Models'' section of the \Xyce{} Users' 
Guide\UsersGuide. 

\end{Command}

