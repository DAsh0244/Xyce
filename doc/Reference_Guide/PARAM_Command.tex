% Sandia National Laboratories is a multimission laboratory managed and
% operated by National Technology & Engineering Solutions of Sandia, LLC, a
% wholly owned subsidiary of Honeywell International Inc., for the U.S.
% Department of Energy’s National Nuclear Security Administration under
% contract DE-NA0003525.

% Copyright 2002-2021 National Technology & Engineering Solutions of Sandia,
% LLC (NTESS).


%%
%% Function Table
%%

User defined parameter that can be used in expressions throughout the netlist.

\begin{Command}

\format
.PARAM [<name>=<value>]*

\examples
\begin{alltt}
.PARAM A_Param=1K
.PARAM B_Param=\{A_Param*3.1415926535\}
\end{alltt}

\arguments

\begin{Arguments}
\argument{\vbox{\hbox{name\hfil}\hbox{value}}}

The name of a parameter and its value.  The value may be an expression.
\medskip

\end{Arguments}

\comments
Parameters defined using \texttt{.PARAM} only have a few restrictions on their 
  usage.  In earlier versions of \Xyce{} they were handled as constants that 
  were evaluated during parsing.  This is no longer the case, and parameters 
  can now have their values change throughout the calculation.   A \texttt{.PARAM} 
  defined in the top level netlist is equivalent to 
  a \texttt{.GLOBAL\_PARAM}, and they can be combined as needed.
Thus, you may use parameters defined by \texttt{.PARAM} in expressions used to
define global parameters, and you may also use global parameters in
\texttt{.PARAM} definitions.  

It is legal for parameters to depend on special variabls such as 
  TIME, FREQ, TEMP and VT variables.  However, it is not legal for parameters 
  to depend on solution variables such as voltage nodes or independent source currents.

Parameters defined using \texttt{.PARAM} can be modified directly by various analyses, such as \texttt{.DC},   
  \texttt{.STEP}, \texttt{.SAMPLING} and \texttt{.EMBEDDEDSAMPLING}, subject to scoping rules. 

To load an external data file with time voltage pairs of data on each 
line into a global parameter, use this syntax:

\texttt{.GLOBAL\_PARAM extdata = \{tablefile("filename")\}}

or

\texttt{.GLOBAL\_PARAM extdata = \{table("filename")\}}

where \texttt{filename} would be the name of the file to load.  
Other interpolators that can read in a data table from a file 
include \texttt{fasttable},\texttt{spline}, \texttt{akima}, \texttt{cubic}, 
\texttt{wodicka} and \texttt{bli}.  See \ref{ExpressionDocumentation} 
for further information.  

There are several reserved words that may not be used as names for parameters.  These reserved words are:
\begin{XyceItemize}
\item \verb+Time+
\item \verb+Freq+ 
\item \verb+Hertz+ 
\item \verb+Vt+
\item \verb+Temp+
\item \verb+Temper+
\item \verb+GMIN+
\end{XyceItemize}

\index{\texttt{.PARAM}!subcircuit scoping}The scoping rules for parameters are:
\begin{XyceItemize}
\item If a \texttt{.PARAM}, statement is included in the main circuit 
netlist, then it is accessible from the main circuit and all subcircuits. 
\item \texttt{.PARAM} statements defined within a subcircuit are scoped 
to that subciruit definition.  So, their parameters are only accessible within 
that subcircuit definition, as well as within ``nested subcircuits'' also 
defined within that subcircuit definition.
\item Parameters defined via \texttt{.PARAM} statements can be modified by the 
  various UQ analysis techniques (\texttt{.STEP}, \texttt{.SAMPLING}, etc) but
  this only works for \texttt{.PARAM} that have been defined in the top level netlist.
  Parameters defined inside of subcircuits cannot be modified directly 
  by these analyes, but they can be modified indirectly via dependence on other 
  globally scoped parameter.
\end{XyceItemize}

Additional illustative examples of scoping are given in the
``Working with Subcircuits and Models'' section of the \Xyce{} Users' 
Guide\UsersGuide. 

\end{Command}

