% Sandia National Laboratories is a multimission laboratory managed and
% operated by National Technology & Engineering Solutions of Sandia, LLC, a
% wholly owned subsidiary of Honeywell International Inc., for the U.S.
% Department of Energy’s National Nuclear Security Administration under
% contract DE-NA0003525.

% Copyright 2002-2020 National Technology & Engineering Solutions of Sandia,
% LLC (NTESS).


%%
%% PDE Device Description Table
%%
\small
\begin{longtable}[Hh]{>{\setlength{\hsize}{.3\hsize}}YY}
\hline

\underline{\bf General Form, 1D:} &
\verb|YPDE <name> <node> [node] [model name]|
\verb|[na=<value>] [nd=<value>] |
\verb|[nx=<value>] [area=<value>]|
\verb|[graded=<value>]|
\verb|[wj=<value>] [l=<value>] [w=<value>]|
\verb|[tecplotlevel=<value>]|
\verb|[gnuplotlevel=<value>]|
\verb|[node=<tabular data>]|
\verb|[region=<tabular data>]|
\verb|[bulkmaterial=<string>]|
\verb|[temp=<value>]|
\\ 
\hline
\\
\hline
\underline{\bf General Form, 2D:} &
\verb|YPDE <name> <node> <node> [node][node] [model name]|
\verb|[na=<value>] [nd=<value>] |
\verb|[meshfile=<filename.msh>]|
\verb|[nx=<value>][ny=<value>]|
\verb|[l=<value>][w=<value>]|
\verb|[type=<string>|
\verb|[node=<tabular data>]|
\verb|[region=<tabular data>]|
\verb|[x0=<value>] [cyl=<value>]|
\verb|[tecplotlevel=<value>]|
\verb|[gnuplotlevel=<value>] [txtdatalevel=<value>]| 
\verb|[ph.a1=<value>] [ph.type=<string>]|
\verb|[ph.tstart=<value>] [ph.tstop=<value>]|
\verb|[photogen=<value>]|
\verb|[ph.td=<value>] [ph.tr=<value>]|
\verb|[ph.tf=<value>] [ph.pw=<value>]|
\verb|[ph.per=<value>]|
\verb|[bulkmaterial=<string>]|
\verb|[temp=<value>]|

\\ \hline
\\
\hline
\underline{\bf Comments:} & Most of the PDE parameters are specified on the
instance level. At this point the model statement is only used for specifying
if the device is 1D or 2D, via the level parameter.  Both the 1D and the 
2D devices can construct evenly spaced meshes, internally. The 2D device also 
has the option of reading in an unstructured mesh from an external mesh file.  
\linebreak \linebreak
The electrode tabular data specification is explained in detail in 
table~\ref {PDE_Electrode_Params} Similarly, the doping region tabular 
data specification is explained in detail in 
table~\ref{PDE_Doping_Params}. \linebreak
\\ \hline
\end{longtable}
% Eric Keiter.  1/19/04:
% As this table does not have a caption, it should not have a number
% assigned to it.  This line is needed to fix latex so that the tables in
% chapter 2, which actually have their numbers printed in table captions, 
% and the list of tables,  do not start at 2.9, but start at 2.1.
\addtocounter{table}{-1}


