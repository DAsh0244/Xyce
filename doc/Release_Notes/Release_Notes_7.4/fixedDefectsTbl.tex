% Sandia National Laboratories is a multimission laboratory managed and
% operated by National Technology & Engineering Solutions of Sandia, LLC, a
% wholly owned subsidiary of Honeywell International Inc., for the U.S.
% Department of Energy’s National Nuclear Security Administration under
% contract DE-NA0003525.

% Copyright 2002-2021 National Technology & Engineering Solutions of Sandia,
% LLC (NTESS).

% Sandia National Laboratories is a multimission laboratory managed and
% operated by National Technology & Engineering Solutions of Sandia, LLC, a
% wholly owned subsidiary of Honeywell International Inc., for the U.S.
% Department of Energy’s National Nuclear Security Administration under
% contract DE-NA0003525.

% Copyright 2002-2021 National Technology & Engineering Solutions of Sandia,
% LLC (NTESS).


%%
%% Fixed Defects.
%%
{
\small

\begin{longtable}[h] {>{\raggedright\small}m{2in}|>{\raggedright\let\\\tabularnewline\small}m{3.5in}}
     \caption{Fixed Defects.  Note that we have multiple issue
     tracking systems for Sandia users.  SON, which bugzilla on the
     open network, and SRN, which is bugzilla on the restricted
     network.  We are also transitioning from bugzilla to gitlab issue
     tracking.  Further, some issues are reported by open source users
     on GitHub and these issues may be tracked using multiple issue
     numbers.} \\ \hline
     \rowcolor{XyceDarkBlue} \color{white}\textbf{Defect} & \color{white}\textbf{Description} \\ \hline
     \endfirsthead
     \caption[]{Fixed Defects.  Note that we have two multiple issue tracking systems for Sandia Users.
     SON and SRN refer to our legacy open- and restricted-network Bugzilla system, and Gitlab refers to issues in our gitlab repositories.  } \\ \hline
     \rowcolor{XyceDarkBlue} \color{white}\textbf{Defect} & \color{white}\textbf{Description} \\ \hline
     \endhead
\textbf{placeholder issue number}: placeholder description &
     placeholder explanation \\ \hline
\textbf{Gitlab-ex issue 274}: Xyce/ADMS generages incorrect code for derivatives of ``min'' function &
A logic error in the Xyce/ADMS code generation back-end was producing
incorrect code for derivatives of expressions involving the ``min''
function whose arguments were dependent on probes.  This had minor
impact on the convergence behavior some internal models in \Xyce{} due
to errors in the Jacobians of those models that used this
function.  \\ \hline
\textbf{Gitlab-ex issue 264}: Add multiplicity to levels 1 and 2 diodes &
 An instance parameter ``M'' has been added to the level 1 and 2
diodes to implement multiplicity (number of devices in parallel).
This instance parameter must not be confused with the model parameter
of the same name, which implements the P-N junction grading
parameter. \\ \hline

\textbf{Gitlab-ex issue 257}: Improve error handling for invalid
FREQ values on .FOUR lines and invalid AT values on FOUR measure
lines & If a \Xyce{} user entered an invalid \texttt{FREQ} value
on a \texttt{.FOUR} line or an invalid \texttt{AT} value on a
\texttt{FOUR} measure line then that would cause the simulation
to exit before all of the post-processing output files had been
made at the end of the transient simulation. \\ \hline

\textbf{Gitlab-ex issue 243/Github issue 29}: I(*) not expanded in external output &
The design of the code that handled output wildcards involving
currents (including I(*), P(*), and W(*)) was such that wildcards
requested by external output interfaces through the library API would
not have currents provided unless a \texttt{.PRINT} line in the
netlist also requested that wildcard.  The handling of this feature
has been reworked so that wildcards in external output requests
function correctly irrespective of any \texttt{.PRINT} lines in the
netlist, and even if there are no \texttt{.PRINT} lines at
all.\\ \hline

\textbf{Gitlab-ex issue 204/Github issue 24}: Interlibrary dependency problems &
For user-compiled and installed Xyce builds beginning with release 6.1
and up to release 7.3, Xyce created and installed multiple,
interdependent libraries such as ``libxyce'', ``libADMS'' and
``libNeuronModels'', all of which had to be linked into any user code
that used Xyce.  As of release 7.4, only libxyce is installed, and it
contains all functions previously placed into the other libraries.  If
you are building codes to be linked against Xyce libraries, you must
simplify your link line to link in just ``-lxyce''.\\ \hline

  \textbf{Gitlab-ex issue 221}: \texttt{.OPTIONS MEASURE MEASFAIL=1}
does not work properly for successful \texttt{TRIG-TARG} measures &
If \texttt{.OPTIONS MEASURE MEASFAIL=1} was used in a netlist then the
results for a successful \texttt{TRIG-TARG} measure would be correct
in the descriptive information sent to stdout.  However, the measure's
value would be shown as FAILED in the \texttt{<netlistName>.mt0}
file. \\ \hline

  \textbf{Gitlab-ex issue 229}: Parsing error when V() expression
contains extraneous trailing spaces & The new expression library did
not handle whitespace correctly, in certain contexts.  This has been
fixed in the expression lexer.  \\ \hline

  \textbf{Gitlab-ex issue 235}: Extend the R=0 capability to allow use
of resistor .MODELs & Resistors that have a resistance value of zero
are treated different inside of Xyce.  This different handling did not
allow for such resistors to refer to \texttt{.MODEL} statements.  That
has been fixed and now this use case works.  \\ \hline

  \textbf{Gitlab-ex issue 239}: \texttt{.options parser scale} doesn't
play nice with dependent parameters & If Xyce was used with a PDK that
used \texttt{.option scale} (or Xyce's version \texttt{.options parser
scale}), Xyce would apply this scale factor in the
Instance::processParams function of the relevant MOSFET devices.
However, if the device instance contained any dependent parameters,
then every time those parameters were updated, this function was
called again, and thus the scale factor was applied each time. This
resulted in the wrong L and W values.  This has been corrected and
scale factors are now applied no more than once.  \\ \hline

  \textbf{Gitlab-ex issue 245}: An endless loop in parsing can be caused 
if a \texttt{.SUBCKT} in an include file is not terminated properly &
If a subcircuit was defined in an include file that was not terminated
properly, the Xyce parser would continue parsing without acknowledging
that the subcircuit had ended.  This would result in an error in the
parsing later or an endless loop.  This has been corrected. \\ \hline
 
  \textbf{Gitlab-ex issue 262}: Bug in \texttt{.PREPROCESS ADDRESISTORS} 
logic that generates file output & If \texttt{.PREPROCESS ADDRESISTORS}
is used in a file to add resistors to one-terminal connections or to
make DC paths to ground, then all \texttt{.PREPROCESS} calls are 
commented out in the file outputted that includes those resistors.
This included \texttt{.PREPROCESS} calls that are executed to replace
ground synonyms or remove unused devices, which need to be called each
time the circuit is simulated.  This no longer happens. \\ \hline

  \textbf{1322-SON}: Preserve ordering of .PRINT line wildcards &
In previous \Xyce{} versions, these two \texttt{.PRINT} lines would
both output the \texttt{V(*)} wildcard results before the \texttt{I(*)}
wildcard results.
\begin{verbatim}
.PRINT TRAN V(*) I(*)
.PRINT TRAN I(*) V(*)
\end{verbatim}
\\ \hline

  \textbf{1145-SON}: \texttt{.GLOBAL\_PARAM} does not work
in \texttt{.IC} statements & This has been resolved and global
parameters can be used to set the value of an \texttt{.IC}
statement.\\ \hline

  \textbf{1029-SON}: Improve checking of \texttt{.PRINT} type versus
analysis type & The checking of the \texttt{.PRINT} type versus the
analysis type had some deficiencies. It would falsely report an error
if a \texttt{.PRINT TRANADJOINT} line preceeded the \texttt{.PRINT
TRAN} line.  It could falsely not report an error if the \texttt{.OP}
statement preceeded the primary analysis statement (e.g.,
\texttt{.TRAN}), or if a valid \texttt{.PRINT} line was the first \texttt{.PRINT}
line in the netlist but a subsequent \texttt{.PRINT} line was not allowed
for the requested primary analysis type. \\ \hline

\end{longtable}
}
