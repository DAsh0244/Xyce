% Sandia National Laboratories is a multimission laboratory managed and
% operated by National Technology & Engineering Solutions of Sandia, LLC, a
% wholly owned subsidiary of Honeywell International Inc., for the U.S.
% Department of Energy’s National Nuclear Security Administration under
% contract DE-NA0003525.

% Copyright 2002-2021 National Technology & Engineering Solutions of Sandia,
% LLC (NTESS).

% Sandia National Laboratories is a multimission laboratory managed and
% operated by National Technology & Engineering Solutions of Sandia, LLC, a
% wholly owned subsidiary of Honeywell International Inc., for the U.S.
% Department of Energy’s National Nuclear Security Administration under
% contract DE-NA0003525.

% Copyright 2002-2021 National Technology & Engineering Solutions of Sandia,
% LLC (NTESS).


%%
%% Fixed Defects.
%%
{
\small

\begin{longtable}[h] {>{\raggedright\small}m{2in}|>{\raggedright\let\\\tabularnewline\small}m{3.5in}}
     \caption{Fixed Defects.  Note that we have multiple issue
     tracking systems for Sandia users.  SON, which bugzilla on the
     open network, and SRN, which is bugzilla on the restricted
     network.  We are also transitioning from bugzilla to gitlab issue
     tracking.  Further, some issues are reported by open source users
     on GitHub and these issues may be tracked using multiple issue
     numbers.} \\ \hline
     \rowcolor{XyceDarkBlue} \color{white}\textbf{Defect} & \color{white}\textbf{Description} \\ \hline
     \endfirsthead
     \caption[]{Fixed Defects.  Note that we have two multiple issue tracking systems for Sandia Users.
     SON and SRN refer to our legacy open- and restricted-network Bugzilla system, and Gitlab refers to issues in our gitlab repositories.  } \\ \hline
     \rowcolor{XyceDarkBlue} \color{white}\textbf{Defect} & \color{white}\textbf{Description} \\ \hline
     \endhead
\textbf{placeholder issue number}: placeholder description &
     placeholder explanation \\ \hline
\textbf{Gitlab-ex issue 204/Github issue 24}: Interlibrary dependency problems &
     For user-compiled and installed Xyce builds beginning with
     release 6.1 and up to release 7.3, Xyce created and installed
     multiple, interdependent libraries such as ``libxyce'',
     ``libADMS'' and ``libNeuronModels'', all of which had to be
     linked into any user code that used Xyce.  As of release 7.4,
     only libxyce is installed, and it contains all functions
     previously placed into the other libraries.  If you are building
     codes to be linked against Xyce libraries, you must simplify your
     link line to link in just ``-lxyce''.\\ \hline

  \textbf{Gitlab-ex issue 204}: \texttt{.OPTIONS MEASURE MEASFAIL=1} does not work
properly for successful \texttt{TRIG-TARG} measures  & If \texttt{.OPTIONS MEASURE MEASFAIL=1}
was used in a netlist then the results for a successful \texttt{TRIG-TARG} measure
would be correct in the descriptive information sent to stdout.  However, the measure's
value would be shown as FAILED in the \texttt{<netlistName>.mt0} file. \\ \hline

  \textbf{Gitlab-ex issue 229}: Parsing error when V() expression contains extraneous trailing spaces &
The new expression library did not handle whitespace correctly, in certain contexts.  
  This has been fixed in the expression lexer.
  \\ \hline

  \textbf{Gitlab-ex issue 235}: Extend the R=0 capability to allow use of resistor .MODELs & 
Resistors that have a resistance value of zero are treated different inside of Xyce.  
  This different handling did not allow for such resistors to refer to \texttt{.MODEL} statements.  
  That has been fixed and now this use case works.
  \\ \hline

  \textbf{1145-SON}: \texttt{.GLOBAL\_PARAM} does not work in \texttt{.IC} statements &  
      This has been resolved and global parameters can be used to set the 
      value of an \texttt{.IC} statement.\\ \hline
\end{longtable}
}
