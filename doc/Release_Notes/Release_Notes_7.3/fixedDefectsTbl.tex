% Sandia National Laboratories is a multimission laboratory managed and
% operated by National Technology & Engineering Solutions of Sandia, LLC, a
% wholly owned subsidiary of Honeywell International Inc., for the U.S.
% Department of Energy’s National Nuclear Security Administration under
% contract DE-NA0003525.

% Copyright 2002-2020 National Technology & Engineering Solutions of Sandia,
% LLC (NTESS).

% Sandia National Laboratories is a multimission laboratory managed and
% operated by National Technology & Engineering Solutions of Sandia, LLC, a
% wholly owned subsidiary of Honeywell International Inc., for the U.S.
% Department of Energy’s National Nuclear Security Administration under
% contract DE-NA0003525.

% Copyright 2002-2020 National Technology & Engineering Solutions of Sandia,
% LLC (NTESS).


%%
%% Fixed Defects.
%%
{
\small

\begin{longtable}[h] {>{\raggedright\small}m{2in}|>{\raggedright\let\\\tabularnewline\small}m{3.5in}}
     \caption{Fixed Defects.  Note that we have two different Bugzilla systems for Sandia users.
     SON, which is on the open network, and SRN, which is on the restricted network. } \\ \hline
     \rowcolor{XyceDarkBlue} \color{white}\textbf{Defect} & \color{white}\textbf{Description} \\ \hline
     \endfirsthead
     \caption[]{Fixed Defects.  Note that we have two multiple issue tracking systems for Sandia Users.
     SON and SRN refer to our legacy open- and restricted-network Bugzilla system, and Gitlab refers to issues in our gitlab repositories.  } \\ \hline
     \rowcolor{XyceDarkBlue} \color{white}\textbf{Defect} & \color{white}\textbf{Description} \\ \hline
     \endhead

\textbf{ISSUE NUMBER}:  Placeholder title&
Placeholder description.
\\ \hline
\textbf{Gitlab-ex issue 128}: Fix Xyce/ADMS handling of some while and for loops &
Xyce/ADMS was emitting improper code for ``for'' or ``while'' loops
that contained what ADMS flagged as ``nonlinear'' conditions.  This
turned out to be due to an error in the implicit rules of ADMS that
has been present for at least 10 years, in which this unusual type of
condition resulted in an inappropriate double call to an early
dependency-tracking template.  The result was that contents of the
code block controlled by the loop construct was in backwards order,
and all function precomputation was happening twice (due to the
presence of duplicate entries in the ``function'' data structure).

Xyce/ADMS uses a modified version of ADMS's implicit rules file
instead of the one that ships with ADMS, and this error has now been
fixed in Xyce's modified version.

No standard Verilog-A models make use of ``for'' or ``while'' loops
that would have tripped this bug, but such a usage was found in a
Verilog-A model inside a proprietary vendor PDK. \\ \hline

\textbf{Gitlab-ex issue 103}: Fix glitches in transmission line at breakpoints &
The lossless transmission line is an ideal behavioral device that
operates by keeping a record of the histories of the two ends of the
line, and using interpolation on that history to look up prior
behavior at a time in the past in order to determine the current
behavior.  This interpolation is generally done using three-point
quadratic Lagrange interpolation.  It has been determined that Xyce
has historically been producing small ``glitches'' at the corners of
pulsed signals because quadratic interpolation is not really
appropriate with piecewise linear signals at the corners where the
slope changes signficantly.  Xyce now detects when such
discontinuities in derivative occur and performs linear interpolation
instead.  This prevents the small overshoots that have been observed
in these cases in all prior versions of Xyce. \\ \hline
\textbf{1203-SON}: Issues with temperature-dependent device
instance-parameters that were expression-based & The values for
temperature-dependent device instance parameters could be incorrect
if their expression used a global parameter. This could occur if the
default temperature of 27C was used.  It could also occur if the
temperature was set with a \texttt{.OPTIONS DEVICE} line or via
\texttt{.STEP}.  A simple example is:
\begin{verbatim}
.GLOBAL_PARAM GTEMP=37
R1 1 0 1  TC=0.1 temp={GTEMP}
\end{verbatim}
If the temperature was set with a \texttt{.OPTIONS DEVICE} line then
the use of a ``normal'' parameter in their expression could also give
an incorrect answer.  An example is:
\begin{verbatim}
.PARAM ptemp=10
R2 2 0 1  TC=0.1 temp={temp+ptemp}
\end{verbatim}

\\ \hline
\textbf{Gitlab-ex Issue 101, SON BUG 274}:  Remove deprecated parsing of quoted filename to table &
Double quoted strings were once parsed as filenames into a lookup table.  That syntax 
has been deprecated for some time and has now been removed. Use the \texttt{tablefile(``filename'')}
syntax to read a file into a table.  A double quoted string is 
now only interpreted as a string.  Additionally, the syntax \texttt{string(``'')} 
has also been removed as it was used to signify a string.
\\ \hline
\end{longtable}
}
