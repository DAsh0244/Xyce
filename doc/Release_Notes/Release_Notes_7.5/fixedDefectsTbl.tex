% Sandia National Laboratories is a multimission laboratory managed and
% operated by National Technology & Engineering Solutions of Sandia, LLC, a
% wholly owned subsidiary of Honeywell International Inc., for the U.S.
% Department of Energy’s National Nuclear Security Administration under
% contract DE-NA0003525.

% Copyright 2002-2022 National Technology & Engineering Solutions of Sandia,
% LLC (NTESS).

% Sandia National Laboratories is a multimission laboratory managed and
% operated by National Technology & Engineering Solutions of Sandia, LLC, a
% wholly owned subsidiary of Honeywell International Inc., for the U.S.
% Department of Energy’s National Nuclear Security Administration under
% contract DE-NA0003525.

% Copyright 2002-2022 National Technology & Engineering Solutions of Sandia,
% LLC (NTESS).


%%
%% Fixed Defects.
%%
{
\small

\begin{longtable}[h] {>{\raggedright\small}m{2in}|>{\raggedright\let\\\tabularnewline\small}m{3.5in}}
     \caption{Fixed Defects.  Note that we have multiple issue
     tracking systems for Sandia users.  SON, which bugzilla on the
     open network, and SRN, which is bugzilla on the restricted
     network.  We are also transitioning from bugzilla to gitlab issue
     tracking.  Further, some issues are reported by open source users
     on GitHub and these issues may be tracked using multiple issue
     numbers.} \\ \hline
     \rowcolor{XyceDarkBlue} \color{white}\textbf{Defect} & \color{white}\textbf{Description} \\ \hline
     \endfirsthead
     \caption[]{Fixed Defects.  Note that we have two multiple issue tracking systems for Sandia Users.
     SON and SRN refer to our legacy open- and restricted-network Bugzilla system, and Gitlab refers to issues in our gitlab repositories.  } \\ \hline
     \rowcolor{XyceDarkBlue} \color{white}\textbf{Defect} & \color{white}\textbf{Description} \\ \hline
     \endhead

\textbf{placeholder issue number}: placeholder description &
     placeholder explanation \\ \hline
\textbf{Gitlab-ex issue 358/GitHub issue 49}: Ternary operator broken in
analog function context & There were some use cases in which Xyce/ADMS
would emit bad code for ternary operator expressions in analog
function context.  Correct code is now emitted in all
cases.  \\ \hline
\textbf{Gitlab-ex issue 319}:  Improve error checking for invalid
.MEASURE lines & The error checking for invalid \texttt{DERIV-AT},
\texttt{DERIV-WHEN}, \texttt{FIND-AT}, \texttt{FIND-WHEN} and
\texttt{WHEN} measures has been improved.  Some invalid measure
lines that would previously be reported as ``FAILED'' are now
correctly reported as parsing errors. \\ \hline

\textbf{Gitlab-ex issue 289} Improvements to TRIG-TARG measure &
Separate \texttt{TD} (time delay) and/or \texttt{AT} qualifiers
are now allowed for both the \texttt{TRIG} and \texttt{TARG} 
clauses.  Expression support has also been improved.  Previously,
expressions did not work correctly in the \texttt{TARG} clause.
Examples are as follows:

\begin{verbatim}
.TRAN 0 1
.MEASURE TRAN M1 TRIG V(1)=0.5 CROSS=1 TD=0.3
+ TARG V(1)=0.5 CROSS=1 TD=0.8
.MEASURE TRAN M2 TRIG AT=0.05 
+ TARG V(1)=0.5 CROSS=1
.MEASURE TRAN M3 TRIG V(1)=0.5
+ CROSS=1 TARG AT=0.8
\end{verbatim}

Xyce will now properly report M1 = 0.5 with targ = 0.875 and
trig = 0.375, M2 = 0.075 with targ = 0.125 and trig = 0.05, and
M3 = 0.675 with targ = 0.8 and trig = 0.125.
\\ \hline 

\textbf{Gitlab-ex issue 220}:  .MEASURE trig/targ doesn't match ngspice
when there is negative setup time. & Previous \Xyce{} versions would
not return a negative value from a \texttt{TRIG-TARG} measure. The 
\texttt{TARG} clause would only be evaluated if the \texttt{TRIG} clause
was satisfied.  A measure line such as this will now properly return 
M4 = -0.5 with targ = 0.25 and trig = 0.75, instead of
failing to find the targ time.

\begin{verbatim}
VPWL2 2 0 pwl(0 0 0.5 1 1 0)
.MEASURE TRAN M4 TRIG V(2)=0.5 CROSS=2
+ TARG V(2)=0.5 CROSS=1
\end{verbatim}
 \\ \hline

\textbf{Gitlab-ex issue 328}:  Support C for temperature units & 
  When specifying temperature, \Xyce{} would inappropriately exit 
  with error if the units of Celsius were explicitly used in the netlist.  
  This has been corrected, and now a line such as \texttt{.options device temp=25C} 
  will work correctly.   The units for temperature in \Xyce{} have always been 
  Celsius, so this was only a parsing issue. \\ \hline

\textbf{Gitlab-ex issue 348}: \texttt{xyce\_getTimeVoltagePairsADC()} could overwrite memory &
  The general access function \texttt{xyce\_getTimeVoltagePairsADC()} could overwrite
  memory if the caller did not preallocate sufficient space.  A new function has been
  added to the \texttt{XyceCInterface} called \texttt{xyce\_getTimeVoltagePairsADCLimitData()}
  which limits the data copied to the caller allocated space to whatever maximum 
  allocation length is provided.  See the Application note, 
  Mixed Signal Simulation with \Xyce{} 7.5 for further details.  \\ \hline

\textbf{Gitlab-ex issue 351}: Support Python 3 in Xyce &
  The Python interface to Xyce in \texttt{xyce\_interface.py}
  has been updated to support Python 3.x.  See the Application note, 
  Mixed Signal Simulation with \Xyce{} 7.5 for further details.  \\ \hline

\textbf{Gitlab-ex issue 340}: Bug in breakpointing, when an expression combines a ternary and a table &
  Under rare circumstances, if an expression combined a ternary operator and
  a time-dependent table, an error in breakpointing could cause the time integrator 
  to get stuck in a near-infinite loop.  This has been fixed.  \\ \hline

\end{longtable}
}
