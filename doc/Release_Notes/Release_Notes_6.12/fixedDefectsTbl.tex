% Sandia National Laboratories is a multimission laboratory managed and
% operated by National Technology & Engineering Solutions of Sandia, LLC, a
% wholly owned subsidiary of Honeywell International Inc., for the U.S.
% Department of Energy’s National Nuclear Security Administration under
% contract DE-NA0003525.

% Copyright 2002-2019 National Technology & Engineering Solutions of Sandia,
% LLC (NTESS).

% Sandia National Laboratories is a multimission laboratory managed and
% operated by National Technology & Engineering Solutions of Sandia, LLC, a
% wholly owned subsidiary of Honeywell International Inc., for the U.S.
% Department of Energy’s National Nuclear Security Administration under
% contract DE-NA0003525.

% Copyright 2002-2019 National Technology & Engineering Solutions of Sandia,
% LLC (NTESS).


%%
%% Fixed Defects.
%%
{
\small

\begin{longtable}[h] {>{\raggedright\small}m{2in}|>{\raggedright\let\\\tabularnewline\small}m{3.5in}}
     \caption{Fixed Defects.  Note that we have two different Bugzilla systems for Sandia users.
     SON, which is on the open network, and SRN, which is on the restricted network. } \\ \hline
     \rowcolor{XyceDarkBlue} \color{white}\textbf{Defect} & \color{white}\textbf{Description} \\ \hline
     \endfirsthead
     \caption[]{Fixed Defects.  Note that we have two different Bugzilla systems for Sandia Users.
     SON, which is on the open network, and SRN, which is on the restricted network. } \\ \hline
     \rowcolor{XyceDarkBlue} \color{white}\textbf{Defect} & \color{white}\textbf{Description} \\ \hline
     \endhead

\textbf{xxx-SON}: place holder title &
place holder description  \\ \hline

\textbf{1224-SON}: Improved error handling for .SAMPLING &
In the previous release of Xyce, a \texttt{.SAMPLING} analysis would run
even if the user-specified \texttt{STD\_DEVIATIONS}, \texttt{ALPHA} and
\texttt{BETA} values were invalid.  For Gamma distributions, the
user-specified \texttt{ALPHA} and \texttt{BETA} values must now be
strictly positive.  The \texttt{STD\_DEVIATIONS} values specified for
Gaussian distributions must now be non-negative. \\ \hline

\textbf{1223-SON}: ADMS and Xyce/ADMS do not correctly handle analog functions with output arguments  &
A long-standing bug in ADMS fails to propagate probe and variable
dependence from analog function call input arguments into the output
arguments.  This leads to incorrect code generation by Xyce/ADMS when
such functions are used.  Xyce/ADMS now uses a modified version of
ADMS's ``implicit'' templates that it invokes explicitly instead.
Analog function output variables should now be processed just as
correctly as if they had appeared on the left hand side of an
assignment.  The relevant fixes to ADMS's implicit templates have also been
provided upstream to the ADMS maintainers.  \\ \hline

\textbf{1219-SON}: Add PSP 103 model with self-heating &
This model has been added as the level 1031 MOSFET.  \\ \hline

\textbf{1210-SON}: Sign of negative values in parameter files ignored &
When the \texttt{-prf} command line argument was used to pass Xyce a
parameter file the leading negative signs of parameter values was
ignored.  This has been fixed.  \\ \hline

\textbf{1204-SON}: Fix issue with extra blank lines in Touchstone 1 formatted output
for .LIN analysis &

There could be extra blank lines inserted in the Touchstone 1
formatted output, between the S-parameter values output for each
frequency, if a
\texttt{.LIN} analysis used any number of ports other than 2.  The Touchstone 2
formatted output, which is the default output format for \texttt{.LIN} analyses,
was not affected by this bug. \\ \hline

\textbf{1202-SON}: Fix segfaults caused by using .LIN SPARCALC=1 in a netlist
without any port (P) devices & The use of a \texttt{.LIN} line with the
parameter \texttt{SPARCALC} set to 1 in a netlist that requested a \texttt{.AC}
analysis without any valid port (P) devices in the netlist would cause
\Xyce{} to segfault. \\ \hline

\textbf{1198-SON}: Binning doesn't work for BSIM6 and other models in the ADMS directory &
Binning (the kind enabled by \texttt{.options parser
model\_binning=true} did not work properly for several MOSFET models
derived from Verilog-A, because the authors of those models had not
included the dummy
parameters \textbf{LMIN}, \textbf{LMAX}, \textbf{WMIN},
and \textbf{WMAX}.  These dummy parameters are not used directly by
any model, but must be present in order for the binning code to select
the correct model card from a collection of cards based on
the \textbf{L} and \textbf{W} parameter on the instance line.  These
models were BSIM6, PSP102, MVSG-HV, and the non-free EKV models.
Dummy parameters have been added for all these models and they should
now work with binning. \\ \hline

\textbf{1197-SON}: Allow dependent parameters, other than C and Q, for
the C-device that use expressions without solution variables in them &
The C and Q instance parameters are allowed to be solution-dependent
for the C-device.  In \Xyce{} 6.11, or earlier, an attempt to use an
expression (that did not depend on solution variables such as nodal voltages)
in the definition of other parameters, such as TEMP, for the C device
would cause a parsing error.  This is fixed now.
\\ \hline

\textbf{1196-SON}: -hspice-ext math command line option fails on .FUNC
statements the use Logical OR and Logical AND & The \texttt{-hspice-ext}
command line option would fail on \texttt{.FUNC} statements that contained
logical OR or logical AND operators.  Examples were as follows, where
\texttt{P1} and \texttt{P2} were previously defined on \texttt{.PARAM}
lines:
\begin{verbatim}
.FUNC ANDFUNC(p1,p2) {(P1&&P2) ? 1.0 : 2.0}
.FUNC ORFUNC(p1,p2)  {(P1||P2) ? 1.0 : 2.0}
\end{verbatim}
\\ \hline

\textbf{1188-SON}: Various ill-formed .DATA, .DC and .STEP lines could cause
\Xyce{} to segfault & An attempt to use \texttt{.DATA}, \texttt{.DC} or
\texttt{.STEP} lines with less than the required number of elements on
those lines could cause \Xyce{} to segfault.  In addition, the error
messages emitted for invalid uses of \texttt{DATA=<tableName>} on
\texttt{.AC} \texttt{.DC} and \texttt{.STEP} lines, or for ill-formed
\texttt{.DATA} statements, could be cryptic and unhelpful.
\\ \hline

\textbf{989-SON}: I(*) will not print branch currents that are part of a Y
device & Until this release of \Xyce{}, the I(*) feature of
the \texttt{.PRINT} would not output any branch currents (extra
solution variables describing internal currents of a device)
associated with Y devices.  No Y devices in \Xyce{} have such branch
currents, but Y devices imported from Verilog-A plugins could.  These
branch currents are now output by I(*).

\\ \hline



\textbf{1242-SON}: Allow S-parameter analysis to handle DC and negative  frequencies &
The non-positive frequencies were not allowed in the S-parameter analysis previously.
The linear frequency sweep can now handle DC and negative frequencies.    
\\ \hline

\end{longtable}
}
