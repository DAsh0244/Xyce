% Sandia National Laboratories is a multimission laboratory managed and
% operated by National Technology & Engineering Solutions of Sandia, LLC, a
% wholly owned subsidiary of Honeywell International Inc., for the U.S.
% Department of Energy’s National Nuclear Security Administration under
% contract DE-NA0003525.

% Copyright 2002-2019 National Technology & Engineering Solutions of Sandia,
% LLC (NTESS).

% Sandia National Laboratories is a multimission laboratory managed and
% operated by National Technology & Engineering Solutions of Sandia, LLC, a
% wholly owned subsidiary of Honeywell International Inc., for the U.S.
% Department of Energy’s National Nuclear Security Administration under
% contract DE-NA0003525.

% Copyright 2002-2019 National Technology & Engineering Solutions of Sandia,
% LLC (NTESS).


%%
%% Fixed Defects.
%%
{
\small

\begin{longtable}[h] {>{\raggedright\small}m{2in}|>{\raggedright\let\\\tabularnewline\small}m{3.5in}}
     \caption{Fixed Defects.  Note that we have two different Bugzilla systems for Sandia users.
     SON, which is on the open network, and SRN, which is on the restricted network. } \\ \hline
     \rowcolor{XyceDarkBlue} \color{white}\textbf{Defect} & \color{white}\textbf{Description} \\ \hline
     \endfirsthead
     \caption[]{Fixed Defects.  Note that we have two different Bugzilla systems for Sandia Users.
     SON, which is on the open network, and SRN, which is on the restricted network. } \\ \hline
     \rowcolor{XyceDarkBlue} \color{white}\textbf{Defect} & \color{white}\textbf{Description} \\ \hline
     \endhead

\textbf{xxx-SON}: place holder title &
place holder description  \\ \hline
\textbf{1198-SON}: Binning doesn't work for BSIM6 and other models in the ADMS directory &
Binning (the kind enabled by \texttt{.options parser
model\_binning=true} did not work properly for several MOSFET models
derived from Verilog-A, because the authors of those models had not
included the dummy
parameters \textbf{LMIN}, \textbf{LMAX}, \textbf{WMIN},
and \textbf{WMAX}.  These dummy parameters are not used directly by
any model, but must be present in order for the binning code to select
the correct model card from a collection of cards based on
the \textbf{L} and \textbf{W} parameter on the instance line.  These
models were BSIM6, PSP102, MVSG-HV, and the non-free EKV models.
Dummy parameters have been added for all these models and they should
now work with binning. \\ \hline

\textbf{1197-SON}: Allow dependent parameters, other than C and Q, for
the C-device that use expressions without solution variables in them &
The C and Q instance parameters are allowed to be solution-dependent
for the C-device.  In \Xyce{} 6.11, or earlier, an attempt to use an
expression (that did not depend on solution variables such as nodal voltages)
in the definition of other parameters, such as TEMP, for the C device
would cause a parsing error.  This is fixed now.
\\ \hline

\textbf{1196-SON}: -hspice-ext math command line option fails on .FUNC
statements the use Logical OR and Logical AND & The \texttt{-hspice-ext}
command line option would fail on \texttt{.FUNC} statements that contained
logical OR or logical AND operators.  Examples were as follows, where
\texttt{P1} and \texttt{P2} were previously defined on \texttt{.PARAM}
lines:
\begin{verbatim}
.FUNC ANDFUNC(p1,p2) {(P1&&P2) ? 1.0 : 2.0}
.FUNC ORFUNC(p1,p2)  {(P1||P2) ? 1.0 : 2.0}
\end{verbatim}
\\ \hline

\textbf{989-SON}: I(*) will not print branch currents that are part of a Y
device & Until this release of \Xyce{}, the I(*) feature of
the \texttt{.PRINT} would not output any branch currents (extra
solution variables describing internal currents of a device)
associated with Y devices.  No Y devices in \Xyce{} have such branch
currents, but Y devices imported from Verilog-A plugins could.  These
branch currents are now output by I(*).

\\ \hline



\end{longtable}
}
