% Sandia National Laboratories is a multimission laboratory managed and
% operated by National Technology & Engineering Solutions of Sandia, LLC, a
% wholly owned subsidiary of Honeywell International Inc., for the U.S.
% Department of Energy’s National Nuclear Security Administration under
% contract DE-NA0003525.

% Copyright 2002-2020 National Technology & Engineering Solutions of Sandia,
% LLC (NTESS).

% Sandia National Laboratories is a multimission laboratory managed and
% operated by National Technology & Engineering Solutions of Sandia, LLC, a
% wholly owned subsidiary of Honeywell International Inc., for the U.S.
% Department of Energy’s National Nuclear Security Administration under
% contract DE-NA0003525.

% Copyright 2002-2020 National Technology & Engineering Solutions of Sandia,
% LLC (NTESS).


%%
%% Fixed Defects.
%%
{
\small

\begin{longtable}[h] {>{\raggedright\small}m{2in}|>{\raggedright\let\\\tabularnewline\small}m{3.5in}}
     \caption{Fixed Defects.  Note that we have two different Bugzilla systems for Sandia users.
     SON, which is on the open network, and SRN, which is on the restricted network. } \\ \hline
     \rowcolor{XyceDarkBlue} \color{white}\textbf{Defect} & \color{white}\textbf{Description} \\ \hline
     \endfirsthead
     \caption[]{Fixed Defects.  Note that we have two different Bugzilla systems for Sandia Users.
     SON, which is on the open network, and SRN, which is on the restricted network. } \\ \hline
     \rowcolor{XyceDarkBlue} \color{white}\textbf{Defect} & \color{white}\textbf{Description} \\ \hline
     \endhead

\textbf{1283-SON}: Make RMS measure results more compliant with the HSPICE method &
The \Xyce{} method for calculating the results of \texttt{RMS} measures, for \texttt{AC},
\texttt{DC} and \texttt{TRAN} measure modes, now matches the HSPICE approach of using the
``the square root of the area under the out\_var curve, divided by the period of interest''.
\\ \hline

\textbf{1275-SON}: Bug with handling SIMPLE and PAUSE breakpoints that occur at the same time &
\Xyce{} supports two types of ``breakpoints'' -- namely SIMPLE and PAUSE breakpoints.  The SIMPLE
breakpoints are, for example, set based on the times of the slope discontinuities in any PULSE
and PWL sources in the netlist.  The PAUSE breakpoints are set, for example, by the simulation
end time and any \texttt{simulateUntil(requested\_time)} calls used by the Mixed Signal Interface.
In previous versions of \Xyce{}, if a SIMPLE and PAUSE breakpoint occurred at the same time
then the PAUSE breakpoint would be ignored and the \texttt{simulateUntil()} calls would work
incorrectly.
\\ \hline

\textbf{1273-SON}: Bug fixes for FIND-WHEN and WHEN measures for all analysis modes &
Several issues with the \texttt{FIND-WHEN} and \texttt{WHEN} measures, for \texttt{AC},
\texttt{DC} and \texttt{TRAN} measure modes, were addressed for this release.  First,
the right-hand side of the equality in the \texttt{WHEN} clause can now be an expression.
An example is:
\begin{verbatim}
.MEASURE DC whenExample1 when v(2)={v(1)+1}
\end{verbatim}
The second issue was that the interpolation algorithm for determining the time
(or frequency or DC sweep value) at which the \texttt{WHEN} clause was satisifed
could be inaccurate if the \texttt{WHEN} clause used varying quantities.
An example is:
\begin{verbatim}
.MEASURE AC whenExample2 when vr(1)=vi(1)
\end{verbatim}
The third issue was that, especially for \texttt{TRAN} measures, the \texttt{WHEN}
clause could miss some of the crossings if that clause used varying quantities.
\\ \hline

\textbf{1271-SON}: Incorrect error handling for .PRINT AC when .LIN is used with
.STEP & An attempt to print out an S-,Y- or Z-parameter value via a .PRINT AC line
would cause a parsing error when .LIN was used with .STEP.
\\ \hline

\textbf{xxx-SRN}: Place holder title &
Place holder description.  \\ \hline

\end{longtable}
}
