% Sandia National Laboratories is a multimission laboratory managed and
% operated by National Technology & Engineering Solutions of Sandia, LLC, a
% wholly owned subsidiary of Honeywell International Inc., for the U.S.
% Department of Energy’s National Nuclear Security Administration under
% contract DE-NA0003525.

% Copyright 2002-2020 National Technology & Engineering Solutions of Sandia,
% LLC (NTESS).

% Sandia National Laboratories is a multimission laboratory managed and
% operated by National Technology & Engineering Solutions of Sandia, LLC, a
% wholly owned subsidiary of Honeywell International Inc., for the U.S.
% Department of Energy’s National Nuclear Security Administration under
% contract DE-NA0003525.

% Copyright 2002-2020 National Technology & Engineering Solutions of Sandia,
% LLC (NTESS).


%%
%% Fixed Defects.
%%
{
\small

\begin{longtable}[h] {>{\raggedright\small}m{2in}|>{\raggedright\let\\\tabularnewline\small}m{3.5in}}
     \caption{Fixed Defects.  Note that we have two different Bugzilla systems for Sandia users.
     SON, which is on the open network, and SRN, which is on the restricted network. } \\ \hline
     \rowcolor{XyceDarkBlue} \color{white}\textbf{Defect} & \color{white}\textbf{Description} \\ \hline
     \endfirsthead
     \caption[]{Fixed Defects.  Note that we have two different Bugzilla systems for Sandia Users.
     SON, which is on the open network, and SRN, which is on the restricted network. } \\ \hline
     \rowcolor{XyceDarkBlue} \color{white}\textbf{Defect} & \color{white}\textbf{Description} \\ \hline
     \endhead

\textbf{1299-SON}: Xyce incorrectly outputs multiple unwanted solution results when 'tstart' given on tran line and .options output also given &
When a non-zero third parameter (``TSTART'') was given to
a \texttt{.TRAN} analysis statement and a \texttt{.OPTIONS OUTPUT}
line was also used to specify detailed printing intervals, \Xyce{}
would output unwanted information for times prior
to \texttt{TSTART}.
\\ \hline

\textbf{203-SON}: Add BSIM 4 SOI model to Xyce &
The BSIM SOI version 4.6.1 has been added as the level 70 MOSFET.
\\ \hline

\textbf{1296-SON}: Add multiplicity factor to BSIM 6 &
The BSIM 6 in Xyce prior to this release did not support a
multiplicity factor (``M'' instance parameter).  Support for
multiplicity in the BSIM 6 has been added in this release.
\\ \hline

\textbf{788-SON}: Support output variables from Verilog-A models &
The Verilog-A Language Reference Manual (LRM) states that module
scoped variables that have ``desc'' or ``units'' attributes should be
considered output variables, and users should have access to these
values during simulation.  \Xyce{} now supports this usage, and
Xyce/ADMS will generate appropriate code so that these output
variables may be printed with the ``N()'' notation on \texttt{.PRINT}
lines. \\ \hline

\textbf{410-SON}: Implement ddx() in Xyce/ADMS Verilog-A compiler &
Xyce/ADMS did not support use of the Verilog-A construct ``ddx()'',
which provides symbolic differentiation capabilities.  ddx() is now
supported with minor limitations that impact no published Verilog-A
models we have encountered.  See the Xyce/ADMS Users' Guide for more
details. \\ \hline

\textbf{1298-SON}: Bug in handling FREQ special variable for AC measures &
There was an error in handling the \texttt{FREQ} special variable in
\texttt{.MEASURE} statements that can be illustrated with this simple
netlist fragment:

\begin{verbatim}
.ac dec 5 100Hz 1e6
.print AC FREQ EQNFREQ vm(b)
.MEASURE AC EQNFREQ EQN {FREQ}
\end{verbatim}
The value of \texttt{FREQ}, generated by the \texttt{.PRINT AC} line,
in the output file \texttt{<netlistName>.FD.prn} would be correct.  However,
the value reported for the AC measure \texttt{EQNFREQ}, in that output file, would
be the value from the previous \texttt{AC} sweep value.  So, in this example,
the final value reported for \texttt{EQNFREQ} in the file \texttt{<netlistName>.ma0}
would be 6.31e+05 rather than 1e+06.
\\ \hline

\textbf{1284-SON}: Rewrite Xyce/ADMS to generate code that does not use Sacado  at all &
With release 6.11 of \Xyce{}, Xyce/ADMS was rewritten to require
Sacado only for the functions that are used during sensitivity
analysis, resulting in a dramatic performance improvement for all
analyses other than sensitivity.  This effort has concluded in release
7.1 by implementing sensitivity code without Sacado as well.  This
change has little impact on performance of sensitivity analysis, but
does significantly reduce memory requirements at compile
time. \\ \hline

\textbf{1283-SON}: Make RMS measure results more compliant with the HSPICE method &
The \Xyce{} method for calculating the results of \texttt{RMS} measures, for \texttt{AC},
\texttt{DC} and \texttt{TRAN} measure modes, now matches the HSPICE approach of using the
``the square root of the area under the out\_var curve, divided by the period of interest''.
\\ \hline

\textbf{1275-SON}: Bug with handling SIMPLE and PAUSE breakpoints that occur at the same time &
\Xyce{} supports two types of ``breakpoints'' -- namely SIMPLE and PAUSE breakpoints.  The SIMPLE
breakpoints are, for example, set based on the times of the slope discontinuities in any PULSE
and PWL sources in the netlist.  The PAUSE breakpoints are set, for example, by the simulation
end time and any \texttt{simulateUntil(requested\_time)} calls used by the Mixed Signal Interface.
In previous versions of \Xyce{}, if a SIMPLE and PAUSE breakpoint occurred at the same time
then the PAUSE breakpoint would be ignored and the \texttt{simulateUntil()} calls would work
incorrectly.
\\ \hline

\textbf{1273-SON}: Bug fixes for DERIV, FIND-WHEN and WHEN measures for all analysis modes &
Several issues with the \texttt{DERIV}, \texttt{FIND-WHEN} and \texttt{WHEN} measures, for \texttt{AC},
\texttt{DC} and \texttt{TRAN} measure modes, were addressed for this release.  First,
the right-hand side of the equality in the \texttt{WHEN} clause can now be an expression.
An example is:
\begin{verbatim}
.MEASURE DC whenExample1 when v(2)={v(1)+1}
\end{verbatim}
The second issue was that the interpolation algorithm for determining the time
(or frequency or DC sweep value) at which the \texttt{WHEN} clause was satisifed
could be inaccurate if the \texttt{WHEN} clause used varying quantities.
An example is:
\begin{verbatim}
.MEASURE AC whenExample2 when vr(1)=vi(1)
\end{verbatim}
The third issue was that, especially for \texttt{TRAN} measures, the \texttt{WHEN}
clause could miss some of the crossings if that clause used varying quantities.  All of
the previous discussion applies to \texttt{DERIV} and \texttt{FIND-WHEN} measures.
\\ \hline

\textbf{1271-SON}: Incorrect error handling for .PRINT AC when .LIN is used with
.STEP & An attempt to print out an S-,Y- or Z-parameter value via a .PRINT AC line
would cause a parsing error when .LIN was used with .STEP.
\\ \hline

\textbf{1292-SON, Gitlab Issue \#6}: Inductor coupling (mutual inductor) code does not respect inductor initial condition specification &
Linear and non-linear mutual inductors were ignoring the initial conditions set on 
component inductors.  This was a code deficiency as the initial conditions were not 
passed into the mutual inductor device.  This issue has been fixed.
\\ \hline

\textbf{1287-SON, Github Issue \#6}: Ill-formed .IC line leads to segfaults on some Linuxen
& Empty .IC/.DCVOLT/.NODESET lines were causing Xyce to seg fault on certain platforms.
A warning is now emitted for empty .IC/.DCVOLT/.NODESET lines and the line is effectively ignored.
\\ \hline

\textbf{xxx-SRN}: Place holder title &
Place holder description.  \\ \hline
\end{longtable}
}
