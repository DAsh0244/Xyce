% Sandia National Laboratories is a multimission laboratory managed and
% operated by National Technology & Engineering Solutions of Sandia, LLC, a
% wholly owned subsidiary of Honeywell International Inc., for the U.S.
% Department of Energy’s National Nuclear Security Administration under
% contract DE-NA0003525.

% Copyright 2002-2020 National Technology & Engineering Solutions of Sandia,
% LLC (NTESS).

% Sandia National Laboratories is a multimission laboratory managed and
% operated by National Technology & Engineering Solutions of Sandia, LLC, a
% wholly owned subsidiary of Honeywell International Inc., for the U.S.
% Department of Energy’s National Nuclear Security Administration under
% contract DE-NA0003525.

% Copyright 2002-2020 National Technology & Engineering Solutions of Sandia,
% LLC (NTESS).


%%
%% Fixed Defects.
%%
{
\small

\begin{longtable}[h] {>{\raggedright\small}m{2in}|>{\raggedright\let\\\tabularnewline\small}m{3.5in}}
     \caption{Fixed Defects.  Note that we have two different Bugzilla systems for Sandia users.
     SON, which is on the open network, and SRN, which is on the restricted network. } \\ \hline
     \rowcolor{XyceDarkBlue} \color{white}\textbf{Defect} & \color{white}\textbf{Description} \\ \hline
     \endfirsthead
     \caption[]{Fixed Defects.  Note that we have two different Bugzilla systems for Sandia Users.
     SON, which is on the open network, and SRN, which is on the restricted network. } \\ \hline
     \rowcolor{XyceDarkBlue} \color{white}\textbf{Defect} & \color{white}\textbf{Description} \\ \hline
     \endhead

\textbf{812-SON}: Incorrect response when N or V is used as a global param name &
Use of a \texttt{.GLOBAL\_PARAM} statement to define a parameter which
had a name identical to a valid print line accessor function
(e.g. ``V'' or ``N'') would result in error messages of the form
``Function or variable ... is not defined'' when that accessor was
used on a print line, with the undefined variable being the node or
device name inside the parentheses of the accessor.  This was a bug in
handling of global parameters, and the only workaround had been to
avoid such parameter names.  This workaround is no longer
necessary. \\ \hline

\textbf{1310-SON}: Improve FROM and TO info printed to stdout for AC,
DC and Noise measures & The information printed to stdout for the
start and end points of the measurement window, for AC, DC and NOISE
measures, is now correct for cases where the FROM and/or TO values
are not equal to a sweep value.  That information is also now
correct for DERIV-WHEN, FIND-WHEN and WHEN measures.
\\ \hline

\textbf{1307-SON}: Improve Compatibility of .MEASURE with .DATA &
If a \texttt{.DATA} statement is used to define a table-based sweep
on a \texttt{.DC} line then \texttt{DC} measures now correctly use
the row index in that table as their ``swept variable''.  The
\texttt{.MEASURE} section of the \Xyce{} Reference Guide provides
an example of this use case.
\\ \hline

\textbf{1306-SON}: Fix FROM and TO qualifiers for DC mode measures &
The \texttt{FROM} and \texttt{TO} qualifiers would work correctly if
either both were used on a \texttt{.MEASURE DC} line, or if neither
was used.  Measures such as these now also work correctly for all
supported DC measure types:
\begin{verbatim}
.measure dc maxFrom max v(1) FROM=4
.measure dc minTo min v(1) TO=4
\end{verbatim}
\\ \hline

\textbf{1303-SON}: Segfault in mixed signal interface &
The Python method \texttt{getDeviceNames()}, and the underlying
\texttt{xyce\_getDeviceNames()} method in the mixed signal interface,
would segfault if invoked for some model groups (e.g, D, L and M) when
there were no devices from that model group in the \Xyce{} netlist.
\\ \hline

\textbf{xxx-SRN}: Place holder title &
Place holder description.  \\ \hline
\end{longtable}
}
