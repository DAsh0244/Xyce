% Sandia National Laboratories is a multimission laboratory managed and
% operated by National Technology & Engineering Solutions of Sandia, LLC, a
% wholly owned subsidiary of Honeywell International Inc., for the U.S.
% Department of Energy’s National Nuclear Security Administration under
% contract DE-NA0003525.

% Copyright 2002-2020 National Technology & Engineering Solutions of Sandia,
% LLC (NTESS).

% Sandia National Laboratories is a multimission laboratory managed and
% operated by National Technology & Engineering Solutions of Sandia, LLC, a
% wholly owned subsidiary of Honeywell International Inc., for the U.S.
% Department of Energy’s National Nuclear Security Administration under
% contract DE-NA0003525.

% Copyright 2002-2020 National Technology & Engineering Solutions of Sandia,
% LLC (NTESS).


%%
%% Fixed Defects.
%%
{
\small

\begin{longtable}[h] {>{\raggedright\small}m{2in}|>{\raggedright\let\\\tabularnewline\small}m{3.5in}}
     \caption{Fixed Defects.  Note that we have two different Bugzilla systems for Sandia users.
     SON, which is on the open network, and SRN, which is on the restricted network. } \\ \hline
     \rowcolor{XyceDarkBlue} \color{white}\textbf{Defect} & \color{white}\textbf{Description} \\ \hline
     \endfirsthead
     \caption[]{Fixed Defects.  Note that we have two multiple issue tracking systems for Sandia Users.
     SON and SRN refer to our legacy open- and restricted-network Bugzilla system, and Gitlab refers to issues in our gitlab repositories.  } \\ \hline
     \rowcolor{XyceDarkBlue} \color{white}\textbf{Defect} & \color{white}\textbf{Description} \\ \hline
     \endhead

\textbf{Gitlab-ex 56}: Xyce header files unnecessarily include Trilinos header files&
\Xyce{} provides an API for coupling circuit simulation to external
programs. The headers and shared libraries installed by ``make
install'' should be sufficient to provide access to this feature, but
some Xyce header files unnecessarily included Trilinos header files.
This meant that users of Xyce shared libraries also needed to have
access to complete sets of installed Trilinos headers.  The header
files needed to access the coupling API have been cleaned up and are
now sufficiently self-contained to allow coupling of Xyce to an
external code without the need for Trilinos headers or
libraries.  Trilinos is still required to build Xyce itself, and
is still needed to build shared library plugins from Verilog-A models.
\\ \hline

\textbf{Gitlab-ex 54}: Xyce/ADMS incorrectly sets ``dependency'' and ``OPdependent'' for analog function output variables &
The internal ADMS data tree has an element called ``dependency'' that
may be ``constant'', ``noprobe'', ``linear'', or ``nonlinear''.  This
element was being incorrectly set for analog function output variables
(other than the return value).  At the moment Xyce/ADMS does not use
those elements for anything important, so the incorrect setting had no
impact on generated C++ code.  The incorrect setting could have had
impact on future development of the Xyce/ADMS back-end.  \\ \hline

\textbf{Gitlab-ex 50}: Xyce/ADMS mishandles potential contributions in presence of flow contributions &
When Xyce/ADMS processed a module that contained both flow and
potential contributions between the same two nodes, it was generating
incorrect code that would not converge.  This use case now emits code
that puts the flow sources in parallel with the potential sources.
This is not in compliance with the value retention rules of the
Verilog-A language reference manual, which would have the second
contribution of such a use case discard the information of the first
contribution.  The prior behavior, however, was not only
non-compliant, but also broken.

This issue impacted no standard Verilog-A models, as none make use of
this type of contribution or rely on the value retention rules of the LRM.
\\ \hline

\textbf{Gitlab-ex 42}: Xyce/ADMS unnecessarily outputting second derivative code &
An error in logic caused Xyce/ADMS to emit unnecessary code to compute
second derivatives of variables.  Second derivatives are only required
when a variable whose value is computed via ddx() is subsequently used
in non-noise contributions (a usage that is strongly discouraged and
not present in any common models).  This unnecessary code was often as
large or larger than the required code, and resulted in extraordinary
compilation times and compiler memory use. \\ \hline

\textbf{Gitlab-ex 40}: Xyce/ADMS incorrectly omitting variable declaration in obscure use cases &
As a result of a design flaw that has been fixed in this release,
previous versions of Xyce/ADMS would incorrectly omit the declaration
of a module-scoped variable if the only use of that variable was in a
lower-level block and an earlier block had a local variable of the
same name. \\ \hline

\textbf{Gitlab-ex 33}: Lossless transmission line unnecessarily sets maximum time step &
The \Xyce{} lossless transmission line was implemented in a manner
that restricted the time integrator's time step to never
be longer than the transmission line's delay.  This had been done to
avoid applying extrapolation to the device history.  This restriction
severely impacted run time whenever a circuit included very short
delay lines.  This unnecessary restriction has been removed.  \\ \hline

\textbf{Gitlab-ex 24, 390-SON}: Xyce/ADMS does not support \$port\_connected &
The Verilog-A construct ``\$port\_connected'' tests whether an
optional node has been specified on the instance line of a device.
Using this construct anywhere in a Verilog-A model implicitly
designates the named node as optional.  Xyce/ADMS now supports this
construct in a limited sense.  See the Xyce/ADMS users guide for
details. \\ \hline

\textbf{812-SON}: Incorrect response when N or V is used as a global param name &
Use of a \texttt{.GLOBAL\_PARAM} statement to define a parameter which
had a name identical to a valid print line accessor function
(e.g. ``V'' or ``N'') would result in error messages of the form
``Function or variable ... is not defined'' when that accessor was
used on a print line, with the undefined variable being the node or
device name inside the parentheses of the accessor.  This was a bug in
handling of global parameters, and the only workaround had been to
avoid such parameter names.  This workaround is no longer
necessary. \\ \hline

\textbf{1318-SON}: Allow for numerical roundoff when FROM, TO or TD
are expressions on .MEASURE lines & \Xyce{} could give the incorrect
answer when the \texttt{FROM}, \texttt{TO} or \texttt{TD} qualifiers
on a .MEASURE line were expressions.  This was caused by small
numerical-roundoff errors in the calculation of the expression
values.
\\ \hline

\textbf{1310-SON}: Improve FROM and TO info printed to stdout for AC,
DC and Noise measures & The information printed to stdout for the
start and end points of the measurement window, for AC, DC and NOISE
measures, is now correct for cases where the FROM and/or TO values
are not equal to a sweep value.  That information is also now
correct for DERIV-WHEN, FIND-WHEN and WHEN measures.
\\ \hline

\textbf{1307-SON}: Improve Compatibility of .MEASURE with .DATA &
If a \texttt{.DATA} statement is used to define a table-based sweep
on a \texttt{.DC} line then \texttt{DC} measures now correctly use
the row index in that table as their ``swept variable''.  The
\texttt{.MEASURE} section of the \Xyce{} Reference Guide provides
an example of this use case.
\\ \hline

\textbf{1306-SON}: Fix FROM and TO qualifiers for DC mode measures &
The \texttt{FROM} and \texttt{TO} qualifiers would work correctly if
either both were used on a \texttt{.MEASURE DC} line, or if neither
was used.  Measures such as these now also work correctly for all
supported DC measure types:
\begin{verbatim}
.measure dc maxFrom max v(1) FROM=4
.measure dc minTo min v(1) TO=4
\end{verbatim}
\\ \hline

\textbf{1304-SON}: Fix issues with FROM-TO qualifiers for DERIV-WHEN,
FIND-WHEN and WHEN measures & These measure types could get the incorrect
answer, for all measure modes, if the FROM or TO qualifiers
were given.  The interpolated WHEN time was not being properly compared with
specified the FROM-TO window.  Now, if the interpolated WHEN time is within
the FROM-TO window then the measure value is valid.  If the interpolated
time is outside of the FROM-TO window then the measure is "FAILED".
\\ \hline

\textbf{1303-SON}: Segfault in mixed signal interface &
The Python method \texttt{getDeviceNames()}, and the underlying
\texttt{xyce\_getDeviceNames()} method in the mixed signal interface,
would segfault if invoked for some model groups (e.g, D, L and M) when
there were no devices from that model group in the \Xyce{} netlist.
\\ \hline

\textbf{1176-SON}: Expression tables have an efficiency bottleneck when tables are large &
When a really large table (thousands of points) was used to specify a Bsrc 
via the expression library, there was a bottleneck which caused really slow execution.
\\ \hline

\textbf{1827-SRN}: Expression library selection of minimum breakpoint distance is too large &
The breakpointing algorithm for expression-based sources had a flaw in it that 
imposed an artificially large minimum distance between breakpoints.  This 
prevented really fast rise times (for example) from working correctly.
\\ \hline

\textbf{794-SON}: Bug in TABLE Form of \Xyce{} Controlled Sources & 
In some case, a \Xyce{} netlist that contains a controlled source that uses the 
TABLE form would get the correct answer at first.  However, it may then "stall" 
(e.g, keep taking really small time-steps) and never complete the simulation run.
\\ \hline

\textbf{1222-SON}: Near-infinite loop in expression library, uncovered by modern PDK files  & 
Some modern process design kits (PDKs) contain very complex use of expressions, 
involving many nested function calls.  In some cases, this was not parsable with
the old \Xyce{} expression library.
\\ \hline

\textbf{xxx-SRN}: Place holder title &
Place holder description.  \\ \hline
\end{longtable}
}
