% Sandia National Laboratories is a multimission laboratory managed and
% operated by National Technology & Engineering Solutions of Sandia, LLC, a
% wholly owned subsidiary of Honeywell International Inc., for the U.S.
% Department of Energy’s National Nuclear Security Administration under
% contract DE-NA0003525.

% Copyright 2002-2019 National Technology & Engineering Solutions of Sandia,
% LLC (NTESS).

% Sandia National Laboratories is a multimission laboratory managed and
% operated by National Technology & Engineering Solutions of Sandia, LLC, a
% wholly owned subsidiary of Honeywell International Inc., for the U.S.
% Department of Energy’s National Nuclear Security Administration under
% contract DE-NA0003525.

% Copyright 2002-2019 National Technology & Engineering Solutions of Sandia,
% LLC (NTESS).


%%
%% Fixed Defects.
%%
{
\small

\begin{longtable}[h] {>{\raggedright\small}m{2in}|>{\raggedright\let\\\tabularnewline\small}m{3.5in}}
     \caption{Fixed Defects.  Note that we have two different Bugzilla systems for Sandia users.
     SON, which is on the open network, and SRN, which is on the restricted network. } \\ \hline
     \rowcolor{XyceDarkBlue} \color{white}\textbf{Defect} & \color{white}\textbf{Description} \\ \hline
     \endfirsthead
     \caption[]{Fixed Defects.  Note that we have two different Bugzilla systems for Sandia Users.
     SON, which is on the open network, and SRN, which is on the restricted network. } \\ \hline
     \rowcolor{XyceDarkBlue} \color{white}\textbf{Defect} & \color{white}\textbf{Description} \\ \hline
     \endhead

\textbf{1162-SON}: \Xyce{} behavior is counter-intuitive when \texttt{.DC} or \texttt{.STEP} parameters are inconsistent with syntax &
\Xyce{}'s .DC and .STEP can sweep linearly, by decade, or by octave in
addition to other sweep types.  When a linear, decade, or octave sweep
was specified in such a manner that the ``stop'' value could not be
reached from the ``start'' value by applying the requested step
algorithm, \Xyce{} would not perform the same simulation that other
SPICE-like simulators would have.  For the linear case, if the stop
value was smaller than start, but the step value was positive, \Xyce{}
would perform no simulation at all.  For the decade and octave sweeps
if stop was smaller than start, \Xyce{} would perform an incorrect
upward sweep starting at the start value.

\Xyce{} now performs better error checking on the arguments of \texttt{.DC}
and \texttt{.STEP}, and will now always compute only the first DC or
STEP value, even if the stepping algorithm cannot advance from
``start'' to ``stop''.  This brings \Xyce{}'s behavior in line with
those of other SPICE-like simulators.  \\ \hline

\textbf{1160-SON} .DATA did not work with -remeasure for .TRAN and .DC analyses &
The \texttt{.DATA} netlist command, which was introduced in release
6.10, did not work with the \texttt{-remeasure} command line
option. This is fixed now. \\ \hline

\textbf{1144-SON} Fix discontinuity in diode model at high temperature & Since
2005, the \Xyce{} diode model has supported a parameter
called \textbf{IRF}, which allowed users to apply a scaling factor to
the reverse current of the diode in order to match experimental data.
Not only was this feature never properly validated, it was also not
correctly falling back to SPICE3F5 compatibility as the documentation
claimed it should have done, nor was its temperature dependence
correctly documented.  This has been corrected in \Xyce{} 6.11, and
the feature is now disabled correctly if \textbf{IRF} is not specified
in the diode's model card, and the diode does in fact match SPICE3F5
as originally intended.  The reference guide discussion of the reverse
current equations in the level 1 and 2 diode models now correctly
documents how \textbf{IRF} is actually used and what temperature
dependence is applied to it.

Furthermore, since it has been observed that use of \textbf{IRF} can
introduce a discontinuity that could have significant impact on
simulation results, even resulting in simulation failure, the \Xyce{}
team intends to remove this feature in a future release.  Any use of
the \textbf{IRF} parameter in a diode model card will now result
in \Xyce{} emitting a deprecation warning even if the parameter is
being set to its default value of 1.0.  That is because
when \textbf{IRF} is specified at all, a temperature-dependent factor
is applied to it that could also cause a discontinuity.  For maximum
compatibility with other SPICE-like products, users should never
specify \textbf{IRF} in any diode model.\\ \hline

\textbf{1146-SON} Fix and test ill-defined behavior from ill-formed .AC lines &
Some ill-formed \texttt{.AC} lines would produce the ``sub-optimal''
behavior of running the simulation to completion without any errors or
warnings.  However, no output files would be made because the .AC line
had invalid parameters.  \Xyce{} now correctly emits a parsing error
if the
\texttt{<start frequency value>}, \texttt{<end frequency value>} or
\texttt{<points value>} parameters specifed on the \texttt{.AC} line are
not allowed. This also applies to the error handling for \texttt{.DATA}
commands that are used to specify the frequency steps for a \texttt{.AC}
analysis.  \\ \hline

\textbf{1112-SON}: Support for HSpice/LTSpice charge-based capacitor needed &
 Xyce now supports a ``Q'' parameter for its capacitor.  This
 parameter should be set to a solution-dependent expression that will
 be used to evaluate the capacitor charge instead of using a
 capacitance value. \\ \hline

\textbf{1087-SON}: Incorrectly formatted V and I device lines would cause \Xyce{} to segfault  &
 These netlist lines, where the value is missing after the \texttt{DC}
 parameter at the end of the netlist line, would cause \Xyce{} to
 segfault.
\begin{verbatim}
V1 1 0 DC
I2 2 0 DC
\end{verbatim}
This is fixed now.  \\ \hline

\textbf{1037-SON}: Issues with the use of TIME, TEMP and VT in .PARAM statements &
The use of the reserved words \texttt{TIME}, \texttt{TEMP} and \texttt{VT} on
\texttt{.PARAM} lines would result in the simulation running, but the resultant
parameters would evaluate to 0 without generating any warning or error message.
Parameters defined on \texttt{.PARAM} lines are supposed to evaluate to
constants in \Xyce{}, so the use of \texttt{TIME}, \texttt{TEMP} and \texttt{VT}
should not be allowed on \texttt{.PARAM} lines.  That usage will now cause a
parsing error.
\\ \hline

\textbf{1033-SON}: Issues with use of reserved words TEMP and VT in expressions
in device instance parameters and .GLOBAL PARAM statements & Prior to Release 6.10
of \Xyce{}, if the reserved words \texttt{TEMP} or \texttt{VT} were used in
expressions for device instance parameters or in \texttt{.GLOBAL\_PARAM}
statements then their values would always be 0.  For Release 6.10, it was fixed
for netlists that had a \texttt{.STEP} command.  It is now also fixed for the
``non-step case'' and for the case where \texttt{.OPTION DEVICE TEMP=<val>}
is used to set the circuit temperature.
\\ \hline

\textbf{1034-SON}: Issues with processing expressions that contain arithmetic
operators (e.g., + - or *) within an otherwise legal node or device name &
Something like this G-source instance line would produce a parsing error:
\begin{verbatim}
G1 IOUT+ IOUT- VALUE={V(VIN+)}
\end{verbatim}
The issue was that ``arithmetic operators'', like +, were not valid in device and
node names in as many expression contexts as in other Spice-like simulators.  They are now
allowed in device and node names in expressions, when they are enclosed within \Xyce{} 
operators such as V() and I().  See the ``Legal Characters in Node
and Device Names'' subsection of the \Xyce{} Reference Guide for more details.
\\ \hline

\textbf{944-SON}: Xyce-generated .csd Files for .AC with .STEP do not 
open correctly in the PSpice A/D waveform viewer &  Some of the header 
fields (e.g., \texttt{XBEGIN} and \texttt{XEND}) were incorrect.  Also, an 
attempt to view the signals in the \texttt{<netlistName>.TD.csd} file would 
cause the PSpice A/D waveform viewer to crash.  These issues are fixed now.  
\\ \hline

\textbf{855-SON}: Missing error message when a netlist uses an operator (e.g.,
IR or P) that is not supported for .AC analyses &  \Xyce{} would output all
zeroes or all NaNs, for the requested quantity, when a netlist used an operator (e.g.,
\texttt{IR} or \texttt{P}) that was unsupported for .AC or .NOISE analyses.

For .AC and .NOISE analyses,  \Xyce{} now emits a parsing error for the \texttt{P}
operator.  It also emits a parsing error for the \texttt{I} operator for all devices,
except the V, E, H and L devices and the voltage-form of the B-device.  Those five
devices do support branch currents for .AC and .NOISE analyses. \\ \hline

\textbf{697-SON} Re-measure did not support the  I() operator & In previous \Xyce{} 
releases, the \texttt{-remeasure} command line option only worked with the V() 
and N() operators.  It is now also supported for the I() operator. \\ \hline 

\textbf{649-SON}: \Xyce{} does not support traditional ternary operators in expressions &
 Most other simulators accept C-style ternary operators (``?:'') in
 expressions, and until release 6.11 \Xyce{} did not.  These operators
 are now valid in \Xyce{} expressions. \\ \hline

\textbf{2039-SRN}: \Xyce{} will not accept expressions that contain curly braces &
  Until release 6.11, \Xyce{} would not accept expressions that
  contained curly braces (other than those delimiting the expression
  itself).  As of this release, these internal braces are now treated
  as equivalent to parentheses.  The only remaining exception is expressions
  used in table definitions for E and G sources.\\ \hline


\textbf{1173-SON}: breakpoint handling is really inefficient when there is a large number of breakpoints &  It was observed that when an independent source is specified from a PWL file, and that file contains a really large number of points (thousands), that the circuit runs much more slowly than it should.  This happened because breakpoints were stored and sorted using inefficient methods.  This has been fixed for this release.  \\ \hline

\textbf{1092-SON}: Analytic and finite difference sensivities for the bsim6 and bsim3 (and possibly others) don't match well in transient &   The formula used for finite difference device parameter derivatives was not accurate.  It has been fixed and now it produces much more accurate results.  
\\ \hline

\textbf{1086-SON}: .OP and .PRINT SENS, without a .PRINT DC present, causes Xyce to exit with error  &  This was due to a small logic mistake in the parser and has been fixed.
\\ \hline

\textbf{1080-SON}: Adjoint transient sensitivities fail if using finite difference derivatives  &  This was due to the finite difference derivatives being computed and stored in an inefficient manner.  For transient adjoints, a time history of device parameter derivatives has to be stored, and this storage must be sparse in order to work well.  The code has been rewritten to use sparse storage and works correctly with transient adjoints now.
\\ \hline

\end{longtable}
}
