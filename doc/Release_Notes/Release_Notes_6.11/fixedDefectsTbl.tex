% Sandia National Laboratories is a multimission laboratory managed and
% operated by National Technology & Engineering Solutions of Sandia, LLC, a
% wholly owned subsidiary of Honeywell International Inc., for the U.S.
% Department of Energy’s National Nuclear Security Administration under
% contract DE-NA0003525.

% Copyright 2002-2019 National Technology & Engineering Solutions of Sandia,
% LLC (NTESS).

% Sandia National Laboratories is a multimission laboratory managed and
% operated by National Technology & Engineering Solutions of Sandia, LLC, a
% wholly owned subsidiary of Honeywell International Inc., for the U.S.
% Department of Energy’s National Nuclear Security Administration under
% contract DE-NA0003525.

% Copyright 2002-2019 National Technology & Engineering Solutions of Sandia,
% LLC (NTESS).


%%
%% Fixed Defects.
%%
{
\small

\begin{longtable}[h] {>{\raggedright\small}m{2in}|>{\raggedright\let\\\tabularnewline\small}m{3.5in}}
     \caption{Fixed Defects.  Note that we have two different Bugzilla systems for Sandia users.
     SON, which is on the open network, and SRN, which is on the restricted network. } \\ \hline
     \rowcolor{XyceDarkBlue} \color{white}\textbf{Defect} & \color{white}\textbf{Description} \\ \hline
     \endfirsthead
     \caption[]{Fixed Defects.  Note that we have two different Bugzilla systems for Sandia Users.
     SON, which is on the open network, and SRN, which is on the restricted network. } \\ \hline
     \rowcolor{XyceDarkBlue} \color{white}\textbf{Defect} & \color{white}\textbf{Description} \\ \hline
     \endhead

\textbf{1162-SON}: \Xyce{} behavior is counter-intuitive when \texttt{.DC} or \texttt{.STEP} parameters are inconsistent with syntax &
\Xyce{}'s .DC and .STEP can sweep linearly, by decade, or by octave in
addition to other sweep types.  When a linear, decade, or octave sweep
was specified in such a manner that the ``stop'' value could not be
reached from the ``start'' value by applying the requested step
algorithm, \Xyce{} would not perform the same simulation that other
SPICE-like simulators would have.  For the linear case, if the stop
value was smaller than start, but the step value was positive, \Xyce{}
would perform no simulation at all.  For the decade and octave sweeps
if stop was smaller than start, \Xyce{} would perform an incorrect
upward sweep starting at the start value.

\Xyce{} now performs better error checking on the arguments of \texttt{.DC}
and \texttt{.STEP}, and will now always compute only the first DC or
STEP value, even if the stepping algorithm cannot advance from
``start'' to ``stop''.  This brings \Xyce{}'s behavior in line with
those of other SPICE-like simulators.  Unlike other
simulators, \\ \hline

\textbf{1160-SON} .DATA did not work with -remeasure for .TRAN and .DC analyses &
The \texttt{.DATA} netlist command, which was introduced in release
6.10, did not work with the \texttt{-remeasure} command line
option. This is fixed now. \\ \hline

\textbf{1112-SON}: Support for HSpice/LTSpice charge-based capacitor needed &
 Xyce now supports a ``Q'' parameter for its capacitor.  This
 parameter should be set to a solution-dependent expression that will
 be used to evaluate the capacitor charge instead of using a
 capacitance value. \\ \hline

\textbf{1087-SON}: Incorrectly formatted V and I device lines would cause \Xyce{} to segfault  &
 These netlist lines, where the value is missing after the \texttt{DC}
 parameter at the end of the netlist line, would cause \Xyce{} to
 segfault.
\begin{verbatim}
V1 1 0 DC
I2 2 0 DC
\end{verbatim}
This is fixed now.  \\ \hline

\textbf{1034-SON}: Issues with processing expressions that contain arithmetic
operators (e.g., + - or *) within an otherwise legal node or device name &
Something like this G-source instance line would produce a parsing error:
\begin{verbatim}
G1 IOUT+ IOUT- VALUE={V(VIN+)}
\end{verbatim}
The issue was that ``arithmetic operators'', like +, were not valid in device and
node names in as many expression contexts as in other Spice-like simulators.  They are now
allowed in device and node names in expressions, when they are enclosed within \Xyce{} 
operators such as V() and I().  See the ``Legal Characters in Node
and Device Names'' subsection of the \Xyce{} Reference Guide for more details.
\\ \hline

\textbf{944-SON}: Xyce-generated .csd Files for .AC with .STEP do not 
open correctly in the PSpice A/D waveform viewer &  Some of the header 
fields (e.g., \texttt{XBEGIN} and \texttt{XEND}) were incorrect.  Also, an 
attempt to view the signals in the \texttt{<netlistName>.TD.csd} file would 
cause the PSpice A/D waveform viewer to crash.  These issues are fixed now.  
\\ \hline

\textbf{855-SON}: Missing error message when a netlist uses an operator (e.g.,
IR or P) that is not supported for .AC analyses &  \Xyce{} would output all
zeroes or all NaNs, for the requested quantity, when a netlist used an operator (e.g.,
\texttt{IR} or \texttt{P}) that was unsupported for .AC or .NOISE analyses.

For .AC and .NOISE analyses,  \Xyce{} now emits a parsing error for the \texttt{P}
operator.  It also emits a parsing error for the \texttt{I} operator for all devices,
except the V, E, H and L devices and the voltage-form of the B-device.  Those five
devices do support branch currents for .AC and .NOISE analyses. \\ \hline

\textbf{697-SON} Re-measure did not support the  I() operator & In previous \Xyce{} 
releases, the \texttt{-remeasure} command line option only worked with the V() 
and N() operators.  It is now also supported for the I() operator. \\ \hline 

\textbf{649-SON}: \Xyce{} does not support traditional ternary operators in expressions &
 Most other simulators accept C-style ternary operators (``?:'') in
 expressions, and until release 6.11 \Xyce{} did not.  These operators
 are now valid in \Xyce{} expressions. \\ \hline

\textbf{2039-SRN}: \Xyce{} will not accept expressions that contain curly braces &
  Until release 6.11, \Xyce{} would not accept expressions that
  contained curly braces (other than those delimiting the expression
  itself).  As of this release, these internal braces are now treated
  as equivalent to parentheses.  \\ \hline

\end{longtable}
}
