% Sandia National Laboratories is a multimission laboratory managed and
% operated by National Technology & Engineering Solutions of Sandia, LLC, a
% wholly owned subsidiary of Honeywell International Inc., for the U.S.
% Department of Energy’s National Nuclear Security Administration under
% contract DE-NA0003525.

% Copyright 2002-2020 National Technology & Engineering Solutions of Sandia,
% LLC (NTESS).

% Sandia National Laboratories is a multimission laboratory managed and
% operated by National Technology & Engineering Solutions of Sandia, LLC, a
% wholly owned subsidiary of Honeywell International Inc., for the U.S.
% Department of Energy’s National Nuclear Security Administration under
% contract DE-NA0003525.

% Copyright 2002-2020 National Technology & Engineering Solutions of Sandia,
% LLC (NTESS).


%%
%% Known defects and workarounds.
%%
%% This section should highlight significant defects that were not fixed in
%% the release

{
\small

\begin{longtable}[h] {>{\raggedright\small}m{2in}|>{\raggedright\let\\\tabularnewline\small}m{3.5in}}
  \caption{Known Defects and Workarounds.} \\ \hline
  \rowcolor{XyceDarkBlue} \color{white}\textbf{Defect} & \color{white}\textbf{Description} \\ \hline \endfirsthead
  \caption[]{Known Defects and Workarounds.} \\ \hline
  \rowcolor{XyceDarkBlue} \color{white}\textbf{Defect} & \color{white}\textbf{Description}
  \\ \hline \endhead
% EXAMPLE:
%\textbf{bug number-SRN}: bug title & Description of KNOWN BUG THAT HAS NOT BEEN
  %FIXED.

%\textbf{\textit{Workaround}}: Describe how to work around this bug.
%\\ \hline
%
%
\textbf{1262-SON}: Duplicate L device definitions are not a parsing error
when one of the duplicate L devices is part of a K device &
As an example, this netlist will not produce a parsing error.  Instead,
the first L1 definition will be used in the K1 device definition.
\begin{verbatim}
* parsing fails to detect duplicate L1 devices
V1 1 0 SIN(0 1 1KHz)
L1 1 2 1e-3
R1 2 0 1
C1 2 0 1e-9
* mutual inductor definition, with duplicate L1 device
L1 4 0 1e-6
L2 3 0 1mH
K1 L1 L2 0.75
.TRAN 0 1ms
.PRINT TRAN V(1) v(2)
.END
\end{verbatim}

\textbf{\textit{Workaround}}:
There is none.
\\ \hline

\textbf{1241-SON}: Expression library parsing bottleneck on large expressions &
It has been determined that the expression library in Xyce can be the
source of a severe parsing bottleneck when expressions are large and
complex.  Expressions of this sort show up most often when parsing
large PDKs with complex use of the \texttt{.FUNC} feature, and when
using the ``tablefile'' feature to import a large file of time/voltage
pairs for use in a \texttt{B} source.

\textbf{\textit{Workaround}}:
There is currently no workaround for the issue of complex PDK function
use, and the team is working on fixing this issue by redesigning the
way Xyce handles expressions with user defined functions.  For the
``tablefile'' issue, one should avoid using \texttt{B} sources with
``tablefile'' to read in large tables, and instead use the ``PWL
FILE'' option of the \texttt{V} source, which does not have this
parsing issue.
\\ \hline

\textbf{1085-SON}: Expression library mishandles \texttt{.FUNC} definitions of functions that begin with ``I'' and are two characters long &
\Xyce{}'s expression library assumes that all terms of the form
``\texttt{Ix(<arguments>)}'' are lead current expressions, where ``x'' is
either a lead designator such as ``D'', ``G'' , or ''S'' for a MOSFET
or ``C'', ``B'', ``E'' for a BJT, or a digit indicating the pin number
of the device associated with the lead.  This assumption makes it
impossible for users to define a function with a two-character name
starts with ``I''.  Unfortunately, the parser does not warn of this
problem should a user define such a function, and the first indication
of something being wrong is an unhelpful error message about an
``undefined parameter or function'' where the problematic function is
used.

\textbf{\textit{Workaround}}: Do not use function names of two character
length that begin with the letter ``I''.  If you are making use of a
vendor-supplied library that includes definitions of functions such as
``I0'', you will have to modify the library to change the function
name and all the instances of its use.
\\ \hline

\textbf{1037-SON}: The use of non-constant values in .PARAM
statements may lead to unexpected results &  This netlist line
(\texttt{.PARAM PA = \{TEMP\}}) is forbidden in Xyce since
the special variable \texttt{TEMP} is not constant.  However,
that netlist line will not produce a \Xyce{} parsing error, and
the value of \texttt{PA} in the simulation may then be set to 
zero in some contexts.

\textbf{\textit{Workaround}}: Non-constant values should only
be used in \texttt{.GLOBAL PARAM} statements in \Xyce{}. This
restriction may be different than in other Spice-like 
simulators.
\\ \hline

\textbf{1031-SON}: .OP output is incomplete in parallel & When \Xyce{}
is run in parallel, the \texttt{.OP} output may be incomplete. 

\textbf{\textit{Workaround}}: One workaround is to run the netlist in 
serial.  Another one is to use these command line options: \texttt{
-per-processor -l output}.  In that case, the per-processor log files 
will have the \texttt{.OP} information for the devices that were 
instantiated on each processor.
\\ \hline 

  \textbf{1009-SON}:  Transient adjoint sensitivities don't work with \texttt{.STEP}
  & Transient adjoint sensitivities require backward integrations to be performed after the primary transient forward integration.  Doing this properly requires information to be stored during the forward solve, and for certain bookkeeping to be performed.  Currently, these extra operations to support transient adjoints are not properly set up for \texttt{.STEP} analysis.

\textbf{\textit{Workaround}}: None
\\ \hline

  \textbf{1006-SON}:   SDT (expression library time integration) derivatives are not supported, so SDT can't be used for sensitivity analysis objective functions &
  SDT is a function supported by the \Xyce{} expression library to compute numerical time integration.  When this function is used, the expression library does not produce correct derivatives.  This impacts Jacobian matrix entries, when SDT is used with a Bsrc, and it also impacts sensitivity analysis, when SDT is used in an objective function.  For the former case, this can result in a lack of robustness for circuits that contain SDT-Bsrc devices.  For the latter case, the objective function will simply be incorrect.

\textbf{\textit{Workaround}}: None
\\ \hline

\textbf{1004-SON}: Ill-defined .STEP behavior for "default parameters" for 
transient sources (SIN, EXP, PWL, PULSE and SFFM) & If, for example,
these netlist lines are used in a transient (\texttt{.TRAN}) simulation:
\begin{verbatim}
V1 1 0 SIN(0 1 1)
.STEP V1 1 2 1
\end{verbatim}
then \Xyce{} will run the simulation without warnings or errors, but
no instance parameter of source \texttt{V1} will be stepped.  

\textbf{\textit{Workaround}}: Explicitly use the desired stepped parameter
(e.g., \texttt{V0}) on the \texttt{.STEP} line.  For example, 
\texttt{.STEP V1:V0 1 2 1} would work correctly.
\\ \hline

\textbf{991-SON}: Non-physical BH Loops in non-linear mutual inductor &
Nonlinear mutual inductors that have high coupling coefficients (i.e. 
model parameter \texttt{ALPHA} over 1.0e-4) and low loss characteristics 
(i.e. zero \texttt{GAP}) can produce B-H loops with nonphysical hysteresis.

\textbf{\textit{Workaround}}: Lower \texttt{ALPHA} values or larger 
\texttt{GAP} values can ameliorate this issue, but the root cause is 
still under investigation. 
\\ \hline

\textbf{800-SON}: Use of global parameters in expressions on .MEASURE lines
will yield incorrect results & The use of global parameters in expressions
on .MEASURE lines is not allowed, as documented in the \Xyce{} Reference Guide.
However, instead of producing a parsing error the measure statement will be
evaluated with the specified qualifier value (e.g., \texttt{FROM}) being left
at its default value.

\textbf{\textit{Workaround}}: None, other than not doing this.
\\ \hline

\textbf{970-SON}: Some devices do not work in frequency-domain analysis &
Devices that may be expected to work in AC or HB analysis do not at
this time.  For AC this includes, but is not limited to, the lossy
transmission line (LTRA) and lossless transmission line (TRA).  For
HB, the transmission lines do work but the nonlinear dependent sources
(B source and nonlinear E, F, G, or H source) do not.

\textbf{\textit{Workaround}}: The LTRA and TRA models will need to be replaced
with lumped transmission line models (YTRANSLINE) for AC analysis.
There is not yet a workaround for the B source in harmonic balance.
\\ \hline

\textbf{967-SON}: Zoltan segmentation fault with OpenMPI 2.1.x and 3.0.0 on
some systems &

It has been observed that when \Xyce{} and Trilinos are built with
OpenMPI 2.1.x or 3.0.0 on certain unsupported operating systems, a
small number of test cases in the regression suite crash with a
segmentation fault inside the Zoltan library.

The \Xyce{} team has determined that this is not a bug in
either \Xyce{} or Zoltan, but is instead due to some pre-packaged OpenMPI
binaries on some operating systems having been built with an
inappropriate option.  This option, ``--enable-heterogeneous'' is
explicitly documented in OpenMPI documentation as broken and unusable
since 2013, but some package managers have OpenMPI binaries built with
this option explicitly enabled.  Turning on this option causes the
resulting OpenMPI build to perform certain communication operations in
a way that does not adhere to the MPI standard.  There is nothing that
can be done in \Xyce{} or Zoltan to fix this issue --- it is entirely
a bug in the OpenMPI library as built on that system.

A new test case has been added to the \Xyce{} test suite in order to
detect this problem.  The test is ``MPI\_Test/bug\_967'', and it will
be run whenever the test suite is invoked with the ``+parallel'' tag
as described in the documentation for the test suite
at \url{https://xyce.sandia.gov/documentation/RunningTheTests.html}.
If this test fails, your system has a broken OpenMPI build that cannot
be used with \Xyce{}.

At the time of this writing, this issue is present in Ubuntu Linux
versions 17.10 and later, and there is an open bug report for it
at \url{https://bugs.launchpad.net/ubuntu/+source/openmpi/+bug/1731938}.

The issue may be present in other distros of Linux that are derived
from Debian (as is Ubuntu), but we cannot confirm this.

\textbf{\textit{Workaround}}:
The only workaround for this problem is to build OpenMPI from source
yourself, and not to include ``--enable-heterogeneous'' in its
configure options.

You should also post a bug report in your operating system's issue
tracker requesting that they rebuild their OpenMPI binaries without
the ``--enable-heterogeneous'' option.  If you are using Ubuntu, you
should register with that issue tracking system and add yourself to
the list of people it affects in the existing bug report (doing so
increases the ``heat'' of the bug, which may increase the likelihood
of it being fixed).

\\ \hline

\textbf{964-SON}: Compatibility of .PRINT TRANADJOINT with .STEP & The use of
\texttt{.PRINT TRANADJOINT} is not compatible with \texttt{.STEP}.  The
resultant \Xyce{} output will not be correct.

\textbf{\textit{Workaround}}: There is none.
\\ \hline

\textbf{932-SON}: Analysis lines do not support expressions for their
operating parameters & The \Xyce{} parser and analysis handlers do not
yet support the use of expressions on netlist analysis lines such
as \texttt{.TRAN}.  The parameters of these analysis lines (such as
stop time for \texttt{.TRAN} or fundamental frequency
for \texttt{.HB}) may only be expressed as literal numbers.

\textbf{\textit{Workaround}}: There is no workaround internal to \Xyce{}.
Use of an external netlist preprocessor would be required. \\ \hline

\textbf{883-SON} .PREPROCESS REPLACEGROUND does not work on nodes referenced
in expressions & The \texttt{.PREPROCESS REPLACEGROUND} feature does
not replace ground synonyms if they appear in B source expressions.

\textbf{\textit{Workaround}}: Do not use ground synonyms (\texttt{GND},
\texttt{GROUND}, etc.) in expressions.  Use a literal ``0'' when
referring to the ground node in expressions.\\ \hline

\textbf{812-SON}: Undocumented limitations on, and bugs with, parameter and
global parameter names & Based on external customer input and
pre-release testing, there are some bugs and undocumented limitations
on parameter and global parameter names in
\Xyce{}. Parameters and global parameters should start with a letter, rather
than with a number or ``special'' character like \#.  In addition, the
use of a single character $V$ as a global parameter name can result in
either netlist parsing failures or incorrect results
from \texttt{.PRINT} lines.  \\ \hline

\textbf{807-SON}: BSIM4 convergence problems with non-zero rgatemod value &
There have been reports of convergence problems (e.g., the \Xyce{}
simulation fails part way through and says that the ``time step is too
small'') when the \texttt{rgatemod} parameter is non-zero. \\ \hline

\textbf{794-SON}: Bug in TABLE Form of \Xyce{} Controlled Sources & In some case, a \Xyce{}
netlist that contains a controlled source that uses the TABLE form will get
the correct answer at first.  However, it may then "stall" (e.g, keep
taking really small time-steps) and never complete the simulation run.

\textbf{\textit{Workaround}}: In some cases, the TABLE specification for the controlled
source can be replaced with a Piecewise Linear (PWL) source that uses
nested IF statements. \\ \hline

\textbf{783-SON}: Use of ddt in a B-Source definition may produce incorrect
results & The \texttt{DDT()} function from the \Xyce{} expression
package, which implements a time derivative, may not function
correctly in a B-Source definition.

\textbf{\textit{Workaround}}: None. \\ \hline

\textbf{727-SON}: \Xyce{} parallel builds hang randomly on OS X & During
Sandia's internal nightly testing of the OSX parallel builds, we see
that \Xyce{} ``hangs on exit'' with an estimated frequency of less
than 1-in-5000 simulation runs.  We have not seen this issue with
parallel builds for either RHEL6 or BSD.  The hang is on exit, whether
on a successful exit or on an error exit.  The hang occurs after all
of the \Xyce{} output has occurred though.  So, the user will get
their sim results, but might have trouble if the individual \Xyce{}
runs are part of a larger script.

\textbf{\textit{Workaround}}: None. \\ \hline

\textbf{661-SON} Lead currents and power accessors (I(), P() and W()) do
not work properly in .RESULT Statements & There are two issues.
First, \texttt{.RESULT} statements will fail netlist parsing if the
requested lead current is omitted from the \texttt{.PRINT TRAN}
line.  As an example, this statement (\texttt{.RESULT I(R1)}) requires
either \texttt{I(R1)},
\texttt{P(R1)} or \texttt{W(R1)} to be on the \texttt{.PRINT TRAN} line.
Second, the output value, in the \texttt{.res} file, for the lead
current or power calculation will always be zero.
\\ \hline

\textbf{583-SON}: Switch with RON=0 leads to convergence failure. &
The switch device does not prevent a user from
specifying \texttt{RON=0} in its model, but then takes the inverse of
this value to get the ``on'' conductance.  The resulting invalid
division will either lead to a division by zero error on platforms
that throw such errors, or produce a conductance with ``Not A Number''
or ``Infinity'' as value.  This will lead to a convergence failure.

\textbf{\textit{Workaround}}: Do not specify an identically zero resistance
for the switch's ``on'' value.  A small value of resistance such as
1e-15 or smaller will generally work well as a substitute. \\ \hline


\textbf{469-SON}: Belos memory consumption on FreeBSD and excessive CPU on other
platforms & Memory or thread bloat can result when using multithreaded
dense linear algebra libraries, which are employed by Belos.  If this
situation is observed, either build
\Xyce{} with a serial dense linear algebra library or use environment variables
to control the number of spawned threads in a multithreaded library.
\\ \hline


\textbf{468-SON}: It should be legal to have two model cards with the same model
name, but different model types. & SPICE3F5 and ngspice only require
that model cards of the same type have unique model names. They accept
model cards of different types with the same name.  \Xyce{} requires
that all model card names be unique.
\\ \hline


\textbf{250-SON}: NODESET in \Xyce{} is not equivalent to NODESET in SPICE & As
currently implemented, \texttt{.NODESET} applies the initial
conditions given throughout a full nonlinear solve for the operating
point, then uses the result as an initial guess for a second nonlinear
solve with no constraints.  This is not the same as SPICE, which
merely applies the given initial conditions to a single nonlinear
solve for the first two iterations, then lets the problem converge
with no further constraints.  This can lead to
a \Xyce{} \texttt{.NODESET} failing where the same netlist in SPICE
might not, if the initial conditions are such that a full nonlinear
solve cannot converge with those constraints in place.  There is no
workaround.
\\ \hline

\textbf{247-SON}: Expressions don't work on .options lines & Expressions
enclosed in braces (\{ \}) are handled specially throughout \Xyce{},
and may only be used in certain contexts such as in device model or
instance parameters or on \texttt{.PRINT} lines.
\\ \hline


\textbf{49-SON} \Xyce{} BSIM models recognize the model TNOM, but not the
instance TNOM & Some simulators allow the model parameter TNOM of BSIM
devices to be specified on the instance line, overriding the model
parameter TNOM.  \Xyce{} does not support this.
\\ \hline


\textbf{27-SON}: Fix handling of .options parameters & When specifying .options
for a particular package, what gets applied as the non-specified
default options might change.  \\ \hline

\textbf{2119-SRN}: Voltages from interface nodes for subcircuits do not 
work in expressions used in device instance parameters & This bug can be
illustrated with this netlist fragment:
\begin{alltt}
X1 1 2 MySub
.SUBCKT MYSUB a c
R1   a b 0.5
R2   b c 0.5
.ENDS
B1 3 0 V=\{V(X1:a)\}
\end{alltt}
This fragment will produce the netlist parsing error \texttt{Directory 
node not found: X1:A}.  The workaround is to use \texttt{V=\{V(1)\}} 
in the B-source expression instead.  This bug also affects the 
solution-dependent capacitor.
\\ \hline

\textbf{1923-SRN}: LC lines run out of memory, even if equivalent (larger) RLC
lines do not. & In some cases, circuits that run fine using an RLC
approximation for a transmission line, exit with an out-of-memory
error if the (supposedly smaller) LC approximation is used.
\\ \hline

\textbf{1595-SRN}: \Xyce{} won't allow access to inductors within subcircuits
for mutual inductors external to subcircuits & It is not possible to
have a mutual inductor outside of a subcircuit couple to inductors in
a subcircuit.

\textbf{\textit{Workaround}}: Put all inductors and mutual inductance lines
that couple to them together at the same level of circuit hierarchy.
\\ \hline

\end{longtable}
}
