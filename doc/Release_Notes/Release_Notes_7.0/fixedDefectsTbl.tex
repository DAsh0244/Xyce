% Sandia National Laboratories is a multimission laboratory managed and
% operated by National Technology & Engineering Solutions of Sandia, LLC, a
% wholly owned subsidiary of Honeywell International Inc., for the U.S.
% Department of Energy’s National Nuclear Security Administration under
% contract DE-NA0003525.

% Copyright 2002-2021 National Technology & Engineering Solutions of Sandia,
% LLC (NTESS).

% Sandia National Laboratories is a multimission laboratory managed and
% operated by National Technology & Engineering Solutions of Sandia, LLC, a
% wholly owned subsidiary of Honeywell International Inc., for the U.S.
% Department of Energy’s National Nuclear Security Administration under
% contract DE-NA0003525.

% Copyright 2002-2021 National Technology & Engineering Solutions of Sandia,
% LLC (NTESS).


%%
%% Fixed Defects.
%%
{
\small

\begin{longtable}[h] {>{\raggedright\small}m{2in}|>{\raggedright\let\\\tabularnewline\small}m{3.5in}}
     \caption{Fixed Defects.  Note that we have two different Bugzilla systems for Sandia users.
     SON, which is on the open network, and SRN, which is on the restricted network. } \\ \hline
     \rowcolor{XyceDarkBlue} \color{white}\textbf{Defect} & \color{white}\textbf{Description} \\ \hline
     \endfirsthead
     \caption[]{Fixed Defects.  Note that we have two different Bugzilla systems for Sandia Users.
     SON, which is on the open network, and SRN, which is on the restricted network. } \\ \hline
     \rowcolor{XyceDarkBlue} \color{white}\textbf{Defect} & \color{white}\textbf{Description} \\ \hline
     \endhead
\textbf{1200-SON}: Model binning was not supported for the BSIM CMG model &
As of this release model binning is now supported for
the \texttt{LMIN}, \texttt{LMAX}, \texttt{NFINMIN}
and \texttt{NFINMAX} binning parameters needed by the BSIM CMG model.
\\ \hline


\textbf{1257-SON}: Let Xyce/ADMS support JFET model types &
In prior versions, Xyce/ADMS supported use of
the \texttt{xyceModelGroup} module attribute to define a model as
being a MOSFET, BJT, Diode, resistor, or capacitor, and plug into Xyce
using the same spice format as other models of that type.  As of this
release, JFET is also supported. \\ \hline

\textbf{1256-SON}: Xyce/ADMS analog function error &
Xyce/ADMS would emit an error about an undefined variable if one
used \$vt as the input argument to an analog function that had both
input and output arguments. \\ \hline

\textbf{1252-SON}: Xyce segfaults when .OPTIONS OUTPUT PRINTHEADER=FALSE
is used  &  Xyce would segfault if a \texttt{.OPTIONS OUTPUT PRINTHEADER=FALSE}
line was used and the netlist contained either a \texttt{.PRINT SENS} line
for an AC analysis or a \texttt{.PRINT ES} line.
\\ \hline

\textbf{971-SON}: Use of default device parameter syntax on a .PRINT line causes
\Xyce{} to print 0 for that parameter & This line (\texttt{.PRINT TRAN R1})
would cause \Xyce{} to print 0 for the resistance value of R1.  This line
(\texttt{.PRINT TRAN X1:R1}) would be a netlist parsing error.  This bug
also affected the C, L, I and V devices.
\\ \hline

\textbf{928-SON}: The .hb\_ic.prn file could be incorrect when .STEP is
used with .HB & \Xyce{} should only output the initial condition (IC)
data for the accepted tolerance in the \texttt{<netlist-name>.hb\_ic.prn}
file.  However, in previous releases, it would output all of the
intermediate IC data while harmonic balance tried to find a good
tolerance if \texttt{.STEP} was used with \texttt{.HB}. For that
case, the IC data will now be initially output to a ``tmp file''
(e.g., \texttt{<netlist-name>.hb\_ic.prn.tmp}). If that IC data meets
the required tolerance then it will be copied to the end of the
\texttt{<netlist-name>.hb\_ic.prn} file, and the tmp file 
will be deleted. \\ \hline

\end{longtable}
}
