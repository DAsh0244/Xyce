% Sandia National Laboratories is a multimission laboratory managed and
% operated by National Technology & Engineering Solutions of Sandia, LLC, a
% wholly owned subsidiary of Honeywell International Inc., for the U.S.
% Department of Energy’s National Nuclear Security Administration under
% contract DE-NA0003525.

% Copyright 2002-2019 National Technology & Engineering Solutions of Sandia,
% LLC (NTESS).


%%-------------------------------------------------------------------------
%% Purpose        : Release notes for Xyce version 6.0.
%% Special Notes  : Graphic files (pdf format) work with pdflatex.  
%% Creation Date  : {03/24/2006}
%%-------------------------------------------------------------------------

\documentclass[11pt,report,strict]{SANDreport}
\usepackage[hyperindex=true, colorlinks=false]{hyperref}
\usepackage{makeidx, ltxtable, multirow}
\usepackage{Xyce}
\pdfcatalog {/PageMode /UseNone}

% ---------------------------------------------------------------------------- %
%
% Set the title, author, and date
%
    \title{\XyceTitle{} Parallel Electronic Simulator Release Notes\\Release 6.0}
    \XyceVersion{6.0}

    % There is a "Printed" date on the title page of a SAND report, so
    % the generic \date should generally be empty.
    \date{}

% ---------------------------------------------------------------------------- %
% Set some things we need for SAND reports. These are mandatory
%
%Submitted to R&A 28 Aug 2013 Tracking number 5327111,  approved 29 Aug

\SANDnum{SAND2013-7298}
\SANDprintDate{August 2013}
\author{Eric R. Keiter, 
            Ting Mei, 
            Thomas V. Russo, \\
            Richard L. Schiek, 
            Peter E. Sholander,
            Heidi K. Thornquist,
            Jason C. Verley,\\
            Electrical Models and Simulation \\[1ex]
\\
            Sandia National Laboratories\\
            P.O. Box 5800\\
            Albuquerque, NM 87185-1177 \\
\\
\and 
David G. Baur\\
  Raytheon\\
  1300 Eubank Blvd\\
  Albuquerque, NM 87123 \\
}
\SANDauthor{Eric R. Keiter, Ting Mei, Thomas V. Russo, Richard L. Schiek, Peter E. Sholander, Heidi K. Thornquist, Jason C. Verley, David G. Baur}
% ---------------------------------------------------------------------------- %
% Start the document

\begin{document}
\maketitle
%\draft

% ---------------------------------------------------------------------- %
% This is where the body of the report begins; usually with an Introduction
%
\SANDmain		% Start the main part of the report
\section{Scope/Product Definition}

The \Xyce{} Parallel Electronic Simulator has been written to support, in a
rigorous manner, the simulation needs of the Sandia National Laboratories
electrical designers.  Specific requirements include, among others, the
ability to solve extremely large circuit problems by supporting large-scale
parallel computing platforms, improved numerical performance and
object-oriented code design and implementation.

The \Xyce{} release notes describe:
\begin{XyceItemize}
  \item Hardware and software requirements
  \item New features and enhancements
  \item Any defects fixed since the last release
  \item Current known defects and defect workarounds
\end{XyceItemize}

For up-to-date information not available at the time these notes 
were produced, please visit the \Xyce{} web page at 
{\color{XyceDeepRed} http://xyce.sandia.gov/}.


\section{Hardware/Software}
This section gives basic information on supported platforms and hardware
and software requirements for running \Xyce{} 6.0.

\subsection{Supported Platforms (Certified Support)}
\Xyce{} 6.0 currently supports any of the following operating system (all
versions imply the earliest supported -- \Xyce{} generally works on later
versions as well) platforms.  These platforms are supported in the sense
that they have been subject to certification testing for the \Xyce{} version 
6.0 release.  

\begin{XyceItemize}
\item Red Hat Enterprise Linux${}^{\mbox{\textregistered}}$ 5, x86 (serial only) and x86-64 (serial and parallel)
\item Red Hat Enterprise Linux${}^{\mbox{\textregistered}}$ 6, x86-64 (serial and parallel)
\item Microsoft Windows 7${}^{\mbox{\textregistered}}$, x86 (serial)
\item Apple${}^{\mbox{\textregistered}}$ OS X, x86-64 (serial and parallel)
\item TLCC (serial and parallel)
\item Red Sky (serial and parallel)
\end{XyceItemize}

\subsection{Build Supported Platforms (not Certified)}
The platforms listed in this section are ``not supported'' in the
sense that they are not subject to nightly regression testing, and
they also were not subject to certification testing for the \Xyce{}
version 6.0 release.  Despite this lack of testing rigor, \Xyce{} has
been known to run under these configurations.

\begin{XyceItemize}
\item FreeBSD 9.x on Intel x86 and x86-64 architectures (serial and parallel)
\item Distributions of Linux other than Red Hat Enterprise Linux
\item Microsoft Windows under Cygwin and MinGW.
\end{XyceItemize}

Please contact the Xyce development team for platform and configuration availability.

\subsection{Hardware Requirements}

The \Xyce{} team has not determined a minimum memory or processor
speed requirement.  Any modern computer should have enough memory and
processor power to run moderately sized circuits in serial with
\Xyce{}.  Naturally, memory requirements grow with problem size.  

Running \Xyce{} in parallel will require a system with at least two
processors. For problems of the size where parallel operation is
beneficial (thousands of devices per processor), one can expect to
need several gigabytes of memory per processor. 

For the very largest problems that \Xyce{} can run (millions of
devices), one will require a cluster system with a sufficient number
of nodes to distribute the problem efficiently, and sufficient memory
per node to support the distributed problem.  

\subsection{Software Requirements}
Several libraries are required to run \Xyce{} or build \Xyce{} from
source on a platform.  Serial versions of the \Xyce{} binary have no
run-time software requirements, as they ship with all the shared
libraries they need.  However, parallel versions require the following
software at run time:
\begin{XyceItemize}
\item Open MPI ({\color{XyceDeepRed} http://www.open-mpi.org/}) (version 1.4 or higher)    
\item TLCC and Red Sky users can load the {\bf xyce} module to properly set the environment 
\end{XyceItemize}

Several libraries (all freely available from Sandia National
Laboratories or other sites) are always required when building \Xyce{}
from source.  These are:
\begin{XyceItemize}
\item Trilinos Solver Library version 11.2.4 (Sandia, 
{\color{XyceDeepRed} http://trilinos.sandia.gov}) .  This is a suite
of libraries including Amesos, AztecOO, Belos, Teuchos, Epetra, EpetraExt, Ifpack, NOX, LOCA, Sacado, Zoltan.
\item UMFPACK version 4.1 and AMD version 1.0 (libumfpack.a, libamd.a)
  ({\color{XyceDeepRed}
  \url{http://www.cise.ufl.edu/research/sparse/umfpack/}}).  This may be provided by the SuiteSparse package that is available on many systems.
\item LAPACK 
\item BLAS
\end{XyceItemize}
For parallel builds, the following libraries are additionally required:
\begin{XyceItemize}
\item Open MPI ({\color{XyceDeepRed} http://www.open-mpi.org}) library
  for message passing (version 1.4 or higher).  The version used to
  build Xyce must be the same that is used for building Trilinos.
\item ParMETIS ({\color{XyceDeepRed}
  http://glaros.dtc.umn.edu/gkhome/views/metis}) library for graph
  partitioning (version 3.1 or higher).  The MPI compiler used to
  compile ParMETIS must be the same that is used for Trilinos and
  Xyce.
\end{XyceItemize}

\section{\Xyce{} Release 6.0 Documentation}
The following \Xyce{} documentation is available at the \Xyce{}
website in pdf form.
\begin{XyceItemize}
\item \Xyce{} Release Notes, Version 6.0
\item \Xyce{} Users' Guide, Version 6.0
\item \Xyce{} Reference Guide, Version 6.0
\item \Xyce{} Building Guide, Version 6.0
\item \Xyce{} Theory Document
\item EKV MOSFET version 3.0.1 model documentation.
\end{XyceItemize}

\section{New Features and Enhancements}

Highlights for \Xyce{} Release 6.0 include:

\begin{XyceItemize}
\item \Xyce{} is being released as open source software under the Gnu Public License, version 3.
\item A new variable order Gear time integration method in addition to the existing Trapezoid and BDF integrators.
\item Improved support for small-signal frequency domain (.AC) analysis for improved compatibility with SPICE and related analog circuit simulation tools.
\item Improved support for harmonic balance (.HB) analysis.
\item Extended support for .MEASURE that allows post-processing of output data, including Fourier analysis post-processing.
\item A SPICE3F5-compatible lossy transmission line model has been added.
\item Direct and adjoint sensitivities are now available for DC calculations.
\end{XyceItemize}

For details of each of these new features, see the \Xyce{} Users'
Guide, the \Xyce{} Installation Guide, and the \Xyce{} Reference Guide.  Also, a more complete listing of
new features and improvements are given in the following sections.

\subsection{Device Support}

Table~\ref{deviceListTable} contains a complete list of devices 
for \Xyce{} Release 6.0.

% device list 
\LTXtable{\textwidth}{deviceListTbl}

\subsection{Robustness Improvements}
\begin{XyceItemize}
\item Solver defaults have been modified for increased robustness of
  parallel runs with a minimum of user-specified options.
\end{XyceItemize}

% new device types:
\subsection{New Devices}
\begin{XyceItemize}
\item FBH HBT\_X (version 2.1) Heterojunction Bipolar Transistor model
\item SPICE3F5-compatible lossy transmission line
\end{XyceItemize}

\subsection{Enhanced Solver Stability, Performance and Features}
\begin{XyceItemize}
\item Xyce has been updated to use Trilinos 11.2, which has resulted in significant performance improvements.
\end{XyceItemize}

%\subsection{Interface Improvements}
%\begin{XyceItemize}
%\item   
%\end{XyceItemize}

\newpage
\section{Defects of Release 5.3 Fixed in this Release}
\LTXtable{\textwidth}{fixedDefectsTbl}

\newpage

\section{Known Defects and Workarounds}
\LTXtable{\textwidth}{knownDefectsTbl}
\newpage

\section{Incompatibilities With Other Circuit Simulators}
\LTXtable{\textwidth}{incompTbl}

\newpage
\section{Important Changes to \Xyce{} Usage Since the Release 5.3.}
Table ~\ref{newUsage} lists some usage changes for \Xyce{}.
\LTXtable{\textwidth}{changesInputTbl}

\clearpage
% acknowledgements, trademarks, contact information.
\LTXtable{\textwidth} {acktrade}
\begin{SANDdistribution}[NM]% or [CA]
    % \SANDdistCRADA	% If this report is about CRADA work
    % \SANDdistPatent	% If this report has a Patent Caution or Patent Interest
    % \SANDdistLDRD	% If this report is about LDRD work

    % External Address Format: {num copies}{Address}
%    \SANDdistExternal{}{}
    \bigskip

    % The following MUST BE between the external and internal distributions!
    % \SANDdistClassified % If this report is classified

    % Internal Address Format: {num copies}{Mail stop}{Name}{Org}
%    \SANDdistInternal{}{}{}{}

    % Mail Channel Address Format: {num copies}{Mail Channel}{Name}{Org}
%    \SANDdistInternalM{}{}{}{}
\end{SANDdistribution}


\end{document}
