% Sandia National Laboratories is a multimission laboratory managed and
% operated by National Technology & Engineering Solutions of Sandia, LLC, a
% wholly owned subsidiary of Honeywell International Inc., for the U.S.
% Department of Energy’s National Nuclear Security Administration under
% contract DE-NA0003525.

% Copyright 2002-2019 National Technology & Engineering Solutions of Sandia,
% LLC (NTESS).


%%
%% Fixed Defects.
%%
{
\small

\begin{longtable}[h] {>{\raggedright\small}m{2in}|>{\raggedright\let\\\tabularnewline\small}m{3.5in}}
     \caption{Fixed Defects.  Note that we have two different Bugzilla systems for Sandia users.
     SON, which is on the open network, and SRN, which is on the restricted network. } \\ \hline
     \rowcolor{XyceDarkBlue} \color{white}\bf Defect & \color{white}\bf Description \\ \hline
     \endfirsthead
     \caption[]{Fixed Defects.  Note that we have two different Bugzilla systems for Sandia Users.
     SON, which is on the open network, and SRN, which is on the restricted network. } \\ \hline
     \rowcolor{XyceDarkBlue} \color{white}\bf Defect & \color{white}\bf Description \\ \hline
     \endhead

     %\textbf{bug xxxx }: Describe issue & Describe fix.  \\ \hline
     \textbf{302-SON}:  Support re-measure of existing \Xyce{} simulation data  & 
     Re-measure was documented in the \Xyce{} 6.3 Reference Guide.  Subsequent testing showed
     that re-measure did not work correctly for: a) parallel execution of \Xyce{}; or b)
     when the measure line used an expression. This has been fixed.  
     \\ \hline 

     \textbf{308-SON}: \texttt{.STEP} analysis has a subtle bug for transient
     simulation & \texttt{.STEP} analysis applies an outer parameter stepper
     loop around another analysis type such as transient.  When \texttt{.STEP}
     was applied to transient, the transient output was not identical to the
     output from equivalent stand-alone transient analyses. This bug was only
     present in the BDF time integrator.  This has been fixed.  
     \\ \hline 

     \textbf{414-SON}:  Fix the nonlinear solver convergence tests to be
     simpler and make more sense & The nonlinear solver in \Xyce{} had an
     overly-complex set of convergence tests that were not necessary.
     Occasionally, this complexity would prevent convergence of circuits, or under very
     rare instances, allow solver failures to be declared as passed.  This
     has been fixed.  
     \\ \hline 

     \textbf{440-SON}: \texttt{.MEASURE} should properly support signals with DC offsets and
     measures that use resistor lead currents & \texttt{.MEASURE} statements did not 
     function correctly if the measured quantity either did not start at 0 at time=0 or 
     used a resistor lead current.  
     Simple examples were measuring the maximum level of the voltage source defined as
     \texttt{VCOS 1 1a SIN(0 5 100K -2.5U)} or measuring the maximum current level
     through a resistor with \texttt{.MEASURE TRAN MAXIR MAX I(RTEST)}.  
     \\ \hline 

     \textbf{540-SON }: \Xyce{} incorrectly added extensions to file names
     specified on print lines & In \Xyce{} 6.2, the behavior of the
     \texttt{FILE=} option to \texttt{.PRINT} lines was inadvertently changed.
     In versions 6.2 and 6.3, the code incorrectly added format-specific
     extensions to the file name specified, e.g. \texttt{FILE=myfile} would
     actually result in a file called \texttt{myfile.prn} for the standard
     output format.  This was not intended, and was reported by users too soon
     before the 6.3 release to be corrected in that release.  \Xyce{} 6.4 does
     not add this undesired extension, and the file name specified is once
     again used exactly as specified.

     {\bf NOTE:} The file name specified by \texttt{FILE=} is used only for the
     primary output file associated with the print line.  For print lines that
     result in multiple output files the secondary outputs are still given
     default file names based on the netlist name and output format.  As an
     example, if the input file is called ``netlist.cir'', \texttt{.print HB
     FILE=a} will output the primary data (frequency-domain solution) to the
     file ``a'', but the secondary data (time-domain solution) to
     ``netlist.cir.HB.TD.prn'' rather than using the file name specified as a
     base name.  {\bf This choice of file name for secondary outputs may change
     in future releases.} Additional \texttt{.PRINT} options exist to specify
     file names for these secondary outputs, such as \texttt{.print HB\_TD
     FILE=b} to output the time domain data to file ``b''.  
     \\ \hline

     \textbf{554-SON}: PDE\_1D\_BJT/gilbert\_cell\_pde\_four test failing on
     parallel RHEL5 since mid-October & The ``gilbert\_cell\_pde\_four'' test
     was failing on Windows and RHEL5 builds at various times prior to the
     release of \Xyce{} 6.3. The problem was with the test circuit not having
     tight enough tolerances, so it had variable results. The tolerances were
     tightened and the failures stopped.
     \\ \hline

     \textbf{573-SON}: Continuation does not work in HB solves & Prior to \Xyce{}
     6.3, specification of LOCA continuation options in \texttt{.OPTIONS
     NONLIN-HB} would be accepted but continuation would not actually be
     applied.  This was fixed in \Xyce{} 6.3, but had not been documented in the
     6.3 Release Notes.  
     \\ \hline


     \textbf{592-SON}: Make \texttt{.IC} and \texttt{.NODESET} be able to use expressions & 
     In \Xyce{} 6.3, if one attempted to use expressions on the \texttt{.IC} line,
     the simulation would fail, as the code was not instrumented to handle this
     properly.  The same was true for \texttt{.NODESET}.  This has been fixed
     for \Xyce{} 6.4.
     \\ \hline

     \textbf{606-SON and 654-SON}: Make TEMP and Vt work correctly in B source
     expressions & It was observed that if a B source expression used the TEMP
     variable, the value of the expression was not correctly being updated if
     TEMP was swept in a \texttt{.STEP} loop.  Furthermore, the \Xyce{}
     documentation has always stated that Vt is a reserved variable name, but
     \Xyce{} neither defined the Vt variable nor allowed it to be used in B
     source expressions.  Release 6.4 fixes both of these issues, allowing TEMP
     and Vt to be used in B source expressions, and correctly updating them in
     \texttt{.STEP} loops.
     \\ \hline


     \textbf{614-SON}: PDE (TCAD) Device nonlinear Poisson initialization needs 
     to work properly with LOCA & Prior to \Xyce{} 6.3, specification of LOCA 
     continuation in \texttt{.OPTIONS NONLIN} for a PDE(TCAD) problem would lead 
     to a logic error that often resulted in nonconvergence.  The TCAD devices
     by default solve a nonlinear Poisson problem first, followed by the full 
     drift-diffusion formulation at the DCOP.  The nonlinear Poisson calculation 
     can be manually turned off, but typically is not, as it provides a good initial 
     guess.  The logic error was that if running a continuation loop, the nonlinear 
     Poisson calculation would be re-computed at each step of the continuation, and 
     it only should have been computed for the very first step.  This has been fixed 
     for \Xyce{} 6.4.
     \\ \hline

     \textbf{624-SON}: GMIN stepping output gets put into TRAN output file under 
     some circumstances  &  This issue happened when the user specified an 
     output filename on the \texttt{.PRINT} line.  \Xyce{} would incorrectly 
     use that filename for all the outputs, even if it had only been specified for 
     one type of analysis (such as transient).  When this happened, output that 
     should have been in separate files was all concatenated into the same file.  
     This was not the intended behavior, and it has been fixed in \Xyce{} 6.4.
     For example, if the user specifies a output file name for the transient 
     \texttt{.PRINT} line, the resulting file with this name will only contain 
     transient output.  Other outputs, such as output for GMIN stepping will 
     be directed to a separate file using the default filename for that analysis
     type.
     \\ \hline

     \textbf{629-SON}: Power accessors (P() and W()) do not work in expressions
     & P() and W() now work in expressions in both top-level circuits and
     subcircuits.
     \\ \hline

     \textbf{632-SON}: Fix LOCA interface so that we can use arclength 
     continuation with voltage limiting on & Arclength continuation only worked
     if voltage limiting was turned off by setting \texttt{.options device voltlim=0}.
     This was because the arclength equation was required the sensitivity 
     of the system with respect to the continuation parameter.  This was computed 
     using a finite-difference calculation, that conflicted with voltage limiting.
     This issue has been fixed in \Xyce{} 6.4.  Many devices are now instrumented
     to provide analytic parameter sensitivities to support \texttt{.SENS}.  The
     interface to LOCA has been refactored to use these sensitivities when available.
     When not available, a new finite-difference has been implemented that temporarily
     disables voltage limiting.  
     \\ \hline


     \textbf{636-SON}: \Xyce{} core dumps on invalid \texttt{.TRAN} line & A
     \texttt{.TRAN} line with only one of the two required arguments caused
     \Xyce{} versions 6.1 through 6.3 to crash with a core dump.  This error was
     fixed in \Xyce{} 6.4.  
     \\ \hline

     \textbf{641-SON}: Core Dumps Caused by OBJFUNC Contents in .SENS & This occurred
     if an invalid objective function was specified for a sensitivity analysis.  For 
     example, sensitivity analysis objective functions currently cannot process
     output functions such as the power accessors, so a specification such as
     \texttt{.SENS objfunc={P(R1)}}  would cause a core dump in \Xyce{} 6.3.  This
     has been fixed for \Xyce{} 6.4.
     \\ \hline

     \textbf{644-SON}: Wrong number of arguments on .DC line causes segfault &
     This line (\texttt{ .DC V1 -8.0 -4.0 0.0 4.0}) would cause a core dump,
     since the last 4.0 is extraneous.  Incorrect .DC lines should now produce
     useful error messages and graceful exits.  
     \\ \hline

     \textbf{646-SON}: .csd format for PROBE output is incorrect for specific
     numbers of output variables & This line (\texttt{.PRINT TRAN FORMAT=PROBE
     V(1) V(2) V(3) V(4)}) would produce an incorrectly formatted .csd file
     that would not open in PSpice.  In general, the bug occurred if the number
     of output variables (N) on the \texttt{FORMAT=PROBE} print line satisfied
     this equation: (N+1)mod5 = 0.  It works correctly now.  
     \\ \hline

     \textbf{647-SON }: Instance parameters copied from model at wrong time,
     breaking step loops  & Some devices have default values of instance
     parameters that are computed from model parameters.  If a device is
     instantiated without specifying these instance parameters, the default is
     computed and used.  In versions of \Xyce{} prior to 6.4, this computation
     was performed once, when the device was first created.  This is incorrect
     if any of the model parameters used in the computation are being varied in
     a \texttt{.STEP} loop.  In such a case, the instance parameter would not
     properly be updated as the model parameter was varied.  This has been
     fixed in \Xyce{} 6.4 for all devices that had the problem, which included
     most MOSFET devices and all devices that had been generated from Verilog-A
     sources.  
     \\ \hline

     \textbf{648-SON}: .print HOMOTOPY doesn't work with .DC or .TRAN &
     An error trap in the \Xyce{} 6.3 parser was too agressive.  It was attempting
     to check that the specified \texttt{.print} lines matched the specified 
     analysis lines.  As the keyword \texttt{HOMOTOPY} didn't match \texttt{DC}
     or \texttt{TRAN}, the parser incorrectly threw an error.  Continuation
     methods are often used to compute DC operating points, that are then used
     as the starting point for a transient analysis.  So, it was natural that one might request
     such output.  This has been fixed for \Xyce{} 6.4.
     \\ \hline

     \textbf{652-SON (partially) }: HB output is buggy & Release 6.3 of
     \Xyce{} had a bug that would garble harmonic balance output if a
     \texttt{FILE} keyword was used on the \texttt{.print HB} line to redirect
     output to a specified file name instead of the default file name.  The
     result was that time- and frequency-domain data were improperly placed on
     the same line of the same output file, and the output file contained both
     time- and frequency-domain header lines.  This part of bug 652 was
     resolved: the \texttt{FILE} option of the \texttt{.print HB} line now
     redirects only the frequency-domain data, and the time-domain data is sent
     to the default-named file (the netlist name with a ``.HB.TD'' suffix and
     an extension.

     A further bug in HB output from release 6.3 was fixed, in that the special
     \texttt{.print HB\_TD} and \texttt{.print HB\_FD} output lines may be used
     as documented in the Reference Guide in order to specify different
     variables to be printed for time- and frequency-domain harmonic balance
     output, and to redirect that output to non-default file names with
     \texttt{FILE} keywords.  But an unresolved part of this bug means that
     these options must be used as a pair --- \Xyce{} 6.4 will not output
     anything at all if only time- or frequency-domain data is requested.  See
     also the ``Known Defects'' table for further description of this bug.  \\
     \hline

     \textbf{657-SON}: \Xyce{} core dumps during PWL source parsing if the
     time-voltage pair list is missing & A piecewise linear (PWL) source
     definition without a list of time-voltage pairs caused a core dump.  An
     example instance line was: \texttt{V4 4 0 PWL TD=1 R=1}. This is now
     fixed.  
     \\ \hline

     \textbf{658-SON }: Multiple print lines to same file should work
     intuitively & In prior releases of \Xyce{}, having multiple print lines
     that output to the same file was not supported, and if attempted would not
     do what a user would have expected --- and the specific behavior observed
     would depend on which version of \Xyce{} was used.  

     The feature has been implemented, and if multiple \texttt{.PRINT} lines
     for the same analysis type are specified that output to the same file, the
     variable lists of each are aggregated so that the net effect is the same
     as having a single print line with all the specified variables.   Multiple
     print lines for the same analysis to different files do not aggregate in
     this manner.  Multiple print lines for analyses that result in multiple
     files, such as \texttt{.print HB}, aggregate the variable lists for all
     the secondary files as well as the primary file.  It is an error to
     attempt to have multiple print lines to the same file with different
     \texttt{FORMAT} specifications.  
     \\ \hline

     \textbf{663-SON}:  Lead currents for the method=trap integrator have a
     mistake in the first transient time step  &  The lead current feature in
     \Xyce{} has undergone numerous internal changes recently.  As a result, a bug
     was introduced into the trapezoid rule integration version of lead
     currents, which resulted in incorrect lead currents in some circuits, in
     the initial time step out of the DCOP calculation.  Time derivatives
     should have always been near zero at this phase of the calculation, but
     this bug caused them to be much larger under some circumstances.  This has
     been fixed.  
     \\ \hline 

     \textbf{664-SON}: Fix the old nonlinear solver's convergence tests to be 
     consistent with the NOX convergence test & This is approximately the same
     bug as 414-SON (described above).  The only difference is that it is applied
     to the non-default, older nonlinear solver.  As with the default solver (NOX),
     the convergence tests were too complex and occasionally resulted in 
     nonlinear solver failures, or cases where failed solves were declared
     as passed.  This has been fixed for \Xyce{} 6.4.
     \\ \hline 

     \textbf{668-SON}: \Xyce{} Core dumps on some invald .OPTIONS lines &
     Examples of the offending, incorrect .OPTIONS lines included:
     \texttt{.OPTIONS} and \texttt{.OPTIONS LINSOL type=}. This is fixed now.
     \\ \hline

     \textbf{669-SON}: Sensitivity analysis does not recognize default device parameters &
     If performing sensitivity analysis, it was not possible to use default device parameters.
     For example, \texttt{.sens param=Rtest:R} would work, but \texttt{.sens param=Rtest} would not.
     This is fixed now.
     \\ \hline

     \textbf{673-SON}: Core dumps caused by incomplete \texttt{.MEASURE} lines & \Xyce{}
     would core dump on some incomplete \texttt{.MEASURE} lines.  Examples included:
     \texttt{.MEASURE TRAN} and \texttt{.MEASURE}.  Another cause was when the
     \texttt{<variable>=<val>} portion of the line was missing, such as
     \texttt{.MEASURE TRAN DERIV1 DERIV}.  This is fixed now.  
     \\ \hline


\textbf{674-SON}: HB startup output is broken and HB tecplot output
has a bug   & The HB startup output contains wrong information. It
was the same information as HB IC output. The HB IC output for tecplot
format is not correct if the transient analysis needs to be re-run
with tighter tolerance during the initial guess calculation. This has
been fixed.
\\ \hline

\textbf{675-SON}: HB has a bug when the tolerance is tightened during
the initial guess calculation &  
When the transient analysis is re-run with a tighter tolerance during
the initial guess calculation, the second transient run could fail at
DCOP or after a few steps. However, the equivalent stand-alone
transient analysis runs to completion. This has been fixed.
\\ \hline

     \textbf{679-SON}: TRIG V(A)=V(B) or TARG V(C)=V(D) syntax did not work in
     \texttt{.MEASURE} & The TRIG and TARG clauses did not work correctly if the trig or
     targ level was a variable level rather than a fixed level (e.g.,
     \texttt{TRIG V(A)=V(B) TARG V(C)=0.5}). In this example, the TRIG clause
     uses a variable level (\texttt{V(B)}), while the TARG clause uses a fixed
     level of 0.5V.  This is fixed now.  
     \\ \hline

     \textbf{681-SON }: Stepping over BSIM3 and BSIM4 model parameters gives
     wrong results  & The mechanism used in the BSIM3 and BSIM4 to share
     size-dependent quantities derived from model parameters between several
     instances of the same size was not set up to work with \texttt{.STEP}
     loops. If the \texttt{.STEP} loop was over a BSIM3 or BSIM4 model
     parameter, the derived quantities would not be updated after the initial
     computation.  This is similar, but not identical, to bug 647-SON, above .
     The BSIM3 and BSIM4 devices now clear out their saved lists of derived
     parameters if any model parameters change.
     \\ \hline

     \textbf{682-SON}: ``Spice strategy'' only attempted on first DC bias point
     & Since \Xyce{} 6.2, \Xyce{} will attempt to solve a DC bias point using a
     three-step strategy:  Newton iteration, GMIN-stepping, and
     source-stepping.  This is the same strategy that SPICE 3F5 uses.  In \Xyce{} 6.2 and
     \Xyce{} 6.3 there was a bug that caused this strategy only to be applied at
     the first step of a \texttt{.DC} sweep.  If subsequent points in the sweep
     failed Newton iteration, no further attempts were made to obtain a
     solution for those points.  This error has been corrected in release 6.4;
     the strategy will be followed any time a DC bias point is computed.  \\
     \hline

     \textbf{690-SON}: Fix various WHEN syntaxes for the DERIV measure &  The
     DERIV measure did not work correctly for some syntaxes in the WHEN clause.
     This included having a variable as the target level (e.g., \texttt{WHEN
     V(A)=V(B)}).  It also included the syntax \texttt{DERIV V(A) WHEN
     V(B)=0.5} if A and B were different.  This is fixed now.  
     \\ \hline 

     \textbf{694-SON}: Core dumps from invalid \texttt{.PRINT} lines & Examples of the
     offending, incorrect \texttt{.PRINT} lines included: \texttt{.PRINT V(1} and
     \texttt{.PRINT V()}. This is fixed now.  
     \\ \hline

     \textbf{699-SON}: Fix core dumps in \texttt{.FOUR} and FOURIER measure
     statements & \Xyce{} would core dump on several invalid or poorly
     chosen \texttt{.FOUR} lines and \texttt{FOURIER} measure statements.
     The first was when a negative fundamental frequency was requested, for
     either \texttt{.FOUR} or \texttt{.MEASURE}.  The second was when
     the measurement window, for a \texttt{.MEASURE} statement, only contained
     one time point.  The last was when a netlist contained multiple 
     \texttt{.FOUR} lines.  These issues have been fixed.  
     \\ \hline

     \textbf{700-SON}: Random number use in level 3 synapse limited to one
     instance & Random number use within this synapse device has been fixed so
     that individual instances share the same underlying random number
     generator, and will get a unique random number each time the generator is
     called.  Thus, they will no longer follow the same sequence of random
     numbers.
     \\ \hline

     \textbf{705-SON}: Bad argument to \texttt{.SAVE} causes \Xyce{} to go into an infinite loop &
     This invalid line (\texttt{.SAVE filename=<name>}) would cause \Xyce{} to 
     enter an infinite loop during netlist parsing.  This is fixed now.  The overall
     error handling and error messages for invalid and ill-formatted \texttt{.SAVE} lines 
     were improved.
     \\ \hline

     \textbf{710-SON}: dQdx derivatives in the BSIM3 and BSIM4 devices need to be 
     loaded into the matrix unconditionally &
     In some devices, the capacitive terms are excluded from the DCOP Jacobian, and
     also from the Jacobian of the first Newton iteration of the first time step.
     This was done because the capacitive currents at the DCOP are by definition zero.
     However, there are use cases (such as using a MOSFET device as a capacitor), where
     doing it this way caused singular matrices.  This has been fixed for \Xyce{} 6.4.
     \\ \hline

     \textbf{1794-SRN}: Fix core dump in outputting when the number of devices is less
     than number of processors & This issue only applies to a parallel build of \Xyce{}.
     If the devices are distributed such that one processor does not have any
     devices, then the code for outputting data could attempt to access an empty 
     container resulting in a segmentation fault.  This issue has been fixed.
     \\ \hline

     \textbf{1958-SRN}: LOCA fails to reset properly after a failed step, for
     PDE devices & This issue was specific to PDE devices.  If attempting to
     obtain a steady-state solution using a homotopy method, PDE devices were
     not having their previous state restored after failed steps.  As a result,
     all subsequent attempts at the step would fail, and the steady-state
     solution would not be obtained.  This was due to a logic error in these
     device models and has been fixed.  The issue applied to all PDE devices.
     \\ \hline 

\end{longtable}
}
