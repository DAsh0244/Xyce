% Sandia National Laboratories is a multimission laboratory managed and
% operated by National Technology & Engineering Solutions of Sandia, LLC, a
% wholly owned subsidiary of Honeywell International Inc., for the U.S.
% Department of Energy’s National Nuclear Security Administration under
% contract DE-NA0003525.

% Copyright 2002-2019 National Technology & Engineering Solutions of Sandia,
% LLC (NTESS).


%%
%% Changes to Xyce input since the last release.
%%
{
\small

\begin{longtable}[h] {>{\raggedright\small}m{2in}|>{\raggedright\let\\\tabularnewline\small}m{3.5in}}
  \caption{Changes to netlist specification since the last release.\label{newUsage}} \\ \hline
  \rowcolor{XyceDarkBlue}
  \color{white}\bf Change &
  \color{white}\bf Detail \\ \hline \endfirsthead
  \caption[]{Changes to netlist specification since the last release.\label{newUsage}} \\ \hline
  \rowcolor{XyceDarkBlue}
  \color{white}\bf Change &
  \color{white}\bf Detail \\ \hline \endhead

.MEASURE output is in scientific notation. &  The .MEASURE output, in both the
.mt0 file and standard output, is in scientific notation now.   \\ \hline

The SIN() transient specification for the V and I devices supports a
\texttt{PHASE} parameter. & The \texttt{PHASE} parameter was supported 
in past \Xyce{} releases but was not documented in the Reference Guide. 
\\ \hline 

\Xyce{} now supports the DIGINITSTATE option.  &  PSpice supports a
DIGINITSTATE option that controls the behavior of latches and flip-flops during
the DC Operating Point calculation. This capability is now in \Xyce{}, but for
backwards compatibility, its default behavior differs from PSpice.
\\ \hline

The behavior of a \Xyce{} piecewise linear (PWL) source when the time-voltage
list omits an entry for time=0. & The \Xyce{} behavior in this case now
matches PSpice and HSpice.  See the V and I device sections of the Reference
Guide for more details. \\ \hline 

The behavior of \texttt{.PRINT HB FILE=} was broken in previous releases. & It
was discovered that \Xyce{} 6.3 would output both time-domain and
frequency-domain data into the same file when the \texttt{FILE} option was used
on harmonic balance print lines.  This was never intended behavior.  \Xyce{}
6.4 outputs time and frequency domain data to separate files, whether or not
the \texttt{FILE} option is given.  In this release of \Xyce{}, the named file
is where the frequency-domain data is sent.  The time-domain data is sent to a
file whose name is the netlist name with the suffix ``.HB.FD'' followed by an
extension determined by the output format.  To send the time-domain data to a
file with a different name, use the \texttt{.print HB\_TD FILE=} line {\em in
addition to\/} either a \texttt{.print HB} or \texttt{.print HB\_FD} line.
\\ \hline

In previous releases, \Xyce{} treated all lines with leading whitespace as
comments. & The \Xyce{} behavior now is to treat lines that begin with
whitespace as comments UNLESS the first non-whitespace character is a ``+'', in
which case it treats the line as a continuation.  This is consistent with how
Spice3F5 handles lines with leading whitespace.
\\ \hline

Character limit of top-level netlist title line. & In previous releases,
\Xyce{} limited the first line of any top-level netlist file (the title line)
to 256 characters.  This limitation has been removed.
\\ \hline

\end{longtable}
}
