% Sandia National Laboratories is a multimission laboratory managed and
% operated by National Technology & Engineering Solutions of Sandia, LLC, a
% wholly owned subsidiary of Honeywell International Inc., for the U.S.
% Department of Energy’s National Nuclear Security Administration under
% contract DE-NA0003525.

% Copyright 2002-2022 National Technology & Engineering Solutions of Sandia,
% LLC (NTESS).

%%-------------------------------------------------------------------------
%% Purpose        : Main LaTeX Xyce Users' Guide
%% Special Notes  : Graphic files (pdf format) work with pdflatex.  To use
%%                  LaTeX, we need to use postcript versions.  Not sure why.
%% Creator        : Robert Hoektra, Computational Sciences, SNL
%% Creation Date  : {12/22/2003}
%%
%%-------------------------------------------------------------------------

\chapter{Guidance for Running Xyce in Parallel}
\label{Parallel}
\index{\Xyce{}!running in parallel}

\chapteroverview{Chapter Overview}
{
This chapter provides guidance for running a parallel version of \Xyce{}, and includes the following sections:
\begin{XyceItemize}
\item Section~\ref{Parallel_Introduction}, {\em Introduction}
\item Section~\ref{paffinity}, {\em Processor Affinity on Linux systems}
\item Section~\ref{ProblemSize_Guidance}, {\em Problem Size}
\item Section~\ref{LinearSolver_Options}, {\em Linear Solver Options}
\item Section~\ref{Transformation_Options}, {\em Transformation Options}
\item Section~\ref{Device_Distribution_Options}, {\em Device Distribution Options}
\end{XyceItemize}
}

\section{Introduction}
\label{Parallel_Introduction}

\Xyce{} is designed from the ground up to be distributed-memory parallel, supported by the message-passing interface (MPI)
standard. Although many of the issues pertinent to running in parallel are still being researched, \Xyce{} is mature
enough that some general principles have emerged for efficiently running
problems in a parallel environment.  In addition to the information in this chapter, reference~\cite{xyceBookChapter:2011} provides
supplemental information about \Xyce{} parallel performance. 

Parallel simulations must be run from the command line.  Section~\ref{command_line_simulation} provides 
information about the parallel execution syntax for \Xyce{}.

\section{Processor Affinity on Linux systems}
\label{paffinity}

Beginning with release 1.8 of OpenMPI, the default behavior on Linux
systems is for OpenMPI to apply ``processor affinity'' to runs invoked
with mpirun.  This is intended to improve performance of parallel runs
by preventing the processes from moving around the system, possibly
degrading memory access efficiency.  Since most users of OpenMPI are
trying to maximize the performance of their systems, this is generally
a good change.

Prior to that release, the default behavior was to allow
MPI processes to be moved from processor to processor by the Linux
system as needed.

If you are the only person running MPI jobs on a system, and you are
only running one job, this change will have no impact on you, but if
you are running more than one MPI job at a time, or more than one user
on your system is running MPI jobs and the system is not using a
resource manager such as slurm, then this change has a profound impact
that must be understood and adapted to with mpirun options.

\subsection{Default OpenMPI Behavior with Processor Affinity Support}

At the time of this writing, OpenMPI's mpirun will, by default, bind
processes to a core if the number of processors requested is 2 or
less.  If the number of processes requested is greater than 2, it
will bind processes to a socket.

What this means is that once your run starts, each process of your
parallel job will be locked onto the processor (core or socket) on
which it started.

\subsection{Why You Have to Know About This}
Unfortunately, OpenMPI by default allocates processors to processes in
a round-robin fashion starting from the lowest numbered processor on
the system, and each mpirun makes this determination without any
access to what other mpirun invocations have done.

The effect of this is that if you start five identical 2-processor
parallel runs of \Xyce{} simultaneously on a 16-core system and don't add
extra mpirun options, each of these five jobs will be locked to the
\emph{same\/} two processors (processors 0 and 1) on the system.

{\bf Rather than getting five jobs run in the time it takes to run one,
each will only get 20\% of the two processors and take at least five
times longer to run than a single job would have.}

The same thing
would be true if five different users each ran one 2-processor \Xyce{}
job.  All jobs would be running on processors 0 and 1 of this
16-processor system.

Therefore, unless you are the only person on your system and you are
only running one job, you must be aware if OpenMPI is using processor
affinity on your system, and take appropriate steps to avoid
oversubscribing cores.

\subsection{Affected Systems}

At the time of this writing, OpenMPI supports processor affinity only
on Linux systems, and so the issues of this section apply only to
Linux.  The OS X kernel does not support jobs setting their own
processor affinity.  Some other systems have processor affinity
support in their kernels, but it is not yet supported by OpenMPI on
those systems.

Systems that are running a resource manager such as slurm (which
includes all of Sandia's high performance computing systems) are not
impacted by this issue, because slurm will allocate a specific set of
CPUs to be exclusive to your job, and no other jobs of yours or other
users can run on these CPUs.

\subsection{mpirun Command Line Options to Change Default Behavior}

If squeezing maximum performance out of the hardware is not important,
then the simplest way to avoid the oversubscription issue described
above is to use the \texttt{--bind-to none} option to mpirun.  This
was the default behavior of mpirun prior to release 1.8 of OpenMPI and
the default behavior of OpenMPI on every system other than Linux.
While the \texttt{--bind-to none} option will potentially allow your 
mpi processes to move around the
system and have degraded memory performance, it will \emph{NOT\/}
accidentally stack multiple jobs onto the same small set of
processors.  Any number of mpirun jobs may be run with this option,
and they will not compete for resources unless the Linux task
scheduler makes them do so.

If you do not want to do without the benefits that processor affinity
can bring, you can manually specify the set of CPUs that OpenMPI will
use for your job by using the \texttt{-cpu-set} option to mpirun,
chosing as your CPU set some numbered processors that you know are not
being used by other MPI jobs.  For example, \texttt{mpirun -np 2
  -cpu-set 2-3} will run your 2 processor job with the jobs locked to
processors 2 and 3 instead of to processors 0 and 1.  You will have to
make sure to use a different CPU set for each job, and make sure that
you are not using the same CPU set as some other user on the system.

There are other options for modifying OpenMPI's processor binding
behavior, so consult OpenMPI documentation if you wish to understand
them better.

Finally, if you really want to solve the problem correctly, reaping
both the benefits of processor affinity and the simplicity of using
mpirun's defaults, you can install and configure a resource manager on
your system.  This is a topic that is far outside the scope of \Xyce{}
documentation.  See \url{http://slurm.schedmd.com/} for documentation
on one such resource manager.


\section{Problem Size}
\label{ProblemSize_Guidance}

Running \Xyce{} in parallel is often useful for circuits with thousands of devices or more.  However, due to the overhead of interprocessor communication,
 there is an optimal number of processors that will achieve the best performance.  This number is dependent upon many factors, including the number and 
type of devices, the topology of the circuit, and the characteristics of the computing architecture.  It is difficult to know a priori what this optimal 
number of processors is.  However, it is apparent when that optimal number is exceeded because, as the number of processors is increased, 
the total simulation time will also increase.  This is due to the increasing amount of required communication and decreasing amount of work per processor.
In other words, the benefit of distributing the problem is outweighed by the communication overhead, so increasing the processor count beyond this
optimal point is counterproductive.

\subsection{Ideal Problem Size}
In general, a circuit needs to be relatively large to take full advantage of the parallel capability of \Xyce{}. 
However, parallelism is achieved in two distinct phases of the code:  the device evaluation and the linear solve.
The device evaluation is, as the name implies, the evaluation of all the device equations in order to compute the residual vector 
and Jacobian entries for Newton's method.  \Xyce{} distributes the number of devices over the number of processors in parallel, 
so their evaluation enables speedups in the total simulation time even for thousands of devices. 

The linear solve phase is more computationally complex.  The Jacobian matrix generated by most circuits is sparse and has heterogeneous structure, in 
that there is not a regular sparsity pattern in the matrix nonzeros.  Sparse, direct linear solvers have proven to be efficient on these types of 
linear systems up into the tens to hundreds of thousands of unknowns.  They become less efficient for linear systems in the hundreds of thousands of unknowns.  
This is where iterative linear solvers can provide scalable performance because of their inherent parallelism.  Unfortunately, the effectiveness of 
iterative linear solvers is dependent upon preconditioning the linear system (see Section~\ref{Preconditioning_Options}).  
The benefit of direct over iterative linear solvers is that they rarely fail to compute a solution, so direct
linear solvers are the more robust option for enabling simulations to complete.  

In general, there are three modes in which \Xyce{} can
be executed:  ``Serial load, serial solve'', ``Parallel load, serial solve'', and ``Parallel load, parallel solve''.  
Each of these modes optimizes the amount of available parallelism for a given linear system size, as summarized in Table~\ref{tab:sim:modes}.  The ``load'' refers to the device evaluation phase combined with the assembly of the Jacobian matrix and residual vector, while
the ``solve'' refers to the linear solve phase. 
``Serial load, serial solve'' is the only mode of computation that a serial version of \Xyce{} will perform, but it can also be
obtained in a parallel version of \Xyce{} by using only one MPI processor.  Both of the ``Parallel load'' simulation modes require a parallel build of \Xyce{}, where the linear solver method can be a direct method (``serial solve'') or an iterative method (``parallel solve'') using
the options discussed in Section~\ref{LinearSolver_Options}.
Hybrid linear solvers, which combine the best attributes of both direct and iterative methods, provide a robust and scalable option. 
They are not reflected in Table~\ref{tab:sim:modes}, but more information about these types of linear solvers will be discussed in 
Section~\ref{HybridLinearSolver_Options}.  

\begin{table}[htp]
\caption[\Xyce{} Simulation Modes]{Xyce simulation modes.}
\label{tab:sim:modes}
\begin{center}
\begin{tabular}{| p{5cm} | p{3.5cm} | p{7cm} |}
\hline
Mode & Linear System Size & Reason \\
\hline
``Serial load, serial solve'' & $10^0$ - $10^3$ & MPI overhead cannot speed up device evaluation or linear solve. \\
``Parallel load, serial solve'' & $10^3$ - $10^4$ & Distributed device evaluations can speed up the simulation, but 
iterative linear solvers are not more efficient than direct methods.\\
``Parallel load, parallel solve'' & $10^4$ or more & Distributed device evaluations can speed up the simulation and
so can iterative linear solvers, if an efficient preconditioner is available. \\
\hline
\end{tabular}
\end{center}
\end{table}

\subsection{Smallest Possible Problem Size}
Circuits consist of a discrete set of components (voltage nodes, devices, etc.). For parallel simulation, it is preferable that \Xyce{} be able to put 
at least one discrete component of the problem on each processor. In practice, this means the circuit should be distributed across fewer processors than 
the number of nodes and devices it contains.

\section{Linear Solver Options}
\label{LinearSolver_Options}

The different linear solvers available in \Xyce{} are:

\begin{XyceItemize}
  \item KLU
  \item KSparse
  \item SuperLU and SuperLU DIST (optional)
  \item The AztecOO iterative solver library
  \item The Belos iterative solver library
  \item The ShyLU hybrid solver library (optional)
\end{XyceItemize}

AztecOO and Belos are the parallel iterative solvers and KLU, KSparse, and SuperLU (optional)
are the serial direct solvers that are available for both serial and parallel builds of \Xyce{}.  
If KLU, KSparse, or SuperLU is used with a parallel version of \Xyce{}, the devices are evaluated and
linear problem is assembled in parallel, but the linear system is solved in serial
on one processor.  This can be quite effective for circuits with tens of thousands of devices
or fewer (see Table~\ref{tab:sim:modes}). The ShyLU hybrid linear solver, which combines the robustness
of a direct solver with the scalability of an iterative solver, will be discussed in 
Section~\ref{HybridLinearSolver_Options}.  

The user can specify the solver through the \texttt{.OPTIONS LINSOL} control line 
in the netlist.  The default linear solver used by \Xyce{} is described in Table~\ref{tab:default:solver}.
By default, a parallel version of \Xyce{} uses AztecOO as the linear solver when the linear
system is larger than ten thousand unknowns.  For any linear system smaller than ten thousand unknowns,
\Xyce{} uses KLU as the linear solver.  A serial version of \Xyce{} uses KLU as its default linear 
solver.  To use a solver other than the default the user needs to add the option 
``\texttt{TYPE=<solver>}'' to the \texttt{.OPTIONS LINSOL}
control line in the netlist, where \texttt{<solver>}
is `\texttt{KLU},' `\texttt{KSPARSE},' `\texttt{SUPERLU},' `\texttt{SUPERLUDIST},' `\texttt{AZTECOO},' `\texttt{BELOS},' or `\texttt{SHYLU}.'

\begin{table}[htp]
\caption[\Xyce{} Default Linear Solver]{Xyce default linear solver.}
\label{tab:default:solver}
\begin{center}
\begin{tabular}{| p{3cm} | p{3cm} | p{4cm} |}
\hline
Solver & Version & Linear System Size\\
\hline
KLU & Serial & {\it all} \\
KLU & Parallel & $< 10^4$ unknowns \\
AztecOO & Parallel & $\geq 10^4$ unknowns \\
\hline
\end{tabular}
\end{center}
\end{table}

\subsection{KLU}
KLU is a serial, sparse direct solver native to the Amesos package in Trilinos~\cite{trilinos:toms} and is the default solver for serial builds of \Xyce{}. 
KLU is the default solver for small circuits in parallel builds of \Xyce{} as well, but this requires the linear system to be solved 
on one processor and the solution communicated back to all processors. As long as the linear system can fit on one processor, KLU is 
often a superior approach to using an iterative linear solver.  So, if a parallel build of \Xyce{} is run in serial on a circuit that generates a linear system
larger than ten thousand unknowns and the simulation fails to converge, then specifying KLU as the linear solver may fix that problem.  

Some of the solver parameters for KLU can be altered through the `\texttt{.OPTIONS LINSOL}' control line in the netlist.  
Table \ref{tab:klu:options} lists solver parameters and their default values for KLU.
   
\begin{table}[htp]
\caption[ KLU linear solver options.] {KLU linear solver options.}
\label{tab:klu:options}
\begin{center}
\begin{tabular}{| p{3.5cm} | p{9cm} | p{2.5cm} |}
\hline
Option & Description & Default Value \\
\hline
{\tt KLU\_repivot}         & Recompute pivot order each solve & 1 (true) \\
{\tt output\_ls}           & Write out linear systems solved by KLU to file every \# solves & 0 (no output)\\
{\tt output\_base\_ls}     & Write out linear systems before any transformations to file every \# solves & 0 (no output)\\
{\tt output\_failed\_ls}   & Write out linear systems KLU failed to solve to file & 0 (no output) \\
\hline
\end{tabular}
\end{center}
\end{table}

\subsection{KSparse}
KSparse is a serial, sparse direct solver based on Ken Kundert's sparse solver, Sparse 1.3.  Kundert's sparse solver was developed 
as part of the SPICE circuit simulation code.  KSparse is built, by default, in \Xyce{}.  Similar to KLU,
KSparse can be used in a parallel version of \Xyce{}, but the linear system is solved on one processor.

\subsection{SuperLU and SuperLU DIST}
SuperLU is a serial, sparse direct solver and SuperLU DIST is a parallel, sparse direct solver with an interface in the 
Amesos package.  SuperLU and SuperLU DIST support are {\it optionally} built in \Xyce{},
so they are not available by default in any \Xyce{} build or provided binary.  Furthermore, to enable SuperLU and 
SuperLU DIST support in \Xyce{}, it is necessary to build SuperLU and SuperLU DIST support in Amesos/Trilinos.  Similar to KLU, 
SuperLU can be used in a parallel version of \Xyce{}, but the linear system is 
solved on one processor.  SuperLU DIST can only be used in a parallel version of \Xyce{}, the Amesos interface handles the redistribution
of the matrix into the format required by SuperLU DIST.  
\Xyce{} does not allow modifications to SuperLU and SuperLU DIST solver parameters. 


\subsection{AztecOO}
AztecOO is a package in Trilinos~\cite{trilinos:toms} that offers an assortment of iterative linear solver algorithms.  
\Xyce{} uses the Generalized Minimal Residual (GMRES) method~\cite{sasc86} from this suite of iterative solvers.  
Some of the solver parameters for GMRES can be altered through the `\texttt{.OPTIONS LINSOL}' control line in the netlist. 
Table~\ref{tab:aztecoo:options} provides a list of solver parameters for AztecOO and their default values.

\begin{table}[htp]
\caption[AztecOO linear solver options.]{AztecOO linear solver options.}
\label{tab:aztecoo:options}
\begin{center}
\begin{tabular}{| p{3cm} | p{9cm} | p{2.5cm} |}
\hline
Option & Description & Default Value \\
\hline
{\tt AZ\_max\_iter}        & Maximum allowed iterations & 200 \\
{\tt AZ\_tol}              & Iterative solver (relative residual) tolerance & 1.0e-9 \\
{\tt AZ\_kspace}           & Krylov subspace size & 50 \\
{\tt output\_ls}           & Write out linear systems solved by AztecOO to file every \# solves & 0 (no output)\\
{\tt output\_base\_ls}     & Write out linear systems before any transformations to file every \# solves & 0 (no output)\\
\hline
\end{tabular}
\end{center}
\end{table}

\subsubsection{Common AztecOO Warnings}

If \Xyce{} is built with the verbosity enabled for the linear algebra package, it is not
uncommon to see warnings from AztecOO usually indicating the solver returned unconverged due to a numerical issue.

\begin{center}
\begin{minipage}{0.85\textwidth}
\color{XyceRed} {\bf NOTE:  }\color{black}  AztecOO warnings {\em do not} indicate the entire simulation has failed, \Xyce{} uses a hierarchy of solvers so if the iterative linear solver fails, the nonlinear solver or time integrator will usually make adjustments and attempt the step again; so the warnings can often be ignored. If the entire simulation eventually fails (i.e., gets a ``time-step-too-small'' error), then the AztecOO warnings might contain clues as to what went wrong.
\end{minipage}
\end{center}

The simplest reason for AztecOO to return unconverged would be when the maximum number of 
iterations is reached, resulting in the following warning:
\begin{verbatim}
***************************************************************
Warning: maximum number of iterations exceeded without convergence
***************************************************************
\end{verbatim}
Another reason AztecOO may return unconverged is when the GMRES Hessenberg
matrix is ill-conditioned, which is usually a sign that the matrix and/or
preconditioner is nearly singular, resulting in the following warning:
\begin{verbatim}
***************************************************************
Warning: the GMRES Hessenberg matrix is ill-conditioned.  This may
indicate that the application matrix is singular. In this case, GMRES
may have a least-squares solution.
***************************************************************
\end{verbatim}
It is also common to lose accuracy when either the matrix or preconditioner, or both,
are nearly singular.  GMRES relies on an estimate of the residual norm,
called the recursive residual, to determine convergence.  \Xyce{} uses the recursive
residual instead of the actual residual for computational efficiency.
However, numerical issues can cause the recursive residual to differ
from the actual residual.  When AztecOO detects but
cannot rectify this situation, it outputs the following warning:
\begin{verbatim}
***************************************************************
Warning: recursive residual indicates convergence
though the true residual is too large.

Sometimes this occurs when storage is overwritten (e.g. the
solution vector was not dimensioned large enough to hold
external variables). Other times, this is due to roundoff. In
this case, the solution has either converged to the accuracy
of the machine or intermediate roundoff errors occurred
preventing full convergence. In the latter case, try solving
again using the new solution as an initial guess.
***************************************************************
\end{verbatim}

\subsection{Belos}
\label{Belos_Options}
Belos is a package in Trilinos~\cite{trilinos:toms} that offers an assortment of iterative linear
solver algorithms.  Many of the algorithms available in Belos can also be found in AztecOO.  However, Belos
offers a few computational advantages because its solvers are implemented using templated C++.  
In particular, AztecOO can solve linear systems only in double-precision arithmetic, while Belos
can solve linear systems that are complex-valued or in extended-precision arithmetic.  At this time,
\Xyce{} is using a subset of Belos capabilities, the default method is GMRES, and the 
interface to Belos will recognize most of the AztecOO linear solver options, as shown in 
Table~\ref{tab:belos:options}.

\begin{table}[htp]
\caption[Belos linear solver options.]{Belos linear solver options.}
\label{tab:belos:options}
\begin{center}
\begin{tabular}{| p{3cm} | p{9cm} | p{2.5cm} |}
\hline
Option & Description & Default Value \\
\hline
{\tt AZ\_max\_iter}        & Maximum allowed iterations & 200 \\
{\tt AZ\_tol}              & Iterative solver (relative residual) tolerance & 1.0e-9 \\
{\tt AZ\_kspace}           & Krylov subspace size & 50 \\
{\tt output\_ls}           & Write out linear systems solved by Belos to file every \# solves & 0 (no output)\\
{\tt output\_base\_ls}     & Write out linear systems before any transformations to file every \# solves & 0 (no output)\\
\hline
\end{tabular}
\end{center}
\end{table}

\subsection{Preconditioning Options}
\label{Preconditioning_Options}

Iterative linear solvers often require the assistance of a preconditioner
to efficiently compute a solution of the linear system
\begin{equation}
\label{axb}
Ax=b
\end{equation}
\noindent to the requested accuracy.  
A preconditioner, $M$, is an approximation to the original matrix $A$ that is inexpensive
to solve.  Then (\ref{axb}) can be rewritten to include this (right) preconditioner as
\begin{equation}
\label{axb:prec}
AM^{-1}y=b,
\end{equation}
\noindent where $x=M^{-1}y$ is the solution to the original linear system.
If $M=A$, then the solution to the linear system is found in one iteration.
In practice, $M$ is a good approximation to $A$, then it will take few iterations
to compute the solution of the linear system to the requested accuracy.
By default, \Xyce{} uses a non-overlapped additive Schwarz preconditioner with 
an incomplete LU factorization on each subdomain~\cite{Saad:2003:IMSLS}.  
The parameters of the incomplete LU factorization are found in Table~\ref{tab:prec_options}.  
This is a simple preconditioner that always works, but is not always the most effective, 
so other preconditioning options will be presented in this section.    

\Xyce{} provides access to preconditioning packages in Trilinos~\cite{trilinos:toms}, 
such as Ifpack, through an expanded preconditioning interface.  
If modifications to the preconditioner are necessary, the user may specify the preconditioner 
through the `\texttt{.OPTIONS LINSOL}' control line in the netlist. 
Table \ref{tab:prec_options} provides a list of preconditioner parameters and
their default values.  

\begin{table}[htp]
\caption[Preconditioner options.]{Preconditioner options.}
\label{tab:prec_options}
\begin{center}
\begin{tabular}{| p{3.5cm} | p{8cm} | p{2.5cm} |}
\hline
Option & Description & Default Value \\
\hline
{\tt prec\_type}           & Preconditioner & Ifpack \\
{\tt AZ\_ilut\_fill}       & ILU fill level & 2.0 \\
{\tt AZ\_drop}             & ILU drop tolerance & 1.0e-3 \\
{\tt AZ\_overlap}          & ILU subdomain overlap & 0 \\
{\tt AZ\_athresh}          & ILU absolute threshold & 0.0001 \\
{\tt AZ\_rthresh}          & ILU relative threshold & 1.0001 \\
{\tt use\_aztec\_precond}  & Use native ILU from AztecOO package & 0 (false) \\
{\tt use\_ifpack\_factory} & Use Ifpack factory to create preconditioner & 0 (false) \\
{\tt ifpack\_type}         & Control which preconditioner Ifpack factory creates (ILU, ILUT, Amesos) & Amesos \\
\hline
\end{tabular}
\end{center}
\end{table}

In practice, the choice of an effective preconditioner is highly problem dependent.  By default,
\Xyce{} provides a preconditioner that works for most circuits, but is not the best preconditioner
for all circuits.  One simple modification to the default preconditioner that often makes it more
effective is the use of a sparse direct solver on each subdomain, instead of an inexact factorization: \\[0.5em] 
\noindent \verb|.OPTIONS LINSOL USE_IFPACK_FACTORY=1| \\[0.5em]
This preconditioner will fail if there is a singular subdomain matrix because the KLU solver on that subdomain will fail.
If numerical difficulties are not encountered during the simulation, this preconditioner is superior to inexact factorizations.
A more advanced preconditioner that has been effective for certain types of circuits uses the block triangular form (BTF)
permutation of the original matrix before generating the additive Schwarz preconditioner.  This preconditioner, which
is published in ~\cite{ICCAD09_precond}, will be presented in Section~\ref{BTF_Precond}.


\subsection{ShyLU}
\label{HybridLinearSolver_Options}
ShyLU is a package in Trilinos~\cite{trilinos:toms} that provides a hybrid linear solver 
designed to be a black-box algebraic solver~\cite{ShyLU-IPDPS}. 
ShyLU support is {\it optionally} built in \Xyce{},
so it is not available by default in any \Xyce{} build or provided binary.
Furthermore, to enable ShyLU support in \Xyce{}, it is necessary to build 
the ShyLU package in Trilinos.

ShyLU is hybrid in both the
parallel programming sense - using MPI and threads - and in the mathematical
sense - using features from direct and iterative methods. \Xyce{} uses ShyLU as a global
Schur complement solver~\cite{Saad:2003:IMSLS}.  This solver can be expensive, but also
has proven to be a robust and scalable approach for some circuit matrices~\cite{bomhof00}. 

ShyLU is under active development and testing in \Xyce{}, so a minimum number of options are 
provided to the user for controlling this solver.  The solution approach is static, 
the diagonal blocks of the partitioned matrix are solved using KLU, while the Schur complement 
is solved using an iterative method (AztecOO's GMRES specifically). 
The matrix partitioning is generated using a wide separator, which is a conventional 
vertex separator where all the vertices that are adjacent to the separator in one of the subgraphs
are added in.  The only options that can be modified are shown
in Table~\ref{tab:shylu:options}.  This includes the maximum number of iterations and solver tolerance
used by GMRES and the dropping threshold that ShyLU uses to generate a preconditioner for GMRES.

\begin{table}[htp]
\caption[ShyLU linear solver options.]{ShyLU linear solver options.}
\label{tab:shylu:options}
\begin{center}
\begin{tabular}{| p{3cm} | p{9cm} | p{2.5cm} |}
\hline
Option & Description & Default Value \\
\hline
{\tt AZ\_max\_iter}        & Maximum allowed iterations & 30 \\
{\tt AZ\_tol}              & Iterative solver (relative residual) tolerance & 1.0e-12 \\
{\tt ShyLU\_rthresh}       & Relative dropping threshold for Schur complement preconditioner & 1.0e-3 \\ 
{\tt output\_ls}           & Write out linear systems solved by ShyLU to file every \# solves & 0 (no output)\\
{\tt output\_base\_ls}     & Write out linear systems before any transformations to file every \# solves & 0 (no output)\\
\hline
\end{tabular}
\end{center}
\end{table}



\section{Transformation Options}
\label{Transformation_Options}

Transformations are often used to permute the original linear system to one that
is easier or more efficient for direct or iterative linear solvers.  \Xyce{} has many different permutations 
that can be applied to remove dense rows and columns from a matrix, reduce fill-in, find a block triangular form, 
or partition the linear system for improved parallel performance. 

\subsection{Removing Dense Rows and Columns}
The transformation that reduces the linear system through removal of all rows and 
columns with single non-zero entries in the matrix is called singleton filtering.  The values
associated with these removed entries can be resolved in a pre- or post-processing
phase with the linear solve.
A by-product of this transformation is a more tractable and sparse 
linear system for the load balancing and linear solver
algorithms.  This functionality can be turned on by adding 
`\texttt{TR\_SINGLETON\_FILTER=1}' to the `\texttt{.OPTIONS LINSOL}'
control line in the netlist.  This option is enabled by default whenever iterative
solvers are used in \Xyce{}.

\subsection{Reordering the Linear System}
Approximate Minimum Degree (AMD) ordering is a symmetric permutation that 
reduces the fill-in for direct factorizations.  If given a nonsymmetric
matrix $A$, the transformation computes the AMD ordering of $A + A^T$.  
This functionality may be turned on by adding `\texttt{TR\_AMD=1}' 
to the `\texttt{.OPTIONS LINSOL}' control line in the netlist. 
For parallel builds of \Xyce{}, AMD ordering is enabled by default whenever iterative solvers
are used.  In parallel, the AMD ordering is performed only on the local graph for
each processor, not the global graph.  This is to reduce the fill-in for the incomplete
LU factorization used by the additive Schwarz preconditioner, see Section~\ref{Preconditioning_Options}.

\subsection{Partitioning the Linear System}
\label{Partitioning_Linear_System}

Partitioning subdivides the linear system and 
then distributes it to the available processors.  A good partition can have a dramatic 
effect on the parallel performance of a circuit simulation tool.  
There are two key components to a good partition:

\begin{XyceItemize}
  \item Effective load balance\index{parallel!load balance}
  \item Minimizing communication\index{parallel!communication} overhead.  
\end{XyceItemize}

An effective load balance ensures the computational load of the calculation is equally distributed among 
available processors. Minimizing communication overhead seeks to distribute the problem in a way to reduce 
impacts of underlying message passing during the simulation run. For runs with a small number of devices 
per processor the communication overhead becomes the critical issue, while for runs with larger numbers 
of devices per processor the load balancing becomes more important.  

\Xyce{} provides hypergraph partitioning via the \textbf{Zoltan}\index{ZOLTAN} library of parallel 
partitioning heuristics integrated into \Xyce{}.  The Isorropia package in Trilinos provides access to \textbf{Zoltan}\index{ZOLTAN} and 
can be controlled through the `\texttt{.OPTIONS LINSOL}'\index{\texttt{.OPTIONS}!\texttt{LINSOL}} 
control line in the netlist. Table \ref{tab:partitioning:options} provides
the partitioning options and their default parameters. For parallel builds of 
\Xyce{}, when iterative solvers are used, \textbf{Isorropia} is enabled by default to use hypergraph partitioning. 
The linear system is statically load balanced at the beginning of the simulation based 
on the graph of the Jacobian matrix.  

\begin{table}[htp]
\caption[Partitioning options.]{Partitioning options.}
\label{tab:partitioning:options}
\begin{center}
\begin{tabular}{| p{3.5cm} | p{6cm} | p{4.5cm} |}
\hline
Option & Description & Default Value \\
\hline
{\tt TR\_PARTITION}        & Partitioning package & 0 (none), serial, \\
                           &                      & 1 (Isorropia), parallel \\
{\tt TR\_PARTITION\_TYPE}  & Isorropia partitioner type & \verb|HYPERGRAPH| \\
\hline
\end{tabular}
\end{center}
\end{table}

\Xyce{} includes an expanded partitioning interface to allow the user to access multiple partitioners through Isorropia. 
Users may change the partitioner provided by adding `\texttt{TR\_PARTITION\_TYPE}' to the `\texttt{.OPTIONS LINSOL}' control line in
the netlist. There are two options for partitioning:  hypergraph
(`\texttt{TR\_PARTITION\_TYPE=HYPERGRAPH}') and, optionally, graph (`\texttt{TR\_PARTITION\_TYPE=GRAPH}') partitioning through ParMETIS.
Occasionally it is desirable to turn off the partitioning option, even for parallel simulations.   To do so, users can add the 
`\texttt{TR\_PARTITION=0}' to the `\texttt{.OPTIONS LINSOL}' control line.

These techniques can be very effective for improving the
efficiency of the iterative linear solvers\index{solvers!iterative linear}.
See the \textbf{Zoltan User Guide}~\cite{zoltan:user} for more details.

\subsection{Permuting the Linear System to Block Triangular Form}
\label{BTF_Precond}
The block triangular form (BTF) permutation is often useful for direct and iterative
solvers, enabling a more efficient computation of the linear system solution.
In particular, the BTF permutation has shown promise when it is combined with
an additive Schwarz preconditioner (see Section~\ref{Preconditioning_Options}) in the simulation of
circuits with unidirectional flow.  

The global BTF transformation computes the permutation of the linear system to
block triangular form, and uses the block structure to partition the linear system. The partitioning can be
a simple linear distribution of block rows, `\texttt{TR\_GLOBAL\_BTF=1}', or a
hypergraph partitioning of block rows, `\texttt{TR\_GLOBAL\_BTF=2}.'
As the global BTF transformation includes elements of other tranformations,
it is imperative to turn off other linear solver options.  
To use the global BTF, the linear solver control line in the netlist should 
contain: \\[0.5em] 
\noindent \verb|.OPTIONS LINSOL TR_GLOBAL_BTF=<1,2> TR_SINGLETON_FILTER=1|\\
\noindent \verb|+ TR_AMD=0 TR_PARTITION=0|

This transformation is only useful in parallel when using a preconditioned iterative solver.
It is often more effective when combined with the exact factorization of each subdomain,
given by the `\texttt{USE\_IFPACK\_FACTORY=1}' option.  In practice, the structure
that this transformation takes advantage of is found in CMOS memory circuits~\cite{ICCAD09_precond}.


\section{Device Distribution Options}
\label{Device_Distribution_Options} 
\Xyce{} uses different parallel distributions for the objects that evaluate the
device models and the linear system.  As discussed in Section~\ref{Partitioning_Linear_System}, a good
parallel partition of the linear system can improve load balance and minimize communication. 
The same is true for the parallel distribution of devices, since they have widely varying
computational cost and the circuit connectivity graph is often irregular.  Furthermore, a more efficient
device distribution strategy can overlap netlist processing with the simulation setup, enabling scalable
parsing for larger netlists.

\begin{table}[htp]
\caption[Device Distribution options.]{Distribution options.}
\label{tab:distribution:options}
\begin{center}
\begin{tabular}{| p{3.5cm} | p{6cm} | p{4.5cm} |}
\hline
Option & Description & Default Value \\
\hline
{\tt STRATEGY}        & Distribution strategy & 0 \\
\hline
\end{tabular}
\end{center}
\end{table}

\Xyce{} provides three different device distribution strategies that can be controlled through
the \texttt{.OPTIONS DIST}\index{\texttt{.OPTIONS}!\texttt{DIST}} control line in the netlist,
see Table \ref{tab:distribution:options}.  The default strategy (\texttt{STRATEGY=0}) that \Xyce{} 
uses for parallel device distribution is a first-come-first-served
approach, in that the total number of devices in a circuit is divided evenly over the number
of parallel processors in the order of parsing.  This is the simplest distribution strategy, but
it does not take into account the connectivity of a circuit or balance device model computation.
Therefore, this strategy can exhibit parallel imbalance for post-layout circuits that have a 
substantial portion of parasitic devices.

\Xyce{} provides two other distribution strategies that enable scalable parsing for
flattened circuits or distribute devices in a more balanced manner.  The "flat round robin" strategy
(\texttt{STRATEGY=1}) will generate the same device distribution as the default strategy, but every
parallel processor will participate in reading their portion of the netlist.  In that way, this
strategy provides a more scalable setup than the default strategy, but can only be applied to flattened
(non-hierarchical) netlists.  The "device balanced" strategy (\texttt{STRATEGY=2}) will evenly divide
each of the device types over the number of parallel processors, so each processor will have a balanced
number of each model type.  This allieviates the parallel imbalance in the device model computation that
can be experienced with post-layout circuits.  However, it does not take into account the circuit
connectivity, so the communication will not be minimized by this strategy.




%%% Local Variables:
%%% mode: latex
%%% End:

%% END of Xyce_UG_ch13.tex ************
