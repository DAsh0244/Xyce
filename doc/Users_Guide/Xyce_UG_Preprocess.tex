% Sandia National Laboratories is a multimission laboratory managed and
% operated by National Technology & Engineering Solutions of Sandia, LLC, a
% wholly owned subsidiary of Honeywell International Inc., for the U.S.
% Department of Energy’s National Nuclear Security Administration under
% contract DE-NA0003525.

% Copyright 2002-2019 National Technology & Engineering Solutions of Sandia,
% LLC (NTESS).

%%-------------------------------------------------------------------------
%% Purpose        : Main LaTeX Xyce Users' Guide
%% Special Notes  : Graphic files (pdf format) work with pdflatex.  To use
%%                  LaTeX, we need to use postcript versions.  Not sure why.
%% Creator        : Keith Santarelli
%% Creation Date  : 12/17/2007
%%
%%-------------------------------------------------------------------------

% -------------------------------------------------------------------------
% .PREPROCESS statements Chapter --------------------------------
% -------------------------------------------------------------------------

\chapter{Working with .PREPROCESS Commands}
\label{Preprocess_Chap}
\index{\texttt{.PREPROCESS}}
%%%%%%%%%%%%%%HAVE TO ADD LABEL TO TABLE OF CONTENTS, OR INDEX, OR SOMETHING??

%\index{}
%%%%%%%%%%%%%%HAVE TO ADD IN INDEXING STUFF AT SOME POINT, TOO!!!!!

\chapteroverview{Chapter Overview}
{
This chapter includes the following sections:
\begin{XyceItemize}
\item Section~\ref{PP_Intro}, {\em Introduction}
\item Section~\ref{PP_gndsyn}, {\em Ground Synonym Replacement}
\item Section~\ref{PP_removeunused}, {\em Removal of Unused Components}
\item Section~\ref{PP_dangling}, {\em Adding Resistors to Dangling Nodes}

\end{XyceItemize}
}

\section{Introduction}
\label{PP_Intro}

In an effort to make \Xyce{} more compatible with other commercial circuit 
simulators (e.g., HSPICE), some optional tools have been added to increase the netlist processing capabilities of \Xyce{}.  These options, which occur toward
the beginning of a simulation, have been incorporated not only to make 
\Xyce{} more compatible with different (i.e. non-\Xyce{}) netlist syntaxes, but also 
to help detect and remove certain singular netlist configurations that can often 
cause a \Xyce{} simulation to fail.  Because all of the commands described 
in this section occur as a precursory step to setting up a \Xyce{} simulation, they are 
all invoked in a netlist file via the keyword \texttt{.PREPROCESS}.  This 
chapter describes each of the different functionalities that can be invoked 
via a \texttt{.PREPROCESS} statement in detail and provides examples to illustrate 
their use.



%%%%%%%%%%%%%%%%%%%%%%%%%%%%%%%%%%%
%%GROUND SYNONYM CHECKING
%%%%%%%%%%%%%%%%%%%%%%%%%%%%%%%%%%%

\section{Ground Synonym Replacement}
\label{PP_gndsyn}
\index{\texttt{.PREPROCESS}!\texttt{REPLACEGROUND}}
In certain versions of SPICE, keywords such as \texttt{GROUND}, \texttt{GND}, and 
\texttt{GND!} can be used as node names in a netlist file to represent the ground 
node of a circuit.  \Xyce{}, however, only recognizes node \texttt{0} as an official
name for ground.  Hence, if any of the prior node names is encountered in a 
netlist file, \Xyce{} will treat these as different nodes from ground.  To 
illustrate this point, consider the netlist of figure\ \ref{fig:nlgndrepl2}.  
When the node \texttt{Gnd} is encountered in the definition of resistor \texttt{R3}, 
\Xyce{} instantiates this as a new node.  The schematic diagram corresponding to
this netlist (figure\ \ref{fig:gndreplace2}) shows that the resistor 
\texttt{R3} is ``floating'' between node \texttt{2} and a node with only a 
single device connection, node \texttt{Gnd}.  When \Xyce{} executes the netlist of figure\ 
\ref{fig:nlgndrepl2}, the voltage \texttt{V(2)} will evaluate to 0.5V.  

\begin{figure}[htbp]
\begin{centering}
\shadowbox{
\begin{minipage}{0.8\textwidth}
\begin{vquote}
Circuit with "floating" resistor R3

V1 1 0 1
R1 1 2 1
R2 2 0 1
R3 2 Gnd 1

.DC V1 1 1 0.1
.PRINT DC V(2)
.END
\end{vquote}
\end{minipage}
}
\caption[Example netlist -- \texttt{Gnd} treated {\em different}
from node \texttt{0}.] {Example netlist where \texttt{Gnd} is treated as being {\em different}
from node \texttt{0}.}
\label{fig:nlgndrepl2}
\end{centering}
\end{figure}

\begin{figure}
\centering{\input{gndreplace2.latex}}
\caption[Circuit diagram corresponding to figure\ 
\ref{fig:nlgndrepl2}.] {Circuit diagram corresponding to the netlist of figure\ 
\ref{fig:nlgndrepl2} where node \texttt{Gnd} is treated as being {\em different} 
from node \texttt{0}.} 
\label{fig:gndreplace2}
\end{figure}

If one would rather treat \texttt{Gnd} the same as node \texttt{0} in the above example, use the  figure\ 
\ref{fig:nlgndrepl1} netlist instead.  When the statement \texttt{.PREPROCESS REPLACEGROUND
TRUE} is present in a netlist, \Xyce{} will treat any nodes named \texttt{GND}, \texttt{GND!}, \texttt{GROUND}, or any capital/lowercase variant of these keywords (e.g., \texttt{gROunD}) as synonyms
for node \texttt{0}.  Hence, according to \Xyce{}, the figure\ \ref{fig:nlgndrepl1} netlist corresponds to figure\ 
\ref{fig:gndreplace1} schematic diagram, and the voltage \texttt{V(2)} will evaluate to 0.33V.

\begin{figure}[htbp]
\begin{centering}
\shadowbox{
\begin{minipage}{0.8\textwidth}
\begin{vquote}

Circuit where resistor R3 does *not* float

V1 1 0 1
R1 1 2 1
R2 2 0 1
R3 2 Gnd 1

.PREPROCESS REPLACEGROUND TRUE

.DC V1 1 1 0.1
.PRINT DC V(2)
.END
\end{vquote}
\end{minipage}
}
\caption[Example netlist --- \texttt{Gnd} as a synonym for node 
\texttt{0}.] {Example netlist where \texttt{Gnd} is treated as a synonym for node 
\texttt{0}.}
\label{fig:nlgndrepl1}
\end{centering}
\end{figure}

\begin{figure}
\centering{\input{gndreplace1.latex}}
\caption [Circuit diagram corresponding to figure\ \ref{fig:nlgndrepl1}.]{Circuit diagram corresponding to figure\ \ref{fig:nlgndrepl1} where 
node \texttt{Gnd} is treated as a synonym for node \texttt{0}.} 
\label{fig:gndreplace1}
\end{figure}

NOTE:	Only one \texttt{.PREPROCESS REPLACEGROUND} statement is allowed per netlist file. This constraint prevents the user from setting \texttt{REPLACEGROUND} to \texttt{TRUE} on one line and then to \texttt{FALSE} on 
another line.  Also, there is no way to differentiate between different keywords. So, for example, it is not possible to treat \texttt{GROUND} as a synonym for node \texttt{0} while allowing \texttt{GND} to represent an independent
node).  If \texttt{REPLACEGROUND} is set to \texttt{TRUE}, \Xyce{} will treat {\em both} of these keywords as node \texttt{0}.



%%%%%%%%%%%%%%%%%%%%%%%%%%%%%%%%%%%
%%REMOVING UNUSED COMPONENTS
%%%%%%%%%%%%%%%%%%%%%%%%%%%%%%%%%%%

\section{Removal of Unused Components}
\label{PP_removeunused}
\index{\texttt{.PREPROCESS}!\texttt{REMOVEUNUSED}}

Consider a slight variant of the circuit in figure\ \ref{fig:nlgndrepl1} with
the netlist given in figure\ \ref{fig:nlunused1}.  Here, the resistor \texttt{R3}
is connected in a peculiar configuration:  both terminals of the resistor are 
tied to the same circuit node, as is illustrated in figure\ \ref{fig:unused1}.
Clearly, the presence of this resistor has no effect on the other voltages and
currents in the circuit since, by the very nature of its configuration, it has
no voltage across it and, hence, does not draw any current.  Therefore, in 
some sense, the component can be considered as ``unused.''  The presence of a resistor such as \texttt{R3} is rarely or never introduced by design, rather the presence of such components is the 
result of either human or automated error during netlist creation.

\begin{figure}[htbp]
\begin{centering}
\shadowbox{
\begin{minipage}{0.8\textwidth}
\begin{vquote}
Circuit with an unused resistor R3

V1 1 0 1
R1 1 2 1
R2 2 0 1
R3 2 2 1

.DC V1 1 1 0.1
.PRINT DC V(2)
.END
\end{vquote}
\end{minipage}
}
\caption[Netlist with a resistor with terminals both the
same node.] {Netlist with a resistor \texttt{R3} whose device terminals are both the
same node (node \texttt{2}).}
\label{fig:nlunused1}
\end{centering}
\end{figure}

\begin{figure}
\centering{\input{unused1.latex}}
\caption[Circuit of figure\ \ref{fig:nlunused1}.] {Circuit of figure\ \ref{fig:nlunused1} containing a resistor \texttt{R3} whose terminals are tied to the same node (node \texttt{2}).} 
\label{fig:unused1}
\end{figure}

While the presence of the resistor \texttt{R3} in figure\ \ref{fig:nlgndrepl1} does not 
change the behavior of the circuit, it adds an additional component to 
the netlist  \Xyce{} must include when solving for the voltages and 
currents in the circuit.  If the number of such components in a given netlist
is large, it is potentially desirable to remove them from the
netlist to ease the burden on \Xyce's solver engines.  This, in turn, can help
to avoid possible convergence issues.  For example, even though the netlist in
figure\ \ref{fig:nlunused1} will run properly in \Xyce{}, the netlist of figure\ 
\ref{fig:nlunused3} will abort.  The voltage source \texttt{V2} attempts to place
a 1V difference between its two device terminals; however, as both nodes of
the voltage source are the same, the voltage source is effectively shorted.

\begin{figure}[htbp]
\begin{centering}
\shadowbox{
\begin{minipage}{0.8\textwidth}
\begin{vquote}
Circuit with improperly connected voltage source V2

V1 1 0 1
R1 1 2 1
R2 2 0 1
V2 2 2 1

.DC V1 1 1 0.1
.PRINT DC V(2)
.END
\end{vquote}
\end{minipage}
}
\caption[Circuit with an improperly connected voltage source.] {Circuit with an improperly connected voltage source \texttt{V2}.}
\label{fig:nlunused3}
\end{centering}
\end{figure}

\Xyce{} includes the following command to prevent similar situations:

\texttt{.PREPROCESS REMOVEUNUSED <component list>}

where \texttt{<component list>} is a list of device types separated by commas.
For each device type specified in the list, \Xyce{} checks for instances of
that device type for which all of the device's terminals are connected to the
same node.  If such a device is found, \Xyce{} removes that device from the
netlist.  For instance, when executing the netlist of figure\ \ref{fig:nlunused2},
\Xyce{} will seek out such devices and remove them from the netlist.  This causes the
resistor \texttt{R3} to be removed from the netlist. Figure\ \ref{fig:unused2} presents the schematic of the 
resulting \Xyce{}-simulated circuit.  
NOTE:	The presence of  ``\texttt{C}'' in the \texttt{REMOVEUNUSED} statement does
not cause \Xyce{} to abort even though there are no capacitors in the netlist.
Also, as in the case of a \texttt{REPLACEGROUND} statement, only one 
\texttt{.PREPROCESS REMOVEUNUSED} line may be present in a netlist, or \Xyce{} will abort.

Table \ref{tbl:removeunusedtbl} lists devices that can be removed via a \texttt{REMOVEUNUSED} 
statement.  In the case of MOSFETs and BJTs, three device terminals must be the same (the gate, source,
and drain in the case of a MOSFET; the base, collector, and emitter in the
case of a BJT) to remove either device from the netlist.

\begin{figure}[htbp]
\begin{centering}
\shadowbox{
\begin{minipage}{0.8\textwidth}
\begin{vquote}
Circuit with improperly connected voltage source V2

V1 1 0 1
R1 1 2 1
R2 2 0 1
R3 2 2 1

.PREPROCESS REMOVEUNUSED R,C

.DC V1 1 1 0.1
.PRINT DC V(2)
.END
\end{vquote}
\end{minipage}
}
\caption{Circuit with an ``unused'' resistor R3 removed from the 
netlist.}
\label{fig:nlunused2}
\end{centering}
\end{figure}

\begin{figure}[h]
\centering{\input{unused2.latex}}
\caption[Circuit of figure\ \ref{fig:nlunused2}.] {Circuit of figure\ \ref{fig:nlunused2} where resistor R3 has been removed via the \texttt{.PREPROCESS REMOVEUNUSED} statement.} 
\label{fig:unused2}
\end{figure}

\LTXtable{0.5\textwidth}{removeunusedtbl}



%%%%%%%%%%%%%%%%%%%%%%%%%%%%%%%%%%%%%%%%%%%%%%
%%ADDING RESISTORS TO NO-DC-PATH AND 
%%CONN-TO-ONE-TERMINAL NODES
%%%%%%%%%%%%%%%%%%%%%%%%%%%%%%%%%%%%%%%%%%%%%%

\section{Adding Resistors to Dangling Nodes}
\label{PP_dangling}
\index{\texttt{.PREPROCESS}!\texttt{ADDRESISTORS}}
Consider the netlist of figure\ \ref{fig:nldangling1} and the corresponding
schematic of figure\ \ref{fig:dangle1}.  Nodes \texttt{3} and \texttt{4} of the 
netlist are what we will henceforth refer to as {\em dangling nodes}.  We say
that node \texttt{4} dangles because it is only connected to the terminal of a
single device, while we say that node \texttt{3} dangles because it has no DC
path to ground.  The first of these situations---connection to a single 
device terminal only---can arise, for example, in a netlist which contains 
nodes representing output pins that are not connected to a load device.  For 
instance, the resistance \texttt{R2} in figure\ \ref{fig:nldangling1} could 
represent the resistance of an output pin of a package that is meant to drive 
resistive loads.  Hence, an actual physical implementation of the circuit of 
figure\ \ref{fig:dangle1} would normally include a resistor between node \texttt{4} 
and ground, but, in creating the netlist, the presence of such an output load 
has been (either intentionally or unintentionally) left out.

\begin{figure}[htbp]
\begin{centering}
\shadowbox{
\begin{minipage}{0.8\textwidth}
\begin{vquote}
Circuit with two dangling nodes, nodes 3 and 4

V1 1 0 1
R1 1 2 1
C1 2 3 1 
C2 3 0 1
R2 2 4 1

.DC V1 0 1 0.1
.PRINT DC V(2)
.END
\end{vquote}
\end{minipage}
}
\caption[Netlist of circuit with two dangling nodes.] {Netlist of circuit with two dangling nodes, nodes \texttt{3} and 
\texttt{4}.}
\label{fig:nldangling1}
\end{centering}
\end{figure}

\begin{figure}[h]
\centering{\input{dangle1.latex}}
\caption{Schematic of netlist in figure\ \ref{fig:nldangling1}.} 
\label{fig:dangle1}
\end{figure}

The second situation---where a node has no DC path to ground---is sometimes an
effect that is purposely incorporated into a design (e.g., the design of
switched capacitor integrators (e.g., see \cite{JohnsMartin}, chapter 10), but
oftentimes it is also the result of some form of error in the process of
creating the netlist.  For instance, when graphical user interfaces (GUIs) are
used to create circuit schematics that are then translated into netlists via
software, one very common unintentional error is to fail to connect two nodes
that are intended to be connected.  To illustrate this point, consider the
schematic of figure\ \ref{fig:dangle3}.  The schematic seems to indicate that
the lower terminal of resistor \texttt{R2} should be connected to node
\texttt{3}. This is not the case as there is a small gap between node
\texttt{3} and the line intended to connect node \texttt{3} to the resistor.
Such an error can often go unnoticed when creating a schematic of the netlist
in a GUI.  Thus, when the schematic is translated into a netlist file, the
resulting netlist would {\em not} connect the resistor to node \texttt{3} and
would instead create a new node at the bottom of the resistor, resulting in the
circuit depicted in figure\ \ref{fig:dangle1}.  

While neither of the previous situations is necessarily threatening (\Xyce{}
will run the figure\ \ref{fig:nldangling1} netlist successfully to completion),
there are times when it is desirable to somehow make a dangling node {\em not}
dangle.  For instance, returning to the example in which the resistor
\texttt{R2} represents the resistance of an output pin, one may want to
simulate the circuit when a 1K load is attached between node \texttt{4} and
ground in figure\ \ref{fig:dangle1}.  In the case where a node has no DC path
to ground, the situation is slightly more dangerous if, for instance, the node
in question is also connected to a high-gain device such as the gate of a
MOSFET.  As the DC gate bias has a great impact on the DC current traveling
through the drain and source of the transistor, not having a well-defined DC
gate voltage can greatly degrade the simulated performance of the circuit.

In both prior examples, the only true way to ``fix'' each of these issues is to
find all dangling nodes in a particular netlist file and augment the netlist
at/near these nodes to obtain the desired behavior.  If, however, the number of
components in a circuit is very large (say on the order of hundreds of
thousands of components), manually augmenting the netlist file for each
dangling node becomes a practical impossibility if the number of such nodes is
large.  

Hence, it is desirable for \Xyce{} to be capabable of automatically augmenting
netlist files so as to help remove dangling nodes from a given netlist.  The
command \texttt{.PREPROCESS ADDRESISTORS} is designed to do just this. Assuming
the netlist of figure\ \ref{fig:nldangling2} is stored in the file
\texttt{filename}, the \texttt{.PREPROCESS ADDRESISTORS} statements will cause
\Xyce{} to create a new netlist file called \texttt{filename\_xyce.cir}
(depicted in figure\ \ref{fig:nlnotdangling}).  The line \texttt{.PREPROCESS
ADDRESISTORS NODCPATH 1G} instructs \Xyce{} to create a copy of the netlist
file containing a set of resistors of value 1~G$\mathsf{\Omega}$ that are
connected between ground and the nodes that did not have a DC path to ground.
Similarly, the line \texttt{.PREPROCESS ADDRESISTORS ONETERMINAL 1M} instructs
\Xyce{} to add to the same netlist file a set of resistors of value
1~M$\mathsf{\Omega}$ that are connected between ground and devices that are
connected to only one terminal.  The resistor \texttt{RNODCPATH1} in figure\
\ref{fig:nlnotdangling} achieves the first of these goals while
\texttt{RONETERM1} achieves the second.  Figure\ \ref{fig:dangle2} shows a
schematic of the resulting circuit represented by the netlist in figure\
\ref{fig:nlnotdangling}.

\begin{figure}[h]
\centering{\input{dangle3.latex}}
\caption[Schematic  with an incomplete connection.] {Schematic of a circuit with an incomplete connection between the 
resistor \texttt{R2} and node \texttt{3}.} 
\label{fig:dangle3}
\end{figure}

\begin{figure}[htbp]
\begin{centering}
\shadowbox{
\begin{minipage}{0.8\textwidth}
\begin{vquote}
Circuit with two dangling nodes, nodes 3 and 4

V1 1 0 1
R1 1 2 1
C1 2 3 1 
C2 3 0 1
R2 2 4 1

.PREPROCESS ADDRESISTORS NODCPATH 1G
.PREPROCESS ADDRESISTORS ONETERMINAL 1M

.DC V1 0 1 0.1
.PRINT DC V(2)
.END
\end{vquote}
\end{minipage}
}
\caption[Netlist of circuit with two dangling nodes with \texttt{.PREPROCESS ADDRESISTORS} statements.] {Netlist of circuit with two dangling nodes, nodes \texttt{3} and \texttt{4}, with \texttt{.PREPROCESS ADDRESISTORS} statements.}
\label{fig:nldangling2}
\end{centering}
\end{figure}



\begin{figure}[htbp]
\begin{centering}
\shadowbox{
\begin{minipage}{0.8\textwidth}
\begin{vquote}
XYCE-generated Netlist file copy:  TIME='07:32:31 AM' 
* DATE='Dec 19, 2007' 
*Original Netlist Title:  

*Circuit with two dangling nodes, nodes 3 and 4.


V1 1 0 1
R1 1 2 1
C1 2 3 1 
C2 3 0 1
R2 2 4 1

*.PREPROCESS ADDRESISTORS NODCPATH 1G
*Xyce:  ".PREPROCESS ADDRESISTORS" statement 
* automatically commented out in netlist copy.
*.PREPROCESS ADDRESISTORS ONETERMINAL 1M
*Xyce:  ".PREPROCESS ADDRESISTORS" statement 
* automatically commented out in netlist copy.

.DC V1 0 1 0.1
.PRINT DC V(2)


*XYCE-GENERATED OUTPUT:  Adding resistors between ground 
* and nodes connected to only 1 device terminal:

RONETERM1 4 0 1M


*XYCE-GENERATED OUTPUT:  Adding resistors between ground 
* and nodes with no DC path to ground:

RNODCPATH1 3 0 1G

.END
\end{vquote}
\end{minipage}
}
\caption[Output file resulting from  
\texttt{.PREPROCESS ADDRESISTOR} statements for figure~\ref{fig:dangle3}.] {Output file \texttt{filename\_xyce.cir} which results from the 
\texttt{.PREPROCESS ADDRESISTOR} statements for the netlist of figure\ 
\ref{fig:dangle3} (with assumed file name \texttt{filename}).}
\label{fig:nlnotdangling}
\end{centering}
\end{figure}



\begin{figure}[h]
\centering{\input{dangle2.latex}}
\caption[Schematic corresponding to figure\ 
\ref{fig:nlnotdangling}.] {Schematic corresponding to the \Xyce{}-generated netlist of figure\ 
\ref{fig:nlnotdangling}.} 
\label{fig:dangle2}
\end{figure}

Some general comments regarding the use of \texttt{.PREPROCESS ADDRESISTOR} 
statements include:

\begin{itemize}
\item \Xyce{} does not terminate immediately after the netlist file is created.
In other words, if \Xyce{} is run on the \texttt{filename} of figure\ 
\ref{fig:nldangling2} netlist, it will attempt to execute this netlist as given 
(i.e., it tries to simulate the circuit of figure\ \ref{fig:dangle1}) and 
generates the file \texttt{filename\_xyce.cir} as a byproduct.  It is important to
point out that the resistors that are added at the bottom of the netlist 
file \texttt{filename\_xyce.cir} do {\bf not} get added to the original netlist
when \Xyce{} is running on the file \texttt{filename}.  If one wishes to simulate
\Xyce{} with these resistors in place, one must run \Xyce{} on 
\texttt{filename\_xyce.cir} explicitly.

\item The naming convention for resistors which connect to ground nodes which 
do not have a DC path to ground is \texttt{RNODCPATH<i>}, where \texttt{i} is an 
integer greater than 0; the naming convention is similar for nodes which are 
connected to only one device terminal (i.e., of the form \texttt{RONETERM<i>}).   
\Xyce{} will not change this naming convention if a 
resistor with one of the above names already exists in the netlist.  

Hence, if a resistor named \texttt{RNODCPATH1} exists in netlist file 
\texttt{filename}, and \Xyce{} detects there is a node in this netlist file 
that has no DC path to ground, \Xyce{} will add {\em another} resistor with name 
\texttt{RNODCPATH1} to the netlist file \texttt{filename\_xyce.cir} (assuming that 
either \texttt{.PREPROCESS ADDRESISTORS NODCPATH} or \texttt{.PREPROCESS ADDRESISTORS
ONETERMINAL} are present in \texttt{filename}).  If \Xyce{} is subsequently run on 
\texttt{filename\_xyce.cir}, it will exit in error due to the presence of two 
resistors with the same name.

\item Commands \texttt{.PREPROCESS ADDRESISTORS NODCPATH}  and 
\texttt{.PREPROCESS} \newline \texttt{ADDRESISTORS ONETERMINAL} do {\bf not} have to be 
simultaneously present in a netlist file.  The presence of either command will
generate a file \texttt{filename\_xyce.cir}, and the presence of both will not
generate two separate files.  As with other \texttt{.PREPROCESS} commands, 
however, a netlist file is allowed to contain only one \texttt{NODCPATH} and
one \texttt{ONETERMINAL} command each.  If multiple \texttt{NODCPATH} and/or 
\texttt{ONETERMINAL} lines are found in a single netlist file, \Xyce{} will exit in
error.

\item It is possible that a single node can have no DC path to ground 
{\em and} be connected to only one device terminal.  If a \texttt{NODCPATH}
and \texttt{ONETERMINAL} command are present in a given netlist file, {\bf only}
the resistor corresponding to the \texttt{ONETERMINAL} command is added to 
the netlist file \texttt{filename\_xyce.cir} and the resistor corresponding to 
the \texttt{NODCPATH} command is omitted.  If a \texttt{NODCPATH} command is 
present but a \texttt{ONETERMINAL} command is not, then \Xyce{} will add a resistor corresponding
to the \texttt{NODCPATH} command to the netlist, as usual.
\item In generating the file \texttt{filename\_xyce.cir}, the original 
\texttt{.PREPROCESS ADDRESISTOR} statements are commented out with a warning 
message.  This is to prevent \Xyce{} from creating the file 
\texttt{filename\_xyce.cir\_xyce.cir} when the file \texttt{filename\_xyce.cir} is
run.  

NOTE:	 This feature avoids generating redundant
files.  While \texttt{filename\_xyce.cir\_xyce.cir} would be slightly different 
from \texttt{filename\_xyce.cir} (e.g., a different date and time stamp), both 
files would functionally implement the same netlist.

\end{itemize}



%%% Local Variables:
%%% mode: latex
%%% End:

% END of Xyce_UG_Preprocess.tex ************
