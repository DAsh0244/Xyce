% Sandia National Laboratories is a multimission laboratory managed and
% operated by National Technology & Engineering Solutions of Sandia, LLC, a
% wholly owned subsidiary of Honeywell International Inc., for the U.S.
% Department of Energy’s National Nuclear Security Administration under
% contract DE-NA0003525.

% Copyright 2002-2020 National Technology & Engineering Solutions of Sandia,
% LLC (NTESS).

% -------------------------------------------------------------------------
% Sampling Analysis Section ---------------------------------------------------
% -------------------------------------------------------------------------
\clearpage
\section{Random Sampling Analysis}
\label{SAMPLING_Analysis}
\label{sampling_Overview}
\index{analysis!Sampling} \index{Sampling analysis} \index{Random Sampling}
\index{\texttt{.SAMPLING}}

The \verb|.SAMPLING| command causes \Xyce{} execute the primary analysis (\verb|.DC|,
\verb|.AC|, \verb|.TRAN|, etc.) inside a loop over randomly distributed parameters.  
The user specifies a list of random parameters , their distributions, distribution 
parameters (such as mean and standard deviation), and the total number of 
sample points.  The primary analysis is then executed sequentially for each sample point.

\Xyce{}  can use \verb|.SAMPLING| to modify any device instance or device
model parameter, as well as the circuit temperature.  It is not legal to use 
parameters defined in \texttt{.PARAM} statements, but it is legal to use global 
parameters defined in \texttt{.global\_param} statements.  
Section~\ref{Parameters_Expressions} discusses the differences 
between \texttt{.param} and \texttt{.global\_param} in more detail.

The \verb|.SAMPLING| capability uses a lot of the same code base as \verb|.STEP|, 
so most of the guidance for \verb|.STEP|, including legal parameters, output 
file formats, supported analyses types, etc., also applies to \verb|.SAMPLING|.

\subsection{.SAMPLING Statement}
\label{sampling_statement}
Sampling analysis may be specified by simply adding a \verb|.SAMPLING| 
line to a netlist.  Unlike \verb|.DC|, \verb|.SAMPLING| by itself is not an 
adequate analysis specification, as it merely specifies an outer loop around the 
normal analysis.  A standard analysis line, specifying \verb|.TRAN|, \verb|.AC|, \verb|.HB|, 
and \verb|.DC| analysis, is still required.

Examples of typical \verb|.SAMPLING| lines:

\Example{\\
\texttt{\\
.SAMPLING  \\
+ param=M1:L, M1:W \\
+ type=uniform,uniform \\
+ lower\_bounds=5u,5u \\
+ upper\_bounds=7u,7u \\
}
\texttt{ \\
.SAMPLING \\
+ param=R1,c1 \\
+ type=normal,normal \\
+ means=3K,1uF \\
+ std\_deviations=1K,0.2uF \\
}

}

In the first example, \verb|M1:L| and \verb|M1:W| 
are the names of the
parameters (in this instance, the length and width parameters of the MOSFET
\texttt{M1}), \verb|5u| is the minimum value of both parameters and
\verb|7u| is the maximum value of both parameters.  For both parameters, 
the distribution (specified by \texttt{type}) is uniform.
In the second example, the two parameters are \verb|R1| and \verb|c1|.   In this 
case, it isn't necessary to specify the specific parameter of \verb|R1| or \verb|c1| 
as resistors and capacitors have simple default parameters (see section~\ref{sampling_SpecialCases}).  
The distributions of both parameters are normal, and the means and standard deviations 
are specified for both in comma-separated lists.

For any list of parameters, all the required fields must be provided in comma-separated 
lists.  \Xyce{} will throw an error if the length of any of the lists is different 
than that of the parameter list.

\subsection{.options SAMPLES Statement}
\label{options_samples_statement}

To specify the number of samples, use the netlist command \texttt{.options 
SAMPLES numsamples=<value>}.   For sampling analysis to work correctly, this parameter is required.
Other sampling details, including the sampling type, the covariance matrix, 
the statistical outputs and the random seed can be specified using the 
same \verb|.options SAMPLES| netlist command.  Some examples of this command are:

\Example{\\
\texttt{ \\
.options samples numsamples=1000 \\
}
\texttt{ \\
.options samples numsamples=30 \\
+ covmatrix=1e6,1.0e-3,1.0e-3,4e-14 \\
+ OUTPUTS={R1:R},{C1:C} \\
+ MEASURES=maxSine \\
+ SAMPLE\_TYPE=LHS \\
+ SEED=743190581 \\
}
}
The random seed can be set either in the \texttt{.options SAMPLES} statement, or 
it can be set from the command line using the \texttt{-randseed} option.  If it 
is not set either way, then the random seed is generated internally.

\subsection{Sampling a Device Instance Parameter}
\label{sampling_InstanceParam}
Figure~\ref{Sampling_Netlist_1} provides a simple application of \verb|.SAMPLING| 
to a device instance parameter.  The circuit is a simple voltage divider driven 
by a oscillating voltage source.  The randomly sampled parameter is the resistance 
of \texttt{R1}, which is given a normal distribution with a specified mean and 
standard deviation.

As the circuit is executed multiple times, an output file is generated based on 
the \verb|.PRINT| command.  The \verb|.PRINT| statement is still used in the same 
way that it would be in a non-sampling analysis.  However, the output file 
contains the concatenated output of each \verb|.SAMPLING| step.  
Section~\ref{sampling_output_files} provides more details of how \texttt{.SAMPLING} 
changes \texttt{.PRINT} output files.

\Xyce{} computes statistical outputs for quantities specified by \texttt{OUTPUTS} 
and \texttt{MEASURES} on the \texttt{.options SAMPLES} line.  For these outputs,
moments such as mean and variance are computed and sent to standard output.  More 
details of this type of output, as well as examples, are given in section~\ref{statistical_outputs}.
\begin{figure}[htbp]
\begin{centering}
\shadowbox{
\begin{minipage}{0.8\textwidth}
\begin{vquote}
\color{blue}Voltage Divider with sampling\color{black}
R2  1  0  7k
R1 1 2 3k
VS1  2  0  SIN(0 1.0e3 1KHZ 0 0)

.tran 0.1ms 1ms
.meas tran maxSine max V(1) 

\color{XyceRed}.SAMPLING param=R1:R 
+ type=normal means=1K std\_deviations=0.1K\color{black}

\color{XyceRed}.options SAMPLES numsamples=1000 OUTPUTS=\{R1:R\} 
+ MEASURES=maxSine SAMPLE\_TYPE=MC\color{black}

\end{vquote}
\end{minipage}
}
\caption{Voltage divider circuit netlist with sampling of a device instance parameter.
Sampling commands are in \color{XyceRed}red\color{black}.
\label{Sampling_Netlist_1}}
\end{centering}
\end{figure}

\subsection{Sampling a Device Model Parameter}
\label{sampling_ModelParam}

Sampling a model parameter can be done in an identical manner to an instance parameter.  
Figure~\ref{Sampling_Netlist_2} provides a simple application of \verb|.SAMPLING| to a 
device model and a device instance.  This is the same diode clipper circuit as was used 
in the Transient Analysis section (see section \ref{Transient_Analysis_Sec}), except that 
several lines of sampling commands (in red) have been added.  

The sampling commands will cause \Xyce{} to sample the model parameter, \verb|D1N3940:IS| as well as the 
resistance of the resistor, R4. Both of the parameters are randomly sampled using uniform distributions 
with specified lower and upper bounds.
\begin{figure}[htbp]
\begin{centering}
\shadowbox{
\begin{minipage}{0.8\textwidth}
\begin{vquote}
\color{blue}Transient Diode Clipper Circuit with Sampling Analysis\color{black}

\color{blue}* Primary Analysis Command\color{black}
.tran 2ns 0.5ms
\color{blue}* Output\color{black}
.print tran format=tecplot V(3) V(2) V(4)
.meas tran maxSine max V(4) 

\color{XyceRed}* Sampling statements 
.SAMPLING
+ param=R4:R,D1N3940:IS
+ type=uniform,uniform
+ lower\_bounds=3.0k,2.0e-10
+ upper\_bounds=15.0k,6.0e-10

.options SAMPLES numsamples=10 SAMPLE\_TYPE=LHS
+ OUTPUTS={R4:R},{D1N3940:IS} MEASURES=maxSine\color{black}
\color{blue}* Voltage Sources\color{black}
VCC 1 0 5V
VIN 3 0 SIN(0V 10V 1kHz)
\color{blue}* Diodes\color{black}
D1 2 1 D1N3940
D2 0 2 D1N3940
\color{blue}* Resistors\color{black}
R1 2 3 1K
R2 1 2 3.3K
R3 2 0 3.3K
R4 4 0 5.6K
\color{blue}* Capacitor\color{black}
C1 2 4 0.47u
.MODEL D1N3940 D(
+ IS=4E-10 RS=.105 N=1.48 TT=8E-7
+ CJO=1.95E-11 VJ=.4 M=.38 EG=1.36
+ XTI=-8 KF=0 AF=1 FC=.9 BV=600 IBV=1E-4)
.END
\end{vquote}
\end{minipage}
}
\caption{Diode clipper circuit netlist with sampling analysis.  This example uses a model parameter as one of the sampling parameters.
Sampling commands are in \color{XyceRed}red\color{black}.
  \label{Sampling_Netlist_2}}
\end{centering}
\end{figure}

\subsection{Sampling a Global Parameter}
\label{sampling_GlobalParam}

Sampling can be applied to global parameters.  An example of such usage is given 
in figure~\ref{Sampling_Netlist_3}, which is identical to the circuit in 
figure~\ref{Sampling_Netlist_1}, except that it uses global parameters.
\begin{figure}[htbp]
\begin{centering}
\shadowbox{
\begin{minipage}{0.8\textwidth}
\begin{vquote}
\color{blue}Voltage Divider with sampling, using global parameters\color{black}
.global\_param Vmax=1000.0
.global\_param testNorm=1.5k
.global\_param R1value=\{testNorm*2.0\}

R2  1  0  7k
R1 1 2 \{R1value\}
VS1  2  0  SIN(0 \{Vmax\} 1KHZ 0 0)
.tran 0.1ms 1ms
.meas tran maxSine max V(1) 

\color{XyceRed}.SAMPLING param=testNorm
+ type=normal means=0.5K std\_deviations=0.05K\color{black}

\color{XyceRed}.options SAMPLES numsamples=1000 SAMPLE\_TYPE=MC
+ OUTPUTS=\{R1:R\} MEASURES=maxSine\color{black}

\end{vquote}
\end{minipage}
}
\caption{Voltage divider circuit netlist with sampling of a global parameter.
Sampling commands are in \color{XyceRed}red\color{black}.
\label{Sampling_Netlist_3}}
\end{centering}
\end{figure}

\subsection{Sampling over Temperature}
\label{sampling_Temperature}

It is also possible to sample temperature.  To do so, simply 
specify \verb|temp| as the parameter name.  It will work in the 
same manner as \verb|.SAMPLING| when applied to model and instance 
parameters.

\subsection{Special cases: Sampling Independent Sources, Resistors, Capacitors and Inductors}
\label{sampling_SpecialCases}

For some devices, there is generally only one parameter that one would
want to sample.  For example, a linear resistor's only parameter of
interest is resistance, R.  Similarly, for a DC voltage or current
source, one is usually only interested in the magnitude of the source, DCV0.
Finally, linear capacitors generally only have capacitance, C, as a 
parameter of interest, while inductors generally only have inductance, L,
as a parameter of interest.  

For these simple devices, it is not necessary to specify both the
parameter and device on the \texttt{.SAMPLING} line: only the device name
is strictly required, as these devices have default
parameters that are assumed if no parameter name is given explicitly.

Examples of usage are given below.  The first two lines are equivalent
--- in the first line, the resistance parameter of \texttt{R4} is
named explicitly, and in the second line the resistance parameter is
implicit. A similar example is then shown for the DC voltage of the 
voltage source \texttt{VCC}.  In the remaining lines, parameter names are all 
implicit, and the default parameters of the associated devices are used.

\Example{ \\
\begin{tabular}{lllll}
  \texttt{.SAMPLING param=R4:R,C1:C} &\texttt{type=normal,normal}  &\texttt{means=3k,1u}  &\texttt{std\_deviations=1k,0.1u} \\ 
  \texttt{.SAMPLING param=R4,C1} &\texttt{type=normal,normal}  &\texttt{means=3k,1u}  &\texttt{std\_deviations=1k,0.1u} \\ 
  \texttt{.SAMPLING VCC:DCV0}&\texttt{type=uniform}  &\texttt{means=6.0}  &\texttt{std\_deviations=1.0 } \\  
  \texttt{.SAMPLING VCC}&\texttt{type=uniform}  &\texttt{means=6.0}  &\texttt{std\_deviations=1.0 } \\ 
  \texttt{.SAMPLING param=C1} &\texttt{type=normal}  &\texttt{means=0.5u}  &\texttt{std\_deviations=0.05u} \\ 
  \texttt{.SAMPLING param=L1} &\texttt{type=normal}  &\texttt{means=0.5m}  &\texttt{std\_deviations=0.01m} \\ 
\end{tabular}
        }

Independent sources require further explanation.  Their default
parameter, DCV0, only applies to \texttt{.DC} analyses.  They do not have
default parameters for their transient forms, such as \texttt{SIN}
or \texttt{PULSE}.

\subsection{Output files}
\label{sampling_output_files}
\index{output!\texttt{.SAMPLING}}

Users can think of \texttt{.SAMPLING} simulations as a large number of distinct executions
of the same circuit netlist.  The output data, as specified by a \texttt{.PRINT}
line, however, goes to a single output file.  
\Xyce{} also creates a second auxilliary file with the \texttt{*.res} 
suffix, which just contains the parameter values.  Statistical outputs 
can also be produced (see section~\ref{statistical_outputs}).

Figure~\ref{Sampling_Netlist_1} shows an example file named \verb+clip.cir+, which when run will produce files
\verb+clip.cir.res+ and \verb+clip.cir.prn+.  The file \verb+clip.cir.res+
contains one line for each step, showing what parameter value was used
on that step.  \verb+clip.cir.prn+ is the familiar output format, but
the \verb+INDEX+ field recycles to zero each time a new step begins.
As the default behavior distinguishes each step's output only by recycling 
the \verb+INDEX+ field to zero, it can be beneficial to add the sampling
parameters to the \verb+.PRINT+ line.   For the default file format 
(\texttt{format=std}), \Xyce{} will not automatically include these sampling parameters,
so for plotting it is usually best to specify them by hand.

If using the default \texttt{.prn} file format (\texttt{format=std}), the 
resulting \texttt{.SAMPLING} simulation output file will be a simple concatenation of
each step's underlying analysis output.
If using \texttt{format=probe}, the data for each execution of the circuit
will be in distinct sections of the file, and it should be easy to 
plot the results using PROBE.  If using \texttt{format=tecplot}, 
the results of each \texttt{.SAMPLING} simulation will be in a distinct
Tecplot zone. Finally, format=raw will place the results for each \texttt{.SAMPLING} 
simulation in a distinct ``plot'' region. 

Note that for sampling calculations involving a really large number of sample 
points, the single output file can become prohibitively large.  Be careful when 
using \verb|.PRINT| if the number of samples is large.   If the number is really 
large (thousands) consider excluding any \verb|.PRINT| output statements and 
just rely on statistical outputs, described next in section~\ref{statistical_outputs}.
Similarly, the generation of the measure output can be suppressed with 
\texttt{.OPTIONS MEASURE MEASPRINT}.

\subsection{Statistical Outputs}
\label{statistical_outputs}
When performing uncertainty quantification analysis, the outputs of interest are 
often statistical moments of different circuit metrics.   \Xyce{} will compute these at the
end of a sampling calculation upon request.  The requested statistical outputs are 
specified on the \texttt{.options SAMPLING} line in the netlist.  For an example 
see the \texttt{OUTPUTS} and \texttt{MEASURES} fields in section~\ref{options_samples_statement}.  
Both \texttt{OUTPUTS} and \texttt{MEASURES} are specified as comma-separated 
lists.  The list of \texttt{OUTPUTS} must contain valid expressions that can be 
processed by the \Xyce{} expression library.  The list of \texttt{MEASURES} must 
consist valid \texttt{.MEASURE} statement names, which are present elsewhere in 
the netlist.

The example netlist given by figure~\ref{Sampling_Netlist_1} will produce the result 
given by figure~\ref{Sampling_Netlist_1_output}.  The example netlist given 
by figure~\ref{Sampling_Netlist_2} will produce the result given by 
figure~\ref{Sampling_Netlist_2_output}.  In both examples, the number of samples 
is relatively small, so the exact quantities computed will vary somewhat from run to run.
\begin{figure}[htbp]
\begin{centering}
\shadowbox{
\begin{minipage}{0.8\textwidth}
\begin{vquote}
MC sampling mean of {R1:R} = 1009.58
MC sampling stddev of {R1:R} = 100.131
MC sampling variance of {R1:R} = 10026.3
MC sampling skew of {R1:R} = -0.0208347
MC sampling kurtosis of {R1:R} = 2.79565
MC sampling max of {R1:R} = 1308.03
MC sampling min of {R1:R} = 656.039

MC sampling mean of maxSine = 874.063
MC sampling stddev of maxSine = 10.9115
MC sampling variance of maxSine = 119.061
MC sampling skew of maxSine = 0.08628
MC sampling kurtosis of maxSine = 2.83029
MC sampling max of maxSine = 914.274
MC sampling min of maxSine = 842.547
\end{vquote}
\end{minipage}
}
\caption{Typical statistical outputs for figure~\ref{Sampling_Netlist_1}}
\label{Sampling_Netlist_1_output}
\end{centering}
\end{figure}

\begin{figure}[htbp]
\begin{centering}
\shadowbox{
\begin{minipage}{0.8\textwidth}
\begin{vquote}
LHS sampling mean of {R4:R} = 9198
LHS sampling stddev of {R4:R} = 4293.89
LHS sampling variance of {R4:R} = 1.84375e+07
LHS sampling skew of {R4:R} = -0.0939244
LHS sampling kurtosis of {R4:R} = 1.39595
LHS sampling max of {R4:R} = 14996
LHS sampling min of {R4:R} = 3315.6

LHS sampling mean of {D1N3940:IS} = 4.2571e-10
LHS sampling stddev of {D1N3940:IS} = 9.10744e-11
LHS sampling variance of {D1N3940:IS} = 8.29455e-21
LHS sampling skew of {D1N3940:IS} = 0.0486504
LHS sampling kurtosis of {D1N3940:IS} = 2.05108
LHS sampling max of {D1N3940:IS} = 5.78543e-10
LHS sampling min of {D1N3940:IS} = 2.85753e-10

LHS sampling mean of maxSine = 4.46494
LHS sampling stddev of maxSine = 0.0901329
LHS sampling variance of maxSine = 0.00812394
LHS sampling skew of maxSine = -0.682188
LHS sampling kurtosis of maxSine = 2.07339
LHS sampling max of maxSine = 4.5536
LHS sampling min of maxSine = 4.28396
\end{vquote}
\end{minipage}
}
\caption{Typical statistical outputs for figure~\ref{Sampling_Netlist_2}}
\label{Sampling_Netlist_2_output}
\end{centering}
\end{figure}

\clearpage

