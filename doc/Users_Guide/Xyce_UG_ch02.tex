% Sandia National Laboratories is a multimission laboratory managed and
% operated by National Technology & Engineering Solutions of Sandia, LLC, a
% wholly owned subsidiary of Honeywell International Inc., for the U.S.
% Department of Energy’s National Nuclear Security Administration under
% contract DE-NA0003525.

% Copyright 2002-2022 National Technology & Engineering Solutions of Sandia,
% LLC (NTESS).

%%-------------------------------------------------------------------------
%% Purpose        : Main LaTeX Xyce Users' Guide
%% Special Notes  : Graphic files (pdf format) work with pdflatex.  To use
%%                  LaTeX, we need to use postcript versions.  Not sure why.
%% Creator        : Scott A. Hutchinson, Computational Sciences, SNL
%% Creation Date  : {05/23/2002}
%%
%%-------------------------------------------------------------------------

% -------------------------------------------------------------------------
% Installing and Running Xyce Chapter -------------------------------------
% -------------------------------------------------------------------------

\chapter{Installing and Running Xyce}
\label{Running_Installing_Xyce}
\chapteroverview{Chapter Overview}
{
This chapter describes the basic mechanics of installing and running
\Xyce{}.  It includes the following sections:
\begin{XyceItemize}
\item Section~\ref{Xyce_Installation}, {\em \Xyce{} Installation}
\item Section~\ref{Running_Xyce}, {\em Running \Xyce{}}
\end{XyceItemize}
}

\section{Xyce Installation}
\label{Xyce_Installation}

\Xyce{} is distributed in two ways: source code and binary installers.
Instructions for both methods of installation are available on the \Xyce\ web
site {\color{XyceDeepRed}\url{http://xyce.sandia.gov}}.

\subsection{Postinstallation steps: PATH setting}
In order to run \Xyce{}, either type the path to its binary every time, or
add the installed location of \Xyce{} to the PATH variable.  Do so by editing
the shell start-up file (\texttt{.bashrc} if your shell is bash,
\texttt{.profile} if the shell is sh, or \texttt{.cshrc} if the shell is the
c-shell).  Exact syntax depends on the shell used, but for bash and sh the
syntax is:

\begin{quote}
  \verb+export PATH=$PATH:/usr/local/Xyce-Release-6.9.0/bin+
\end{quote}

The exact path will depend on the installed version of \Xyce{}.  Look in
\texttt{/usr/local} with the following command:
\begin{quote}
  \verb+ls -l /usr/local+
\end{quote}
to identify the actual install directory used by the installed version of
\Xyce{}.  Once entered into the start-up file, the path will be set this way at
the next log in.  The same command can be issued directly in the command line
and it will take effect immediately.

Binary installations on Windows create a ``Command Prompt'' shortcut
for Xyce that sets this path for you.  When the command prompt is
opened with this shortcut, Xyce may simply be invoked by typing its
name.

\section{Running Xyce}
\label{Running_Xyce}
\index{running \Xyce{}}\index{\Xyce{}!running}

While it is possible to connect \Xyce{} to graphical interfaces, such as
gEDA~\cite{geda} or Qucs-s~\cite{qucs-s}, \Xyce{} is not provided with any graphical user interface. It
is primarily used as a command-line-only program across all supported
platforms, including traditionally ``GUI-centered'' platforms such as Mac OS X
and Microsoft Windows.

This section describes how \Xyce{} is run from the command line, for serial and
MPI\index{MPI} parallel simulations.

\subsection{Command Line Simulation}
\index{command line}
\label{command_line_simulation}

The syntax for running \Xyce{} from the command line differs depending
it is the serial or parallel version.

\paragraph{Running \Xyce{}}

Assuming the \texttt{Xyce} executable in in the user's path, then the
commands for running \Xyce\ are:
\begin{XyceItemize}
\item Running serial \Xyce:
\begin{vquote}
  Xyce [options] <netlist filename>
\end{vquote}
\item Running \Xyce\ in parallel:
\begin{vquote}
  mpirun -np <# procs> Xyce [options] <netlist filename>
\end{vquote}
\end{XyceItemize}
Note that \texttt{mpiexec} is sometimes provided as an alternative to
\texttt{mpirun}. With Open MPI, the two commands are identical, but
that is not the case for all MPI implementations.

\paragraph{General comments}
The \texttt{[options]} are the command line arguments for \Xyce{}.  For example,
to log output to a file named \texttt{sample.log} \index{output!log file}type:
\begin{vquote}
  Xyce -l sample.log <netlist filename>
\end{vquote}

The next example runs parallel \Xyce{} on four processors and places
the results into a set of files with the ``basename'' of \texttt{results}
\index{output!specifying file name}.  So, for example, the output from any
\texttt{.PRINT TRAN} and/or \texttt{.MEASURE TRAN} lines in the netlist
would be placed into the files \texttt{results.prn} and \texttt{results.mt0},
respectively (assuming a source code install):
\begin{vquote}
  mpirun -np 4 Xyce -o results <netlist filename>
\end{vquote}

While \Xyce{} is running, simulation progress is output to the
command line\index{command line!output} window along with any error
messages.  

\Xyce\ assumes that \texttt{<netlist filename>} is either in
the current working directory, or includes the path (full or relative)
to the netlist file.  Enclose the filename in quotation
marks~(\texttt{"}) if the path contains spaces.  Help is accessible with
the \texttt{-h} option.

Consult the documentation installed with MPI on the user's platform for more
details concerning MPI options.  The \texttt{-np <\# procs>} denotes the number
of processors\index{parallel!number of processors} to use for the
simulation.\newline \emph{NOTE: It is critical that the number of processors
used must be smaller than the number of devices and voltage nodes in the
netlist.}

\subsubsection{Guidance for Running Xyce in Parallel}
\index{parallelGuidance}

The basic mechanics of running \Xyce{} in parallel have been discussed above.
For general guidance regarding solver options, partitioning options, and
other parallel issues, refer to chapter~\ref{Parallel}.
Distributed memory circuit simulation still contains a number of research
issues, so obtaining an optimal simulation in parallel is a bit of an art. 

\emph{NOTE:  If you are running on Linux, please see chapter~\ref{Parallel}, especially section~\ref{paffinity}, for critical guidance on OpenMPI options that can seriously impact the performance of parallel jobs on a multiuser system.}

\subsection{Command Line Options}
\label{cmd_line_args}
\index{command line!options}

\Xyce{} supports a handful of command line options that must be given
{\em before} the netlist filename. Table~\ref{cmd_line_arg_list} lists \Xyce{} core options.

\LTXtable{\textwidth}{commandlinetbl}

While these options are intended for general use, others may exist for new
features that are disabled by default, and older, deprecated features.  The
\Xyce{} Reference Guide\ReferenceGuide{} provides a comprehensive list, including trial and
deprecated options.


%%% Local Variables:
%%% mode: latex
%%% End:

% END of Xyce_UG_ch02.tex ************
