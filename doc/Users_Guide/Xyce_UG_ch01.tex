% Sandia National Laboratories is a multimission laboratory managed and
% operated by National Technology & Engineering Solutions of Sandia, LLC, a
% wholly owned subsidiary of Honeywell International Inc., for the U.S.
% Department of Energy’s National Nuclear Security Administration under
% contract DE-NA0003525.

% Copyright 2002-2020 National Technology & Engineering Solutions of Sandia,
% LLC (NTESS).

%%-------------------------------------------------------------------------
%% Purpose        : Main LaTeX Xyce Users' Guide
%% Special Notes  : Graphic files (pdf format) work with pdflatex.  To use
%%                  LaTeX, we need to use postcript versions.  Not sure why.
%% Creator        : Scott A. Hutchinson, Computational Sciences, SNL
%% Creation Date  : {05/23/2002}
%%
%%-------------------------------------------------------------------------

%\SANDmain           % Start the main part of the report

\chapter{Introduction}
\label{Introduction}

\chapteroverview{Welcome to \XyceTitle{}}
{
The \XyceTM{} Parallel Electronic Simulator is a
SPICE-compatible~\cite{NagelRohrer}~\cite{Spice3f5-user-guide}
circuit simulator that has been written to support
the unique simulation needs of electrical designers at Sandia National
Laboratories\index{Sandia National Laboratories}.  It is specifically targeted 
to run on large-scale parallel computing
\index{parallel!computing} platforms, but is also available on a variety
of architectures including single processor workstations.  It aims to
support a variety of devices and models specific to Sandia needs, as well
as standard capabilities available from current commercial simulators.
}
\section{Xyce Overview}
\label{Xyce_Overview}

The \Xyce{} Parallel Electronic Simulator project was started in 1999
to support the simulation needs of electrical designers at Sandia
National Laboratories and has evolved into a mature platform for large-scale
circuit simulation.

\Xyce{} includes several unique features.  An important driver 
has been the need to simulate very large-scale
circuits (100,000 devices or more) on the transistor level.  To this end,
scalable algorithms for simulating large circuits in parallel have been
developed.  In addition \Xyce{} includes novel 
approaches to numerical kernels including model-order reduction, continuation algorithms, 
time-integration, nonlinear and linear solvers.  Also, unlike most SPICE-based codes, 
\Xyce{} uses a differential-algebraic-equation (DAE) formulation, which better isolates the 
device model package from solver algorithms. 

\section{Xyce Capabilities}
\label{Xyce_Capabilities}

\subsection{Support for Large-Scale Parallel Computing}
\label{Parallel_Support}
\Xyce{} is a truly parallel simulation code, designed and written from the
ground up to support large-scale\index{parallel!large scale} parallel computing\index{parallel!computing} architectures with up to thousands of processors.
This provides \Xyce{} the capability to solve large circuit problems 
with quick enough runtimes to make these simulations practical.
\Xyce{} uses a message passing
parallel\index{parallel!message passing} implementation, allowing it to run
efficiently on a variety of parallel computing platforms.  These
include serial, shared-memory\index{parallel!shared-memory} and
distributed-memory parallel\index{parallel!distributed-memory}.
Careful attention has been paid to the
specific nature of circuit-simulation problems to ensure optimal parallel
efficiency\index{parallel!efficiency}, even as the number of processors increases.

\subsection{Differential-Algebraic Equation (DAE) formulation}
\Xyce{} has been designed to use a DAE formulation.  Among other advantages, 
this has the benefit of allowing the device models to be nearly independent 
of the type analysis to be performed, and allows a lot of encapsulation between
the models and the solver layers of the source code.  In a SPICE-based code,
new device functions are created for each type of analysis, such as transient 
and AC analysis.  With \Xyce{}'s DAE implementation, this is not necessary.
The same device load functions can be used for all analysis types, resulting in
faster development time for new types of analysis.

\subsection{Device Model Support}
The \Xyce{} development team continually adds new device models to \Xyce{} to meet
the needs of Sandia users.  This includes the full set of models that can be found in 
most SPICE-based codes.  For 
current device availability, consult The \Xyce{} Reference Guide\ReferenceGuide{}.

\section{Reference Guide}
\label{Reference_Guide}
The \Xyce{} User's Guide companion document, the \Xyce{} Reference Guide\ReferenceGuide{},
contains detailed information including a
netlist reference for \Xyce{}-supported input-file commands and elements; a command line reference, which describes the available command line arguments; and quick-references for 
users of other circuit codes, such as Orcad's PSpice~\cite{PSpiceUG:1998}.
\index{Reference Guide}
\index{PSpice}
\index{Users of other circuit codes}

\section{How to Use this Guide}
\label{HowTo_Guide}

This guide is designed to enable one to quickly find the information needed to use \Xyce{}.  It assumes familiarity with basic \index{Unix} Unix-type commands, and how Unix manages applications and files to perform routine tasks (e.g., starting applications, opening files, and saving work).

\subsubsection{Typographical conventions}
Table~\ref{typog} defines the typographical conventions used in this guide.

\begin{table}[htbp]
    \caption{\Xyce{} typographical conventions.}\label{typog}
  \begin{tabularx}{\linewidth}{|Y|Y|Y|}
    \rowcolor{XyceDarkBlue} \color{white}\bf Notation & \color{white}\bf
    Example & \color{white}\bf Description \\ \hline

    \texttt{Typewriter text} & \texttt{mpirun -np 4}
    & Commands entered
    from the keyboard on the command line or text entered in a netlist. \\
    \hline

    \textrmb{Bold Roman Font} & Set nominal temperature using the
    \textrmb{TNOM} option. & SPICE-type parameters used in models, etc. \\
    \hline

    \cellcolor[gray]{0.75} Gray Shaded Text & \cellcolor[gray]{0.75} DEBUGLEVEL
    & Feature that is designed primarily for use by \Xyce{}
    developers. \\ \hline

    \texttt{[text in brackets]} & \texttt{Xyce [options] <netlist>} & Optional parameters. \\ \hline

    \texttt{<text in angle brackets>} & \texttt{Xyce [options] <netlist>} &
    Parameters to be inserted by the user. \\ \hline

    \texttt{<object with asterisk>*} & \texttt{K1 <ind. 1> [<ind. n>*]} &
    Parameter that may be multiply specified. \\ \hline

    \texttt{<TEXT1|TEXT2>}&
    \texttt{.PRINT TRAN}
    \verb-+     DELIMITER=<TAB|COMMA>- & Parameters that may only take specified values. \\ \hline

  \end{tabularx}
\end{table}
% END of Xyce_UG_ch01.tex ************
