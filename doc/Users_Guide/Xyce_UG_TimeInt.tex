% Sandia National Laboratories is a multimission laboratory managed and
% operated by National Technology & Engineering Solutions of Sandia, LLC, a
% wholly owned subsidiary of Honeywell International Inc., for the U.S.
% Department of Energy’s National Nuclear Security Administration under
% contract DE-NA0003525.

% Copyright 2002-2021 National Technology & Engineering Solutions of Sandia,
% LLC (NTESS).

%%-------------------------------------------------------------------------
%% Purpose        : Time integration and MPDE info
%% Special Notes  : Graphic files (pdf format) work with pdflatex.  To use
%%                  LaTeX, we need to use postcript versions.  
%% Creator        : Richard Schiek, Electrical and Microsystem Simulation, SNL
%% Creation Date  : {05/16/2007}
%%
%%-------------------------------------------------------------------------

\chapter{Time Integration and Multi-time Partial Differential Equations (MPDE)}
\label{TimeInt}
\index{Time integration}
\index{MPDE}
\index{DAE}

\chapteroverview{Chapter Overview}
{
This chapter provides an overview and guidance on the time integration algorithms in \Xyce{},
Different time integration options can significantly effect simulation performance and accuracy.  
Additionally, this chapter provides an  overview and using instructions for the 
new Multi-Time Partial Differential Equation (MPDE) capability in \Xyce{}.
This chapter includes the following sections:
\begin{XyceItemize}
  \item Section~\ref{TimeInt}, {\em Differential Algebraic Equation (DAE) Time Integration}
  \item Section~\ref{MPDE_Overview}, {\em Multi-time Partial Differential Equations (MPDE) Overview}
  \item Section~\ref{MPDE_Usage}, {\em MPDE Usage}
\end{XyceItemize}
}

\section{Differential Algebraic Equation (DAE) Time Integration}
\label{TimeInt_Discussion}
\Xyce{} version 4.0 includes a major upgrade to the time integration capability.  
This upgrade will result in transient simulations running more accurately and with 
substantially fewer computed time steps.  This improved accuracy and efficiency comes at
a small cost, as more data structures must be maintained in the code to support this 
capability.  However, this cost has been minimized and the benefits of the new integrator
should be substantial.

The old time integrator was similar to that of Spice3f5, and its descendents.  Spice-style
integrators are based around older ordinary-differential-equation (ODE) 
integrators, which assume every equation in the system of equations include time derivatives.
In circuit problems, there are some fraction of the system of equations are purely algebraic.
For example, the KCL equation for a voltage node that has only linear resistors attached
to it is an algebraic equation, not a differential equation.  Systems of equations
that include some algebraic constraints are referred to as differential algebraic
equations (DAE), and such systems (as can be found in circuit simulation) are better served by 
time integrators that have been specifically designed for DAE's.  

Part of the reason for this is numerical stability.  Systems of DAE's are often characterized
by the DAE index, which is defined as ``the minimum number of times that all or part of [the DAE] 
must be differentiated with respect to $t$'' in order to reduce it to an ODE\cite{Petzold:1996}. 
A system of index=0 is an ODE system.  Many circuits have an index of 1 or more, and 
integrated MOSFET circuits usually have an index of at least 2~\cite{Brachtendorf1}.  This is important
because higher index systems are only stable and accurate for higher orders of 
integration, and are also very sensitive to initial conditions.
In contrast, an ODE (index=0) integrator has the luxury of assuming that any implicit 
time integration scheme will be stable.

\subsection{Solver Options and Guidance}

In general, from a user point of view, the switch to the new time integrator should be transparent,
and \Xyce{} version 4.0 should behave the same with previous \Xyce{} releases.  
Also, over the long term, the new time integrator should be a significant improvement over the old one.  
However, there
will be some rare exception circuits in which the new time integrator will have more difficulty than
the old one.  This is mostly due to the fact that the new time integrator hasn't been 
in the code long enough to be completely mature.  If a transient circuit that ran fine in previous 
releases does not run successfully in \Xyce{} version 4.0, there are several options available.

\begin{XyceItemize}
  \item Loosen the time integrator tolerances.  The default absolute and relative truncation error tolerances are the same as they were before, with the default abstol=1.0e-6 and the default reltol=1.0e-2.  The new time integrator will generally get accurate results with looser tolerances, because the integration scheme is inherently more accurate and less subject to numerical dispersion.  If a simulation exits with a time-step-too-small error, try using larger numbers for abstol and/or reltol.  For example:  \texttt{.options timeint abstol=1e-4 reltol=1e-2}.
  \item Change the maximum order.  In the old time integrator, the maximum integration order by default was 1.  In the new time integrator, the maximum order is, by default, 5.  In some cases, this will make the simulation less robust. It is possible to set the maximum order=1, in the netlist by adding \texttt{.options timeint maxord=1}.
\end{XyceItemize}


\section{Multi-time Partial Differential Equations (MPDE) Overview}
\label{MPDE_Overview}

\Xyce{} version 4.0 includes a new analysis option specially designed
for circuits with two disparate time scales.  Normally, when a circuit
has multiple time scales, the time integrator must take many small steps
to fully resolve the fast time scale which can greatly slow simulation
progress and reduce accuracy.  Now, \Xyce{} can discretize the system on
the fast time scale converting the fast time DAE to a multi-time partial
differential equation, hence MPDE.~\cite{meiMpdeDAC2004}  This dramatically
reduces the number of steps the time integrator must take resulting in a faster
and more accurate simulation.  

In general, best results will be obtained when there is a large
difference between the fast and slow time scales.  The quality of the
solution of the fast time scale depends on the number of points used
in the fast time discretization which also impacts simulation
performance. Typically, if one needs $n$ points to accurately discretize
the fast time scale, then the fast and slow time scales should be
separated by at least a factor of $n$, {\em i.e.} if one needed 10
points to represent the fast time solution, then best performance would
be found when the fast and slow time scales differ by at lest a factor
of 10.  In an MPDE analysis, \Xyce{} is effectively modeling $n$ versions of 
the simulated circuit, so memory usage will grow proportionally with the number 
of points in the fast time domain.

\section{MPDE Usage}
\label{MPDE_Usage}

To use MPDE analysis in a simulation, place the following option line in the 
netlist:

\noindent\texttt{.OPTIONS MPDE OSCSRC=<v1,v2...> IC=<1,2>} [other optional switches]

\begin{XyceItemize}

\item \texttt{OSCSRC=...} A list of voltage and, or current sources that are 
to be considered as changing on the fast time scale.  Typically these will 
be all the sources that change at or more quickly than the fast time scale.

\item \texttt{IC=[1,2]} Specifies the method used to calculate the
initial conditions for the MPDE problem.  1 uses the {\em Sawtooth}
algorithm while 2 uses an initial transient run for the initial
condition.  During the initial condition calculation, the slow sources
are deactivated so one is integrating only along the fast time axis.

\item \texttt{N2=integer} Specifies the number of equally spaced points 
to use in discretizing the fast time scale.  This option can only be used
exclusive of the \texttt{AUTON2} and \texttt{AUTON2MAX} options.

\item \texttt{AUTON2=[true | false]}  False by default.  If it is set to
true, then an initial transient run is done to generate a {\em mesh} of
time points for the fast time scale.  The belief here is that the time
integrator can do a better job picking out where it needs more points to
accurately describe the fast time solution.   Note: you can use
\texttt{AUTON2} with \texttt{IC=1} or texttt{IC=2}.  If you use it with
\texttt{IC=1}, then the time mesh is just used for the set of points the
for the Sawtooth initial conditions.  This option can only be used exclusive 
to the \texttt{N2} option.

\item \texttt{AUTON2MAX=integer}  This sets the maximum number of fast
time points that will be kept from the initial transient run so one has
some control over the size of the simulation.  If the initial transient
run produces 2,000 points and this is set at 50, \Xyce{} will uniformly
sample points from the solution set for the MPDE simulation. This option
can only be used exclusive to the \texttt{N2} option.

\item \texttt{STARTUPPERIODS=integer}  This is the number of fast time
periods that \Xyce{} should integrate through using normal transient analysis
before trying to generate initial conditions for the MPDE analysis. 

\item \texttt{diff=[0,1]}  Specifies the differentiation technique to use in 
calculating time derivatives on the fast time scale.  0 uses backward 
differences for the fast time differentiation while 1 uses central differences.

\end{XyceItemize}

As an example here are two valid options lines for MPDE:


\Example{\texttt{.options MPDE IC=1 N2=21 OSCSRC=Vsine}}
\Example{\texttt{.options MPDE IC=2 AUTON2=true AUTON2MAX=50 OSCSRC=VIN1,VIN2}}


%%% Local Variables:
%%% mode: latex
%%% End:

%% END of Xyce_UG_TimeInt.tex ************
